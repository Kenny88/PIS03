% LaTeX 2.09 file
%
% K.Kodama 1997.11.29
% kodama@kobe-kosen.ac.jp
% 

\documentstyle{article}

%%% Paper size.
\newlength{\pageheight}
\newlength{\pagewidth}

%%%%resetsize: Set \textheight and \textwidth.
%% 
%% Note. 
%% Before use this, set \pageheight,\pagewidth,
%%	\topmargin,\headheight,\headsep,\footskip,
%%	\oddsidemargin and \evensidemargin.
\def\resetsize{
	%% height
	\setlength{\textheight}{\pageheight}
	\addtolength{\textheight}{-2in}
	\addtolength{\textheight}{-2\topmargin}
	\addtolength{\textheight}{-\headheight}
	\addtolength{\textheight}{-\headsep}
	\addtolength{\textheight}{-\footskip}
	%% width
	\setlength{\textwidth}{\pagewidth}
	\addtolength{\textwidth}{-2in}
	\addtolength{\textwidth}{-2\oddsidemargin}
}

%%%% Set page size: A4
\def\paperAfour{
\setlength{\pagewidth}{210mm}
\setlength{\pageheight}{297mm}
	%%%% height
	\topmargin 0pt
	\headheight 0pt %20pt%12pt
	\headsep 0mm %25pt
	\footskip 0pt
	%%%% width
	\oddsidemargin 0pt
	\evensidemargin 0pt
\resetsize
}


\paperAfour
\setlength{\parindent}{0pt}
\binoppenalty=0
\relpenalty=0

\begin{document}

Table of polynomials (up to 10 crossings) \\
{K.Kodama (kodama@kobe-kosen.ac.jp)\\}

\tableofcontents


\section{Alexander polynomial}
k3c1: $ 1-1+1 $ 

k4c1: $ -1+3-1 $ 

k5c1: $ 1-1+1-1+1 $ 

k5c2: $ 2-3+2 $ 

k6c1: $ -2+5-2 $ 

k6c2: $ -1+3-3+3-1 $ 

k6c3: $ 1-3+5-3+1 $ 

k7c1: $ 1-1+1-1+1-1+1 $ 

k7c2: $ 3-5+3 $ 

k7c3: $ 2-3+3-3+2 $ 

k7c4: $ 4-7+4 $ 

k7c5: $ 2-4+5-4+2 $ 

k7c6: $ -1+5-7+5-1 $ 

k7c7: $ 1-5+9-5+1 $ 

k8c1: $ -3+7-3 $ 

k8c2: $ -1+3-3+3-3+3-1 $ 

k8c3: $ -4+9-4 $ 

k8c4: $ -2+5-5+5-2 $ 

k8c5: $ -1+3-4+5-4+3-1 $ 

k8c6: $ -2+6-7+6-2 $ 

k8c7: $ 1-3+5-5+5-3+1 $ 

k8c8: $ 2-6+9-6+2 $ 

k8c9: $ -1+3-5+7-5+3-1 $ 

k8c10: $ 1-3+6-7+6-3+1 $ 

k8c11: $ -2+7-9+7-2 $ 

k8c12: $ 1-7+13-7+1 $ 

k8c13: $ 2-7+11-7+2 $ 

k8c14: $ -2+8-11+8-2 $ 

k8c15: $ 3-8+11-8+3 $ 

k8c16: $ 1-4+8-9+8-4+1 $ 

k8c17: $ -1+4-8+11-8+4-1 $ 

k8c18: $ -1+5-10+13-10+5-1 $ 

k8c19: $ 1-1+0+1+0-1+1 $ 

k8c20: $ 1-2+3-2+1 $ 

k8c21: $ -1+4-5+4-1 $ 

k9c1: $ 1-1+1-1+1-1+1-1+1 $ 

k9c2: $ 4-7+4 $ 

k9c3: $ 2-3+3-3+3-3+2 $ 

k9c4: $ 3-5+5-5+3 $ 

k9c5: $ 6-11+6 $ 

k9c6: $ 2-4+5-5+5-4+2 $ 

k9c7: $ 3-7+9-7+3 $ 

k9c8: $ -2+8-11+8-2 $ 

k9c9: $ 2-4+6-7+6-4+2 $ 

k9c10: $ 4-8+9-8+4 $ 

k9c11: $ -1+5-7+7-7+5-1 $ 

k9c12: $ -2+9-13+9-2 $ 

k9c13: $ 4-9+11-9+4 $ 

k9c14: $ 2-9+15-9+2 $ 

k9c15: $ -2+10-15+10-2 $ 

k9c16: $ 2-5+8-9+8-5+2 $ 

k9c17: $ 1-5+9-9+9-5+1 $ 

k9c18: $ 4-10+13-10+4 $ 

k9c19: $ 2-10+17-10+2 $ 

k9c20: $ -1+5-9+11-9+5-1 $ 

k9c21: $ -2+11-17+11-2 $ 

k9c22: $ 1-5+10-11+10-5+1 $ 

k9c23: $ 4-11+15-11+4 $ 

k9c24: $ -1+5-10+13-10+5-1 $ 

k9c25: $ -3+12-17+12-3 $ 

k9c26: $ 1-5+11-13+11-5+1 $ 

k9c27: $ -1+5-11+15-11+5-1 $ 

k9c28: $ 1-5+12-15+12-5+1 $ 

k9c29: $ 1-5+12-15+12-5+1 $ 

k9c30: $ -1+5-12+17-12+5-1 $ 

k9c31: $ 1-5+13-17+13-5+1 $ 

k9c32: $ 1-6+14-17+14-6+1 $ 

k9c33: $ -1+6-14+19-14+6-1 $ 

k9c34: $ -1+6-16+23-16+6-1 $ 

k9c35: $ 7-13+7 $ 

k9c36: $ -1+5-8+9-8+5-1 $ 

k9c37: $ 2-11+19-11+2 $ 

k9c38: $ 5-14+19-14+5 $ 

k9c39: $ -3+14-21+14-3 $ 

k9c40: $ 1-7+18-23+18-7+1 $ 

k9c41: $ 3-12+19-12+3 $ 

k9c42: $ -1+2-1+2-1 $ 

k9c43: $ -1+3-2+1-2+3-1 $ 

k9c44: $ 1-4+7-4+1 $ 

k9c45: $ -1+6-9+6-1 $ 

k9c46: $ -2+5-2 $ 

k9c47: $ 1-4+6-5+6-4+1 $ 

k9c48: $ -1+7-11+7-1 $ 

k9c49: $ 3-6+7-6+3 $ 

k10c1: $ -4+9-4 $ 

k10c2: $ -1+3-3+3-3+3-3+3-1 $ 

k10c3: $ -6+13-6 $ 

k10c4: $ -3+7-7+7-3 $ 

k10c5: $ 1-3+5-5+5-5+5-3+1 $ 

k10c6: $ -2+6-7+7-7+6-2 $ 

k10c7: $ -3+11-15+11-3 $ 

k10c8: $ -2+5-5+5-5+5-2 $ 

k10c9: $ -1+3-5+7-7+7-5+3-1 $ 

k10c10: $ 3-11+17-11+3 $ 

k10c11: $ -4+11-13+11-4 $ 

k10c12: $ 2-6+10-11+10-6+2 $ 

k10c13: $ 2-13+23-13+2 $ 

k10c14: $ -2+8-12+13-12+8-2 $ 

k10c15: $ 2-6+9-9+9-6+2 $ 

k10c16: $ -4+12-15+12-4 $ 

k10c17: $ 1-3+5-7+9-7+5-3+1 $ 

k10c18: $ -4+14-19+14-4 $ 

k10c19: $ 2-7+11-11+11-7+2 $ 

k10c20: $ -3+9-11+9-3 $ 

k10c21: $ -2+7-9+9-9+7-2 $ 

k10c22: $ -2+6-10+13-10+6-2 $ 

k10c23: $ 2-7+13-15+13-7+2 $ 

k10c24: $ -4+14-19+14-4 $ 

k10c25: $ -2+8-14+17-14+8-2 $ 

k10c26: $ -2+7-13+17-13+7-2 $ 

k10c27: $ 2-8+16-19+16-8+2 $ 

k10c28: $ 4-13+19-13+4 $ 

k10c29: $ 1-7+15-17+15-7+1 $ 

k10c30: $ -4+17-25+17-4 $ 

k10c31: $ 4-14+21-14+4 $ 

k10c32: $ -2+8-15+19-15+8-2 $ 

k10c33: $ 4-16+25-16+4 $ 

k10c34: $ 3-9+13-9+3 $ 

k10c35: $ 2-12+21-12+2 $ 

k10c36: $ -3+13-19+13-3 $ 

k10c37: $ 4-13+19-13+4 $ 

k10c38: $ -4+15-21+15-4 $ 

k10c39: $ -2+8-13+15-13+8-2 $ 

k10c40: $ 2-8+17-21+17-8+2 $ 

k10c41: $ 1-7+17-21+17-7+1 $ 

k10c42: $ -1+7-19+27-19+7-1 $ 

k10c43: $ -1+7-17+23-17+7-1 $ 

k10c44: $ 1-7+19-25+19-7+1 $ 

k10c45: $ -1+7-21+31-21+7-1 $ 

k10c46: $ -1+3-4+5-5+5-4+3-1 $ 

k10c47: $ 1-3+6-7+7-7+6-3+1 $ 

k10c48: $ 1-3+6-9+11-9+6-3+1 $ 

k10c49: $ 3-8+12-13+12-8+3 $ 

k10c50: $ -2+7-11+13-11+7-2 $ 

k10c51: $ 2-7+15-19+15-7+2 $ 

k10c52: $ 2-7+13-15+13-7+2 $ 

k10c53: $ 6-18+25-18+6 $ 

k10c54: $ 2-6+10-11+10-6+2 $ 

k10c55: $ 5-15+21-15+5 $ 

k10c56: $ -2+8-14+17-14+8-2 $ 

k10c57: $ 2-8+18-23+18-8+2 $ 

k10c58: $ 3-16+27-16+3 $ 

k10c59: $ 1-7+18-23+18-7+1 $ 

k10c60: $ -1+7-20+29-20+7-1 $ 

k10c61: $ -2+5-6+7-6+5-2 $ 

k10c62: $ 1-3+6-8+9-8+6-3+1 $ 

k10c63: $ 5-14+19-14+5 $ 

k10c64: $ -1+3-6+10-11+10-6+3-1 $ 

k10c65: $ 2-7+14-17+14-7+2 $ 

k10c66: $ 3-9+16-19+16-9+3 $ 

k10c67: $ -4+16-23+16-4 $ 

k10c68: $ 4-14+21-14+4 $ 

k10c69: $ 1-7+21-29+21-7+1 $ 

k10c70: $ 1-7+16-19+16-7+1 $ 

k10c71: $ -1+7-18+25-18+7-1 $ 

k10c72: $ -2+9-16+19-16+9-2 $ 

k10c73: $ 1-7+20-27+20-7+1 $ 

k10c74: $ -4+16-23+16-4 $ 

k10c75: $ -1+7-19+27-19+7-1 $ 

k10c76: $ -2+7-12+15-12+7-2 $ 

k10c77: $ 2-7+14-17+14-7+2 $ 

k10c78: $ -1+7-16+21-16+7-1 $ 

k10c79: $ 1-3+7-12+15-12+7-3+1 $ 

k10c80: $ 3-9+15-17+15-9+3 $ 

k10c81: $ -1+8-20+27-20+8-1 $ 

k10c82: $ -1+4-8+12-13+12-8+4-1 $ 

k10c83: $ -2+9-19+25-19+9-2 $ 

k10c84: $ 2-9+20-25+20-9+2 $ 

k10c85: $ 1-4+8-10+11-10+8-4+1 $ 

k10c86: $ 2-9+19-23+19-9+2 $ 

k10c87: $ -2+9-18+23-18+9-2 $ 

k10c88: $ -1+8-24+35-24+8-1 $ 

k10c89: $ 1-8+24-33+24-8+1 $ 

k10c90: $ -2+8-17+23-17+8-2 $ 

k10c91: $ 1-4+9-14+17-14+9-4+1 $ 

k10c92: $ -2+10-20+25-20+10-2 $ 

k10c93: $ 2-8+15-17+15-8+2 $ 

k10c94: $ -1+4-9+14-15+14-9+4-1 $ 

k10c95: $ 2-9+21-27+21-9+2 $ 

k10c96: $ -1+7-22+33-22+7-1 $ 

k10c97: $ -5+22-33+22-5 $ 

k10c98: $ -2+9-18+23-18+9-2 $ 

k10c99: $ 1-4+10-16+19-16+10-4+1 $ 

k10c100: $ 1-4+9-12+13-12+9-4+1 $ 

k10c101: $ 7-21+29-21+7 $ 

k10c102: $ -2+8-16+21-16+8-2 $ 

k10c103: $ 2-8+17-21+17-8+2 $ 

k10c104: $ 1-4+9-15+19-15+9-4+1 $ 

k10c105: $ 1-8+22-29+22-8+1 $ 

k10c106: $ -1+4-9+15-17+15-9+4-1 $ 

k10c107: $ -1+8-22+31-22+8-1 $ 

k10c108: $ 2-8+14-15+14-8+2 $ 

k10c109: $ 1-4+10-17+21-17+10-4+1 $ 

k10c110: $ 1-8+20-25+20-8+1 $ 

k10c111: $ -2+9-17+21-17+9-2 $ 

k10c112: $ -1+5-11+17-19+17-11+5-1 $ 

k10c113: $ 2-11+26-33+26-11+2 $ 

k10c114: $ -2+10-21+27-21+10-2 $ 

k10c115: $ -1+9-26+37-26+9-1 $ 

k10c116: $ -1+5-12+19-21+19-12+5-1 $ 

k10c117: $ 2-10+24-31+24-10+2 $ 

k10c118: $ 1-5+12-19+23-19+12-5+1 $ 

k10c119: $ -2+10-23+31-23+10-2 $ 

k10c120: $ 8-26+37-26+8 $ 

k10c121: $ 2-11+27-35+27-11+2 $ 

k10c122: $ -2+11-24+31-24+11-2 $ 

k10c123: $ 1-6+15-24+29-24+15-6+1 $ 

k10c124: $ 1-1+0+1-1+1+0-1+1 $ 

k10c125: $ 1-2+2-1+2-2+1 $ 

k10c126: $ 1-2+4-5+4-2+1 $ 

k10c127: $ -1+4-6+7-6+4-1 $ 

k10c128: $ 2-3+1+1+1-3+2 $ 

k10c129: $ 2-6+9-6+2 $ 

k10c130: $ 2-4+5-4+2 $ 

k10c131: $ -2+8-11+8-2 $ 

k10c132: $ 1-1+1-1+1 $ 

k10c133: $ -1+5-7+5-1 $ 

k10c134: $ 2-4+4-3+4-4+2 $ 

k10c135: $ 3-9+13-9+3 $ 

k10c136: $ -1+4-5+4-1 $ 

k10c137: $ 1-6+11-6+1 $ 

k10c138: $ 1-5+8-7+8-5+1 $ 

k10c139: $ 1-1+0+2-3+2+0-1+1 $ 

k10c140: $ 1-2+3-2+1 $ 

k10c141: $ -1+3-4+5-4+3-1 $ 

k10c142: $ 2-3+2-1+2-3+2 $ 

k10c143: $ 1-3+6-7+6-3+1 $ 

k10c144: $ -3+10-13+10-3 $ 

k10c145: $ 1+1-3+1+1 $ 

k10c146: $ 2-8+13-8+2 $ 

k10c147: $ -2+7-9+7-2 $ 

k10c148: $ 1-3+7-9+7-3+1 $ 

k10c149: $ -1+5-9+11-9+5-1 $ 

k10c150: $ -1+4-6+7-6+4-1 $ 

k10c151: $ 1-4+10-13+10-4+1 $ 

k10c152: $ 1-1-1+4-5+4-1-1+1 $ 

k10c153: $ 1-1-1+3-1-1+1 $ 

k10c154: $ 1+0-4+7-4+0+1 $ 

k10c155: $ -1+3-5+7-5+3-1 $ 

k10c156: $ 1-4+8-9+8-4+1 $ 

k10c157: $ -1+6-11+13-11+6-1 $ 

k10c158: $ -1+4-10+15-10+4-1 $ 

k10c159: $ 1-4+9-11+9-4+1 $ 

k10c160: $ -1+4-4+3-4+4-1 $ 

k10c161: $ 1+0-2+3-2+0+1 $ 

k10c162: $ 1+0-2+3-2+0+1 $ 

k10c163: $ -3+9-11+9-3 $ 

k10c164: $ 1-5+12-15+12-5+1 $ 

k10c165: $ 3-11+17-11+3 $ 

k10c166: $ -2+10-15+10-2 $ 


\newpage


\section{Conway polynomial}
k3c1: $ 1+z^{2} $ 

k4c1: $ 1-z^{2} $ 

k5c1: $ 1+3z^{2}+z^{4} $ 

k5c2: $ 1+2z^{2} $ 

k6c1: $ 1-2z^{2} $ 

k6c2: $ 1-z^{2}-z^{4} $ 

k6c3: $ 1+z^{2}+z^{4} $ 

k7c1: $ 1+6z^{2}+5z^{4}+z^{6} $ 

k7c2: $ 1+3z^{2} $ 

k7c3: $ 1+5z^{2}+2z^{4} $ 

k7c4: $ 1+4z^{2} $ 

k7c5: $ 1+4z^{2}+2z^{4} $ 

k7c6: $ 1+z^{2}-z^{4} $ 

k7c7: $ 1-z^{2}+z^{4} $ 

k8c1: $ 1-3z^{2} $ 

k8c2: $ 1-3z^{4}-z^{6} $ 

k8c3: $ 1-4z^{2} $ 

k8c4: $ 1-3z^{2}-2z^{4} $ 

k8c5: $ 1-z^{2}-3z^{4}-z^{6} $ 

k8c6: $ 1-2z^{2}-2z^{4} $ 

k8c7: $ 1+2z^{2}+3z^{4}+z^{6} $ 

k8c8: $ 1+2z^{2}+2z^{4} $ 

k8c9: $ 1-2z^{2}-3z^{4}-z^{6} $ 

k8c10: $ 1+3z^{2}+3z^{4}+z^{6} $ 

k8c11: $ 1-z^{2}-2z^{4} $ 

k8c12: $ 1-3z^{2}+z^{4} $ 

k8c13: $ 1+z^{2}+2z^{4} $ 

k8c14: $ 1-2z^{4} $ 

k8c15: $ 1+4z^{2}+3z^{4} $ 

k8c16: $ 1+z^{2}+2z^{4}+z^{6} $ 

k8c17: $ 1-z^{2}-2z^{4}-z^{6} $ 

k8c18: $ 1+z^{2}-z^{4}-z^{6} $ 

k8c19: $ 1+5z^{2}+5z^{4}+z^{6} $ 

k8c20: $ 1+2z^{2}+z^{4} $ 

k8c21: $ 1-z^{4} $ 

k9c1: $ 1+10z^{2}+15z^{4}+7z^{6}+z^{8} $ 

k9c2: $ 1+4z^{2} $ 

k9c3: $ 1+9z^{2}+9z^{4}+2z^{6} $ 

k9c4: $ 1+7z^{2}+3z^{4} $ 

k9c5: $ 1+6z^{2} $ 

k9c6: $ 1+7z^{2}+8z^{4}+2z^{6} $ 

k9c7: $ 1+5z^{2}+3z^{4} $ 

k9c8: $ 1-2z^{4} $ 

k9c9: $ 1+8z^{2}+8z^{4}+2z^{6} $ 

k9c10: $ 1+8z^{2}+4z^{4} $ 

k9c11: $ 1+4z^{2}-z^{4}-z^{6} $ 

k9c12: $ 1+z^{2}-2z^{4} $ 

k9c13: $ 1+7z^{2}+4z^{4} $ 

k9c14: $ 1-z^{2}+2z^{4} $ 

k9c15: $ 1+2z^{2}-2z^{4} $ 

k9c16: $ 1+6z^{2}+7z^{4}+2z^{6} $ 

k9c17: $ 1-2z^{2}+z^{4}+z^{6} $ 

k9c18: $ 1+6z^{2}+4z^{4} $ 

k9c19: $ 1-2z^{2}+2z^{4} $ 

k9c20: $ 1+2z^{2}-z^{4}-z^{6} $ 

k9c21: $ 1+3z^{2}-2z^{4} $ 

k9c22: $ 1-z^{2}+z^{4}+z^{6} $ 

k9c23: $ 1+5z^{2}+4z^{4} $ 

k9c24: $ 1+z^{2}-z^{4}-z^{6} $ 

k9c25: $ 1-3z^{4} $ 

k9c26: $ 1+z^{4}+z^{6} $ 

k9c27: $ 1-z^{4}-z^{6} $ 

k9c28: $ 1+z^{2}+z^{4}+z^{6} $ 

k9c29: $ 1+z^{2}+z^{4}+z^{6} $ 

k9c30: $ 1-z^{2}-z^{4}-z^{6} $ 

k9c31: $ 1+2z^{2}+z^{4}+z^{6} $ 

k9c32: $ 1-z^{2}+z^{6} $ 

k9c33: $ 1+z^{2}-z^{6} $ 

k9c34: $ 1-z^{2}-z^{6} $ 

k9c35: $ 1+7z^{2} $ 

k9c36: $ 1+3z^{2}-z^{4}-z^{6} $ 

k9c37: $ 1-3z^{2}+2z^{4} $ 

k9c38: $ 1+6z^{2}+5z^{4} $ 

k9c39: $ 1+2z^{2}-3z^{4} $ 

k9c40: $ 1-z^{2}-z^{4}+z^{6} $ 

k9c41: $ 1+3z^{4} $ 

k9c42: $ 1-2z^{2}-z^{4} $ 

k9c43: $ 1+z^{2}-3z^{4}-z^{6} $ 

k9c44: $ 1+z^{4} $ 

k9c45: $ 1+2z^{2}-z^{4} $ 

k9c46: $ 1-2z^{2} $ 

k9c47: $ 1-z^{2}+2z^{4}+z^{6} $ 

k9c48: $ 1+3z^{2}-z^{4} $ 

k9c49: $ 1+6z^{2}+3z^{4} $ 

k10c1: $ 1-4z^{2} $ 

k10c2: $ 1+2z^{2}-5z^{4}-5z^{6}-z^{8} $ 

k10c3: $ 1-6z^{2} $ 

k10c4: $ 1-5z^{2}-3z^{4} $ 

k10c5: $ 1+4z^{2}+7z^{4}+5z^{6}+z^{8} $ 

k10c6: $ 1-z^{2}-6z^{4}-2z^{6} $ 

k10c7: $ 1-z^{2}-3z^{4} $ 

k10c8: $ 1-3z^{2}-7z^{4}-2z^{6} $ 

k10c9: $ 1-2z^{2}-7z^{4}-5z^{6}-z^{8} $ 

k10c10: $ 1+z^{2}+3z^{4} $ 

k10c11: $ 1-5z^{2}-4z^{4} $ 

k10c12: $ 1+4z^{2}+6z^{4}+2z^{6} $ 

k10c13: $ 1-5z^{2}+2z^{4} $ 

k10c14: $ 1+2z^{2}-4z^{4}-2z^{6} $ 

k10c15: $ 1+3z^{2}+6z^{4}+2z^{6} $ 

k10c16: $ 1-4z^{2}-4z^{4} $ 

k10c17: $ 1+2z^{2}+7z^{4}+5z^{6}+z^{8} $ 

k10c18: $ 1-2z^{2}-4z^{4} $ 

k10c19: $ 1+z^{2}+5z^{4}+2z^{6} $ 

k10c20: $ 1-3z^{2}-3z^{4} $ 

k10c21: $ 1+z^{2}-5z^{4}-2z^{6} $ 

k10c22: $ 1-4z^{2}-6z^{4}-2z^{6} $ 

k10c23: $ 1+3z^{2}+5z^{4}+2z^{6} $ 

k10c24: $ 1-2z^{2}-4z^{4} $ 

k10c25: $ 1-4z^{4}-2z^{6} $ 

k10c26: $ 1-3z^{2}-5z^{4}-2z^{6} $ 

k10c27: $ 1+2z^{2}+4z^{4}+2z^{6} $ 

k10c28: $ 1+3z^{2}+4z^{4} $ 

k10c29: $ 1-4z^{2}-z^{4}+z^{6} $ 

k10c30: $ 1+z^{2}-4z^{4} $ 

k10c31: $ 1+2z^{2}+4z^{4} $ 

k10c32: $ 1-z^{2}-4z^{4}-2z^{6} $ 

k10c33: $ 1+4z^{4} $ 

k10c34: $ 1+3z^{2}+3z^{4} $ 

k10c35: $ 1-4z^{2}+2z^{4} $ 

k10c36: $ 1+z^{2}-3z^{4} $ 

k10c37: $ 1+3z^{2}+4z^{4} $ 

k10c38: $ 1-z^{2}-4z^{4} $ 

k10c39: $ 1+z^{2}-4z^{4}-2z^{6} $ 

k10c40: $ 1+3z^{2}+4z^{4}+2z^{6} $ 

k10c41: $ 1-2z^{2}-z^{4}+z^{6} $ 

k10c42: $ 1+z^{4}-z^{6} $ 

k10c43: $ 1+2z^{2}+z^{4}-z^{6} $ 

k10c44: $ 1-z^{4}+z^{6} $ 

k10c45: $ 1-2z^{2}+z^{4}-z^{6} $ 

k10c46: $ 1-6z^{4}-5z^{6}-z^{8} $ 

k10c47: $ 1+6z^{2}+8z^{4}+5z^{6}+z^{8} $ 

k10c48: $ 1+4z^{2}+8z^{4}+5z^{6}+z^{8} $ 

k10c49: $ 1+7z^{2}+10z^{4}+3z^{6} $ 

k10c50: $ 1-z^{2}-5z^{4}-2z^{6} $ 

k10c51: $ 1+5z^{2}+5z^{4}+2z^{6} $ 

k10c52: $ 1+3z^{2}+5z^{4}+2z^{6} $ 

k10c53: $ 1+6z^{2}+6z^{4} $ 

k10c54: $ 1+4z^{2}+6z^{4}+2z^{6} $ 

k10c55: $ 1+5z^{2}+5z^{4} $ 

k10c56: $ 1-4z^{4}-2z^{6} $ 

k10c57: $ 1+4z^{2}+4z^{4}+2z^{6} $ 

k10c58: $ 1-4z^{2}+3z^{4} $ 

k10c59: $ 1-z^{2}-z^{4}+z^{6} $ 

k10c60: $ 1-z^{2}+z^{4}-z^{6} $ 

k10c61: $ 1-4z^{2}-7z^{4}-2z^{6} $ 

k10c62: $ 1+5z^{2}+8z^{4}+5z^{6}+z^{8} $ 

k10c63: $ 1+6z^{2}+5z^{4} $ 

k10c64: $ 1-3z^{2}-8z^{4}-5z^{6}-z^{8} $ 

k10c65: $ 1+4z^{2}+5z^{4}+2z^{6} $ 

k10c66: $ 1+7z^{2}+9z^{4}+3z^{6} $ 

k10c67: $ 1-4z^{4} $ 

k10c68: $ 1+2z^{2}+4z^{4} $ 

k10c69: $ 1+2z^{2}-z^{4}+z^{6} $ 

k10c70: $ 1-3z^{2}-z^{4}+z^{6} $ 

k10c71: $ 1+z^{2}+z^{4}-z^{6} $ 

k10c72: $ 1+2z^{2}-3z^{4}-2z^{6} $ 

k10c73: $ 1+z^{2}-z^{4}+z^{6} $ 

k10c74: $ 1-4z^{4} $ 

k10c75: $ 1+z^{4}-z^{6} $ 

k10c76: $ 1-2z^{2}-5z^{4}-2z^{6} $ 

k10c77: $ 1+4z^{2}+5z^{4}+2z^{6} $ 

k10c78: $ 1+3z^{2}+z^{4}-z^{6} $ 

k10c79: $ 1+5z^{2}+9z^{4}+5z^{6}+z^{8} $ 

k10c80: $ 1+6z^{2}+9z^{4}+3z^{6} $ 

k10c81: $ 1+3z^{2}+2z^{4}-z^{6} $ 

k10c82: $ 1-4z^{4}-4z^{6}-z^{8} $ 

k10c83: $ 1-z^{2}-3z^{4}-2z^{6} $ 

k10c84: $ 1+2z^{2}+3z^{4}+2z^{6} $ 

k10c85: $ 1+2z^{2}+4z^{4}+4z^{6}+z^{8} $ 

k10c86: $ 1+z^{2}+3z^{4}+2z^{6} $ 

k10c87: $ 1-3z^{4}-2z^{6} $ 

k10c88: $ 1-z^{2}+2z^{4}-z^{6} $ 

k10c89: $ 1+z^{2}-2z^{4}+z^{6} $ 

k10c90: $ 1-3z^{2}-4z^{4}-2z^{6} $ 

k10c91: $ 1+2z^{2}+5z^{4}+4z^{6}+z^{8} $ 

k10c92: $ 1+2z^{2}-2z^{4}-2z^{6} $ 

k10c93: $ 1+z^{2}+4z^{4}+2z^{6} $ 

k10c94: $ 1-2z^{2}-5z^{4}-4z^{6}-z^{8} $ 

k10c95: $ 1+3z^{2}+3z^{4}+2z^{6} $ 

k10c96: $ 1-3z^{2}+z^{4}-z^{6} $ 

k10c97: $ 1+2z^{2}-5z^{4} $ 

k10c98: $ 1-3z^{4}-2z^{6} $ 

k10c99: $ 1+4z^{2}+6z^{4}+4z^{6}+z^{8} $ 

k10c100: $ 1+4z^{2}+5z^{4}+4z^{6}+z^{8} $ 

k10c101: $ 1+7z^{2}+7z^{4} $ 

k10c102: $ 1-2z^{2}-4z^{4}-2z^{6} $ 

k10c103: $ 1+3z^{2}+4z^{4}+2z^{6} $ 

k10c104: $ 1+z^{2}+5z^{4}+4z^{6}+z^{8} $ 

k10c105: $ 1-z^{2}-2z^{4}+z^{6} $ 

k10c106: $ 1-z^{2}-5z^{4}-4z^{6}-z^{8} $ 

k10c107: $ 1+z^{2}+2z^{4}-z^{6} $ 

k10c108: $ 1+4z^{4}+2z^{6} $ 

k10c109: $ 1+3z^{2}+6z^{4}+4z^{6}+z^{8} $ 

k10c110: $ 1-3z^{2}-2z^{4}+z^{6} $ 

k10c111: $ 1+z^{2}-3z^{4}-2z^{6} $ 

k10c112: $ 1+2z^{2}-z^{4}-3z^{6}-z^{8} $ 

k10c113: $ 1+z^{4}+2z^{6} $ 

k10c114: $ 1+z^{2}-2z^{4}-2z^{6} $ 

k10c115: $ 1+z^{2}+3z^{4}-z^{6} $ 

k10c116: $ 1-2z^{4}-3z^{6}-z^{8} $ 

k10c117: $ 1+2z^{2}+2z^{4}+2z^{6} $ 

k10c118: $ 1+2z^{4}+3z^{6}+z^{8} $ 

k10c119: $ 1-z^{2}-2z^{4}-2z^{6} $ 

k10c120: $ 1+6z^{2}+8z^{4} $ 

k10c121: $ 1+z^{2}+z^{4}+2z^{6} $ 

k10c122: $ 1+2z^{2}-z^{4}-2z^{6} $ 

k10c123: $ 1-2z^{2}-z^{4}+2z^{6}+z^{8} $ 

k10c124: $ 1+8z^{2}+14z^{4}+7z^{6}+z^{8} $ 

k10c125: $ 1+3z^{2}+4z^{4}+z^{6} $ 

k10c126: $ 1+5z^{2}+4z^{4}+z^{6} $ 

k10c127: $ 1+z^{2}-2z^{4}-z^{6} $ 

k10c128: $ 1+7z^{2}+9z^{4}+2z^{6} $ 

k10c129: $ 1+2z^{2}+2z^{4} $ 

k10c130: $ 1+4z^{2}+2z^{4} $ 

k10c131: $ 1-2z^{4} $ 

k10c132: $ 1+3z^{2}+z^{4} $ 

k10c133: $ 1+z^{2}-z^{4} $ 

k10c134: $ 1+6z^{2}+8z^{4}+2z^{6} $ 

k10c135: $ 1+3z^{2}+3z^{4} $ 

k10c136: $ 1-z^{4} $ 

k10c137: $ 1-2z^{2}+z^{4} $ 

k10c138: $ 1-3z^{2}+z^{4}+z^{6} $ 

k10c139: $ 1+9z^{2}+14z^{4}+7z^{6}+z^{8} $ 

k10c140: $ 1+2z^{2}+z^{4} $ 

k10c141: $ 1-z^{2}-3z^{4}-z^{6} $ 

k10c142: $ 1+8z^{2}+9z^{4}+2z^{6} $ 

k10c143: $ 1+3z^{2}+3z^{4}+z^{6} $ 

k10c144: $ 1-2z^{2}-3z^{4} $ 

k10c145: $ 1+5z^{2}+z^{4} $ 

k10c146: $ 1+2z^{4} $ 

k10c147: $ 1-z^{2}-2z^{4} $ 

k10c148: $ 1+4z^{2}+3z^{4}+z^{6} $ 

k10c149: $ 1+2z^{2}-z^{4}-z^{6} $ 

k10c150: $ 1+z^{2}-2z^{4}-z^{6} $ 

k10c151: $ 1+3z^{2}+2z^{4}+z^{6} $ 

k10c152: $ 1+7z^{2}+13z^{4}+7z^{6}+z^{8} $ 

k10c153: $ 1+4z^{2}+5z^{4}+z^{6} $ 

k10c154: $ 1+5z^{2}+6z^{4}+z^{6} $ 

k10c155: $ 1-2z^{2}-3z^{4}-z^{6} $ 

k10c156: $ 1+z^{2}+2z^{4}+z^{6} $ 

k10c157: $ 1+4z^{2}-z^{6} $ 

k10c158: $ 1-3z^{2}-2z^{4}-z^{6} $ 

k10c159: $ 1+2z^{2}+2z^{4}+z^{6} $ 

k10c160: $ 1+3z^{2}-2z^{4}-z^{6} $ 

k10c161: $ 1+7z^{2}+6z^{4}+z^{6} $ 

k10c162: $ 1+7z^{2}+6z^{4}+z^{6} $ 

k10c163: $ 1-3z^{2}-3z^{4} $ 

k10c164: $ 1+z^{2}+z^{4}+z^{6} $ 

k10c165: $ 1+z^{2}+3z^{4} $ 

k10c166: $ 1+2z^{2}-2z^{4} $ 


\newpage


\section{Jones polynomial}
$t V_{L+}-(1/t)V_{L-} = -(\sqrt{t}-1/\sqrt{t})V_{L0}$ \par
k3c1: $t+t^{3}-t^{4}$

k4c1: $t^{-2}-t^{-1}+1-t+t^{2}$

k5c1: $t^{2}+t^{4}-t^{5}+t^{6}-t^{7}$

k5c2: $t-t^{2}+2t^{3}-t^{4}+t^{5}-t^{6}$

k6c1: $t^{-2}-t^{-1}+2-2t+t^{2}-t^{3}+t^{4}$

k6c2: $t^{-1}-1+2t-2t^{2}+2t^{3}-2t^{4}+t^{5}$

k6c3: $-t^{-3}+2t^{-2}-2t^{-1}+3-2t+2t^{2}-t^{3}$

k7c1: $t^{3}+t^{5}-t^{6}+t^{7}-t^{8}+t^{9}-t^{10}$

k7c2: $t-t^{2}+2t^{3}-2t^{4}+2t^{5}-t^{6}+t^{7}-t^{8}$

k7c3: $-t^{-9}+t^{-8}-2t^{-7}+3t^{-6}-2t^{-5}+2t^{-4}-t^{-3}+t^{-2}$

k7c4: $-t^{-8}+t^{-7}-2t^{-6}+3t^{-5}-2t^{-4}+3t^{-3}-2t^{-2}+t^{-1}$

k7c5: $t^{2}-t^{3}+3t^{4}-3t^{5}+3t^{6}-3t^{7}+2t^{8}-t^{9}$

k7c6: $t^{-1}-2+3t-3t^{2}+4t^{3}-3t^{4}+2t^{5}-t^{6}$

k7c7: $t^{-4}-2t^{-3}+3t^{-2}-4t^{-1}+4-3t+3t^{2}-t^{3}$

k8c1: $t^{-2}-t^{-1}+2-2t+2t^{2}-2t^{3}+t^{4}-t^{5}+t^{6}$

k8c2: $1-t+2t^{2}-2t^{3}+3t^{4}-3t^{5}+2t^{6}-2t^{7}+t^{8}$

k8c3: $t^{-4}-t^{-3}+2t^{-2}-3t^{-1}+3-3t+2t^{2}-t^{3}+t^{4}$

k8c4: $t^{-3}-t^{-2}+2t^{-1}-3+3t-3t^{2}+3t^{3}-2t^{4}+t^{5}$

k8c5: $t^{-8}-2t^{-7}+3t^{-6}-4t^{-5}+3t^{-4}-3t^{-3}+3t^{-2}-t^{-1}+1$

k8c6: $t^{-1}-1+3t-4t^{2}+4t^{3}-4t^{4}+3t^{5}-2t^{6}+t^{7}$

k8c7: $-t^{-6}+2t^{-5}-3t^{-4}+4t^{-3}-4t^{-2}+4t^{-1}-2+2t-t^{2}$

k8c8: $-t^{-5}+2t^{-4}-3t^{-3}+4t^{-2}-4t^{-1}+5-3t+2t^{2}-t^{3}$

k8c9: $t^{-4}-2t^{-3}+3t^{-2}-4t^{-1}+5-4t+3t^{2}-2t^{3}+t^{4}$

k8c10: $-t^{-6}+2t^{-5}-4t^{-4}+5t^{-3}-4t^{-2}+5t^{-1}-3+2t-t^{2}$

k8c11: $t^{-1}-2+4t-4t^{2}+5t^{3}-5t^{4}+3t^{5}-2t^{6}+t^{7}$

k8c12: $t^{-4}-2t^{-3}+4t^{-2}-5t^{-1}+5-5t+4t^{2}-2t^{3}+t^{4}$

k8c13: $-t^{-5}+2t^{-4}-3t^{-3}+5t^{-2}-5t^{-1}+5-4t+3t^{2}-t^{3}$

k8c14: $t^{-1}-2+4t-5t^{2}+6t^{3}-5t^{4}+4t^{5}-3t^{6}+t^{7}$

k8c15: $t^{2}-2t^{3}+5t^{4}-5t^{5}+6t^{6}-6t^{7}+4t^{8}-3t^{9}+t^{10}$

k8c16: $-t^{-2}+3t^{-1}-4+6t-6t^{2}+6t^{3}-5t^{4}+3t^{5}-t^{6}$

k8c17: $t^{-4}-3t^{-3}+5t^{-2}-6t^{-1}+7-6t+5t^{2}-3t^{3}+t^{4}$

k8c18: $t^{-4}-4t^{-3}+6t^{-2}-7t^{-1}+9-7t+6t^{2}-4t^{3}+t^{4}$

k8c19: $-t^{-8}+t^{-5}+t^{-3}$

k8c20: $-t^{-1}+2-t+2t^{2}-t^{3}+t^{4}-t^{5}$

k8c21: $2t-2t^{2}+3t^{3}-3t^{4}+2t^{5}-2t^{6}+t^{7}$

k9c1: $t^{4}+t^{6}-t^{7}+t^{8}-t^{9}+t^{10}-t^{11}+t^{12}-t^{13}$

k9c2: $t-t^{2}+2t^{3}-2t^{4}+2t^{5}-2t^{6}+2t^{7}-t^{8}+t^{9}-t^{10}$

k9c3: $-t^{-12}+t^{-11}-2t^{-10}+3t^{-9}-3t^{-8}+3t^{-7}-2t^{-6}+2t^{-5}-t^{-4}+t^{-3}$

k9c4: $t^{2}-t^{3}+2t^{4}-3t^{5}+4t^{6}-3t^{7}+3t^{8}-2t^{9}+t^{10}-t^{11}$

k9c5: $-t^{-10}+t^{-9}-2t^{-8}+3t^{-7}-3t^{-6}+4t^{-5}-3t^{-4}+3t^{-3}-2t^{-2}+t^{-1}$

k9c6: $t^{3}-t^{4}+3t^{5}-3t^{6}+4t^{7}-5t^{8}+4t^{9}-3t^{10}+2t^{11}-t^{12}$

k9c7: $t^{2}-t^{3}+3t^{4}-4t^{5}+5t^{6}-5t^{7}+4t^{8}-3t^{9}+2t^{10}-t^{11}$

k9c8: $t^{-3}-2t^{-2}+3t^{-1}-4+5t-5t^{2}+5t^{3}-3t^{4}+2t^{5}-t^{6}$

k9c9: $t^{3}-t^{4}+3t^{5}-4t^{6}+5t^{7}-5t^{8}+5t^{9}-4t^{10}+2t^{11}-t^{12}$

k9c10: $-t^{-11}+t^{-10}-3t^{-9}+5t^{-8}-5t^{-7}+6t^{-6}-5t^{-5}+4t^{-4}-2t^{-3}+t^{-2}$

k9c11: $-t^{-9}+2t^{-8}-4t^{-7}+5t^{-6}-5t^{-5}+6t^{-4}-4t^{-3}+3t^{-2}-2t^{-1}+1$

k9c12: $t^{-1}-2+4t-5t^{2}+6t^{3}-6t^{4}+5t^{5}-3t^{6}+2t^{7}-t^{8}$

k9c13: $-t^{-11}+2t^{-10}-4t^{-9}+5t^{-8}-6t^{-7}+7t^{-6}-5t^{-5}+4t^{-4}-2t^{-3}+t^{-2}$

k9c14: $t^{-6}-2t^{-5}+3t^{-4}-5t^{-3}+6t^{-2}-6t^{-1}+6-4t+3t^{2}-t^{3}$

k9c15: $-t^{-8}+2t^{-7}-4t^{-6}+6t^{-5}-6t^{-4}+7t^{-3}-6t^{-2}+4t^{-1}-2+t$

k9c16: $-t^{-12}+3t^{-11}-5t^{-10}+6t^{-9}-7t^{-8}+6t^{-7}-5t^{-6}+4t^{-5}-t^{-4}+t^{-3}$

k9c17: $t^{-3}-2t^{-2}+4t^{-1}-5+6t-7t^{2}+6t^{3}-4t^{4}+3t^{5}-t^{6}$

k9c18: $t^{2}-2t^{3}+5t^{4}-6t^{5}+7t^{6}-7t^{7}+6t^{8}-4t^{9}+2t^{10}-t^{11}$

k9c19: $t^{-4}-2t^{-3}+4t^{-2}-6t^{-1}+7-7t+6t^{2}-4t^{3}+3t^{4}-t^{5}$

k9c20: $1-2t+4t^{2}-5t^{3}+7t^{4}-7t^{5}+6t^{6}-5t^{7}+3t^{8}-t^{9}$

k9c21: $-t^{-8}+2t^{-7}-4t^{-6}+6t^{-5}-7t^{-4}+8t^{-3}-6t^{-2}+5t^{-1}-3+t$

k9c22: $-t^{-6}+3t^{-5}-5t^{-4}+7t^{-3}-7t^{-2}+7t^{-1}-6+4t-2t^{2}+t^{3}$

k9c23: $t^{2}-2t^{3}+5t^{4}-6t^{5}+8t^{6}-8t^{7}+6t^{8}-5t^{9}+3t^{10}-t^{11}$

k9c24: $t^{-4}-3t^{-3}+5t^{-2}-7t^{-1}+8-7t+7t^{2}-4t^{3}+2t^{4}-t^{5}$

k9c25: $t^{-1}-2+5t-7t^{2}+8t^{3}-8t^{4}+7t^{5}-5t^{6}+3t^{7}-t^{8}$

k9c26: $t^{-7}-3t^{-6}+5t^{-5}-7t^{-4}+8t^{-3}-8t^{-2}+7t^{-1}-4+3t-t^{2}$

k9c27: $t^{-4}-3t^{-3}+5t^{-2}-7t^{-1}+9-8t+7t^{2}-5t^{3}+3t^{4}-t^{5}$

k9c28: $-t^{-2}+3t^{-1}-5+8t-8t^{2}+9t^{3}-8t^{4}+5t^{5}-3t^{6}+t^{7}$

k9c29: $t^{-3}-3t^{-2}+5t^{-1}-7+9t-8t^{2}+8t^{3}-6t^{4}+3t^{5}-t^{6}$

k9c30: $t^{-4}-3t^{-3}+6t^{-2}-8t^{-1}+9-9t+8t^{2}-5t^{3}+3t^{4}-t^{5}$

k9c31: $-t^{-2}+3t^{-1}-5+8t-9t^{2}+10t^{3}-8t^{4}+6t^{5}-4t^{6}+t^{7}$

k9c32: $t^{-7}-3t^{-6}+6t^{-5}-9t^{-4}+10t^{-3}-10t^{-2}+9t^{-1}-6+4t-t^{2}$

k9c33: $t^{-4}-4t^{-3}+7t^{-2}-9t^{-1}+11-10t+9t^{2}-6t^{3}+3t^{4}-t^{5}$

k9c34: $t^{-4}-4t^{-3}+8t^{-2}-10t^{-1}+12-12t+10t^{2}-7t^{3}+4t^{4}-t^{5}$

k9c35: $t-2t^{2}+3t^{3}-4t^{4}+5t^{5}-3t^{6}+4t^{7}-3t^{8}+t^{9}-t^{10}$

k9c36: $-t^{-9}+2t^{-8}-4t^{-7}+6t^{-6}-6t^{-5}+6t^{-4}-5t^{-3}+4t^{-2}-2t^{-1}+1$

k9c37: $t^{-4}-2t^{-3}+5t^{-2}-7t^{-1}+7-8t+7t^{2}-4t^{3}+3t^{4}-t^{5}$

k9c38: $t^{2}-3t^{3}+7t^{4}-8t^{5}+10t^{6}-10t^{7}+8t^{8}-6t^{9}+3t^{10}-t^{11}$

k9c39: $-t^{-8}+3t^{-7}-6t^{-6}+8t^{-5}-9t^{-4}+10t^{-3}-8t^{-2}+6t^{-1}-3+t$

k9c40: $-t^{-2}+5t^{-1}-8+11t-13t^{2}+13t^{3}-11t^{4}+8t^{5}-4t^{6}+t^{7}$

k9c41: $-t^{-3}+3t^{-2}-5t^{-1}+8-8t+8t^{2}-7t^{3}+5t^{4}-3t^{5}+t^{6}$

k9c42: $t^{-3}-t^{-2}+t^{-1}-1+t-t^{2}+t^{3}$

k9c43: $-t^{-7}+2t^{-6}-2t^{-5}+2t^{-4}-2t^{-3}+2t^{-2}-t^{-1}+1$

k9c44: $t^{-2}-2t^{-1}+3-3t+3t^{2}-2t^{3}+2t^{4}-t^{5}$

k9c45: $2t-3t^{2}+4t^{3}-4t^{4}+4t^{5}-3t^{6}+2t^{7}-t^{8}$

k9c46: $2-t+t^{2}-2t^{3}+t^{4}-t^{5}+t^{6}$

k9c47: $2t^{-5}-4t^{-4}+4t^{-3}-5t^{-2}+5t^{-1}-3+3t-t^{2}$

k9c48: $-2t^{-6}+3t^{-5}-4t^{-4}+6t^{-3}-4t^{-2}+4t^{-1}-3+t$

k9c49: $-2t^{-9}+3t^{-8}-4t^{-7}+5t^{-6}-4t^{-5}+4t^{-4}-2t^{-3}+t^{-2}$

k10c1: $t^{-2}-t^{-1}+2-2t+2t^{2}-2t^{3}+2t^{4}-2t^{5}+t^{6}-t^{7}+t^{8}$

k10c2: $t-t^{2}+2t^{3}-2t^{4}+3t^{5}-3t^{6}+3t^{7}-3t^{8}+2t^{9}-2t^{10}+t^{11}$

k10c3: $t^{-4}-t^{-3}+2t^{-2}-3t^{-1}+4-4t+3t^{2}-3t^{3}+2t^{4}-t^{5}+t^{6}$

k10c4: $t^{-5}-t^{-4}+2t^{-3}-3t^{-2}+3t^{-1}-4+4t-3t^{2}+3t^{3}-2t^{4}+t^{5}$

k10c5: $-t^{-9}+2t^{-8}-3t^{-7}+4t^{-6}-5t^{-5}+5t^{-4}-4t^{-3}+4t^{-2}-2t^{-1}+2-t$

k10c6: $1-t+3t^{2}-4t^{3}+5t^{4}-6t^{5}+6t^{6}-5t^{7}+3t^{8}-2t^{9}+t^{10}$

k10c7: $t^{-1}-2+4t-5t^{2}+7t^{3}-7t^{4}+6t^{5}-5t^{6}+3t^{7}-2t^{8}+t^{9}$

k10c8: $t^{-2}-t^{-1}+2-3t+4t^{2}-4t^{3}+4t^{4}-4t^{5}+3t^{6}-2t^{7}+t^{8}$

k10c9: $t^{-7}-2t^{-6}+3t^{-5}-5t^{-4}+6t^{-3}-6t^{-2}+6t^{-1}-4+3t-2t^{2}+t^{3}$

k10c10: $-t^{-7}+2t^{-6}-3t^{-5}+5t^{-4}-6t^{-3}+7t^{-2}-7t^{-1}+6-4t+3t^{2}-t^{3}$

k10c11: $t^{-3}-t^{-2}+3t^{-1}-5+6t-7t^{2}+7t^{3}-6t^{4}+4t^{5}-2t^{6}+t^{7}$

k10c12: $-t^{-8}+2t^{-7}-4t^{-6}+6t^{-5}-7t^{-4}+8t^{-3}-7t^{-2}+6t^{-1}-3+2t-t^{2}$

k10c13: $t^{-4}-2t^{-3}+5t^{-2}-7t^{-1}+8-9t+8t^{2}-6t^{3}+4t^{4}-2t^{5}+t^{6}$

k10c14: $1-2t+4t^{2}-6t^{3}+9t^{4}-9t^{5}+9t^{6}-8t^{7}+5t^{8}-3t^{9}+t^{10}$

k10c15: $-t^{-6}+2t^{-5}-4t^{-4}+6t^{-3}-6t^{-2}+7t^{-1}-6+5t-3t^{2}+2t^{3}-t^{4}$

k10c16: $t^{-7}-2t^{-6}+4t^{-5}-6t^{-4}+7t^{-3}-8t^{-2}+7t^{-1}-5+4t-2t^{2}+t^{3}$

k10c17: $-t^{-5}+2t^{-4}-3t^{-3}+5t^{-2}-6t^{-1}+7-6t+5t^{2}-3t^{3}+2t^{4}-t^{5}$

k10c18: $t^{-3}-2t^{-2}+4t^{-1}-6+8t-9t^{2}+9t^{3}-7t^{4}+5t^{5}-3t^{6}+t^{7}$

k10c19: $-t^{-4}+2t^{-3}-3t^{-2}+6t^{-1}-7+8t-8t^{2}+7t^{3}-5t^{4}+3t^{5}-t^{6}$

k10c20: $t^{-1}-1+3t-4t^{2}+5t^{3}-6t^{4}+5t^{5}-4t^{6}+3t^{7}-2t^{8}+t^{9}$

k10c21: $1-2t+4t^{2}-5t^{3}+7t^{4}-7t^{5}+7t^{6}-6t^{7}+3t^{8}-2t^{9}+t^{10}$

k10c22: $t^{-6}-2t^{-5}+4t^{-4}-6t^{-3}+7t^{-2}-8t^{-1}+8-6t+4t^{2}-2t^{3}+t^{4}$

k10c23: $-t^{-8}+2t^{-7}-4t^{-6}+7t^{-5}-9t^{-4}+10t^{-3}-9t^{-2}+8t^{-1}-5+3t-t^{2}$

k10c24: $t^{-1}-2+5t-7t^{2}+9t^{3}-9t^{4}+8t^{5}-7t^{6}+4t^{7}-2t^{8}+t^{9}$

k10c25: $1-2t+5t^{2}-7t^{3}+10t^{4}-11t^{5}+10t^{6}-9t^{7}+6t^{8}-3t^{9}+t^{10}$

k10c26: $t^{-6}-2t^{-5}+4t^{-4}-7t^{-3}+9t^{-2}-10t^{-1}+10-8t+6t^{2}-3t^{3}+t^{4}$

k10c27: $-t^{-2}+3t^{-1}-5+9t-11t^{2}+12t^{3}-11t^{4}+9t^{5}-6t^{6}+3t^{7}-t^{8}$

k10c28: $-t^{-7}+2t^{-6}-4t^{-5}+6t^{-4}-7t^{-3}+9t^{-2}-8t^{-1}+7-5t+3t^{2}-t^{3}$

k10c29: $t^{-3}-2t^{-2}+5t^{-1}-7+9t-11t^{2}+10t^{3}-8t^{4}+6t^{5}-3t^{6}+t^{7}$

k10c30: $t^{-1}-3+6t-8t^{2}+11t^{3}-11t^{4}+10t^{5}-8t^{6}+5t^{7}-3t^{8}+t^{9}$

k10c31: $-t^{-5}+2t^{-4}-4t^{-3}+7t^{-2}-8t^{-1}+10-9t+7t^{2}-5t^{3}+3t^{4}-t^{5}$

k10c32: $t^{-4}-3t^{-3}+6t^{-2}-9t^{-1}+11-11t+11t^{2}-8t^{3}+5t^{4}-3t^{5}+t^{6}$

k10c33: $-t^{-5}+3t^{-4}-5t^{-3}+8t^{-2}-10t^{-1}+11-10t+8t^{2}-5t^{3}+3t^{4}-t^{5}$

k10c34: $-t^{-7}+2t^{-6}-3t^{-5}+4t^{-4}-5t^{-3}+6t^{-2}-5t^{-1}+5-3t+2t^{2}-t^{3}$

k10c35: $t^{-6}-2t^{-5}+4t^{-4}-6t^{-3}+7t^{-2}-8t^{-1}+8-6t+4t^{2}-2t^{3}+t^{4}$

k10c36: $t^{-1}-2+4t-6t^{2}+8t^{3}-8t^{4}+8t^{5}-6t^{6}+4t^{7}-3t^{8}+t^{9}$

k10c37: $-t^{-5}+2t^{-4}-4t^{-3}+7t^{-2}-8t^{-1}+9-8t+7t^{2}-4t^{3}+2t^{4}-t^{5}$

k10c38: $t^{-1}-2+5t-7t^{2}+9t^{3}-10t^{4}+9t^{5}-7t^{6}+5t^{7}-3t^{8}+t^{9}$

k10c39: $1-2t+5t^{2}-7t^{3}+9t^{4}-10t^{5}+10t^{6}-8t^{7}+5t^{8}-3t^{9}+t^{10}$

k10c40: $-t^{-8}+3t^{-7}-6t^{-6}+9t^{-5}-12t^{-4}+13t^{-3}-11t^{-2}+10t^{-1}-6+3t-t^{2}$

k10c41: $t^{-3}-3t^{-2}+6t^{-1}-8+11t-12t^{2}+11t^{3}-9t^{4}+6t^{5}-3t^{6}+t^{7}$

k10c42: $-t^{-5}+3t^{-4}-6t^{-3}+10t^{-2}-12t^{-1}+14-13t+10t^{2}-7t^{3}+4t^{4}-t^{5}$

k10c43: $-t^{-5}+3t^{-4}-6t^{-3}+9t^{-2}-11t^{-1}+13-11t+9t^{2}-6t^{3}+3t^{4}-t^{5}$

k10c44: $t^{-3}-3t^{-2}+6t^{-1}-9+12t-13t^{2}+13t^{3}-10t^{4}+7t^{5}-4t^{6}+t^{7}$

k10c45: $-t^{-5}+4t^{-4}-7t^{-3}+11t^{-2}-14t^{-1}+15-14t+11t^{2}-7t^{3}+4t^{4}-t^{5}$

k10c46: $t^{-11}-2t^{-10}+3t^{-9}-4t^{-8}+4t^{-7}-5t^{-6}+4t^{-5}-3t^{-4}+3t^{-3}-t^{-2}+t^{-1}$

k10c47: $-t^{-9}+2t^{-8}-4t^{-7}+5t^{-6}-6t^{-5}+7t^{-4}-5t^{-3}+5t^{-2}-3t^{-1}+2-t$

k10c48: $-t^{-5}+2t^{-4}-4t^{-3}+6t^{-2}-7t^{-1}+9-7t+6t^{2}-4t^{3}+2t^{4}-t^{5}$

k10c49: $t^{3}-2t^{4}+5t^{5}-6t^{6}+9t^{7}-10t^{8}+9t^{9}-8t^{10}+5t^{11}-3t^{12}+t^{13}$

k10c50: $t^{-10}-2t^{-9}+4t^{-8}-7t^{-7}+8t^{-6}-9t^{-5}+8t^{-4}-6t^{-3}+5t^{-2}-2t^{-1}+1$

k10c51: $-t^{-8}+2t^{-7}-5t^{-6}+8t^{-5}-10t^{-4}+12t^{-3}-10t^{-2}+9t^{-1}-6+3t-t^{2}$

k10c52: $-t^{-6}+3t^{-5}-6t^{-4}+8t^{-3}-9t^{-2}+10t^{-1}-8+7t-4t^{2}+2t^{3}-t^{4}$

k10c53: $t^{2}-3t^{3}+7t^{4}-9t^{5}+12t^{6}-12t^{7}+11t^{8}-9t^{9}+5t^{10}-3t^{11}+t^{12}$

k10c54: $-t^{-6}+2t^{-5}-4t^{-4}+6t^{-3}-7t^{-2}+8t^{-1}-6+6t-4t^{2}+2t^{3}-t^{4}$

k10c55: $t^{2}-2t^{3}+5t^{4}-7t^{5}+10t^{6}-10t^{7}+9t^{8}-8t^{9}+5t^{10}-3t^{11}+t^{12}$

k10c56: $t^{-10}-3t^{-9}+6t^{-8}-9t^{-7}+10t^{-6}-11t^{-5}+10t^{-4}-7t^{-3}+5t^{-2}-2t^{-1}+1$

k10c57: $-t^{-8}+3t^{-7}-7t^{-6}+10t^{-5}-12t^{-4}+14t^{-3}-12t^{-2}+10t^{-1}-6+3t-t^{2}$

k10c58: $t^{-4}-2t^{-3}+5t^{-2}-8t^{-1}+10-11t+10t^{2}-8t^{3}+6t^{4}-3t^{5}+t^{6}$

k10c59: $t^{-7}-3t^{-6}+6t^{-5}-10t^{-4}+12t^{-3}-12t^{-2}+12t^{-1}-9+6t-3t^{2}+t^{3}$

k10c60: $t^{-4}-4t^{-3}+8t^{-2}-11t^{-1}+14-14t+13t^{2}-10t^{3}+6t^{4}-3t^{5}+t^{6}$

k10c61: $t^{-8}-2t^{-7}+3t^{-6}-4t^{-5}+5t^{-4}-5t^{-3}+4t^{-2}-4t^{-1}+3-t+t^{2}$

k10c62: $-t^{-9}+2t^{-8}-4t^{-7}+6t^{-6}-7t^{-5}+7t^{-4}-6t^{-3}+6t^{-2}-3t^{-1}+2-t$

k10c63: $t^{2}-2t^{3}+5t^{4}-7t^{5}+9t^{6}-9t^{7}+9t^{8}-7t^{9}+4t^{10}-3t^{11}+t^{12}$

k10c64: $t^{-7}-2t^{-6}+4t^{-5}-7t^{-4}+8t^{-3}-8t^{-2}+8t^{-1}-6+4t-2t^{2}+t^{3}$

k10c65: $-t^{-8}+2t^{-7}-5t^{-6}+8t^{-5}-9t^{-4}+11t^{-3}-10t^{-2}+8t^{-1}-5+3t-t^{2}$

k10c66: $t^{3}-2t^{4}+6t^{5}-8t^{6}+11t^{7}-13t^{8}+12t^{9}-10t^{10}+7t^{11}-4t^{12}+t^{13}$

k10c67: $t^{-1}-2+5t-8t^{2}+10t^{3}-10t^{4}+10t^{5}-8t^{6}+5t^{7}-3t^{8}+t^{9}$

k10c68: $-t^{-3}+3t^{-2}-5t^{-1}+8-9t+9t^{2}-8t^{3}+7t^{4}-4t^{5}+2t^{6}-t^{7}$

k10c69: $-t^{-8}+3t^{-7}-7t^{-6}+11t^{-5}-13t^{-4}+15t^{-3}-14t^{-2}+11t^{-1}-7+4t-t^{2}$

k10c70: $t^{-7}-3t^{-6}+6t^{-5}-9t^{-4}+11t^{-3}-11t^{-2}+10t^{-1}-8+5t-2t^{2}+t^{3}$

k10c71: $-t^{-5}+3t^{-4}-6t^{-3}+10t^{-2}-12t^{-1}+13-12t+10t^{2}-6t^{3}+3t^{4}-t^{5}$

k10c72: $t^{-10}-4t^{-9}+7t^{-8}-10t^{-7}+12t^{-6}-12t^{-5}+11t^{-4}-8t^{-3}+5t^{-2}-2t^{-1}+1$

k10c73: $-t^{-2}+4t^{-1}-7+11t-13t^{2}+14t^{3}-13t^{4}+10t^{5}-6t^{6}+3t^{7}-t^{8}$

k10c74: $t^{-1}-3+6t-8t^{2}+11t^{3}-10t^{4}+9t^{5}-8t^{6}+4t^{7}-2t^{8}+t^{9}$

k10c75: $t^{-6}-3t^{-5}+6t^{-4}-10t^{-3}+12t^{-2}-13t^{-1}+14-10t+7t^{2}-4t^{3}+t^{4}$

k10c76: $t^{-10}-3t^{-9}+6t^{-8}-8t^{-7}+9t^{-6}-10t^{-5}+8t^{-4}-6t^{-3}+4t^{-2}-t^{-1}+1$

k10c77: $-t^{-8}+3t^{-7}-6t^{-6}+8t^{-5}-10t^{-4}+11t^{-3}-9t^{-2}+8t^{-1}-4+2t-t^{2}$

k10c78: $1-3t+6t^{2}-8t^{3}+11t^{4}-11t^{5}+11t^{6}-9t^{7}+5t^{8}-3t^{9}+t^{10}$

k10c79: $-t^{-5}+2t^{-4}-5t^{-3}+8t^{-2}-9t^{-1}+11-9t+8t^{2}-5t^{3}+2t^{4}-t^{5}$

k10c80: $t^{3}-2t^{4}+6t^{5}-8t^{6}+11t^{7}-12t^{8}+11t^{9}-10t^{10}+6t^{11}-3t^{12}+t^{13}$

k10c81: $-t^{-5}+3t^{-4}-7t^{-3}+11t^{-2}-13t^{-1}+15-13t+11t^{2}-7t^{3}+3t^{4}-t^{5}$

k10c82: $t^{-3}-3t^{-2}+5t^{-1}-7+10t-10t^{2}+10t^{3}-8t^{4}+5t^{5}-3t^{6}+t^{7}$

k10c83: $t^{-6}-3t^{-5}+6t^{-4}-10t^{-3}+13t^{-2}-14t^{-1}+14-11t+8t^{2}-4t^{3}+t^{4}$

k10c84: $-t^{-8}+4t^{-7}-8t^{-6}+11t^{-5}-14t^{-4}+15t^{-3}-13t^{-2}+11t^{-1}-6+3t-t^{2}$

k10c85: $-t^{-1}+3-4t+7t^{2}-8t^{3}+9t^{4}-9t^{5}+7t^{6}-5t^{7}+3t^{8}-t^{9}$

k10c86: $-t^{-8}+3t^{-7}-6t^{-6}+10t^{-5}-13t^{-4}+14t^{-3}-13t^{-2}+11t^{-1}-7+4t-t^{2}$

k10c87: $t^{-6}-4t^{-5}+7t^{-4}-10t^{-3}+13t^{-2}-13t^{-1}+13-10t+6t^{2}-3t^{3}+t^{4}$

k10c88: $-t^{-5}+4t^{-4}-8t^{-3}+13t^{-2}-16t^{-1}+17-16t+13t^{2}-8t^{3}+4t^{4}-t^{5}$

k10c89: $-t^{-2}+5t^{-1}-9+13t-16t^{2}+17t^{3}-15t^{4}+12t^{5}-7t^{6}+3t^{7}-t^{8}$

k10c90: $t^{-6}-3t^{-5}+6t^{-4}-9t^{-3}+12t^{-2}-13t^{-1}+12-10t+7t^{2}-3t^{3}+t^{4}$

k10c91: $-t^{-5}+3t^{-4}-6t^{-3}+9t^{-2}-11t^{-1}+13-11t+9t^{2}-6t^{3}+3t^{4}-t^{5}$

k10c92: $t^{-10}-4t^{-9}+8t^{-8}-12t^{-7}+14t^{-6}-15t^{-5}+14t^{-4}-10t^{-3}+7t^{-2}-3t^{-1}+1$

k10c93: $-t^{-4}+3t^{-3}-5t^{-2}+8t^{-1}-10+11t-10t^{2}+9t^{3}-6t^{4}+3t^{5}-t^{6}$

k10c94: $t^{-7}-3t^{-6}+6t^{-5}-9t^{-4}+11t^{-3}-12t^{-2}+11t^{-1}-8+6t-3t^{2}+t^{3}$

k10c95: $-t^{-8}+3t^{-7}-7t^{-6}+11t^{-5}-14t^{-4}+16t^{-3}-14t^{-2}+12t^{-1}-8+4t-t^{2}$

k10c96: $t^{-6}-3t^{-5}+7t^{-4}-11t^{-3}+14t^{-2}-16t^{-1}+15-12t+9t^{2}-4t^{3}+t^{4}$

k10c97: $t^{-9}-4t^{-8}+7t^{-7}-11t^{-6}+14t^{-5}-14t^{-4}+14t^{-3}-11t^{-2}+7t^{-1}-3+t$

k10c98: $1-3t+7t^{2}-9t^{3}+13t^{4}-14t^{5}+12t^{6}-11t^{7}+7t^{8}-3t^{9}+t^{10}$

k10c99: $-t^{-5}+3t^{-4}-7t^{-3}+10t^{-2}-12t^{-1}+15-12t+10t^{2}-7t^{3}+3t^{4}-t^{5}$

k10c100: $-t^{-1}+3-5t+8t^{2}-9t^{3}+11t^{4}-10t^{5}+8t^{6}-6t^{7}+3t^{8}-t^{9}$

k10c101: $t^{-12}-4t^{-11}+7t^{-10}-11t^{-9}+13t^{-8}-14t^{-7}+14t^{-6}-10t^{-5}+7t^{-4}-3t^{-3}+t^{-2}$

k10c102: $t^{-6}-3t^{-5}+6t^{-4}-9t^{-3}+11t^{-2}-12t^{-1}+12-9t+6t^{2}-3t^{3}+t^{4}$

k10c103: $-t^{-2}+3t^{-1}-6+10t-11t^{2}+13t^{3}-12t^{4}+9t^{5}-6t^{6}+3t^{7}-t^{8}$

k10c104: $-t^{-5}+3t^{-4}-6t^{-3}+10t^{-2}-12t^{-1}+13-12t+10t^{2}-6t^{3}+3t^{4}-t^{5}$

k10c105: $t^{-7}-4t^{-6}+8t^{-5}-12t^{-4}+15t^{-3}-15t^{-2}+14t^{-1}-11+7t-3t^{2}+t^{3}$

k10c106: $t^{-7}-3t^{-6}+6t^{-5}-10t^{-4}+12t^{-3}-12t^{-2}+12t^{-1}-9+6t-3t^{2}+t^{3}$

k10c107: $-t^{-5}+3t^{-4}-7t^{-3}+12t^{-2}-14t^{-1}+16-15t+12t^{2}-8t^{3}+4t^{4}-t^{5}$

k10c108: $-t^{-6}+3t^{-5}-5t^{-4}+8t^{-3}-10t^{-2}+10t^{-1}-9+8t-5t^{2}+3t^{3}-t^{4}$

k10c109: $-t^{-5}+3t^{-4}-7t^{-3}+11t^{-2}-13t^{-1}+15-13t+11t^{2}-7t^{3}+3t^{4}-t^{5}$

k10c110: $t^{-3}-3t^{-2}+7t^{-1}-10+13t-14t^{2}+13t^{3}-11t^{4}+7t^{5}-3t^{6}+t^{7}$

k10c111: $t^{-10}-3t^{-9}+6t^{-8}-10t^{-7}+12t^{-6}-13t^{-5}+12t^{-4}-9t^{-3}+7t^{-2}-3t^{-1}+1$

k10c112: $t^{-3}-4t^{-2}+7t^{-1}-10+14t-14t^{2}+14t^{3}-11t^{4}+7t^{5}-4t^{6}+t^{7}$

k10c113: $-t^{-8}+5t^{-7}-10t^{-6}+14t^{-5}-18t^{-4}+19t^{-3}-17t^{-2}+14t^{-1}-8+4t-t^{2}$

k10c114: $t^{-4}-4t^{-3}+8t^{-2}-12t^{-1}+15-15t+15t^{2}-11t^{3}+7t^{4}-4t^{5}+t^{6}$

k10c115: $-t^{-5}+4t^{-4}-9t^{-3}+14t^{-2}-17t^{-1}+19-17t+14t^{2}-9t^{3}+4t^{4}-t^{5}$

k10c116: $t^{-3}-4t^{-2}+8t^{-1}-11+15t-16t^{2}+15t^{3}-12t^{4}+8t^{5}-4t^{6}+t^{7}$

k10c117: $-t^{-8}+4t^{-7}-9t^{-6}+13t^{-5}-16t^{-4}+18t^{-3}-16t^{-2}+13t^{-1}-8+4t-t^{2}$

k10c118: $-t^{-5}+4t^{-4}-8t^{-3}+12t^{-2}-15t^{-1}+17-15t+12t^{2}-8t^{3}+4t^{4}-t^{5}$

k10c119: $t^{-6}-4t^{-5}+8t^{-4}-12t^{-3}+16t^{-2}-17t^{-1}+16-13t+9t^{2}-4t^{3}+t^{4}$

k10c120: $t^{2}-4t^{3}+10t^{4}-13t^{5}+17t^{6}-18t^{7}+16t^{8}-13t^{9}+8t^{10}-4t^{11}+t^{12}$

k10c121: $-t^{-2}+5t^{-1}-10+15t-18t^{2}+20t^{3}-18t^{4}+14t^{5}-9t^{6}+4t^{7}-t^{8}$

k10c122: $t^{-6}-5t^{-5}+9t^{-4}-13t^{-3}+17t^{-2}-17t^{-1}+17-13t+8t^{2}-4t^{3}+t^{4}$

k10c123: $-t^{-5}+5t^{-4}-10t^{-3}+15t^{-2}-19t^{-1}+21-19t+15t^{2}-10t^{3}+5t^{4}-t^{5}$

k10c124: $-t^{-10}+t^{-6}+t^{-4}$

k10c125: $-t^{-4}+t^{-3}-t^{-2}+2t^{-1}-1+2t-t^{2}+t^{3}-t^{4}$

k10c126: $-1+2t-2t^{2}+4t^{3}-3t^{4}+3t^{5}-2t^{6}+t^{7}-t^{8}$

k10c127: $2t^{2}-2t^{3}+4t^{4}-5t^{5}+5t^{6}-5t^{7}+3t^{8}-2t^{9}+t^{10}$

k10c128: $-t^{-10}+t^{-9}-2t^{-8}+2t^{-7}-t^{-6}+2t^{-5}-t^{-4}+t^{-3}$

k10c129: $-t^{-3}+2t^{-2}-3t^{-1}+5-4t+4t^{2}-3t^{3}+2t^{4}-t^{5}$

k10c130: $-t^{-1}+2-2t+3t^{2}-2t^{3}+3t^{4}-2t^{5}+t^{6}-t^{7}$

k10c131: $2t-3t^{2}+5t^{3}-5t^{4}+5t^{5}-5t^{6}+3t^{7}-2t^{8}+t^{9}$

k10c132: $t^{2}+t^{4}-t^{5}+t^{6}-t^{7}$

k10c133: $t-t^{2}+3t^{3}-3t^{4}+3t^{5}-3t^{6}+2t^{7}-2t^{8}+t^{9}$

k10c134: $t^{-11}-3t^{-10}+3t^{-9}-4t^{-8}+4t^{-7}-3t^{-6}+3t^{-5}-t^{-4}+t^{-3}$

k10c135: $-2t^{-3}+4t^{-2}-5t^{-1}+7-6t+6t^{2}-4t^{3}+2t^{4}-t^{5}$

k10c136: $-t^{-4}+2t^{-3}-2t^{-2}+3t^{-1}-2+2t-2t^{2}+t^{3}$

k10c137: $t^{-2}-2t^{-1}+4-4t+4t^{2}-4t^{3}+3t^{4}-2t^{5}+t^{6}$

k10c138: $2t^{-5}-4t^{-4}+5t^{-3}-6t^{-2}+6t^{-1}-5+4t-2t^{2}+t^{3}$

k10c139: $-t^{-12}+t^{-11}-t^{-10}+t^{-9}-t^{-8}+t^{-6}+t^{-4}$

k10c140: $1-t+t^{2}-t^{3}+2t^{4}-t^{5}+t^{6}-t^{7}$

k10c141: $t^{-2}-2t^{-1}+3-3t+4t^{2}-3t^{3}+2t^{4}-2t^{5}+t^{6}$

k10c142: $-2t^{-10}+2t^{-9}-2t^{-8}+3t^{-7}-2t^{-6}+2t^{-5}-t^{-4}+t^{-3}$

k10c143: $-1+3t-3t^{2}+5t^{3}-5t^{4}+4t^{5}-3t^{6}+2t^{7}-t^{8}$

k10c144: $2t^{-1}-3+5t-7t^{2}+7t^{3}-6t^{4}+5t^{5}-3t^{6}+t^{7}$

k10c145: $t^{2}+t^{7}-t^{8}+t^{9}-t^{10}$

k10c146: $-t^{-3}+3t^{-2}-4t^{-1}+6-6t+5t^{2}-4t^{3}+3t^{4}-t^{5}$

k10c147: $t^{-5}-3t^{-4}+4t^{-3}-4t^{-2}+5t^{-1}-4+3t-2t^{2}+t^{3}$

k10c148: $-1+3t-4t^{2}+6t^{3}-5t^{4}+5t^{5}-4t^{6}+2t^{7}-t^{8}$

k10c149: $2t^{2}-3t^{3}+6t^{4}-7t^{5}+7t^{6}-7t^{7}+5t^{8}-3t^{9}+t^{10}$

k10c150: $t^{-8}-3t^{-7}+4t^{-6}-5t^{-5}+5t^{-4}-4t^{-3}+4t^{-2}-2t^{-1}+1$

k10c151: $-2t^{-6}+4t^{-5}-6t^{-4}+8t^{-3}-7t^{-2}+7t^{-1}-5+3t-t^{2}$

k10c152: $t^{4}+t^{6}+t^{7}-2t^{8}+2t^{9}-3t^{10}+2t^{11}-2t^{12}+t^{13}$

k10c153: $-t^{-4}+t^{-3}-t^{-2}+t^{-1}+1+t^{2}-t^{3}+t^{4}-t^{5}$

k10c154: $t^{-12}-2t^{-11}+2t^{-10}-3t^{-9}+2t^{-8}-2t^{-7}+2t^{-6}+t^{-3}$

k10c155: $t^{-6}-2t^{-5}+3t^{-4}-4t^{-3}+4t^{-2}-4t^{-1}+4-2t+t^{2}$

k10c156: $-t^{-2}+3t^{-1}-4+6t-6t^{2}+6t^{3}-5t^{4}+3t^{5}-t^{6}$

k10c157: $t^{-10}-4t^{-9}+6t^{-8}-8t^{-7}+9t^{-6}-8t^{-5}+7t^{-4}-4t^{-3}+2t^{-2}$

k10c158: $2t^{-4}-4t^{-3}+6t^{-2}-8t^{-1}+8-7t+6t^{2}-3t^{3}+t^{4}$

k10c159: $-1+4t-5t^{2}+7t^{3}-7t^{4}+6t^{5}-5t^{6}+3t^{7}-t^{8}$

k10c160: $-2t^{-7}+3t^{-6}-3t^{-5}+4t^{-4}-3t^{-3}+3t^{-2}-2t^{-1}+1$

k10c161: $t^{3}+t^{6}-t^{7}+t^{8}-t^{9}+t^{10}-t^{11}$

k10c162: $-t^{-11}+t^{-10}-t^{-9}+t^{-8}-t^{-7}+t^{-6}+t^{-3}$

k10c163: $2t^{-1}-3+5t-6t^{2}+6t^{3}-6t^{4}+4t^{5}-2t^{6}+t^{7}$

k10c164: $-2t^{-6}+5t^{-5}-7t^{-4}+9t^{-3}-9t^{-2}+8t^{-1}-6+4t-t^{2}$

k10c165: $-2t^{-3}+5t^{-2}-6t^{-1}+8-8t+7t^{2}-5t^{3}+3t^{4}-t^{5}$

k10c166: $t^{-9}-3t^{-8}+4t^{-7}-6t^{-6}+7t^{-5}-6t^{-4}+6t^{-3}-4t^{-2}+2t^{-1}$


\newpage


\section{P-polynomial: $P(v,z)$}
 $(1/v)P_{L+} -v P_{L-} = z P_{L0}$ \par
k3c1: $ (-v^{-4}+2v^{-2})  +v^{-2}z^{2} $ 

k4c1: $ (v^{-2}-1+v^{2})  -z^{2} $ 

k5c1: $ (-2v^{-6}+3v^{-4})  +z^{2}(-v^{-6}+4v^{-4})  +v^{-4}z^{4} $ 

k5c2: $ (-v^{-6}+v^{-4}+v^{-2})  +z^{2}(v^{-4}+v^{-2}) $ 

k6c1: $ (v^{-4}-v^{-2}+v^{2})  +z^{2}(-v^{-2}-1) $ 

k6c2: $ (v^{-4}-2v^{-2}+2)  +z^{2}(v^{-4}-3v^{-2}+1)  -v^{-2}z^{4} $ 

k6c3: $ (-v^{-2}+3-v^{2})  +z^{2}(-v^{-2}+3-v^{2})  +z^{4} $ 

k7c1: $ (-3v^{-8}+4v^{-6})  +z^{2}(-4v^{-8}+10v^{-6})  +z^{4}(-v^{-8}+6v^{-6})  +v^{-6}z^{6} $ 

k7c2: $ (-v^{-8}+v^{-6}+v^{-2})  +z^{2}(v^{-6}+v^{-4}+v^{-2}) $ 

k7c3: $ (v^{4}+2v^{6}-2v^{8})  +z^{2}(3v^{4}+3v^{6}-v^{8})  +z^{4}(v^{4}+v^{6}) $ 

k7c4: $ (2v^{4}-v^{8})  +z^{2}(v^{2}+2v^{4}+v^{6}) $ 

k7c5: $ (-v^{-8}+2v^{-4})  +z^{2}(-v^{-8}+2v^{-6}+3v^{-4})  +z^{4}(v^{-6}+v^{-4}) $ 

k7c6: $ (-v^{-6}+2v^{-4}-v^{-2}+1)  +z^{2}(2v^{-4}-2v^{-2}+1)  -v^{-2}z^{4} $ 

k7c7: $ (2-2v^{2}+v^{4})  +z^{2}(-v^{-2}+2-2v^{2})  +z^{4} $ 

k8c1: $ (v^{-6}-v^{-4}+v^{2})  +z^{2}(-v^{-4}-v^{-2}-1) $ 

k8c2: $ (v^{-6}-3v^{-4}+3v^{-2})  +z^{2}(3v^{-6}-7v^{-4}+4v^{-2})  +z^{4}(v^{-6}-5v^{-4}+v^{-2})  -v^{-4}z^{6} $ 

k8c3: $ (v^{-4}-1+v^{4})  +z^{2}(-v^{-2}-2-v^{2}) $ 

k8c4: $ (v^{-4}-2+2v^{2})  +z^{2}(v^{-4}-2v^{-2}-3+v^{2})  +z^{4}(-v^{-2}-1) $ 

k8c5: $ (4v^{2}-5v^{4}+2v^{6})  +z^{2}(4v^{2}-8v^{4}+3v^{6})  +z^{4}(v^{2}-5v^{4}+v^{6})  -v^{4}z^{6} $ 

k8c6: $ (v^{-6}-v^{-4}-v^{-2}+2)  +z^{2}(v^{-6}-2v^{-4}-2v^{-2}+1)  +z^{4}(-v^{-4}-v^{-2}) $ 

k8c7: $ (-1+4v^{2}-2v^{4})  +z^{2}(-3+8v^{2}-3v^{4})  +z^{4}(-1+5v^{2}-v^{4})  +v^{2}z^{6} $ 

k8c8: $ (-v^{-2}+2+v^{2}-v^{4})  +z^{2}(-v^{-2}+2+2v^{2}-v^{4})  +z^{4}(1+v^{2}) $ 

k8c9: $ (2v^{-2}-3+2v^{2})  +z^{2}(3v^{-2}-8+3v^{2})  +z^{4}(v^{-2}-5+v^{2})  -z^{6} $ 

k8c10: $ (-2+6v^{2}-3v^{4})  +z^{2}(-3+9v^{2}-3v^{4})  +z^{4}(-1+5v^{2}-v^{4})  +v^{2}z^{6} $ 

k8c11: $ (v^{-6}-2v^{-4}+v^{-2}+1)  +z^{2}(v^{-6}-2v^{-4}-v^{-2}+1)  +z^{4}(-v^{-4}-v^{-2}) $ 

k8c12: $ (v^{-4}-v^{-2}+1-v^{2}+v^{4})  +z^{2}(-2v^{-2}+1-2v^{2})  +z^{4} $ 

k8c13: $ (2v^{2}-v^{4})  +z^{2}(-v^{-2}+1+2v^{2}-v^{4})  +z^{4}(1+v^{2}) $ 

k8c14: $ 1  +z^{2}(v^{-6}-v^{-4}-v^{-2}+1)  +z^{4}(-v^{-4}-v^{-2}) $ 

k8c15: $ (v^{-10}-4v^{-8}+3v^{-6}+v^{-4})  +z^{2}(-3v^{-8}+5v^{-6}+2v^{-4})  +z^{4}(2v^{-6}+v^{-4}) $ 

k8c16: $ (-v^{-4}+2v^{-2})  +z^{2}(-2v^{-4}+5v^{-2}-2)  +z^{4}(-v^{-4}+4v^{-2}-1)  +v^{-2}z^{6} $ 

k8c17: $ (v^{-2}-1+v^{2})  +z^{2}(2v^{-2}-5+2v^{2})  +z^{4}(v^{-2}-4+v^{2})  -z^{6} $ 

k8c18: $ (-v^{-2}+3-v^{2})  +z^{2}(v^{-2}-1+v^{2})  +z^{4}(v^{-2}-3+v^{2})  -z^{6} $ 

k8c19: $ (5v^{6}-5v^{8}+v^{10})  +z^{2}(10v^{6}-5v^{8})  +z^{4}(6v^{6}-v^{8})  +v^{6}z^{6} $ 

k8c20: $ (-2v^{-4}+4v^{-2}-1)  +z^{2}(-v^{-4}+4v^{-2}-1)  +v^{-2}z^{4} $ 

k8c21: $ (v^{-6}-3v^{-4}+3v^{-2})  +z^{2}(v^{-6}-3v^{-4}+2v^{-2})  -v^{-4}z^{4} $ 

k9c1: $ (-4v^{-10}+5v^{-8})  +z^{2}(-10v^{-10}+20v^{-8})  +z^{4}(-6v^{-10}+21v^{-8})  +z^{6}(-v^{-10}+8v^{-8})  +v^{-8}z^{8} $ 

k9c2: $ (-v^{-10}+v^{-8}+v^{-2})  +z^{2}(v^{-8}+v^{-6}+v^{-4}+v^{-2}) $ 

k9c3: $ (v^{6}+3v^{8}-3v^{10})  +z^{2}(6v^{6}+7v^{8}-4v^{10})  +z^{4}(5v^{6}+5v^{8}-v^{10})  +z^{6}(v^{6}+v^{8}) $ 

k9c4: $ (-2v^{-10}+2v^{-8}+v^{-4})  +z^{2}(-v^{-10}+3v^{-8}+2v^{-6}+3v^{-4})  +z^{4}(v^{-8}+v^{-6}+v^{-4}) $ 

k9c5: $ (v^{4}+v^{6}-v^{10})  +z^{2}(v^{2}+2v^{4}+2v^{6}+v^{8}) $ 

k9c6: $ (-v^{-10}-v^{-8}+3v^{-6})  +z^{2}(-3v^{-10}+3v^{-8}+7v^{-6})  +z^{4}(-v^{-10}+4v^{-8}+5v^{-6})  +z^{6}(v^{-8}+v^{-6}) $ 

k9c7: $ (-v^{-10}+v^{-8}-v^{-6}+2v^{-4})  +z^{2}(-v^{-10}+2v^{-8}+v^{-6}+3v^{-4})  +z^{4}(v^{-8}+v^{-6}+v^{-4}) $ 

k9c8: $ (-v^{-6}+2v^{-4}-1+v^{2})  +z^{2}(2v^{-4}-v^{-2}-2+v^{2})  +z^{4}(-v^{-2}-1) $ 

k9c9: $ (-2v^{-10}+v^{-8}+2v^{-6})  +z^{2}(-3v^{-10}+4v^{-8}+7v^{-6})  +z^{4}(-v^{-10}+4v^{-8}+5v^{-6})  +z^{6}(v^{-8}+v^{-6}) $ 

k9c10: $ (2v^{6}+v^{8}-2v^{10})  +z^{2}(2v^{4}+5v^{6}+2v^{8}-v^{10})  +z^{4}(v^{4}+2v^{6}+v^{8}) $ 

k9c11: $ (v^{2}-v^{4}+3v^{6}-2v^{8})  +z^{2}(3v^{2}-4v^{4}+6v^{6}-v^{8})  +z^{4}(v^{2}-4v^{4}+2v^{6})  -v^{4}z^{6} $ 

k9c12: $ (-v^{-8}+2v^{-6}-v^{-4}+1)  +z^{2}(2v^{-6}-v^{-4}-v^{-2}+1)  +z^{4}(-v^{-4}-v^{-2}) $ 

k9c13: $ (3v^{6}-v^{8}-v^{10})  +z^{2}(2v^{4}+5v^{6}+v^{8}-v^{10})  +z^{4}(v^{4}+2v^{6}+v^{8}) $ 

k9c14: $ (1+v^{2}-2v^{4}+v^{6})  +z^{2}(-v^{-2}+1+v^{2}-2v^{4})  +z^{4}(1+v^{2}) $ 

k9c15: $ (1-v^{2}+v^{4}+v^{6}-v^{8})  +z^{2}(1-v^{2}+2v^{6})  +z^{4}(-v^{2}-v^{4}) $ 

k9c16: $ (4v^{6}-3v^{8})  +z^{2}(8v^{6}-2v^{10})  +z^{4}(5v^{6}+3v^{8}-v^{10})  +z^{6}(v^{6}+v^{8}) $ 

k9c17: $ (2v^{-2}-3+2v^{2})  +z^{2}(-2v^{-4}+5v^{-2}-6+v^{2})  +z^{4}(-v^{-4}+4v^{-2}-2)  +v^{-2}z^{6} $ 

k9c18: $ (-v^{-10}+v^{-6}+v^{-4})  +z^{2}(-v^{-10}+v^{-8}+4v^{-6}+2v^{-4})  +z^{4}(v^{-8}+2v^{-6}+v^{-4}) $ 

k9c19: $ (v^{-2}-v^{2}+v^{4})  +z^{2}(-v^{-4}+v^{-2}-2v^{2})  +z^{4}(v^{-2}+1) $ 

k9c20: $ (-v^{-8}+2v^{-6}-2v^{-4}+2v^{-2})  +z^{2}(-v^{-8}+5v^{-6}-5v^{-4}+3v^{-2})  +z^{4}(2v^{-6}-4v^{-4}+v^{-2})  -v^{-4}z^{6} $ 

k9c21: $ (v^{2}+v^{6}-v^{8})  +z^{2}(1+2v^{6})  +z^{4}(-v^{2}-v^{4}) $ 

k9c22: $ (2v^{-2}-4+4v^{2}-v^{4})  +z^{2}(v^{-2}-6+6v^{2}-2v^{4})  +z^{4}(-2+4v^{2}-v^{4})  +v^{2}z^{6} $ 

k9c23: $ (-2v^{-8}+2v^{-6}+v^{-4})  +z^{2}(-v^{-10}+4v^{-6}+2v^{-4})  +z^{4}(v^{-8}+2v^{-6}+v^{-4}) $ 

k9c24: $ (-2v^{-4}+5v^{-2}-3+v^{2})  +z^{2}(-v^{-4}+6v^{-2}-6+2v^{2})  +z^{4}(2v^{-2}-4+v^{2})  -z^{6} $ 

k9c25: $ (-v^{-8}+3v^{-6}-3v^{-4}+v^{-2}+1)  +z^{2}(3v^{-6}-4v^{-4}+1)  +z^{4}(-2v^{-4}-v^{-2}) $ 

k9c26: $ (3v^{2}-3v^{4}+v^{6})  +z^{2}(-2+6v^{2}-5v^{4}+v^{6})  +z^{4}(-1+4v^{2}-2v^{4})  +v^{2}z^{6} $ 

k9c27: $ (-v^{-4}+3v^{-2}-2+v^{2})  +z^{2}(-v^{-4}+5v^{-2}-6+2v^{2})  +z^{4}(2v^{-2}-4+v^{2})  -z^{6} $ 

k9c28: $ (v^{-6}-4v^{-4}+5v^{-2}-1)  +z^{2}(v^{-6}-5v^{-4}+7v^{-2}-2)  +z^{4}(-2v^{-4}+4v^{-2}-1)  +v^{-2}z^{6} $ 

k9c29: $ (-2v^{-4}+5v^{-2}-3+v^{2})  +z^{2}(-2v^{-4}+7v^{-2}-5+v^{2})  +z^{4}(-v^{-4}+4v^{-2}-2)  +v^{-2}z^{6} $ 

k9c30: $ (-v^{-4}+4v^{-2}-4+2v^{2})  +z^{2}(-v^{-4}+5v^{-2}-7+2v^{2})  +z^{4}(2v^{-2}-4+v^{2})  -z^{6} $ 

k9c31: $ (-2v^{-4}+4v^{-2}-1)  +z^{2}(v^{-6}-4v^{-4}+7v^{-2}-2)  +z^{4}(-2v^{-4}+4v^{-2}-1)  +v^{-2}z^{6} $ 

k9c32: $ (1+v^{2}-2v^{4}+v^{6})  +z^{2}(-1+3v^{2}-4v^{4}+v^{6})  +z^{4}(-1+3v^{2}-2v^{4})  +v^{2}z^{6} $ 

k9c33: $ (-v^{-4}+2v^{-2})  +z^{2}(-v^{-4}+4v^{-2}-3+v^{2})  +z^{4}(2v^{-2}-3+v^{2})  -z^{6} $ 

k9c34: $ (v^{-2}-1+v^{2})  +z^{2}(-v^{-4}+3v^{-2}-4+v^{2})  +z^{4}(2v^{-2}-3+v^{2})  -z^{6} $ 

k9c35: $ (-v^{-10}-v^{-8}+3v^{-6})  +z^{2}(v^{-8}+3v^{-6}+2v^{-4}+v^{-2}) $ 

k9c36: $ (2v^{2}-3v^{4}+4v^{6}-2v^{8})  +z^{2}(3v^{2}-5v^{4}+6v^{6}-v^{8})  +z^{4}(v^{2}-4v^{4}+2v^{6})  -v^{4}z^{6} $ 

k9c37: $ (2v^{-2}-2+v^{4})  +z^{2}(-v^{-4}+v^{-2}-1-2v^{2})  +z^{4}(v^{-2}+1) $ 

k9c38: $ (-3v^{-8}+4v^{-6})  +z^{2}(-v^{-10}-v^{-8}+7v^{-6}+v^{-4})  +z^{4}(v^{-8}+3v^{-6}+v^{-4}) $ 

k9c39: $ (2v^{2}-2v^{4}+2v^{6}-v^{8})  +z^{2}(1+v^{2}-3v^{4}+3v^{6})  +z^{4}(-v^{2}-2v^{4}) $ 

k9c40: $ (v^{-4}-2v^{-2}+2)  +z^{2}(v^{-6}-2v^{-4})  +z^{4}(-2v^{-4}+2v^{-2}-1)  +v^{-2}z^{6} $ 

k9c41: $ (v^{-6}-3v^{-4}+3v^{-2})  +z^{2}(-3v^{-4}+4v^{-2}-v^{2})  +z^{4}(2v^{-2}+1) $ 

k9c42: $ (2v^{-2}-3+2v^{2})  +z^{2}(v^{-2}-4+v^{2})  -z^{4} $ 

k9c43: $ (3v^{2}-4v^{4}+3v^{6}-v^{8})  +z^{2}(4v^{2}-7v^{4}+4v^{6})  +z^{4}(v^{2}-5v^{4}+v^{6})  -v^{4}z^{6} $ 

k9c44: $ (-v^{-4}+3v^{-2}-2+v^{2})  +z^{2}(-v^{-4}+3v^{-2}-2)  +v^{-2}z^{4} $ 

k9c45: $ (-v^{-8}+2v^{-6}-2v^{-4}+2v^{-2})  +z^{2}(2v^{-6}-2v^{-4}+2v^{-2})  -v^{-4}z^{4} $ 

k9c46: $ (v^{-6}-v^{-4}-v^{-2}+2)  +z^{2}(-v^{-4}-v^{-2}) $ 

k9c47: $ (1+v^{2}-2v^{4}+v^{6})  +z^{2}(-2+4v^{2}-3v^{4})  +z^{4}(-1+4v^{2}-v^{4})  +v^{2}z^{6} $ 

k9c48: $ (3v^{4}-2v^{6})  +z^{2}(1-v^{2}+3v^{4})  -v^{2}z^{4} $ 

k9c49: $ (4v^{6}-3v^{8})  +z^{2}(2v^{4}+6v^{6}-2v^{8})  +z^{4}(v^{4}+2v^{6}) $ 

k10c1: $ (v^{-8}-v^{-6}+v^{2})  +z^{2}(-v^{-6}-v^{-4}-v^{-2}-1) $ 

k10c2: $ (v^{-8}-4v^{-6}+4v^{-4})  +z^{2}(6v^{-8}-14v^{-6}+10v^{-4})  +z^{4}(5v^{-8}-16v^{-6}+6v^{-4})  +z^{6}(v^{-8}-7v^{-6}+v^{-4})  -v^{-6}z^{8} $ 

k10c3: $ (v^{-6}-v^{-2}+v^{4})  +z^{2}(-v^{-4}-2v^{-2}-2-v^{2}) $ 

k10c4: $ (v^{-4}-2v^{2}+2v^{4})  +z^{2}(v^{-4}-2v^{-2}-2-3v^{2}+v^{4})  +z^{4}(-v^{-2}-1-v^{2}) $ 

k10c5: $ (-v^{2}+5v^{4}-3v^{6})  +z^{2}(-6v^{2}+17v^{4}-7v^{6})  +z^{4}(-5v^{2}+17v^{4}-5v^{6})  +z^{6}(-v^{2}+7v^{4}-v^{6})  +v^{4}z^{8} $ 

k10c6: $ (v^{-8}-v^{-6}-2v^{-4}+3v^{-2})  +z^{2}(3v^{-8}-4v^{-6}-4v^{-4}+4v^{-2})  +z^{4}(v^{-8}-4v^{-6}-4v^{-4}+v^{-2})  +z^{6}(-v^{-6}-v^{-4}) $ 

k10c7: $ (v^{-8}-2v^{-6}+v^{-4}+1)  +z^{2}(v^{-8}-2v^{-6}-v^{-2}+1)  +z^{4}(-v^{-6}-v^{-4}-v^{-2}) $ 

k10c8: $ (v^{-6}-3v^{-2}+3)  +z^{2}(3v^{-6}-3v^{-4}-7v^{-2}+4)  +z^{4}(v^{-6}-4v^{-4}-5v^{-2}+1)  +z^{6}(-v^{-4}-v^{-2}) $ 

k10c9: $ (3-4v^{2}+2v^{4})  +z^{2}(7-16v^{2}+7v^{4})  +z^{4}(5-17v^{2}+5v^{4})  +z^{6}(1-7v^{2}+v^{4})  -v^{2}z^{8} $ 

k10c10: $ (1-v^{2}+2v^{4}-v^{6})  +z^{2}(-v^{-2}+1+2v^{4}-v^{6})  +z^{4}(1+v^{2}+v^{4}) $ 

k10c11: $ (v^{-6}-v^{-2}-1+2v^{2})  +z^{2}(v^{-6}-v^{-4}-4v^{-2}-2+v^{2})  +z^{4}(-v^{-4}-2v^{-2}-1) $ 

k10c12: $ (-1+2v^{2}+2v^{4}-2v^{6})  +z^{2}(-3+5v^{2}+5v^{4}-3v^{6})  +z^{4}(-1+4v^{2}+4v^{4}-v^{6})  +z^{6}(v^{2}+v^{4}) $ 

k10c13: $ (v^{-6}-v^{-4}+v^{-2}-1+v^{4})  +z^{2}(-2v^{-4}-1-2v^{2})  +z^{4}(v^{-2}+1) $ 

k10c14: $ (-v^{-6}+v^{-4}+v^{-2})  +z^{2}(2v^{-8}-2v^{-6}-v^{-4}+3v^{-2})  +z^{4}(v^{-8}-3v^{-6}-3v^{-4}+v^{-2})  +z^{6}(-v^{-6}-v^{-4}) $ 

k10c15: $ (-v^{-2}+1+3v^{2}-2v^{4})  +z^{2}(-3v^{-2}+4+5v^{2}-3v^{4})  +z^{4}(-v^{-2}+4+4v^{2}-v^{4})  +z^{6}(1+v^{2}) $ 

k10c16: $ (v^{-2}+1-2v^{2}+v^{6})  +z^{2}(v^{-2}-1-4v^{2}-v^{4}+v^{6})  +z^{4}(-1-2v^{2}-v^{4}) $ 

k10c17: $ (-2v^{-2}+5-2v^{2})  +z^{2}(-7v^{-2}+16-7v^{2})  +z^{4}(-5v^{-2}+17-5v^{2})  +z^{6}(-v^{-2}+7-v^{2})  +z^{8} $ 

k10c18: $ (v^{-4}-v^{-2}+v^{2})  +z^{2}(v^{-6}-3v^{-2}-1+v^{2})  +z^{4}(-v^{-4}-2v^{-2}-1) $ 

k10c19: $ (-v^{-2}+3-v^{2})  +z^{2}(-2v^{-4}+v^{-2}+5-3v^{2})  +z^{4}(-v^{-4}+3v^{-2}+4-v^{2})  +z^{6}(v^{-2}+1) $ 

k10c20: $ (v^{-8}-v^{-6}-v^{-2}+2)  +z^{2}(v^{-8}-2v^{-6}-v^{-4}-2v^{-2}+1)  +z^{4}(-v^{-6}-v^{-4}-v^{-2}) $ 

k10c21: $ (v^{-8}-3v^{-6}+2v^{-4}+v^{-2})  +z^{2}(3v^{-8}-5v^{-6}+3v^{-2})  +z^{4}(v^{-8}-4v^{-6}-3v^{-4}+v^{-2})  +z^{6}(-v^{-6}-v^{-4}) $ 

k10c22: $ (2v^{-2}-1-2v^{2}+2v^{4})  +z^{2}(3v^{-2}-5-5v^{2}+3v^{4})  +z^{4}(v^{-2}-4-4v^{2}+v^{4})  +z^{6}(-1-v^{2}) $ 

k10c23: $ (3v^{4}-2v^{6})  +z^{2}(-2+2v^{2}+6v^{4}-3v^{6})  +z^{4}(-1+3v^{2}+4v^{4}-v^{6})  +z^{6}(v^{2}+v^{4}) $ 

k10c24: $ (v^{-8}-v^{-6}-v^{-4}+v^{-2}+1)  +z^{2}(v^{-8}-v^{-6}-3v^{-4}+1)  +z^{4}(-v^{-6}-2v^{-4}-v^{-2}) $ 

k10c25: $ (v^{-8}-2v^{-6}+2v^{-2})  +z^{2}(2v^{-8}-3v^{-6}-2v^{-4}+3v^{-2})  +z^{4}(v^{-8}-3v^{-6}-3v^{-4}+v^{-2})  +z^{6}(-v^{-6}-v^{-4}) $ 

k10c26: $ (v^{-2}+1-3v^{2}+2v^{4})  +z^{2}(2v^{-2}-2-6v^{2}+3v^{4})  +z^{4}(v^{-2}-3-4v^{2}+v^{4})  +z^{6}(-1-v^{2}) $ 

k10c27: $ (-v^{-6}+v^{-4}+v^{-2})  +z^{2}(-2v^{-6}+3v^{-4}+3v^{-2}-2)  +z^{4}(-v^{-6}+3v^{-4}+3v^{-2}-1)  +z^{6}(v^{-4}+v^{-2}) $ 

k10c28: $ (-1+3v^{2}-v^{6})  +z^{2}(-v^{-2}+4v^{2}+v^{4}-v^{6})  +z^{4}(1+2v^{2}+v^{4}) $ 

k10c29: $ (v^{-6}-v^{-4}+v^{-2}-2+2v^{2})  +z^{2}(v^{-6}-4v^{-4}+3v^{-2}-5+v^{2})  +z^{4}(-2v^{-4}+3v^{-2}-2)  +v^{-2}z^{6} $ 

k10c30: $ (-v^{-4}+2v^{-2})  +z^{2}(v^{-8}-2v^{-4}+v^{-2}+1)  +z^{4}(-v^{-6}-2v^{-4}-v^{-2}) $ 

k10c31: $ (-v^{-2}+2+v^{2}-v^{4})  +z^{2}(-v^{-4}+3+v^{2}-v^{4})  +z^{4}(v^{-2}+2+v^{2}) $ 

k10c32: $ (v^{-2}-1+v^{2})  +z^{2}(2v^{-4}-2v^{-2}-3+2v^{2})  +z^{4}(v^{-4}-3v^{-2}-3+v^{2})  +z^{6}(-v^{-2}-1) $ 

k10c33: $ 1  +z^{2}(-v^{-4}+2-v^{4})  +z^{4}(v^{-2}+2+v^{2}) $ 

k10c34: $ (-v^{-2}+2+v^{4}-v^{6})  +z^{2}(-v^{-2}+2+v^{2}+2v^{4}-v^{6})  +z^{4}(1+v^{2}+v^{4}) $ 

k10c35: $ (v^{-4}-v^{-2}+1-v^{4}+v^{6})  +z^{2}(-2v^{-2}-2v^{4})  +z^{4}(1+v^{2}) $ 

k10c36: $ (-v^{-6}+2v^{-4}-v^{-2}+1)  +z^{2}(v^{-8}-v^{-6}+v^{-4}-v^{-2}+1)  +z^{4}(-v^{-6}-v^{-4}-v^{-2}) $ 

k10c37: $ (-v^{-4}+v^{-2}+1+v^{2}-v^{4})  +z^{2}(-v^{-4}+v^{-2}+3+v^{2}-v^{4})  +z^{4}(v^{-2}+2+v^{2}) $ 

k10c38: $ (v^{-6}-2v^{-4}+v^{-2}+1)  +z^{2}(v^{-8}-3v^{-4}+1)  +z^{4}(-v^{-6}-2v^{-4}-v^{-2}) $ 

k10c39: $ (-v^{-4}+2v^{-2})  +z^{2}(2v^{-8}-2v^{-6}-2v^{-4}+3v^{-2})  +z^{4}(v^{-8}-3v^{-6}-3v^{-4}+v^{-2})  +z^{6}(-v^{-6}-v^{-4}) $ 

k10c40: $ (-1+3v^{2}-v^{6})  +z^{2}(-2+4v^{2}+3v^{4}-2v^{6})  +z^{4}(-1+3v^{2}+3v^{4}-v^{6})  +z^{6}(v^{2}+v^{4}) $ 

k10c41: $ (v^{-6}-2v^{-4}+2v^{-2}-1+v^{2})  +z^{2}(v^{-6}-4v^{-4}+4v^{-2}-4+v^{2})  +z^{4}(-2v^{-4}+3v^{-2}-2)  +v^{-2}z^{6} $ 

k10c42: $ (v^{-2}-2+3v^{2}-v^{4})  +z^{2}(-v^{-4}+3v^{-2}-5+4v^{2}-v^{4})  +z^{4}(2v^{-2}-3+2v^{2})  -z^{6} $ 

k10c43: $ (-v^{-4}+2v^{-2}-1+2v^{2}-v^{4})  +z^{2}(-v^{-4}+4v^{-2}-4+4v^{2}-v^{4})  +z^{4}(2v^{-2}-3+2v^{2})  -z^{6} $ 

k10c44: $ (-v^{-4}+3v^{-2}-2+v^{2})  +z^{2}(v^{-6}-3v^{-4}+5v^{-2}-4+v^{2})  +z^{4}(-2v^{-4}+3v^{-2}-2)  +v^{-2}z^{6} $ 

k10c45: $ (2v^{-2}-3+2v^{2})  +z^{2}(-v^{-4}+3v^{-2}-6+3v^{2}-v^{4})  +z^{4}(2v^{-2}-3+2v^{2})  -z^{6} $ 

k10c46: $ (6v^{4}-8v^{6}+3v^{8})  +z^{2}(11v^{4}-18v^{6}+7v^{8})  +z^{4}(6v^{4}-17v^{6}+5v^{8})  +z^{6}(v^{4}-7v^{6}+v^{8})  -v^{6}z^{8} $ 

k10c47: $ (-3v^{2}+9v^{4}-5v^{6})  +z^{2}(-7v^{2}+21v^{4}-8v^{6})  +z^{4}(-5v^{2}+18v^{4}-5v^{6})  +z^{6}(-v^{2}+7v^{4}-v^{6})  +v^{4}z^{8} $ 

k10c48: $ (-4v^{-2}+9-4v^{2})  +z^{2}(-8v^{-2}+20-8v^{2})  +z^{4}(-5v^{-2}+18-5v^{2})  +z^{6}(-v^{-2}+7-v^{2})  +z^{8} $ 

k10c49: $ (2v^{-12}-7v^{-10}+5v^{-8}+v^{-6})  +z^{2}(v^{-12}-10v^{-10}+12v^{-8}+4v^{-6})  +z^{4}(-3v^{-10}+9v^{-8}+4v^{-6})  +z^{6}(2v^{-8}+v^{-6}) $ 

k10c50: $ (2v^{2}+v^{4}-4v^{6}+2v^{8})  +z^{2}(3v^{2}-v^{4}-6v^{6}+3v^{8})  +z^{4}(v^{2}-3v^{4}-4v^{6}+v^{8})  +z^{6}(-v^{4}-v^{6}) $ 

k10c51: $ (-1+v^{2}+4v^{4}-3v^{6})  +z^{2}(-2+3v^{2}+7v^{4}-3v^{6})  +z^{4}(-1+3v^{2}+4v^{4}-v^{6})  +z^{6}(v^{2}+v^{4}) $ 

k10c52: $ (-2v^{-2}+4-v^{4})  +z^{2}(-3v^{-2}+6+2v^{2}-2v^{4})  +z^{4}(-v^{-2}+4+3v^{2}-v^{4})  +z^{6}(1+v^{2}) $ 

k10c53: $ (v^{-12}-3v^{-10}+3v^{-6})  +z^{2}(-3v^{-10}+2v^{-8}+6v^{-6}+v^{-4})  +z^{4}(2v^{-8}+3v^{-6}+v^{-4}) $ 

k10c54: $ (-2v^{-2}+3+2v^{2}-2v^{4})  +z^{2}(-3v^{-2}+5+5v^{2}-3v^{4})  +z^{4}(-v^{-2}+4+4v^{2}-v^{4})  +z^{6}(1+v^{2}) $ 

k10c55: $ (v^{-12}-3v^{-10}+v^{-8}+v^{-6}+v^{-4})  +z^{2}(-3v^{-10}+3v^{-8}+3v^{-6}+2v^{-4})  +z^{4}(2v^{-8}+2v^{-6}+v^{-4}) $ 

k10c56: $ (2v^{2}-2v^{6}+v^{8})  +z^{2}(3v^{2}-2v^{4}-3v^{6}+2v^{8})  +z^{4}(v^{2}-3v^{4}-3v^{6}+v^{8})  +z^{6}(-v^{4}-v^{6}) $ 

k10c57: $ (-1+2v^{2}+2v^{4}-2v^{6})  +z^{2}(-2+4v^{2}+4v^{4}-2v^{6})  +z^{4}(-1+3v^{2}+3v^{4}-v^{6})  +z^{6}(v^{2}+v^{4}) $ 

k10c58: $ (v^{-6}-2v^{-4}+3v^{-2}-2+v^{4})  +z^{2}(-3v^{-4}+3v^{-2}-2-2v^{2})  +z^{4}(2v^{-2}+1) $ 

k10c59: $ (v^{-2}-2+4v^{2}-3v^{4}+v^{6})  +z^{2}(v^{-2}-4+5v^{2}-4v^{4}+v^{6})  +z^{4}(-2+3v^{2}-2v^{4})  +v^{2}z^{6} $ 

k10c60: $ (v^{-6}-3v^{-4}+4v^{-2}-2+v^{2})  +z^{2}(-3v^{-4}+6v^{-2}-5+v^{2})  +z^{4}(3v^{-2}-3+v^{2})  -z^{6} $ 

k10c61: $ (4-5v^{2}+v^{4}+v^{6})  +z^{2}(4-8v^{2}-3v^{4}+3v^{6})  +z^{4}(1-5v^{2}-4v^{4}+v^{6})  +z^{6}(-v^{2}-v^{4}) $ 

k10c62: $ (-2v^{2}+7v^{4}-4v^{6})  +z^{2}(-7v^{2}+20v^{4}-8v^{6})  +z^{4}(-5v^{2}+18v^{4}-5v^{6})  +z^{6}(-v^{2}+7v^{4}-v^{6})  +v^{4}z^{8} $ 

k10c63: $ (v^{-12}-4v^{-10}+3v^{-8}+v^{-4})  +z^{2}(-3v^{-10}+4v^{-8}+3v^{-6}+2v^{-4})  +z^{4}(2v^{-8}+2v^{-6}+v^{-4}) $ 

k10c64: $ (4-6v^{2}+3v^{4})  +z^{2}(8-19v^{2}+8v^{4})  +z^{4}(5-18v^{2}+5v^{4})  +z^{6}(1-7v^{2}+v^{4})  -v^{2}z^{8} $ 

k10c65: $ (-v^{2}+5v^{4}-3v^{6})  +z^{2}(-2+2v^{2}+7v^{4}-3v^{6})  +z^{4}(-1+3v^{2}+4v^{4}-v^{6})  +z^{6}(v^{2}+v^{4}) $ 

k10c66: $ (v^{-12}-4v^{-10}+2v^{-8}+2v^{-6})  +z^{2}(v^{-12}-8v^{-10}+9v^{-8}+5v^{-6})  +z^{4}(-3v^{-10}+8v^{-8}+4v^{-6})  +z^{6}(2v^{-8}+v^{-6}) $ 

k10c67: $ 1  +z^{2}(v^{-8}-2v^{-4}+1)  +z^{4}(-v^{-6}-2v^{-4}-v^{-2}) $ 

k10c68: $ (-v^{-6}+v^{-4}+v^{-2})  +z^{2}(-v^{-6}+v^{-4}+3v^{-2}-v^{2})  +z^{4}(v^{-4}+2v^{-2}+1) $ 

k10c69: $ (2v^{2}-2v^{4}+2v^{6}-v^{8})  +z^{2}(-1+5v^{2}-5v^{4}+3v^{6})  +z^{4}(-1+3v^{2}-3v^{4})  +v^{2}z^{6} $ 

k10c70: $ (2v^{-2}-3+3v^{2}-2v^{4}+v^{6})  +z^{2}(v^{-2}-5+4v^{2}-4v^{4}+v^{6})  +z^{4}(-2+3v^{2}-2v^{4})  +v^{2}z^{6} $ 

k10c71: $ (-v^{-4}+3v^{-2}-3+3v^{2}-v^{4})  +z^{2}(-v^{-4}+4v^{-2}-5+4v^{2}-v^{4})  +z^{4}(2v^{-2}-3+2v^{2})  -z^{6} $ 

k10c72: $ (2v^{2}-2v^{4}+2v^{6}-v^{8})  +z^{2}(3v^{2}-3v^{4}+v^{6}+v^{8})  +z^{4}(v^{2}-3v^{4}-2v^{6}+v^{8})  +z^{6}(-v^{4}-v^{6}) $ 

k10c73: $ (-v^{-8}+3v^{-6}-4v^{-4}+3v^{-2})  +z^{2}(3v^{-6}-6v^{-4}+5v^{-2}-1)  +z^{4}(-3v^{-4}+3v^{-2}-1)  +v^{-2}z^{6} $ 

k10c74: $ (v^{-8}-2v^{-6}+2v^{-2})  +z^{2}(v^{-8}-v^{-6}-2v^{-4}+v^{-2}+1)  +z^{4}(-v^{-6}-2v^{-4}-v^{-2}) $ 

k10c75: $ (3v^{2}-3v^{4}+v^{6})  +z^{2}(v^{-2}-4+6v^{2}-3v^{4})  +z^{4}(v^{-2}-3+3v^{2})  -z^{6} $ 

k10c76: $ (4v^{2}-4v^{4}+v^{8})  +z^{2}(4v^{2}-6v^{4}-2v^{6}+2v^{8})  +z^{4}(v^{2}-4v^{4}-3v^{6}+v^{8})  +z^{6}(-v^{4}-v^{6}) $ 

k10c77: $ (-2+5v^{2}-v^{4}-v^{6})  +z^{2}(-3+7v^{2}+2v^{4}-2v^{6})  +z^{4}(-1+4v^{2}+3v^{4}-v^{6})  +z^{6}(v^{2}+v^{4}) $ 

k10c78: $ (v^{-10}-4v^{-8}+4v^{-6}-v^{-4}+v^{-2})  +z^{2}(-3v^{-8}+7v^{-6}-3v^{-4}+2v^{-2})  +z^{4}(3v^{-6}-3v^{-4}+v^{-2})  -v^{-4}z^{6} $ 

k10c79: $ (-5v^{-2}+11-5v^{2})  +z^{2}(-9v^{-2}+23-9v^{2})  +z^{4}(-5v^{-2}+19-5v^{2})  +z^{6}(-v^{-2}+7-v^{2})  +z^{8} $ 

k10c80: $ (2v^{-12}-6v^{-10}+3v^{-8}+2v^{-6})  +z^{2}(v^{-12}-9v^{-10}+9v^{-8}+5v^{-6})  +z^{4}(-3v^{-10}+8v^{-8}+4v^{-6})  +z^{6}(2v^{-8}+v^{-6}) $ 

k10c81: $ (-v^{-4}+v^{-2}+1+v^{2}-v^{4})  +z^{2}(-v^{-4}+3v^{-2}-1+3v^{2}-v^{4})  +z^{4}(2v^{-2}-2+2v^{2})  -z^{6} $ 

k10c82: $ 1  +z^{2}(4v^{-4}-8v^{-2}+4)  +z^{4}(4v^{-4}-12v^{-2}+4)  +z^{6}(v^{-4}-6v^{-2}+1)  -v^{-2}z^{8} $ 

k10c83: $ (2-2v^{2}+v^{4})  +z^{2}(v^{-2}-4v^{2}+2v^{4})  +z^{4}(v^{-2}-2-3v^{2}+v^{4})  +z^{6}(-1-v^{2}) $ 

k10c84: $ (-1+4v^{2}-2v^{4})  +z^{2}(-2+5v^{2}-v^{6})  +z^{4}(-1+3v^{2}+2v^{4}-v^{6})  +z^{6}(v^{2}+v^{4}) $ 

k10c85: $ (-v^{-6}+v^{-4}+v^{-2})  +z^{2}(-4v^{-6}+9v^{-4}-3v^{-2})  +z^{4}(-4v^{-6}+12v^{-4}-4v^{-2})  +z^{6}(-v^{-6}+6v^{-4}-v^{-2})  +v^{-4}z^{8} $ 

k10c86: $ (1-v^{2}+2v^{4}-v^{6})  +z^{2}(-1+4v^{4}-2v^{6})  +z^{4}(-1+2v^{2}+3v^{4}-v^{6})  +z^{6}(v^{2}+v^{4}) $ 

k10c87: $ (v^{-2}-2+3v^{2}-v^{4})  +z^{2}(2v^{-2}-4+v^{2}+v^{4})  +z^{4}(v^{-2}-3-2v^{2}+v^{4})  +z^{6}(-1-v^{2}) $ 

k10c88: $ (v^{-2}-1+v^{2})  +z^{2}(-v^{-4}+2v^{-2}-3+2v^{2}-v^{4})  +z^{4}(2v^{-2}-2+2v^{2})  -z^{6} $ 

k10c89: $ (-v^{-8}+2v^{-6}-v^{-4}+1)  +z^{2}(3v^{-6}-4v^{-4}+2v^{-2})  +z^{4}(-3v^{-4}+2v^{-2}-1)  +v^{-2}z^{6} $ 

k10c90: $ (2v^{-2}-2+v^{4})  +z^{2}(2v^{-2}-4-3v^{2}+2v^{4})  +z^{4}(v^{-2}-3-3v^{2}+v^{4})  +z^{6}(-1-v^{2}) $ 

k10c91: $ (-2v^{-2}+5-2v^{2})  +z^{2}(-5v^{-2}+12-5v^{2})  +z^{4}(-4v^{-2}+13-4v^{2})  +z^{6}(-v^{-2}+6-v^{2})  +z^{8} $ 

k10c92: $ (v^{2}+v^{4}-v^{6})  +z^{2}(2v^{2}-v^{6}+v^{8})  +z^{4}(v^{2}-2v^{4}-2v^{6}+v^{8})  +z^{6}(-v^{4}-v^{6}) $ 

k10c93: $ (-v^{-4}+2v^{-2})  +z^{2}(-2v^{-4}+3v^{-2}+2-2v^{2})  +z^{4}(-v^{-4}+3v^{-2}+3-v^{2})  +z^{6}(v^{-2}+1) $ 

k10c94: $ (3-4v^{2}+2v^{4})  +z^{2}(5-12v^{2}+5v^{4})  +z^{4}(4-13v^{2}+4v^{4})  +z^{6}(1-6v^{2}+v^{4})  -v^{2}z^{8} $ 

k10c95: $ (3v^{4}-2v^{6})  +z^{2}(-1+v^{2}+5v^{4}-2v^{6})  +z^{4}(-1+2v^{2}+3v^{4}-v^{6})  +z^{6}(v^{2}+v^{4}) $ 

k10c96: $ (2v^{-2}-3+3v^{2}-2v^{4}+v^{6})  +z^{2}(v^{-2}-6+5v^{2}-3v^{4})  +z^{4}(v^{-2}-3+3v^{2})  -z^{6} $ 

k10c97: $ (2v^{2}-2v^{4}+2v^{6}-v^{8})  +z^{2}(1+2v^{2}-4v^{4}+2v^{6}+v^{8})  +z^{4}(-v^{2}-3v^{4}-v^{6}) $ 

k10c98: $ (2v^{-8}-5v^{-6}+3v^{-4}+v^{-2})  +z^{2}(2v^{-8}-5v^{-6}+v^{-4}+2v^{-2})  +z^{4}(v^{-8}-3v^{-6}-2v^{-4}+v^{-2})  +z^{6}(-v^{-6}-v^{-4}) $ 

k10c99: $ (-4v^{-2}+9-4v^{2})  +z^{2}(-6v^{-2}+16-6v^{2})  +z^{4}(-4v^{-2}+14-4v^{2})  +z^{6}(-v^{-2}+6-v^{2})  +z^{8} $ 

k10c100: $ (-3v^{-6}+5v^{-4}-v^{-2})  +z^{2}(-5v^{-6}+13v^{-4}-4v^{-2})  +z^{4}(-4v^{-6}+13v^{-4}-4v^{-2})  +z^{6}(-v^{-6}+6v^{-4}-v^{-2})  +v^{-4}z^{8} $ 

k10c101: $ (2v^{6}+2v^{8}-4v^{10}+v^{12})  +z^{2}(v^{4}+5v^{6}+5v^{8}-4v^{10})  +z^{4}(v^{4}+3v^{6}+3v^{8}) $ 

k10c102: $ (v^{-2}-v^{2}+v^{4})  +z^{2}(2v^{-2}-3-3v^{2}+2v^{4})  +z^{4}(v^{-2}-3-3v^{2}+v^{4})  +z^{6}(-1-v^{2}) $ 

k10c103: $ (-v^{-6}+3v^{-2}-1)  +z^{2}(-2v^{-6}+3v^{-4}+4v^{-2}-2)  +z^{4}(-v^{-6}+3v^{-4}+3v^{-2}-1)  +z^{6}(v^{-4}+v^{-2}) $ 

k10c104: $ (-v^{-2}+3-v^{2})  +z^{2}(-5v^{-2}+11-5v^{2})  +z^{4}(-4v^{-2}+13-4v^{2})  +z^{6}(-v^{-2}+6-v^{2})  +z^{8} $ 

k10c105: $ (v^{-2}-1+v^{2})  +z^{2}(v^{-2}-3+2v^{2}-2v^{4}+v^{6})  +z^{4}(-2+2v^{2}-2v^{4})  +v^{2}z^{6} $ 

k10c106: $ (2-2v^{2}+v^{4})  +z^{2}(5-11v^{2}+5v^{4})  +z^{4}(4-13v^{2}+4v^{4})  +z^{6}(1-6v^{2}+v^{4})  -v^{2}z^{8} $ 

k10c107: $ (2v^{2}-v^{4})  +z^{2}(-v^{-4}+2v^{-2}-2+3v^{2}-v^{4})  +z^{4}(2v^{-2}-2+2v^{2})  -z^{6} $ 

k10c108: $ 1  +z^{2}(-2v^{-2}+2+2v^{2}-2v^{4})  +z^{4}(-v^{-2}+3+3v^{2}-v^{4})  +z^{6}(1+v^{2}) $ 

k10c109: $ (-3v^{-2}+7-3v^{2})  +z^{2}(-6v^{-2}+15-6v^{2})  +z^{4}(-4v^{-2}+14-4v^{2})  +z^{6}(-v^{-2}+6-v^{2})  +z^{8} $ 

k10c110: $ (v^{-6}-v^{-4}+v^{2})  +z^{2}(v^{-6}-3v^{-4}+v^{-2}-3+v^{2})  +z^{4}(-2v^{-4}+2v^{-2}-2)  +v^{-2}z^{6} $ 

k10c111: $ (v^{2}+2v^{4}-3v^{6}+v^{8})  +z^{2}(2v^{2}+v^{4}-4v^{6}+2v^{8})  +z^{4}(v^{2}-2v^{4}-3v^{6}+v^{8})  +z^{6}(-v^{4}-v^{6}) $ 

k10c112: $ (-2v^{-4}+4v^{-2}-1)  +z^{2}(v^{-4}+1)  +z^{4}(3v^{-4}-7v^{-2}+3)  +z^{6}(v^{-4}-5v^{-2}+1)  -v^{-2}z^{8} $ 

k10c113: $ (3v^{2}-3v^{4}+v^{6})  +z^{2}(-1+3v^{2}-2v^{4})  +z^{4}(-1+2v^{2}+v^{4}-v^{6})  +z^{6}(v^{2}+v^{4}) $ 

k10c114: $ (-v^{-4}+2v^{-2})  +z^{2}(v^{-4}-1+v^{2})  +z^{4}(v^{-4}-2v^{-2}-2+v^{2})  +z^{6}(-v^{-2}-1) $ 

k10c115: $ (-v^{-2}+3-v^{2})  +z^{2}(-v^{-4}+v^{-2}+1+v^{2}-v^{4})  +z^{4}(2v^{-2}-1+2v^{2})  -z^{6} $ 

k10c116: $ 1  +z^{2}(2v^{-4}-4v^{-2}+2)  +z^{4}(3v^{-4}-8v^{-2}+3)  +z^{6}(v^{-4}-5v^{-2}+1)  -v^{-2}z^{8} $ 

k10c117: $ (v^{2}+v^{4}-v^{6})  +z^{2}(-1+2v^{2}+2v^{4}-v^{6})  +z^{4}(-1+2v^{2}+2v^{4}-v^{6})  +z^{6}(v^{2}+v^{4}) $ 

k10c118: $ 1  +z^{2}(-2v^{-2}+4-2v^{2})  +z^{4}(-3v^{-2}+8-3v^{2})  +z^{6}(-v^{-2}+5-v^{2})  +z^{8} $ 

k10c119: $ (v^{-2}-1+v^{2})  +z^{2}(v^{-2}-2-v^{2}+v^{4})  +z^{4}(v^{-2}-2-2v^{2}+v^{4})  +z^{6}(-1-v^{2}) $ 

k10c120: $ (v^{-12}-3v^{-10}+3v^{-6})  +z^{2}(-4v^{-10}+3v^{-8}+7v^{-6})  +z^{4}(3v^{-8}+4v^{-6}+v^{-4}) $ 

k10c121: $ (-v^{-6}+2v^{-4}-v^{-2}+1)  +z^{2}(-v^{-6}+3v^{-4}-v^{-2})  +z^{4}(-v^{-6}+2v^{-4}+v^{-2}-1)  +z^{6}(v^{-4}+v^{-2}) $ 

k10c122: $ (-1+4v^{2}-2v^{4})  +z^{2}(v^{-2}-2+3v^{2})  +z^{4}(v^{-2}-2-v^{2}+v^{4})  +z^{6}(-1-v^{2}) $ 

k10c123: $ (2v^{-2}-3+2v^{2})  +z^{2}(v^{-2}-4+v^{2})  +z^{4}(-2v^{-2}+3-2v^{2})  +z^{6}(-v^{-2}+4-v^{2})  +z^{8} $ 

k10c124: $ (7v^{8}-8v^{10}+2v^{12})  +z^{2}(21v^{8}-14v^{10}+v^{12})  +z^{4}(21v^{8}-7v^{10})  +z^{6}(8v^{8}-v^{10})  +v^{8}z^{8} $ 

k10c125: $ (-3v^{-2}+7-3v^{2})  +z^{2}(-4v^{-2}+11-4v^{2})  +z^{4}(-v^{-2}+6-v^{2})  +z^{6} $ 

k10c126: $ (-4v^{-6}+7v^{-4}-2v^{-2})  +z^{2}(-4v^{-6}+12v^{-4}-3v^{-2})  +z^{4}(-v^{-6}+6v^{-4}-v^{-2})  +v^{-4}z^{6} $ 

k10c127: $ (2v^{-8}-6v^{-6}+5v^{-4})  +z^{2}(3v^{-8}-9v^{-6}+7v^{-4})  +z^{4}(v^{-8}-5v^{-6}+2v^{-4})  -v^{-6}z^{6} $ 

k10c128: $ (2v^{6}+2v^{8}-4v^{10}+v^{12})  +z^{2}(6v^{6}+6v^{8}-5v^{10})  +z^{4}(5v^{6}+5v^{8}-v^{10})  +z^{6}(v^{6}+v^{8}) $ 

k10c129: $ (-v^{-4}+v^{-2}+2-v^{2})  +z^{2}(-v^{-4}+2v^{-2}+2-v^{2})  +z^{4}(v^{-2}+1) $ 

k10c130: $ (-2v^{-6}+2v^{-4}+2v^{-2}-1)  +z^{2}(-v^{-6}+3v^{-4}+3v^{-2}-1)  +z^{4}(v^{-4}+v^{-2}) $ 

k10c131: $ (v^{-8}-2v^{-6}+2v^{-2})  +z^{2}(v^{-8}-2v^{-6}-v^{-4}+2v^{-2})  +z^{4}(-v^{-6}-v^{-4}) $ 

k10c132: $ (-2v^{-6}+3v^{-4})  +z^{2}(-v^{-6}+4v^{-4})  +v^{-4}z^{4} $ 

k10c133: $ (v^{-8}-3v^{-6}+2v^{-4}+v^{-2})  +z^{2}(v^{-8}-3v^{-6}+2v^{-4}+v^{-2})  -v^{-6}z^{4} $ 

k10c134: $ (3v^{6}-3v^{10}+v^{12})  +z^{2}(7v^{6}+3v^{8}-4v^{10})  +z^{4}(5v^{6}+4v^{8}-v^{10})  +z^{6}(v^{6}+v^{8}) $ 

k10c135: $ (-v^{-4}+4-2v^{2})  +z^{2}(-v^{-4}+v^{-2}+5-2v^{2})  +z^{4}(v^{-2}+2) $ 

k10c136: $ (v^{-2}-2+3v^{2}-v^{4})  +z^{2}(v^{-2}-3+2v^{2})  -z^{4} $ 

k10c137: $ (v^{-6}-2v^{-4}+2v^{-2}-1+v^{2})  +z^{2}(-2v^{-4}+2v^{-2}-2)  +v^{-2}z^{4} $ 

k10c138: $ (2v^{-2}-3+3v^{2}-2v^{4}+v^{6})  +z^{2}(v^{-2}-6+5v^{2}-3v^{4})  +z^{4}(-2+4v^{2}-v^{4})  +v^{2}z^{6} $ 

k10c139: $ (6v^{8}-6v^{10}+v^{12})  +z^{2}(21v^{8}-13v^{10}+v^{12})  +z^{4}(21v^{8}-7v^{10})  +z^{6}(8v^{8}-v^{10})  +v^{8}z^{8} $ 

k10c140: $ (-2v^{-6}+4v^{-4}-2v^{-2}+1)  +z^{2}(-v^{-6}+4v^{-4}-v^{-2})  +v^{-4}z^{4} $ 

k10c141: $ (v^{-4}-2v^{-2}+2)  +z^{2}(3v^{-4}-7v^{-2}+3)  +z^{4}(v^{-4}-5v^{-2}+1)  -v^{-2}z^{6} $ 

k10c142: $ (v^{6}+4v^{8}-5v^{10}+v^{12})  +z^{2}(6v^{6}+7v^{8}-5v^{10})  +z^{4}(5v^{6}+5v^{8}-v^{10})  +z^{6}(v^{6}+v^{8}) $ 

k10c143: $ (-2v^{-6}+3v^{-4})  +z^{2}(-3v^{-6}+8v^{-4}-2v^{-2})  +z^{4}(-v^{-6}+5v^{-4}-v^{-2})  +v^{-4}z^{6} $ 

k10c144: $ (2v^{-4}-4v^{-2}+3)  +z^{2}(v^{-6}-5v^{-2}+2)  +z^{4}(-v^{-4}-2v^{-2}) $ 

k10c145: $ (-v^{-10}+v^{-8}-v^{-6}+2v^{-4})  +z^{2}(v^{-8}+4v^{-4})  +v^{-4}z^{4} $ 

k10c146: $ 1  +z^{2}(-v^{-4}+v^{-2}+1-v^{2})  +z^{4}(v^{-2}+1) $ 

k10c147: $ (v^{-2}-1+v^{2})  +z^{2}(v^{-2}-2-v^{2}+v^{4})  +z^{4}(-1-v^{2}) $ 

k10c148: $ (-3v^{-6}+5v^{-4}-v^{-2})  +z^{2}(-3v^{-6}+9v^{-4}-2v^{-2})  +z^{4}(-v^{-6}+5v^{-4}-v^{-2})  +v^{-4}z^{6} $ 

k10c149: $ (v^{-8}-4v^{-6}+4v^{-4})  +z^{2}(2v^{-8}-6v^{-6}+6v^{-4})  +z^{4}(v^{-8}-4v^{-6}+2v^{-4})  -v^{-6}z^{6} $ 

k10c150: $ (2v^{2}-v^{4})  +z^{2}(3v^{2}-4v^{4}+2v^{6})  +z^{4}(v^{2}-4v^{4}+v^{6})  -v^{4}z^{6} $ 

k10c151: $ (-1+3v^{2}-v^{6})  +z^{2}(-2+6v^{2}-v^{4})  +z^{4}(-1+4v^{2}-v^{4})  +v^{2}z^{6} $ 

k10c152: $ (3v^{-12}-10v^{-10}+8v^{-8})  +z^{2}(2v^{-12}-17v^{-10}+22v^{-8})  +z^{4}(-8v^{-10}+21v^{-8})  +z^{6}(-v^{-10}+8v^{-8})  +v^{-8}z^{8} $ 

k10c153: $ (-v^{-4}-v^{-2}+6-3v^{2})  +z^{2}(-v^{-4}-v^{-2}+10-4v^{2})  +z^{4}(6-v^{2})  +z^{6} $ 

k10c154: $ (4v^{6}-2v^{8}-2v^{10}+v^{12})  +z^{2}(9v^{6}-2v^{8}-2v^{10})  +6v^{6}z^{4}  +v^{6}z^{6} $ 

k10c155: $ (3-4v^{2}+2v^{4})  +z^{2}(3-8v^{2}+3v^{4})  +z^{4}(1-5v^{2}+v^{4})  -v^{2}z^{6} $ 

k10c156: $ (-v^{-4}+2v^{-2})  +z^{2}(-2v^{-4}+5v^{-2}-2)  +z^{4}(-v^{-4}+4v^{-2}-1)  +v^{-2}z^{6} $ 

k10c157: $ (2v^{4}-v^{8})  +z^{2}(5v^{4}-2v^{6}+v^{8})  +z^{4}(2v^{4}-3v^{6}+v^{8})  -v^{6}z^{6} $ 

k10c158: $ (2v^{-2}-2+v^{4})  +z^{2}(2v^{-2}-6+v^{2})  +z^{4}(v^{-2}-4+v^{2})  -z^{6} $ 

k10c159: $ (-v^{-6}+v^{-4}+v^{-2})  +z^{2}(-2v^{-6}+5v^{-4}-v^{-2})  +z^{4}(-v^{-6}+4v^{-4}-v^{-2})  +v^{-4}z^{6} $ 

k10c160: $ (v^{2}+v^{6}-v^{8})  +z^{2}(3v^{2}-3v^{4}+3v^{6})  +z^{4}(v^{2}-4v^{4}+v^{6})  -v^{4}z^{6} $ 

k10c161: $ (-v^{-10}-v^{-8}+3v^{-6})  +z^{2}(-v^{-10}-v^{-8}+9v^{-6})  +6v^{-6}z^{4}  +v^{-6}z^{6} $ 

k10c162: $ (3v^{6}-v^{8}-v^{10})  +z^{2}(9v^{6}-v^{8}-v^{10})  +6v^{6}z^{4}  +v^{6}z^{6} $ 

k10c163: $ (v^{-6}-3v^{-2}+3)  +z^{2}(v^{-6}-v^{-4}-5v^{-2}+2)  +z^{4}(-v^{-4}-2v^{-2}) $ 

k10c164: $ (1-v^{2}+2v^{4}-v^{6})  +z^{2}(-1+2v^{2})  +z^{4}(-1+3v^{2}-v^{4})  +v^{2}z^{6} $ 

k10c165: $ (-v^{-2}+3-v^{2})  +z^{2}(-v^{-4}+4-2v^{2})  +z^{4}(v^{-2}+2) $ 

k10c166: $ (v^{2}+v^{4}-v^{6})  +z^{2}(2v^{2}-v^{6}+v^{8})  +z^{4}(-v^{4}-v^{6}) $ 


\newpage


%\section{P-polynomial: $P(l,m)$} 
% $l P_{L+} +(1/l)P_{L-} +m P_{L0} = 0$ \par
%k3c1: $ (-2l^{2}-l^{4})  +l^{2}m^{2} $ 

k4c1: $ (-l^{-2}-1-l^{2})  +m^{2} $ 

k5c1: $ (3l^{4}+2l^{6})  +m^{2}(-4l^{4}-l^{6})  +l^{4}m^{4} $ 

k5c2: $ (-l^{2}+l^{4}+l^{6})  +m^{2}(l^{2}-l^{4}) $ 

k6c1: $ (-l^{-2}+l^{2}+l^{4})  +m^{2}(1-l^{2}) $ 

k6c2: $ (2+2l^{2}+l^{4})  +m^{2}(-1-3l^{2}-l^{4})  +l^{2}m^{4} $ 

k6c3: $ (l^{-2}+3+l^{2})  +m^{2}(-l^{-2}-3-l^{2})  +m^{4} $ 

k7c1: $ (-4l^{6}-3l^{8})  +m^{2}(10l^{6}+4l^{8})  +m^{4}(-6l^{6}-l^{8})  +l^{6}m^{6} $ 

k7c2: $ (-l^{2}-l^{6}-l^{8})  +m^{2}(l^{2}-l^{4}+l^{6}) $ 

k7c3: $ (-2l^{-8}-2l^{-6}+l^{-4})  +m^{2}(l^{-8}+3l^{-6}-3l^{-4})  +m^{4}(-l^{-6}+l^{-4}) $ 

k7c4: $ (-l^{-8}+2l^{-4})  +m^{2}(l^{-6}-2l^{-4}+l^{-2}) $ 

k7c5: $ (2l^{4}-l^{8})  +m^{2}(-3l^{4}+2l^{6}+l^{8})  +m^{4}(l^{4}-l^{6}) $ 

k7c6: $ (1+l^{2}+2l^{4}+l^{6})  +m^{2}(-1-2l^{2}-2l^{4})  +l^{2}m^{4} $ 

k7c7: $ (l^{-4}+2l^{-2}+2)  +m^{2}(-2l^{-2}-2-l^{2})  +m^{4} $ 

k8c1: $ (-l^{-2}-l^{4}-l^{6})  +m^{2}(1-l^{2}+l^{4}) $ 

k8c2: $ (-3l^{2}-3l^{4}-l^{6})  +m^{2}(4l^{2}+7l^{4}+3l^{6})  +m^{4}(-l^{2}-5l^{4}-l^{6})  +l^{4}m^{6} $ 

k8c3: $ (l^{-4}-1+l^{4})  +m^{2}(-l^{-2}+2-l^{2}) $ 

k8c4: $ (-2l^{-2}-2+l^{4})  +m^{2}(l^{-2}+3-2l^{2}-l^{4})  +m^{4}(-1+l^{2}) $ 

k8c5: $ (-2l^{-6}-5l^{-4}-4l^{-2})  +m^{2}(3l^{-6}+8l^{-4}+4l^{-2})  +m^{4}(-l^{-6}-5l^{-4}-l^{-2})  +l^{-4}m^{6} $ 

k8c6: $ (2+l^{2}-l^{4}-l^{6})  +m^{2}(-1-2l^{2}+2l^{4}+l^{6})  +m^{4}(l^{2}-l^{4}) $ 

k8c7: $ (-2l^{-4}-4l^{-2}-1)  +m^{2}(3l^{-4}+8l^{-2}+3)  +m^{4}(-l^{-4}-5l^{-2}-1)  +l^{-2}m^{6} $ 

k8c8: $ (-l^{-4}-l^{-2}+2+l^{2})  +m^{2}(l^{-4}+2l^{-2}-2-l^{2})  +m^{4}(-l^{-2}+1) $ 

k8c9: $ (-2l^{-2}-3-2l^{2})  +m^{2}(3l^{-2}+8+3l^{2})  +m^{4}(-l^{-2}-5-l^{2})  +m^{6} $ 

k8c10: $ (-3l^{-4}-6l^{-2}-2)  +m^{2}(3l^{-4}+9l^{-2}+3)  +m^{4}(-l^{-4}-5l^{-2}-1)  +l^{-2}m^{6} $ 

k8c11: $ (1-l^{2}-2l^{4}-l^{6})  +m^{2}(-1-l^{2}+2l^{4}+l^{6})  +m^{4}(l^{2}-l^{4}) $ 

k8c12: $ (l^{-4}+l^{-2}+1+l^{2}+l^{4})  +m^{2}(-2l^{-2}-1-2l^{2})  +m^{4} $ 

k8c13: $ (-l^{-4}-2l^{-2})  +m^{2}(l^{-4}+2l^{-2}-1-l^{2})  +m^{4}(-l^{-2}+1) $ 

k8c14: $ 1  +m^{2}(-1-l^{2}+l^{4}+l^{6})  +m^{4}(l^{2}-l^{4}) $ 

k8c15: $ (l^{4}-3l^{6}-4l^{8}-l^{10})  +m^{2}(-2l^{4}+5l^{6}+3l^{8})  +m^{4}(l^{4}-2l^{6}) $ 

k8c16: $ (-2l^{2}-l^{4})  +m^{2}(2+5l^{2}+2l^{4})  +m^{4}(-1-4l^{2}-l^{4})  +l^{2}m^{6} $ 

k8c17: $ (-l^{-2}-1-l^{2})  +m^{2}(2l^{-2}+5+2l^{2})  +m^{4}(-l^{-2}-4-l^{2})  +m^{6} $ 

k8c18: $ (l^{-2}+3+l^{2})  +m^{2}(l^{-2}+1+l^{2})  +m^{4}(-l^{-2}-3-l^{2})  +m^{6} $ 

k8c19: $ (-l^{-10}-5l^{-8}-5l^{-6})  +m^{2}(5l^{-8}+10l^{-6})  +m^{4}(-l^{-8}-6l^{-6})  +l^{-6}m^{6} $ 

k8c20: $ (-1-4l^{2}-2l^{4})  +m^{2}(1+4l^{2}+l^{4})  -l^{2}m^{4} $ 

k8c21: $ (-3l^{2}-3l^{4}-l^{6})  +m^{2}(2l^{2}+3l^{4}+l^{6})  -l^{4}m^{4} $ 

k9c1: $ (5l^{8}+4l^{10})  +m^{2}(-20l^{8}-10l^{10})  +m^{4}(21l^{8}+6l^{10})  +m^{6}(-8l^{8}-l^{10})  +l^{8}m^{8} $ 

k9c2: $ (-l^{2}+l^{8}+l^{10})  +m^{2}(l^{2}-l^{4}+l^{6}-l^{8}) $ 

k9c3: $ (3l^{-10}+3l^{-8}-l^{-6})  +m^{2}(-4l^{-10}-7l^{-8}+6l^{-6})  +m^{4}(l^{-10}+5l^{-8}-5l^{-6})  +m^{6}(-l^{-8}+l^{-6}) $ 

k9c4: $ (l^{4}+2l^{8}+2l^{10})  +m^{2}(-3l^{4}+2l^{6}-3l^{8}-l^{10})  +m^{4}(l^{4}-l^{6}+l^{8}) $ 

k9c5: $ (l^{-10}-l^{-6}+l^{-4})  +m^{2}(-l^{-8}+2l^{-6}-2l^{-4}+l^{-2}) $ 

k9c6: $ (-3l^{6}-l^{8}+l^{10})  +m^{2}(7l^{6}-3l^{8}-3l^{10})  +m^{4}(-5l^{6}+4l^{8}+l^{10})  +m^{6}(l^{6}-l^{8}) $ 

k9c7: $ (2l^{4}+l^{6}+l^{8}+l^{10})  +m^{2}(-3l^{4}+l^{6}-2l^{8}-l^{10})  +m^{4}(l^{4}-l^{6}+l^{8}) $ 

k9c8: $ (-l^{-2}-1+2l^{4}+l^{6})  +m^{2}(l^{-2}+2-l^{2}-2l^{4})  +m^{4}(-1+l^{2}) $ 

k9c9: $ (-2l^{6}+l^{8}+2l^{10})  +m^{2}(7l^{6}-4l^{8}-3l^{10})  +m^{4}(-5l^{6}+4l^{8}+l^{10})  +m^{6}(l^{6}-l^{8}) $ 

k9c10: $ (2l^{-10}+l^{-8}-2l^{-6})  +m^{2}(-l^{-10}-2l^{-8}+5l^{-6}-2l^{-4})  +m^{4}(l^{-8}-2l^{-6}+l^{-4}) $ 

k9c11: $ (-2l^{-8}-3l^{-6}-l^{-4}-l^{-2})  +m^{2}(l^{-8}+6l^{-6}+4l^{-4}+3l^{-2})  +m^{4}(-2l^{-6}-4l^{-4}-l^{-2})  +l^{-4}m^{6} $ 

k9c12: $ (1-l^{4}-2l^{6}-l^{8})  +m^{2}(-1-l^{2}+l^{4}+2l^{6})  +m^{4}(l^{2}-l^{4}) $ 

k9c13: $ (l^{-10}-l^{-8}-3l^{-6})  +m^{2}(-l^{-10}-l^{-8}+5l^{-6}-2l^{-4})  +m^{4}(l^{-8}-2l^{-6}+l^{-4}) $ 

k9c14: $ (-l^{-6}-2l^{-4}-l^{-2}+1)  +m^{2}(2l^{-4}+l^{-2}-1-l^{2})  +m^{4}(-l^{-2}+1) $ 

k9c15: $ (-l^{-8}-l^{-6}+l^{-4}+l^{-2}+1)  +m^{2}(2l^{-6}-l^{-2}-1)  +m^{4}(-l^{-4}+l^{-2}) $ 

k9c16: $ (-3l^{-8}-4l^{-6})  +m^{2}(-2l^{-10}+8l^{-6})  +m^{4}(l^{-10}+3l^{-8}-5l^{-6})  +m^{6}(-l^{-8}+l^{-6}) $ 

k9c17: $ (-2l^{-2}-3-2l^{2})  +m^{2}(l^{-2}+6+5l^{2}+2l^{4})  +m^{4}(-2-4l^{2}-l^{4})  +l^{2}m^{6} $ 

k9c18: $ (l^{4}-l^{6}+l^{10})  +m^{2}(-2l^{4}+4l^{6}-l^{8}-l^{10})  +m^{4}(l^{4}-2l^{6}+l^{8}) $ 

k9c19: $ (l^{-4}+l^{-2}-l^{2})  +m^{2}(-2l^{-2}+l^{2}+l^{4})  +m^{4}(1-l^{2}) $ 

k9c20: $ (-2l^{2}-2l^{4}-2l^{6}-l^{8})  +m^{2}(3l^{2}+5l^{4}+5l^{6}+l^{8})  +m^{4}(-l^{2}-4l^{4}-2l^{6})  +l^{4}m^{6} $ 

k9c21: $ (-l^{-8}-l^{-6}-l^{-2})  +m^{2}(2l^{-6}-1)  +m^{4}(-l^{-4}+l^{-2}) $ 

k9c22: $ (-l^{-4}-4l^{-2}-4-2l^{2})  +m^{2}(2l^{-4}+6l^{-2}+6+l^{2})  +m^{4}(-l^{-4}-4l^{-2}-2)  +l^{-2}m^{6} $ 

k9c23: $ (l^{4}-2l^{6}-2l^{8})  +m^{2}(-2l^{4}+4l^{6}-l^{10})  +m^{4}(l^{4}-2l^{6}+l^{8}) $ 

k9c24: $ (-l^{-2}-3-5l^{2}-2l^{4})  +m^{2}(2l^{-2}+6+6l^{2}+l^{4})  +m^{4}(-l^{-2}-4-2l^{2})  +m^{6} $ 

k9c25: $ (1-l^{2}-3l^{4}-3l^{6}-l^{8})  +m^{2}(-1+4l^{4}+3l^{6})  +m^{4}(l^{2}-2l^{4}) $ 

k9c26: $ (-l^{-6}-3l^{-4}-3l^{-2})  +m^{2}(l^{-6}+5l^{-4}+6l^{-2}+2)  +m^{4}(-2l^{-4}-4l^{-2}-1)  +l^{-2}m^{6} $ 

k9c27: $ (-l^{-2}-2-3l^{2}-l^{4})  +m^{2}(2l^{-2}+6+5l^{2}+l^{4})  +m^{4}(-l^{-2}-4-2l^{2})  +m^{6} $ 

k9c28: $ (-1-5l^{2}-4l^{4}-l^{6})  +m^{2}(2+7l^{2}+5l^{4}+l^{6})  +m^{4}(-1-4l^{2}-2l^{4})  +l^{2}m^{6} $ 

k9c29: $ (-l^{-2}-3-5l^{2}-2l^{4})  +m^{2}(l^{-2}+5+7l^{2}+2l^{4})  +m^{4}(-2-4l^{2}-l^{4})  +l^{2}m^{6} $ 

k9c30: $ (-2l^{-2}-4-4l^{2}-l^{4})  +m^{2}(2l^{-2}+7+5l^{2}+l^{4})  +m^{4}(-l^{-2}-4-2l^{2})  +m^{6} $ 

k9c31: $ (-1-4l^{2}-2l^{4})  +m^{2}(2+7l^{2}+4l^{4}+l^{6})  +m^{4}(-1-4l^{2}-2l^{4})  +l^{2}m^{6} $ 

k9c32: $ (-l^{-6}-2l^{-4}-l^{-2}+1)  +m^{2}(l^{-6}+4l^{-4}+3l^{-2}+1)  +m^{4}(-2l^{-4}-3l^{-2}-1)  +l^{-2}m^{6} $ 

k9c33: $ (-2l^{2}-l^{4})  +m^{2}(l^{-2}+3+4l^{2}+l^{4})  +m^{4}(-l^{-2}-3-2l^{2})  +m^{6} $ 

k9c34: $ (-l^{-2}-1-l^{2})  +m^{2}(l^{-2}+4+3l^{2}+l^{4})  +m^{4}(-l^{-2}-3-2l^{2})  +m^{6} $ 

k9c35: $ (-3l^{6}-l^{8}+l^{10})  +m^{2}(l^{2}-2l^{4}+3l^{6}-l^{8}) $ 

k9c36: $ (-2l^{-8}-4l^{-6}-3l^{-4}-2l^{-2})  +m^{2}(l^{-8}+6l^{-6}+5l^{-4}+3l^{-2})  +m^{4}(-2l^{-6}-4l^{-4}-l^{-2})  +l^{-4}m^{6} $ 

k9c37: $ (l^{-4}-2-2l^{2})  +m^{2}(-2l^{-2}+1+l^{2}+l^{4})  +m^{4}(1-l^{2}) $ 

k9c38: $ (-4l^{6}-3l^{8})  +m^{2}(-l^{4}+7l^{6}+l^{8}-l^{10})  +m^{4}(l^{4}-3l^{6}+l^{8}) $ 

k9c39: $ (-l^{-8}-2l^{-6}-2l^{-4}-2l^{-2})  +m^{2}(3l^{-6}+3l^{-4}+l^{-2}-1)  +m^{4}(-2l^{-4}+l^{-2}) $ 

k9c40: $ (2+2l^{2}+l^{4})  +m^{2}(2l^{4}+l^{6})  +m^{4}(-1-2l^{2}-2l^{4})  +l^{2}m^{6} $ 

k9c41: $ (-3l^{2}-3l^{4}-l^{6})  +m^{2}(-l^{-2}+4l^{2}+3l^{4})  +m^{4}(1-2l^{2}) $ 

k9c42: $ (-2l^{-2}-3-2l^{2})  +m^{2}(l^{-2}+4+l^{2})  -m^{4} $ 

k9c43: $ (-l^{-8}-3l^{-6}-4l^{-4}-3l^{-2})  +m^{2}(4l^{-6}+7l^{-4}+4l^{-2})  +m^{4}(-l^{-6}-5l^{-4}-l^{-2})  +l^{-4}m^{6} $ 

k9c44: $ (-l^{-2}-2-3l^{2}-l^{4})  +m^{2}(2+3l^{2}+l^{4})  -l^{2}m^{4} $ 

k9c45: $ (-2l^{2}-2l^{4}-2l^{6}-l^{8})  +m^{2}(2l^{2}+2l^{4}+2l^{6})  -l^{4}m^{4} $ 

k9c46: $ (2+l^{2}-l^{4}-l^{6})  +m^{2}(-l^{2}+l^{4}) $ 

k9c47: $ (-l^{-6}-2l^{-4}-l^{-2}+1)  +m^{2}(3l^{-4}+4l^{-2}+2)  +m^{4}(-l^{-4}-4l^{-2}-1)  +l^{-2}m^{6} $ 

k9c48: $ (2l^{-6}+3l^{-4})  +m^{2}(-3l^{-4}-l^{-2}-1)  +l^{-2}m^{4} $ 

k9c49: $ (-3l^{-8}-4l^{-6})  +m^{2}(2l^{-8}+6l^{-6}-2l^{-4})  +m^{4}(-2l^{-6}+l^{-4}) $ 

k10c1: $ (-l^{-2}+l^{6}+l^{8})  +m^{2}(1-l^{2}+l^{4}-l^{6}) $ 

k10c2: $ (4l^{4}+4l^{6}+l^{8})  +m^{2}(-10l^{4}-14l^{6}-6l^{8})  +m^{4}(6l^{4}+16l^{6}+5l^{8})  +m^{6}(-l^{4}-7l^{6}-l^{8})  +l^{6}m^{8} $ 

k10c3: $ (l^{-4}+l^{2}-l^{6})  +m^{2}(-l^{-2}+2-2l^{2}+l^{4}) $ 

k10c4: $ (2l^{-4}+2l^{-2}+l^{4})  +m^{2}(-l^{-4}-3l^{-2}+2-2l^{2}-l^{4})  +m^{4}(l^{-2}-1+l^{2}) $ 

k10c5: $ (3l^{-6}+5l^{-4}+l^{-2})  +m^{2}(-7l^{-6}-17l^{-4}-6l^{-2})  +m^{4}(5l^{-6}+17l^{-4}+5l^{-2})  +m^{6}(-l^{-6}-7l^{-4}-l^{-2})  +l^{-4}m^{8} $ 

k10c6: $ (-3l^{2}-2l^{4}+l^{6}+l^{8})  +m^{2}(4l^{2}+4l^{4}-4l^{6}-3l^{8})  +m^{4}(-l^{2}-4l^{4}+4l^{6}+l^{8})  +m^{6}(l^{4}-l^{6}) $ 

k10c7: $ (1+l^{4}+2l^{6}+l^{8})  +m^{2}(-1-l^{2}-2l^{6}-l^{8})  +m^{4}(l^{2}-l^{4}+l^{6}) $ 

k10c8: $ (3+3l^{2}-l^{6})  +m^{2}(-4-7l^{2}+3l^{4}+3l^{6})  +m^{4}(1+5l^{2}-4l^{4}-l^{6})  +m^{6}(-l^{2}+l^{4}) $ 

k10c9: $ (2l^{-4}+4l^{-2}+3)  +m^{2}(-7l^{-4}-16l^{-2}-7)  +m^{4}(5l^{-4}+17l^{-2}+5)  +m^{6}(-l^{-4}-7l^{-2}-1)  +l^{-2}m^{8} $ 

k10c10: $ (l^{-6}+2l^{-4}+l^{-2}+1)  +m^{2}(-l^{-6}-2l^{-4}-1-l^{2})  +m^{4}(l^{-4}-l^{-2}+1) $ 

k10c11: $ (-2l^{-2}-1+l^{2}-l^{6})  +m^{2}(l^{-2}+2-4l^{2}+l^{4}+l^{6})  +m^{4}(-1+2l^{2}-l^{4}) $ 

k10c12: $ (2l^{-6}+2l^{-4}-2l^{-2}-1)  +m^{2}(-3l^{-6}-5l^{-4}+5l^{-2}+3)  +m^{4}(l^{-6}+4l^{-4}-4l^{-2}-1)  +m^{6}(-l^{-4}+l^{-2}) $ 

k10c13: $ (l^{-4}-1-l^{2}-l^{4}-l^{6})  +m^{2}(-2l^{-2}+1+2l^{4})  +m^{4}(1-l^{2}) $ 

k10c14: $ (-l^{2}+l^{4}+l^{6})  +m^{2}(3l^{2}+l^{4}-2l^{6}-2l^{8})  +m^{4}(-l^{2}-3l^{4}+3l^{6}+l^{8})  +m^{6}(l^{4}-l^{6}) $ 

k10c15: $ (-2l^{-4}-3l^{-2}+1+l^{2})  +m^{2}(3l^{-4}+5l^{-2}-4-3l^{2})  +m^{4}(-l^{-4}-4l^{-2}+4+l^{2})  +m^{6}(l^{-2}-1) $ 

k10c16: $ (-l^{-6}+2l^{-2}+1-l^{2})  +m^{2}(l^{-6}+l^{-4}-4l^{-2}+1+l^{2})  +m^{4}(-l^{-4}+2l^{-2}-1) $ 

k10c17: $ (2l^{-2}+5+2l^{2})  +m^{2}(-7l^{-2}-16-7l^{2})  +m^{4}(5l^{-2}+17+5l^{2})  +m^{6}(-l^{-2}-7-l^{2})  +m^{8} $ 

k10c18: $ (-l^{-2}+l^{2}+l^{4})  +m^{2}(l^{-2}+1-3l^{2}+l^{6})  +m^{4}(-1+2l^{2}-l^{4}) $ 

k10c19: $ (l^{-2}+3+l^{2})  +m^{2}(-3l^{-2}-5+l^{2}+2l^{4})  +m^{4}(l^{-2}+4-3l^{2}-l^{4})  +m^{6}(-1+l^{2}) $ 

k10c20: $ (2+l^{2}+l^{6}+l^{8})  +m^{2}(-1-2l^{2}+l^{4}-2l^{6}-l^{8})  +m^{4}(l^{2}-l^{4}+l^{6}) $ 

k10c21: $ (-l^{2}+2l^{4}+3l^{6}+l^{8})  +m^{2}(3l^{2}-5l^{6}-3l^{8})  +m^{4}(-l^{2}-3l^{4}+4l^{6}+l^{8})  +m^{6}(l^{4}-l^{6}) $ 

k10c22: $ (2l^{-4}+2l^{-2}-1-2l^{2})  +m^{2}(-3l^{-4}-5l^{-2}+5+3l^{2})  +m^{4}(l^{-4}+4l^{-2}-4-l^{2})  +m^{6}(-l^{-2}+1) $ 

k10c23: $ (2l^{-6}+3l^{-4})  +m^{2}(-3l^{-6}-6l^{-4}+2l^{-2}+2)  +m^{4}(l^{-6}+4l^{-4}-3l^{-2}-1)  +m^{6}(-l^{-4}+l^{-2}) $ 

k10c24: $ (1-l^{2}-l^{4}+l^{6}+l^{8})  +m^{2}(-1+3l^{4}-l^{6}-l^{8})  +m^{4}(l^{2}-2l^{4}+l^{6}) $ 

k10c25: $ (-2l^{2}+2l^{6}+l^{8})  +m^{2}(3l^{2}+2l^{4}-3l^{6}-2l^{8})  +m^{4}(-l^{2}-3l^{4}+3l^{6}+l^{8})  +m^{6}(l^{4}-l^{6}) $ 

k10c26: $ (2l^{-4}+3l^{-2}+1-l^{2})  +m^{2}(-3l^{-4}-6l^{-2}+2+2l^{2})  +m^{4}(l^{-4}+4l^{-2}-3-l^{2})  +m^{6}(-l^{-2}+1) $ 

k10c27: $ (-l^{2}+l^{4}+l^{6})  +m^{2}(2+3l^{2}-3l^{4}-2l^{6})  +m^{4}(-1-3l^{2}+3l^{4}+l^{6})  +m^{6}(l^{2}-l^{4}) $ 

k10c28: $ (l^{-6}-3l^{-2}-1)  +m^{2}(-l^{-6}-l^{-4}+4l^{-2}-l^{2})  +m^{4}(l^{-4}-2l^{-2}+1) $ 

k10c29: $ (-2l^{-2}-2-l^{2}-l^{4}-l^{6})  +m^{2}(l^{-2}+5+3l^{2}+4l^{4}+l^{6})  +m^{4}(-2-3l^{2}-2l^{4})  +l^{2}m^{6} $ 

k10c30: $ (-2l^{2}-l^{4})  +m^{2}(-1+l^{2}+2l^{4}-l^{8})  +m^{4}(l^{2}-2l^{4}+l^{6}) $ 

k10c31: $ (-l^{-4}-l^{-2}+2+l^{2})  +m^{2}(l^{-4}+l^{-2}-3+l^{4})  +m^{4}(-l^{-2}+2-l^{2}) $ 

k10c32: $ (-l^{-2}-1-l^{2})  +m^{2}(2l^{-2}+3-2l^{2}-2l^{4})  +m^{4}(-l^{-2}-3+3l^{2}+l^{4})  +m^{6}(1-l^{2}) $ 

k10c33: $ 1  +m^{2}(l^{-4}-2+l^{4})  +m^{4}(-l^{-2}+2-l^{2}) $ 

k10c34: $ (l^{-6}+l^{-4}+2+l^{2})  +m^{2}(-l^{-6}-2l^{-4}+l^{-2}-2-l^{2})  +m^{4}(l^{-4}-l^{-2}+1) $ 

k10c35: $ (-l^{-6}-l^{-4}+1+l^{2}+l^{4})  +m^{2}(2l^{-4}-2l^{2})  +m^{4}(-l^{-2}+1) $ 

k10c36: $ (1+l^{2}+2l^{4}+l^{6})  +m^{2}(-1-l^{2}-l^{4}-l^{6}-l^{8})  +m^{4}(l^{2}-l^{4}+l^{6}) $ 

k10c37: $ (-l^{-4}-l^{-2}+1-l^{2}-l^{4})  +m^{2}(l^{-4}+l^{-2}-3+l^{2}+l^{4})  +m^{4}(-l^{-2}+2-l^{2}) $ 

k10c38: $ (1-l^{2}-2l^{4}-l^{6})  +m^{2}(-1+3l^{4}-l^{8})  +m^{4}(l^{2}-2l^{4}+l^{6}) $ 

k10c39: $ (-2l^{2}-l^{4})  +m^{2}(3l^{2}+2l^{4}-2l^{6}-2l^{8})  +m^{4}(-l^{2}-3l^{4}+3l^{6}+l^{8})  +m^{6}(l^{4}-l^{6}) $ 

k10c40: $ (l^{-6}-3l^{-2}-1)  +m^{2}(-2l^{-6}-3l^{-4}+4l^{-2}+2)  +m^{4}(l^{-6}+3l^{-4}-3l^{-2}-1)  +m^{6}(-l^{-4}+l^{-2}) $ 

k10c41: $ (-l^{-2}-1-2l^{2}-2l^{4}-l^{6})  +m^{2}(l^{-2}+4+4l^{2}+4l^{4}+l^{6})  +m^{4}(-2-3l^{2}-2l^{4})  +l^{2}m^{6} $ 

k10c42: $ (-l^{-4}-3l^{-2}-2-l^{2})  +m^{2}(l^{-4}+4l^{-2}+5+3l^{2}+l^{4})  +m^{4}(-2l^{-2}-3-2l^{2})  +m^{6} $ 

k10c43: $ (-l^{-4}-2l^{-2}-1-2l^{2}-l^{4})  +m^{2}(l^{-4}+4l^{-2}+4+4l^{2}+l^{4})  +m^{4}(-2l^{-2}-3-2l^{2})  +m^{6} $ 

k10c44: $ (-l^{-2}-2-3l^{2}-l^{4})  +m^{2}(l^{-2}+4+5l^{2}+3l^{4}+l^{6})  +m^{4}(-2-3l^{2}-2l^{4})  +l^{2}m^{6} $ 

k10c45: $ (-2l^{-2}-3-2l^{2})  +m^{2}(l^{-4}+3l^{-2}+6+3l^{2}+l^{4})  +m^{4}(-2l^{-2}-3-2l^{2})  +m^{6} $ 

k10c46: $ (3l^{-8}+8l^{-6}+6l^{-4})  +m^{2}(-7l^{-8}-18l^{-6}-11l^{-4})  +m^{4}(5l^{-8}+17l^{-6}+6l^{-4})  +m^{6}(-l^{-8}-7l^{-6}-l^{-4})  +l^{-6}m^{8} $ 

k10c47: $ (5l^{-6}+9l^{-4}+3l^{-2})  +m^{2}(-8l^{-6}-21l^{-4}-7l^{-2})  +m^{4}(5l^{-6}+18l^{-4}+5l^{-2})  +m^{6}(-l^{-6}-7l^{-4}-l^{-2})  +l^{-4}m^{8} $ 

k10c48: $ (4l^{-2}+9+4l^{2})  +m^{2}(-8l^{-2}-20-8l^{2})  +m^{4}(5l^{-2}+18+5l^{2})  +m^{6}(-l^{-2}-7-l^{2})  +m^{8} $ 

k10c49: $ (-l^{6}+5l^{8}+7l^{10}+2l^{12})  +m^{2}(4l^{6}-12l^{8}-10l^{10}-l^{12})  +m^{4}(-4l^{6}+9l^{8}+3l^{10})  +m^{6}(l^{6}-2l^{8}) $ 

k10c50: $ (2l^{-8}+4l^{-6}+l^{-4}-2l^{-2})  +m^{2}(-3l^{-8}-6l^{-6}+l^{-4}+3l^{-2})  +m^{4}(l^{-8}+4l^{-6}-3l^{-4}-l^{-2})  +m^{6}(-l^{-6}+l^{-4}) $ 

k10c51: $ (3l^{-6}+4l^{-4}-l^{-2}-1)  +m^{2}(-3l^{-6}-7l^{-4}+3l^{-2}+2)  +m^{4}(l^{-6}+4l^{-4}-3l^{-2}-1)  +m^{6}(-l^{-4}+l^{-2}) $ 

k10c52: $ (-l^{-4}+4+2l^{2})  +m^{2}(2l^{-4}+2l^{-2}-6-3l^{2})  +m^{4}(-l^{-4}-3l^{-2}+4+l^{2})  +m^{6}(l^{-2}-1) $ 

k10c53: $ (-3l^{6}+3l^{10}+l^{12})  +m^{2}(-l^{4}+6l^{6}-2l^{8}-3l^{10})  +m^{4}(l^{4}-3l^{6}+2l^{8}) $ 

k10c54: $ (-2l^{-4}-2l^{-2}+3+2l^{2})  +m^{2}(3l^{-4}+5l^{-2}-5-3l^{2})  +m^{4}(-l^{-4}-4l^{-2}+4+l^{2})  +m^{6}(l^{-2}-1) $ 

k10c55: $ (l^{4}-l^{6}+l^{8}+3l^{10}+l^{12})  +m^{2}(-2l^{4}+3l^{6}-3l^{8}-3l^{10})  +m^{4}(l^{4}-2l^{6}+2l^{8}) $ 

k10c56: $ (l^{-8}+2l^{-6}-2l^{-2})  +m^{2}(-2l^{-8}-3l^{-6}+2l^{-4}+3l^{-2})  +m^{4}(l^{-8}+3l^{-6}-3l^{-4}-l^{-2})  +m^{6}(-l^{-6}+l^{-4}) $ 

k10c57: $ (2l^{-6}+2l^{-4}-2l^{-2}-1)  +m^{2}(-2l^{-6}-4l^{-4}+4l^{-2}+2)  +m^{4}(l^{-6}+3l^{-4}-3l^{-2}-1)  +m^{6}(-l^{-4}+l^{-2}) $ 

k10c58: $ (l^{-4}-2-3l^{2}-2l^{4}-l^{6})  +m^{2}(-2l^{-2}+2+3l^{2}+3l^{4})  +m^{4}(1-2l^{2}) $ 

k10c59: $ (-l^{-6}-3l^{-4}-4l^{-2}-2-l^{2})  +m^{2}(l^{-6}+4l^{-4}+5l^{-2}+4+l^{2})  +m^{4}(-2l^{-4}-3l^{-2}-2)  +l^{-2}m^{6} $ 

k10c60: $ (-l^{-2}-2-4l^{2}-3l^{4}-l^{6})  +m^{2}(l^{-2}+5+6l^{2}+3l^{4})  +m^{4}(-l^{-2}-3-3l^{2})  +m^{6} $ 

k10c61: $ (-l^{-6}+l^{-4}+5l^{-2}+4)  +m^{2}(3l^{-6}+3l^{-4}-8l^{-2}-4)  +m^{4}(-l^{-6}-4l^{-4}+5l^{-2}+1)  +m^{6}(l^{-4}-l^{-2}) $ 

k10c62: $ (4l^{-6}+7l^{-4}+2l^{-2})  +m^{2}(-8l^{-6}-20l^{-4}-7l^{-2})  +m^{4}(5l^{-6}+18l^{-4}+5l^{-2})  +m^{6}(-l^{-6}-7l^{-4}-l^{-2})  +l^{-4}m^{8} $ 

k10c63: $ (l^{4}+3l^{8}+4l^{10}+l^{12})  +m^{2}(-2l^{4}+3l^{6}-4l^{8}-3l^{10})  +m^{4}(l^{4}-2l^{6}+2l^{8}) $ 

k10c64: $ (3l^{-4}+6l^{-2}+4)  +m^{2}(-8l^{-4}-19l^{-2}-8)  +m^{4}(5l^{-4}+18l^{-2}+5)  +m^{6}(-l^{-4}-7l^{-2}-1)  +l^{-2}m^{8} $ 

k10c65: $ (3l^{-6}+5l^{-4}+l^{-2})  +m^{2}(-3l^{-6}-7l^{-4}+2l^{-2}+2)  +m^{4}(l^{-6}+4l^{-4}-3l^{-2}-1)  +m^{6}(-l^{-4}+l^{-2}) $ 

k10c66: $ (-2l^{6}+2l^{8}+4l^{10}+l^{12})  +m^{2}(5l^{6}-9l^{8}-8l^{10}-l^{12})  +m^{4}(-4l^{6}+8l^{8}+3l^{10})  +m^{6}(l^{6}-2l^{8}) $ 

k10c67: $ 1  +m^{2}(-1+2l^{4}-l^{8})  +m^{4}(l^{2}-2l^{4}+l^{6}) $ 

k10c68: $ (-l^{2}+l^{4}+l^{6})  +m^{2}(-l^{-2}+3l^{2}-l^{4}-l^{6})  +m^{4}(1-2l^{2}+l^{4}) $ 

k10c69: $ (-l^{-8}-2l^{-6}-2l^{-4}-2l^{-2})  +m^{2}(3l^{-6}+5l^{-4}+5l^{-2}+1)  +m^{4}(-3l^{-4}-3l^{-2}-1)  +l^{-2}m^{6} $ 

k10c70: $ (-l^{-6}-2l^{-4}-3l^{-2}-3-2l^{2})  +m^{2}(l^{-6}+4l^{-4}+4l^{-2}+5+l^{2})  +m^{4}(-2l^{-4}-3l^{-2}-2)  +l^{-2}m^{6} $ 

k10c71: $ (-l^{-4}-3l^{-2}-3-3l^{2}-l^{4})  +m^{2}(l^{-4}+4l^{-2}+5+4l^{2}+l^{4})  +m^{4}(-2l^{-2}-3-2l^{2})  +m^{6} $ 

k10c72: $ (-l^{-8}-2l^{-6}-2l^{-4}-2l^{-2})  +m^{2}(-l^{-8}+l^{-6}+3l^{-4}+3l^{-2})  +m^{4}(l^{-8}+2l^{-6}-3l^{-4}-l^{-2})  +m^{6}(-l^{-6}+l^{-4}) $ 

k10c73: $ (-3l^{2}-4l^{4}-3l^{6}-l^{8})  +m^{2}(1+5l^{2}+6l^{4}+3l^{6})  +m^{4}(-1-3l^{2}-3l^{4})  +l^{2}m^{6} $ 

k10c74: $ (-2l^{2}+2l^{6}+l^{8})  +m^{2}(-1+l^{2}+2l^{4}-l^{6}-l^{8})  +m^{4}(l^{2}-2l^{4}+l^{6}) $ 

k10c75: $ (-l^{-6}-3l^{-4}-3l^{-2})  +m^{2}(3l^{-4}+6l^{-2}+4+l^{2})  +m^{4}(-3l^{-2}-3-l^{2})  +m^{6} $ 

k10c76: $ (l^{-8}-4l^{-4}-4l^{-2})  +m^{2}(-2l^{-8}-2l^{-6}+6l^{-4}+4l^{-2})  +m^{4}(l^{-8}+3l^{-6}-4l^{-4}-l^{-2})  +m^{6}(-l^{-6}+l^{-4}) $ 

k10c77: $ (l^{-6}-l^{-4}-5l^{-2}-2)  +m^{2}(-2l^{-6}-2l^{-4}+7l^{-2}+3)  +m^{4}(l^{-6}+3l^{-4}-4l^{-2}-1)  +m^{6}(-l^{-4}+l^{-2}) $ 

k10c78: $ (-l^{2}-l^{4}-4l^{6}-4l^{8}-l^{10})  +m^{2}(2l^{2}+3l^{4}+7l^{6}+3l^{8})  +m^{4}(-l^{2}-3l^{4}-3l^{6})  +l^{4}m^{6} $ 

k10c79: $ (5l^{-2}+11+5l^{2})  +m^{2}(-9l^{-2}-23-9l^{2})  +m^{4}(5l^{-2}+19+5l^{2})  +m^{6}(-l^{-2}-7-l^{2})  +m^{8} $ 

k10c80: $ (-2l^{6}+3l^{8}+6l^{10}+2l^{12})  +m^{2}(5l^{6}-9l^{8}-9l^{10}-l^{12})  +m^{4}(-4l^{6}+8l^{8}+3l^{10})  +m^{6}(l^{6}-2l^{8}) $ 

k10c81: $ (-l^{-4}-l^{-2}+1-l^{2}-l^{4})  +m^{2}(l^{-4}+3l^{-2}+1+3l^{2}+l^{4})  +m^{4}(-2l^{-2}-2-2l^{2})  +m^{6} $ 

k10c82: $ 1  +m^{2}(-4-8l^{2}-4l^{4})  +m^{4}(4+12l^{2}+4l^{4})  +m^{6}(-1-6l^{2}-l^{4})  +l^{2}m^{8} $ 

k10c83: $ (l^{-4}+2l^{-2}+2)  +m^{2}(-2l^{-4}-4l^{-2}+l^{2})  +m^{4}(l^{-4}+3l^{-2}-2-l^{2})  +m^{6}(-l^{-2}+1) $ 

k10c84: $ (-2l^{-4}-4l^{-2}-1)  +m^{2}(-l^{-6}+5l^{-2}+2)  +m^{4}(l^{-6}+2l^{-4}-3l^{-2}-1)  +m^{6}(-l^{-4}+l^{-2}) $ 

k10c85: $ (-l^{2}+l^{4}+l^{6})  +m^{2}(-3l^{2}-9l^{4}-4l^{6})  +m^{4}(4l^{2}+12l^{4}+4l^{6})  +m^{6}(-l^{2}-6l^{4}-l^{6})  +l^{4}m^{8} $ 

k10c86: $ (l^{-6}+2l^{-4}+l^{-2}+1)  +m^{2}(-2l^{-6}-4l^{-4}+1)  +m^{4}(l^{-6}+3l^{-4}-2l^{-2}-1)  +m^{6}(-l^{-4}+l^{-2}) $ 

k10c87: $ (-l^{-4}-3l^{-2}-2-l^{2})  +m^{2}(-l^{-4}+l^{-2}+4+2l^{2})  +m^{4}(l^{-4}+2l^{-2}-3-l^{2})  +m^{6}(-l^{-2}+1) $ 

k10c88: $ (-l^{-2}-1-l^{2})  +m^{2}(l^{-4}+2l^{-2}+3+2l^{2}+l^{4})  +m^{4}(-2l^{-2}-2-2l^{2})  +m^{6} $ 

k10c89: $ (1-l^{4}-2l^{6}-l^{8})  +m^{2}(2l^{2}+4l^{4}+3l^{6})  +m^{4}(-1-2l^{2}-3l^{4})  +l^{2}m^{6} $ 

k10c90: $ (l^{-4}-2-2l^{2})  +m^{2}(-2l^{-4}-3l^{-2}+4+2l^{2})  +m^{4}(l^{-4}+3l^{-2}-3-l^{2})  +m^{6}(-l^{-2}+1) $ 

k10c91: $ (2l^{-2}+5+2l^{2})  +m^{2}(-5l^{-2}-12-5l^{2})  +m^{4}(4l^{-2}+13+4l^{2})  +m^{6}(-l^{-2}-6-l^{2})  +m^{8} $ 

k10c92: $ (l^{-6}+l^{-4}-l^{-2})  +m^{2}(-l^{-8}-l^{-6}+2l^{-2})  +m^{4}(l^{-8}+2l^{-6}-2l^{-4}-l^{-2})  +m^{6}(-l^{-6}+l^{-4}) $ 

k10c93: $ (-2l^{2}-l^{4})  +m^{2}(-2l^{-2}-2+3l^{2}+2l^{4})  +m^{4}(l^{-2}+3-3l^{2}-l^{4})  +m^{6}(-1+l^{2}) $ 

k10c94: $ (2l^{-4}+4l^{-2}+3)  +m^{2}(-5l^{-4}-12l^{-2}-5)  +m^{4}(4l^{-4}+13l^{-2}+4)  +m^{6}(-l^{-4}-6l^{-2}-1)  +l^{-2}m^{8} $ 

k10c95: $ (2l^{-6}+3l^{-4})  +m^{2}(-2l^{-6}-5l^{-4}+l^{-2}+1)  +m^{4}(l^{-6}+3l^{-4}-2l^{-2}-1)  +m^{6}(-l^{-4}+l^{-2}) $ 

k10c96: $ (-l^{-6}-2l^{-4}-3l^{-2}-3-2l^{2})  +m^{2}(3l^{-4}+5l^{-2}+6+l^{2})  +m^{4}(-3l^{-2}-3-l^{2})  +m^{6} $ 

k10c97: $ (-l^{-8}-2l^{-6}-2l^{-4}-2l^{-2})  +m^{2}(-l^{-8}+2l^{-6}+4l^{-4}+2l^{-2}-1)  +m^{4}(l^{-6}-3l^{-4}+l^{-2}) $ 

k10c98: $ (-l^{2}+3l^{4}+5l^{6}+2l^{8})  +m^{2}(2l^{2}-l^{4}-5l^{6}-2l^{8})  +m^{4}(-l^{2}-2l^{4}+3l^{6}+l^{8})  +m^{6}(l^{4}-l^{6}) $ 

k10c99: $ (4l^{-2}+9+4l^{2})  +m^{2}(-6l^{-2}-16-6l^{2})  +m^{4}(4l^{-2}+14+4l^{2})  +m^{6}(-l^{-2}-6-l^{2})  +m^{8} $ 

k10c100: $ (l^{2}+5l^{4}+3l^{6})  +m^{2}(-4l^{2}-13l^{4}-5l^{6})  +m^{4}(4l^{2}+13l^{4}+4l^{6})  +m^{6}(-l^{2}-6l^{4}-l^{6})  +l^{4}m^{8} $ 

k10c101: $ (l^{-12}+4l^{-10}+2l^{-8}-2l^{-6})  +m^{2}(-4l^{-10}-5l^{-8}+5l^{-6}-l^{-4})  +m^{4}(3l^{-8}-3l^{-6}+l^{-4}) $ 

k10c102: $ (l^{-4}+l^{-2}-l^{2})  +m^{2}(-2l^{-4}-3l^{-2}+3+2l^{2})  +m^{4}(l^{-4}+3l^{-2}-3-l^{2})  +m^{6}(-l^{-2}+1) $ 

k10c103: $ (-1-3l^{2}+l^{6})  +m^{2}(2+4l^{2}-3l^{4}-2l^{6})  +m^{4}(-1-3l^{2}+3l^{4}+l^{6})  +m^{6}(l^{2}-l^{4}) $ 

k10c104: $ (l^{-2}+3+l^{2})  +m^{2}(-5l^{-2}-11-5l^{2})  +m^{4}(4l^{-2}+13+4l^{2})  +m^{6}(-l^{-2}-6-l^{2})  +m^{8} $ 

k10c105: $ (-l^{-2}-1-l^{2})  +m^{2}(l^{-6}+2l^{-4}+2l^{-2}+3+l^{2})  +m^{4}(-2l^{-4}-2l^{-2}-2)  +l^{-2}m^{6} $ 

k10c106: $ (l^{-4}+2l^{-2}+2)  +m^{2}(-5l^{-4}-11l^{-2}-5)  +m^{4}(4l^{-4}+13l^{-2}+4)  +m^{6}(-l^{-4}-6l^{-2}-1)  +l^{-2}m^{8} $ 

k10c107: $ (-l^{-4}-2l^{-2})  +m^{2}(l^{-4}+3l^{-2}+2+2l^{2}+l^{4})  +m^{4}(-2l^{-2}-2-2l^{2})  +m^{6} $ 

k10c108: $ 1  +m^{2}(2l^{-4}+2l^{-2}-2-2l^{2})  +m^{4}(-l^{-4}-3l^{-2}+3+l^{2})  +m^{6}(l^{-2}-1) $ 

k10c109: $ (3l^{-2}+7+3l^{2})  +m^{2}(-6l^{-2}-15-6l^{2})  +m^{4}(4l^{-2}+14+4l^{2})  +m^{6}(-l^{-2}-6-l^{2})  +m^{8} $ 

k10c110: $ (-l^{-2}-l^{4}-l^{6})  +m^{2}(l^{-2}+3+l^{2}+3l^{4}+l^{6})  +m^{4}(-2-2l^{2}-2l^{4})  +l^{2}m^{6} $ 

k10c111: $ (l^{-8}+3l^{-6}+2l^{-4}-l^{-2})  +m^{2}(-2l^{-8}-4l^{-6}-l^{-4}+2l^{-2})  +m^{4}(l^{-8}+3l^{-6}-2l^{-4}-l^{-2})  +m^{6}(-l^{-6}+l^{-4}) $ 

k10c112: $ (-1-4l^{2}-2l^{4})  +m^{2}(-1-l^{4})  +m^{4}(3+7l^{2}+3l^{4})  +m^{6}(-1-5l^{2}-l^{4})  +l^{2}m^{8} $ 

k10c113: $ (-l^{-6}-3l^{-4}-3l^{-2})  +m^{2}(2l^{-4}+3l^{-2}+1)  +m^{4}(l^{-6}+l^{-4}-2l^{-2}-1)  +m^{6}(-l^{-4}+l^{-2}) $ 

k10c114: $ (-2l^{2}-l^{4})  +m^{2}(l^{-2}+1-l^{4})  +m^{4}(-l^{-2}-2+2l^{2}+l^{4})  +m^{6}(1-l^{2}) $ 

k10c115: $ (l^{-2}+3+l^{2})  +m^{2}(l^{-4}+l^{-2}-1+l^{2}+l^{4})  +m^{4}(-2l^{-2}-1-2l^{2})  +m^{6} $ 

k10c116: $ 1  +m^{2}(-2-4l^{2}-2l^{4})  +m^{4}(3+8l^{2}+3l^{4})  +m^{6}(-1-5l^{2}-l^{4})  +l^{2}m^{8} $ 

k10c117: $ (l^{-6}+l^{-4}-l^{-2})  +m^{2}(-l^{-6}-2l^{-4}+2l^{-2}+1)  +m^{4}(l^{-6}+2l^{-4}-2l^{-2}-1)  +m^{6}(-l^{-4}+l^{-2}) $ 

k10c118: $ 1  +m^{2}(-2l^{-2}-4-2l^{2})  +m^{4}(3l^{-2}+8+3l^{2})  +m^{6}(-l^{-2}-5-l^{2})  +m^{8} $ 

k10c119: $ (-l^{-2}-1-l^{2})  +m^{2}(-l^{-4}-l^{-2}+2+l^{2})  +m^{4}(l^{-4}+2l^{-2}-2-l^{2})  +m^{6}(-l^{-2}+1) $ 

k10c120: $ (-3l^{6}+3l^{10}+l^{12})  +m^{2}(7l^{6}-3l^{8}-4l^{10})  +m^{4}(l^{4}-4l^{6}+3l^{8}) $ 

k10c121: $ (1+l^{2}+2l^{4}+l^{6})  +m^{2}(-l^{2}-3l^{4}-l^{6})  +m^{4}(-1-l^{2}+2l^{4}+l^{6})  +m^{6}(l^{2}-l^{4}) $ 

k10c122: $ (-2l^{-4}-4l^{-2}-1)  +m^{2}(3l^{-2}+2+l^{2})  +m^{4}(l^{-4}+l^{-2}-2-l^{2})  +m^{6}(-l^{-2}+1) $ 

k10c123: $ (-2l^{-2}-3-2l^{2})  +m^{2}(l^{-2}+4+l^{2})  +m^{4}(2l^{-2}+3+2l^{2})  +m^{6}(-l^{-2}-4-l^{2})  +m^{8} $ 

k10c124: $ (2l^{-12}+8l^{-10}+7l^{-8})  +m^{2}(-l^{-12}-14l^{-10}-21l^{-8})  +m^{4}(7l^{-10}+21l^{-8})  +m^{6}(-l^{-10}-8l^{-8})  +l^{-8}m^{8} $ 

k10c125: $ (3l^{-2}+7+3l^{2})  +m^{2}(-4l^{-2}-11-4l^{2})  +m^{4}(l^{-2}+6+l^{2})  -m^{6} $ 

k10c126: $ (2l^{2}+7l^{4}+4l^{6})  +m^{2}(-3l^{2}-12l^{4}-4l^{6})  +m^{4}(l^{2}+6l^{4}+l^{6})  -l^{4}m^{6} $ 

k10c127: $ (5l^{4}+6l^{6}+2l^{8})  +m^{2}(-7l^{4}-9l^{6}-3l^{8})  +m^{4}(2l^{4}+5l^{6}+l^{8})  -l^{6}m^{6} $ 

k10c128: $ (l^{-12}+4l^{-10}+2l^{-8}-2l^{-6})  +m^{2}(-5l^{-10}-6l^{-8}+6l^{-6})  +m^{4}(l^{-10}+5l^{-8}-5l^{-6})  +m^{6}(-l^{-8}+l^{-6}) $ 

k10c129: $ (l^{-2}+2-l^{2}-l^{4})  +m^{2}(-l^{-2}-2+2l^{2}+l^{4})  +m^{4}(1-l^{2}) $ 

k10c130: $ (-1-2l^{2}+2l^{4}+2l^{6})  +m^{2}(1+3l^{2}-3l^{4}-l^{6})  +m^{4}(-l^{2}+l^{4}) $ 

k10c131: $ (-2l^{2}+2l^{6}+l^{8})  +m^{2}(2l^{2}+l^{4}-2l^{6}-l^{8})  +m^{4}(-l^{4}+l^{6}) $ 

k10c132: $ (3l^{4}+2l^{6})  +m^{2}(-4l^{4}-l^{6})  +l^{4}m^{4} $ 

k10c133: $ (-l^{2}+2l^{4}+3l^{6}+l^{8})  +m^{2}(l^{2}-2l^{4}-3l^{6}-l^{8})  +l^{6}m^{4} $ 

k10c134: $ (l^{-12}+3l^{-10}-3l^{-6})  +m^{2}(-4l^{-10}-3l^{-8}+7l^{-6})  +m^{4}(l^{-10}+4l^{-8}-5l^{-6})  +m^{6}(-l^{-8}+l^{-6}) $ 

k10c135: $ (2l^{-2}+4-l^{4})  +m^{2}(-2l^{-2}-5+l^{2}+l^{4})  +m^{4}(2-l^{2}) $ 

k10c136: $ (-l^{-4}-3l^{-2}-2-l^{2})  +m^{2}(2l^{-2}+3+l^{2})  -m^{4} $ 

k10c137: $ (-l^{-2}-1-2l^{2}-2l^{4}-l^{6})  +m^{2}(2+2l^{2}+2l^{4})  -l^{2}m^{4} $ 

k10c138: $ (-l^{-6}-2l^{-4}-3l^{-2}-3-2l^{2})  +m^{2}(3l^{-4}+5l^{-2}+6+l^{2})  +m^{4}(-l^{-4}-4l^{-2}-2)  +l^{-2}m^{6} $ 

k10c139: $ (l^{-12}+6l^{-10}+6l^{-8})  +m^{2}(-l^{-12}-13l^{-10}-21l^{-8})  +m^{4}(7l^{-10}+21l^{-8})  +m^{6}(-l^{-10}-8l^{-8})  +l^{-8}m^{8} $ 

k10c140: $ (1+2l^{2}+4l^{4}+2l^{6})  +m^{2}(-l^{2}-4l^{4}-l^{6})  +l^{4}m^{4} $ 

k10c141: $ (2+2l^{2}+l^{4})  +m^{2}(-3-7l^{2}-3l^{4})  +m^{4}(1+5l^{2}+l^{4})  -l^{2}m^{6} $ 

k10c142: $ (l^{-12}+5l^{-10}+4l^{-8}-l^{-6})  +m^{2}(-5l^{-10}-7l^{-8}+6l^{-6})  +m^{4}(l^{-10}+5l^{-8}-5l^{-6})  +m^{6}(-l^{-8}+l^{-6}) $ 

k10c143: $ (3l^{4}+2l^{6})  +m^{2}(-2l^{2}-8l^{4}-3l^{6})  +m^{4}(l^{2}+5l^{4}+l^{6})  -l^{4}m^{6} $ 

k10c144: $ (3+4l^{2}+2l^{4})  +m^{2}(-2-5l^{2}+l^{6})  +m^{4}(2l^{2}-l^{4}) $ 

k10c145: $ (2l^{4}+l^{6}+l^{8}+l^{10})  +m^{2}(-4l^{4}-l^{8})  +l^{4}m^{4} $ 

k10c146: $ 1  +m^{2}(-l^{-2}-1+l^{2}+l^{4})  +m^{4}(1-l^{2}) $ 

k10c147: $ (-l^{-2}-1-l^{2})  +m^{2}(-l^{-4}-l^{-2}+2+l^{2})  +m^{4}(l^{-2}-1) $ 

k10c148: $ (l^{2}+5l^{4}+3l^{6})  +m^{2}(-2l^{2}-9l^{4}-3l^{6})  +m^{4}(l^{2}+5l^{4}+l^{6})  -l^{4}m^{6} $ 

k10c149: $ (4l^{4}+4l^{6}+l^{8})  +m^{2}(-6l^{4}-6l^{6}-2l^{8})  +m^{4}(2l^{4}+4l^{6}+l^{8})  -l^{6}m^{6} $ 

k10c150: $ (-l^{-4}-2l^{-2})  +m^{2}(2l^{-6}+4l^{-4}+3l^{-2})  +m^{4}(-l^{-6}-4l^{-4}-l^{-2})  +l^{-4}m^{6} $ 

k10c151: $ (l^{-6}-3l^{-2}-1)  +m^{2}(l^{-4}+6l^{-2}+2)  +m^{4}(-l^{-4}-4l^{-2}-1)  +l^{-2}m^{6} $ 

k10c152: $ (8l^{8}+10l^{10}+3l^{12})  +m^{2}(-22l^{8}-17l^{10}-2l^{12})  +m^{4}(21l^{8}+8l^{10})  +m^{6}(-8l^{8}-l^{10})  +l^{8}m^{8} $ 

k10c153: $ (3l^{-2}+6+l^{2}-l^{4})  +m^{2}(-4l^{-2}-10-l^{2}+l^{4})  +m^{4}(l^{-2}+6)  -m^{6} $ 

k10c154: $ (l^{-12}+2l^{-10}-2l^{-8}-4l^{-6})  +m^{2}(-2l^{-10}+2l^{-8}+9l^{-6})  -6l^{-6}m^{4}  +l^{-6}m^{6} $ 

k10c155: $ (2l^{-4}+4l^{-2}+3)  +m^{2}(-3l^{-4}-8l^{-2}-3)  +m^{4}(l^{-4}+5l^{-2}+1)  -l^{-2}m^{6} $ 

k10c156: $ (-2l^{2}-l^{4})  +m^{2}(2+5l^{2}+2l^{4})  +m^{4}(-1-4l^{2}-l^{4})  +l^{2}m^{6} $ 

k10c157: $ (-l^{-8}+2l^{-4})  +m^{2}(-l^{-8}-2l^{-6}-5l^{-4})  +m^{4}(l^{-8}+3l^{-6}+2l^{-4})  -l^{-6}m^{6} $ 

k10c158: $ (l^{-4}-2-2l^{2})  +m^{2}(l^{-2}+6+2l^{2})  +m^{4}(-l^{-2}-4-l^{2})  +m^{6} $ 

k10c159: $ (-l^{2}+l^{4}+l^{6})  +m^{2}(-l^{2}-5l^{4}-2l^{6})  +m^{4}(l^{2}+4l^{4}+l^{6})  -l^{4}m^{6} $ 

k10c160: $ (-l^{-8}-l^{-6}-l^{-2})  +m^{2}(3l^{-6}+3l^{-4}+3l^{-2})  +m^{4}(-l^{-6}-4l^{-4}-l^{-2})  +l^{-4}m^{6} $ 

k10c161: $ (-3l^{6}-l^{8}+l^{10})  +m^{2}(9l^{6}+l^{8}-l^{10})  -6l^{6}m^{4}  +l^{6}m^{6} $ 

k10c162: $ (l^{-10}-l^{-8}-3l^{-6})  +m^{2}(-l^{-10}+l^{-8}+9l^{-6})  -6l^{-6}m^{4}  +l^{-6}m^{6} $ 

k10c163: $ (3+3l^{2}-l^{6})  +m^{2}(-2-5l^{2}+l^{4}+l^{6})  +m^{4}(2l^{2}-l^{4}) $ 

k10c164: $ (l^{-6}+2l^{-4}+l^{-2}+1)  +m^{2}(2l^{-2}+1)  +m^{4}(-l^{-4}-3l^{-2}-1)  +l^{-2}m^{6} $ 

k10c165: $ (l^{-2}+3+l^{2})  +m^{2}(-2l^{-2}-4+l^{4})  +m^{4}(2-l^{2}) $ 

k10c166: $ (l^{-6}+l^{-4}-l^{-2})  +m^{2}(-l^{-8}-l^{-6}+2l^{-2})  +m^{4}(l^{-6}-l^{-4}) $ 


%\newpage


%\section{P-polynomial: $P(x,y,z)$}
%$x P_{L+} -y P_{L-} = z P_{L0}$ \par
%k3c1: $ (2(x/y)-(x/y)^{2})  +(z/y)^{2} $ 

k4c1: $ ((x/y)^{-1}-1+(x/y))  -(x/y)^{-1}(z/y)^{2} $ 

k5c1: $ (3(x/y)^{2}-2(x/y)^{3})  +(z/y)^{2}(4(x/y)-(x/y)^{2})  +(z/y)^{4} $ 

k5c2: $ ((x/y)+(x/y)^{2}-(x/y)^{3})  +(z/y)^{2}(1+(x/y)) $ 

k6c1: $ ((x/y)^{-1}-(x/y)+(x/y)^{2})  +(z/y)^{2}(-(x/y)^{-1}-1) $ 

k6c2: $ (2-2(x/y)+(x/y)^{2})  +(z/y)^{2}((x/y)^{-1}-3+(x/y))  -(x/y)^{-1}(z/y)^{4} $ 

k6c3: $ (-(x/y)^{-1}+3-(x/y))  +(z/y)^{2}(-(x/y)^{-2}+3(x/y)^{-1}-1)  +(x/y)^{-2}(z/y)^{4} $ 

k7c1: $ (4(x/y)^{3}-3(x/y)^{4})  +(z/y)^{2}(10(x/y)^{2}-4(x/y)^{3})  +(z/y)^{4}(6(x/y)-(x/y)^{2})  +(z/y)^{6} $ 

k7c2: $ ((x/y)+(x/y)^{3}-(x/y)^{4})  +(z/y)^{2}(1+(x/y)+(x/y)^{2}) $ 

k7c3: $ (-2(x/y)^{-4}+2(x/y)^{-3}+(x/y)^{-2})  +(z/y)^{2}(-(x/y)^{-5}+3(x/y)^{-4}+3(x/y)^{-3})  +(z/y)^{4}((x/y)^{-5}+(x/y)^{-4}) $ 

k7c4: $ (-(x/y)^{-4}+2(x/y)^{-2})  +(z/y)^{2}((x/y)^{-4}+2(x/y)^{-3}+(x/y)^{-2}) $ 

k7c5: $ (2(x/y)^{2}-(x/y)^{4})  +(z/y)^{2}(3(x/y)+2(x/y)^{2}-(x/y)^{3})  +(z/y)^{4}(1+(x/y)) $ 

k7c6: $ (1-(x/y)+2(x/y)^{2}-(x/y)^{3})  +(z/y)^{2}((x/y)^{-1}-2+2(x/y))  -(x/y)^{-1}(z/y)^{4} $ 

k7c7: $ ((x/y)^{-2}-2(x/y)^{-1}+2)  +(z/y)^{2}(-2(x/y)^{-2}+2(x/y)^{-1}-1)  +(x/y)^{-2}(z/y)^{4} $ 

k8c1: $ ((x/y)^{-1}-(x/y)^{2}+(x/y)^{3})  +(z/y)^{2}(-(x/y)^{-1}-1-(x/y)) $ 

k8c2: $ (3(x/y)-3(x/y)^{2}+(x/y)^{3})  +(z/y)^{2}(4-7(x/y)+3(x/y)^{2})  +(z/y)^{4}((x/y)^{-1}-5+(x/y))  -(x/y)^{-1}(z/y)^{6} $ 

k8c3: $ ((x/y)^{-2}-1+(x/y)^{2})  +(z/y)^{2}(-(x/y)^{-2}-2(x/y)^{-1}-1) $ 

k8c4: $ (2(x/y)^{-1}-2+(x/y)^{2})  +(z/y)^{2}((x/y)^{-2}-3(x/y)^{-1}-2+(x/y))  +(z/y)^{4}(-(x/y)^{-2}-(x/y)^{-1}) $ 

k8c5: $ (2(x/y)^{-3}-5(x/y)^{-2}+4(x/y)^{-1})  +(z/y)^{2}(3(x/y)^{-4}-8(x/y)^{-3}+4(x/y)^{-2})  +(z/y)^{4}((x/y)^{-5}-5(x/y)^{-4}+(x/y)^{-3})  -(x/y)^{-5}(z/y)^{6} $ 

k8c6: $ (2-(x/y)-(x/y)^{2}+(x/y)^{3})  +(z/y)^{2}((x/y)^{-1}-2-2(x/y)+(x/y)^{2})  +(z/y)^{4}(-(x/y)^{-1}-1) $ 

k8c7: $ (-2(x/y)^{-2}+4(x/y)^{-1}-1)  +(z/y)^{2}(-3(x/y)^{-3}+8(x/y)^{-2}-3(x/y)^{-1})  +(z/y)^{4}(-(x/y)^{-4}+5(x/y)^{-3}-(x/y)^{-2})  +(x/y)^{-4}(z/y)^{6} $ 

k8c8: $ (-(x/y)^{-2}+(x/y)^{-1}+2-(x/y))  +(z/y)^{2}(-(x/y)^{-3}+2(x/y)^{-2}+2(x/y)^{-1}-1)  +(z/y)^{4}((x/y)^{-3}+(x/y)^{-2}) $ 

k8c9: $ (2(x/y)^{-1}-3+2(x/y))  +(z/y)^{2}(3(x/y)^{-2}-8(x/y)^{-1}+3)  +(z/y)^{4}((x/y)^{-3}-5(x/y)^{-2}+(x/y)^{-1})  -(x/y)^{-3}(z/y)^{6} $ 

k8c10: $ (-3(x/y)^{-2}+6(x/y)^{-1}-2)  +(z/y)^{2}(-3(x/y)^{-3}+9(x/y)^{-2}-3(x/y)^{-1})  +(z/y)^{4}(-(x/y)^{-4}+5(x/y)^{-3}-(x/y)^{-2})  +(x/y)^{-4}(z/y)^{6} $ 

k8c11: $ (1+(x/y)-2(x/y)^{2}+(x/y)^{3})  +(z/y)^{2}((x/y)^{-1}-1-2(x/y)+(x/y)^{2})  +(z/y)^{4}(-(x/y)^{-1}-1) $ 

k8c12: $ ((x/y)^{-2}-(x/y)^{-1}+1-(x/y)+(x/y)^{2})  +(z/y)^{2}(-2(x/y)^{-2}+(x/y)^{-1}-2)  +(x/y)^{-2}(z/y)^{4} $ 

k8c13: $ (-(x/y)^{-2}+2(x/y)^{-1})  +(z/y)^{2}(-(x/y)^{-3}+2(x/y)^{-2}+(x/y)^{-1}-1)  +(z/y)^{4}((x/y)^{-3}+(x/y)^{-2}) $ 

k8c14: $ 1  +(z/y)^{2}((x/y)^{-1}-1-(x/y)+(x/y)^{2})  +(z/y)^{4}(-(x/y)^{-1}-1) $ 

k8c15: $ ((x/y)^{2}+3(x/y)^{3}-4(x/y)^{4}+(x/y)^{5})  +(z/y)^{2}(2(x/y)+5(x/y)^{2}-3(x/y)^{3})  +(z/y)^{4}(1+2(x/y)) $ 

k8c16: $ (2(x/y)-(x/y)^{2})  +(z/y)^{2}(-2(x/y)^{-1}+5-2(x/y))  +(z/y)^{4}(-(x/y)^{-2}+4(x/y)^{-1}-1)  +(x/y)^{-2}(z/y)^{6} $ 

k8c17: $ ((x/y)^{-1}-1+(x/y))  +(z/y)^{2}(2(x/y)^{-2}-5(x/y)^{-1}+2)  +(z/y)^{4}((x/y)^{-3}-4(x/y)^{-2}+(x/y)^{-1})  -(x/y)^{-3}(z/y)^{6} $ 

k8c18: $ (-(x/y)^{-1}+3-(x/y))  +(z/y)^{2}((x/y)^{-2}-(x/y)^{-1}+1)  +(z/y)^{4}((x/y)^{-3}-3(x/y)^{-2}+(x/y)^{-1})  -(x/y)^{-3}(z/y)^{6} $ 

k8c19: $ ((x/y)^{-5}-5(x/y)^{-4}+5(x/y)^{-3})  +(z/y)^{2}(-5(x/y)^{-5}+10(x/y)^{-4})  +(z/y)^{4}(-(x/y)^{-6}+6(x/y)^{-5})  +(x/y)^{-6}(z/y)^{6} $ 

k8c20: $ (-1+4(x/y)-2(x/y)^{2})  +(z/y)^{2}(-(x/y)^{-1}+4-(x/y))  +(x/y)^{-1}(z/y)^{4} $ 

k8c21: $ (3(x/y)-3(x/y)^{2}+(x/y)^{3})  +(z/y)^{2}(2-3(x/y)+(x/y)^{2})  -(z/y)^{4} $ 

k9c1: $ (5(x/y)^{4}-4(x/y)^{5})  +(z/y)^{2}(20(x/y)^{3}-10(x/y)^{4})  +(z/y)^{4}(21(x/y)^{2}-6(x/y)^{3})  +(z/y)^{6}(8(x/y)-(x/y)^{2})  +(z/y)^{8} $ 

k9c2: $ ((x/y)+(x/y)^{4}-(x/y)^{5})  +(z/y)^{2}(1+(x/y)+(x/y)^{2}+(x/y)^{3}) $ 

k9c3: $ (-3(x/y)^{-5}+3(x/y)^{-4}+(x/y)^{-3})  +(z/y)^{2}(-4(x/y)^{-6}+7(x/y)^{-5}+6(x/y)^{-4})  +(z/y)^{4}(-(x/y)^{-7}+5(x/y)^{-6}+5(x/y)^{-5})  +(z/y)^{6}((x/y)^{-7}+(x/y)^{-6}) $ 

k9c4: $ ((x/y)^{2}+2(x/y)^{4}-2(x/y)^{5})  +(z/y)^{2}(3(x/y)+2(x/y)^{2}+3(x/y)^{3}-(x/y)^{4})  +(z/y)^{4}(1+(x/y)+(x/y)^{2}) $ 

k9c5: $ (-(x/y)^{-5}+(x/y)^{-3}+(x/y)^{-2})  +(z/y)^{2}((x/y)^{-5}+2(x/y)^{-4}+2(x/y)^{-3}+(x/y)^{-2}) $ 

k9c6: $ (3(x/y)^{3}-(x/y)^{4}-(x/y)^{5})  +(z/y)^{2}(7(x/y)^{2}+3(x/y)^{3}-3(x/y)^{4})  +(z/y)^{4}(5(x/y)+4(x/y)^{2}-(x/y)^{3})  +(z/y)^{6}(1+(x/y)) $ 

k9c7: $ (2(x/y)^{2}-(x/y)^{3}+(x/y)^{4}-(x/y)^{5})  +(z/y)^{2}(3(x/y)+(x/y)^{2}+2(x/y)^{3}-(x/y)^{4})  +(z/y)^{4}(1+(x/y)+(x/y)^{2}) $ 

k9c8: $ ((x/y)^{-1}-1+2(x/y)^{2}-(x/y)^{3})  +(z/y)^{2}((x/y)^{-2}-2(x/y)^{-1}-1+2(x/y))  +(z/y)^{4}(-(x/y)^{-2}-(x/y)^{-1}) $ 

k9c9: $ (2(x/y)^{3}+(x/y)^{4}-2(x/y)^{5})  +(z/y)^{2}(7(x/y)^{2}+4(x/y)^{3}-3(x/y)^{4})  +(z/y)^{4}(5(x/y)+4(x/y)^{2}-(x/y)^{3})  +(z/y)^{6}(1+(x/y)) $ 

k9c10: $ (-2(x/y)^{-5}+(x/y)^{-4}+2(x/y)^{-3})  +(z/y)^{2}(-(x/y)^{-6}+2(x/y)^{-5}+5(x/y)^{-4}+2(x/y)^{-3})  +(z/y)^{4}((x/y)^{-6}+2(x/y)^{-5}+(x/y)^{-4}) $ 

k9c11: $ (-2(x/y)^{-4}+3(x/y)^{-3}-(x/y)^{-2}+(x/y)^{-1})  +(z/y)^{2}(-(x/y)^{-5}+6(x/y)^{-4}-4(x/y)^{-3}+3(x/y)^{-2})  +(z/y)^{4}(2(x/y)^{-5}-4(x/y)^{-4}+(x/y)^{-3})  -(x/y)^{-5}(z/y)^{6} $ 

k9c12: $ (1-(x/y)^{2}+2(x/y)^{3}-(x/y)^{4})  +(z/y)^{2}((x/y)^{-1}-1-(x/y)+2(x/y)^{2})  +(z/y)^{4}(-(x/y)^{-1}-1) $ 

k9c13: $ (-(x/y)^{-5}-(x/y)^{-4}+3(x/y)^{-3})  +(z/y)^{2}(-(x/y)^{-6}+(x/y)^{-5}+5(x/y)^{-4}+2(x/y)^{-3})  +(z/y)^{4}((x/y)^{-6}+2(x/y)^{-5}+(x/y)^{-4}) $ 

k9c14: $ ((x/y)^{-3}-2(x/y)^{-2}+(x/y)^{-1}+1)  +(z/y)^{2}(-2(x/y)^{-3}+(x/y)^{-2}+(x/y)^{-1}-1)  +(z/y)^{4}((x/y)^{-3}+(x/y)^{-2}) $ 

k9c15: $ (-(x/y)^{-4}+(x/y)^{-3}+(x/y)^{-2}-(x/y)^{-1}+1)  +(z/y)^{2}(2(x/y)^{-4}-(x/y)^{-2}+(x/y)^{-1})  +(z/y)^{4}(-(x/y)^{-4}-(x/y)^{-3}) $ 

k9c16: $ (-3(x/y)^{-4}+4(x/y)^{-3})  +(z/y)^{2}(-2(x/y)^{-6}+8(x/y)^{-4})  +(z/y)^{4}(-(x/y)^{-7}+3(x/y)^{-6}+5(x/y)^{-5})  +(z/y)^{6}((x/y)^{-7}+(x/y)^{-6}) $ 

k9c17: $ (2(x/y)^{-1}-3+2(x/y))  +(z/y)^{2}((x/y)^{-2}-6(x/y)^{-1}+5-2(x/y))  +(z/y)^{4}(-2(x/y)^{-2}+4(x/y)^{-1}-1)  +(x/y)^{-2}(z/y)^{6} $ 

k9c18: $ ((x/y)^{2}+(x/y)^{3}-(x/y)^{5})  +(z/y)^{2}(2(x/y)+4(x/y)^{2}+(x/y)^{3}-(x/y)^{4})  +(z/y)^{4}(1+2(x/y)+(x/y)^{2}) $ 

k9c19: $ ((x/y)^{-2}-(x/y)^{-1}+(x/y))  +(z/y)^{2}(-2(x/y)^{-2}+1-(x/y))  +(z/y)^{4}((x/y)^{-2}+(x/y)^{-1}) $ 

k9c20: $ (2(x/y)-2(x/y)^{2}+2(x/y)^{3}-(x/y)^{4})  +(z/y)^{2}(3-5(x/y)+5(x/y)^{2}-(x/y)^{3})  +(z/y)^{4}((x/y)^{-1}-4+2(x/y))  -(x/y)^{-1}(z/y)^{6} $ 

k9c21: $ (-(x/y)^{-4}+(x/y)^{-3}+(x/y)^{-1})  +(z/y)^{2}(2(x/y)^{-4}+(x/y)^{-1})  +(z/y)^{4}(-(x/y)^{-4}-(x/y)^{-3}) $ 

k9c22: $ (-(x/y)^{-2}+4(x/y)^{-1}-4+2(x/y))  +(z/y)^{2}(-2(x/y)^{-3}+6(x/y)^{-2}-6(x/y)^{-1}+1)  +(z/y)^{4}(-(x/y)^{-4}+4(x/y)^{-3}-2(x/y)^{-2})  +(x/y)^{-4}(z/y)^{6} $ 

k9c23: $ ((x/y)^{2}+2(x/y)^{3}-2(x/y)^{4})  +(z/y)^{2}(2(x/y)+4(x/y)^{2}-(x/y)^{4})  +(z/y)^{4}(1+2(x/y)+(x/y)^{2}) $ 

k9c24: $ ((x/y)^{-1}-3+5(x/y)-2(x/y)^{2})  +(z/y)^{2}(2(x/y)^{-2}-6(x/y)^{-1}+6-(x/y))  +(z/y)^{4}((x/y)^{-3}-4(x/y)^{-2}+2(x/y)^{-1})  -(x/y)^{-3}(z/y)^{6} $ 

k9c25: $ (1+(x/y)-3(x/y)^{2}+3(x/y)^{3}-(x/y)^{4})  +(z/y)^{2}((x/y)^{-1}-4(x/y)+3(x/y)^{2})  +(z/y)^{4}(-(x/y)^{-1}-2) $ 

k9c26: $ ((x/y)^{-3}-3(x/y)^{-2}+3(x/y)^{-1})  +(z/y)^{2}((x/y)^{-4}-5(x/y)^{-3}+6(x/y)^{-2}-2(x/y)^{-1})  +(z/y)^{4}(-2(x/y)^{-4}+4(x/y)^{-3}-(x/y)^{-2})  +(x/y)^{-4}(z/y)^{6} $ 

k9c27: $ ((x/y)^{-1}-2+3(x/y)-(x/y)^{2})  +(z/y)^{2}(2(x/y)^{-2}-6(x/y)^{-1}+5-(x/y))  +(z/y)^{4}((x/y)^{-3}-4(x/y)^{-2}+2(x/y)^{-1})  -(x/y)^{-3}(z/y)^{6} $ 

k9c28: $ (-1+5(x/y)-4(x/y)^{2}+(x/y)^{3})  +(z/y)^{2}(-2(x/y)^{-1}+7-5(x/y)+(x/y)^{2})  +(z/y)^{4}(-(x/y)^{-2}+4(x/y)^{-1}-2)  +(x/y)^{-2}(z/y)^{6} $ 

k9c29: $ ((x/y)^{-1}-3+5(x/y)-2(x/y)^{2})  +(z/y)^{2}((x/y)^{-2}-5(x/y)^{-1}+7-2(x/y))  +(z/y)^{4}(-2(x/y)^{-2}+4(x/y)^{-1}-1)  +(x/y)^{-2}(z/y)^{6} $ 

k9c30: $ (2(x/y)^{-1}-4+4(x/y)-(x/y)^{2})  +(z/y)^{2}(2(x/y)^{-2}-7(x/y)^{-1}+5-(x/y))  +(z/y)^{4}((x/y)^{-3}-4(x/y)^{-2}+2(x/y)^{-1})  -(x/y)^{-3}(z/y)^{6} $ 

k9c31: $ (-1+4(x/y)-2(x/y)^{2})  +(z/y)^{2}(-2(x/y)^{-1}+7-4(x/y)+(x/y)^{2})  +(z/y)^{4}(-(x/y)^{-2}+4(x/y)^{-1}-2)  +(x/y)^{-2}(z/y)^{6} $ 

k9c32: $ ((x/y)^{-3}-2(x/y)^{-2}+(x/y)^{-1}+1)  +(z/y)^{2}((x/y)^{-4}-4(x/y)^{-3}+3(x/y)^{-2}-(x/y)^{-1})  +(z/y)^{4}(-2(x/y)^{-4}+3(x/y)^{-3}-(x/y)^{-2})  +(x/y)^{-4}(z/y)^{6} $ 

k9c33: $ (2(x/y)-(x/y)^{2})  +(z/y)^{2}((x/y)^{-2}-3(x/y)^{-1}+4-(x/y))  +(z/y)^{4}((x/y)^{-3}-3(x/y)^{-2}+2(x/y)^{-1})  -(x/y)^{-3}(z/y)^{6} $ 

k9c34: $ ((x/y)^{-1}-1+(x/y))  +(z/y)^{2}((x/y)^{-2}-4(x/y)^{-1}+3-(x/y))  +(z/y)^{4}((x/y)^{-3}-3(x/y)^{-2}+2(x/y)^{-1})  -(x/y)^{-3}(z/y)^{6} $ 

k9c35: $ (3(x/y)^{3}-(x/y)^{4}-(x/y)^{5})  +(z/y)^{2}(1+2(x/y)+3(x/y)^{2}+(x/y)^{3}) $ 

k9c36: $ (-2(x/y)^{-4}+4(x/y)^{-3}-3(x/y)^{-2}+2(x/y)^{-1})  +(z/y)^{2}(-(x/y)^{-5}+6(x/y)^{-4}-5(x/y)^{-3}+3(x/y)^{-2})  +(z/y)^{4}(2(x/y)^{-5}-4(x/y)^{-4}+(x/y)^{-3})  -(x/y)^{-5}(z/y)^{6} $ 

k9c37: $ ((x/y)^{-2}-2+2(x/y))  +(z/y)^{2}(-2(x/y)^{-2}-(x/y)^{-1}+1-(x/y))  +(z/y)^{4}((x/y)^{-2}+(x/y)^{-1}) $ 

k9c38: $ (4(x/y)^{3}-3(x/y)^{4})  +(z/y)^{2}((x/y)+7(x/y)^{2}-(x/y)^{3}-(x/y)^{4})  +(z/y)^{4}(1+3(x/y)+(x/y)^{2}) $ 

k9c39: $ (-(x/y)^{-4}+2(x/y)^{-3}-2(x/y)^{-2}+2(x/y)^{-1})  +(z/y)^{2}(3(x/y)^{-4}-3(x/y)^{-3}+(x/y)^{-2}+(x/y)^{-1})  +(z/y)^{4}(-2(x/y)^{-4}-(x/y)^{-3}) $ 

k9c40: $ (2-2(x/y)+(x/y)^{2})  +(z/y)^{2}(-2(x/y)+(x/y)^{2})  +(z/y)^{4}(-(x/y)^{-2}+2(x/y)^{-1}-2)  +(x/y)^{-2}(z/y)^{6} $ 

k9c41: $ (3(x/y)-3(x/y)^{2}+(x/y)^{3})  +(z/y)^{2}(-(x/y)^{-2}+4-3(x/y))  +(z/y)^{4}((x/y)^{-2}+2(x/y)^{-1}) $ 

k9c42: $ (2(x/y)^{-1}-3+2(x/y))  +(z/y)^{2}((x/y)^{-2}-4(x/y)^{-1}+1)  -(x/y)^{-2}(z/y)^{4} $ 

k9c43: $ (-(x/y)^{-4}+3(x/y)^{-3}-4(x/y)^{-2}+3(x/y)^{-1})  +(z/y)^{2}(4(x/y)^{-4}-7(x/y)^{-3}+4(x/y)^{-2})  +(z/y)^{4}((x/y)^{-5}-5(x/y)^{-4}+(x/y)^{-3})  -(x/y)^{-5}(z/y)^{6} $ 

k9c44: $ ((x/y)^{-1}-2+3(x/y)-(x/y)^{2})  +(z/y)^{2}(-2(x/y)^{-1}+3-(x/y))  +(x/y)^{-1}(z/y)^{4} $ 

k9c45: $ (2(x/y)-2(x/y)^{2}+2(x/y)^{3}-(x/y)^{4})  +(z/y)^{2}(2-2(x/y)+2(x/y)^{2})  -(z/y)^{4} $ 

k9c46: $ (2-(x/y)-(x/y)^{2}+(x/y)^{3})  +(z/y)^{2}(-1-(x/y)) $ 

k9c47: $ ((x/y)^{-3}-2(x/y)^{-2}+(x/y)^{-1}+1)  +(z/y)^{2}(-3(x/y)^{-3}+4(x/y)^{-2}-2(x/y)^{-1})  +(z/y)^{4}(-(x/y)^{-4}+4(x/y)^{-3}-(x/y)^{-2})  +(x/y)^{-4}(z/y)^{6} $ 

k9c48: $ (-2(x/y)^{-3}+3(x/y)^{-2})  +(z/y)^{2}(3(x/y)^{-3}-(x/y)^{-2}+(x/y)^{-1})  -(x/y)^{-3}(z/y)^{4} $ 

k9c49: $ (-3(x/y)^{-4}+4(x/y)^{-3})  +(z/y)^{2}(-2(x/y)^{-5}+6(x/y)^{-4}+2(x/y)^{-3})  +(z/y)^{4}(2(x/y)^{-5}+(x/y)^{-4}) $ 

k10c1: $ ((x/y)^{-1}-(x/y)^{3}+(x/y)^{4})  +(z/y)^{2}(-(x/y)^{-1}-1-(x/y)-(x/y)^{2}) $ 

k10c2: $ (4(x/y)^{2}-4(x/y)^{3}+(x/y)^{4})  +(z/y)^{2}(10(x/y)-14(x/y)^{2}+6(x/y)^{3})  +(z/y)^{4}(6-16(x/y)+5(x/y)^{2})  +(z/y)^{6}((x/y)^{-1}-7+(x/y))  -(x/y)^{-1}(z/y)^{8} $ 

k10c3: $ ((x/y)^{-2}-(x/y)+(x/y)^{3})  +(z/y)^{2}(-(x/y)^{-2}-2(x/y)^{-1}-2-(x/y)) $ 

k10c4: $ (2(x/y)^{-2}-2(x/y)^{-1}+(x/y)^{2})  +(z/y)^{2}((x/y)^{-3}-3(x/y)^{-2}-2(x/y)^{-1}-2+(x/y))  +(z/y)^{4}(-(x/y)^{-3}-(x/y)^{-2}-(x/y)^{-1}) $ 

k10c5: $ (-3(x/y)^{-3}+5(x/y)^{-2}-(x/y)^{-1})  +(z/y)^{2}(-7(x/y)^{-4}+17(x/y)^{-3}-6(x/y)^{-2})  +(z/y)^{4}(-5(x/y)^{-5}+17(x/y)^{-4}-5(x/y)^{-3})  +(z/y)^{6}(-(x/y)^{-6}+7(x/y)^{-5}-(x/y)^{-4})  +(x/y)^{-6}(z/y)^{8} $ 

k10c6: $ (3(x/y)-2(x/y)^{2}-(x/y)^{3}+(x/y)^{4})  +(z/y)^{2}(4-4(x/y)-4(x/y)^{2}+3(x/y)^{3})  +(z/y)^{4}((x/y)^{-1}-4-4(x/y)+(x/y)^{2})  +(z/y)^{6}(-(x/y)^{-1}-1) $ 

k10c7: $ (1+(x/y)^{2}-2(x/y)^{3}+(x/y)^{4})  +(z/y)^{2}((x/y)^{-1}-1-2(x/y)^{2}+(x/y)^{3})  +(z/y)^{4}(-(x/y)^{-1}-1-(x/y)) $ 

k10c8: $ (3-3(x/y)+(x/y)^{3})  +(z/y)^{2}(4(x/y)^{-1}-7-3(x/y)+3(x/y)^{2})  +(z/y)^{4}((x/y)^{-2}-5(x/y)^{-1}-4+(x/y))  +(z/y)^{6}(-(x/y)^{-2}-(x/y)^{-1}) $ 

k10c9: $ (2(x/y)^{-2}-4(x/y)^{-1}+3)  +(z/y)^{2}(7(x/y)^{-3}-16(x/y)^{-2}+7(x/y)^{-1})  +(z/y)^{4}(5(x/y)^{-4}-17(x/y)^{-3}+5(x/y)^{-2})  +(z/y)^{6}((x/y)^{-5}-7(x/y)^{-4}+(x/y)^{-3})  -(x/y)^{-5}(z/y)^{8} $ 

k10c10: $ (-(x/y)^{-3}+2(x/y)^{-2}-(x/y)^{-1}+1)  +(z/y)^{2}(-(x/y)^{-4}+2(x/y)^{-3}+(x/y)^{-1}-1)  +(z/y)^{4}((x/y)^{-4}+(x/y)^{-3}+(x/y)^{-2}) $ 

k10c11: $ (2(x/y)^{-1}-1-(x/y)+(x/y)^{3})  +(z/y)^{2}((x/y)^{-2}-2(x/y)^{-1}-4-(x/y)+(x/y)^{2})  +(z/y)^{4}(-(x/y)^{-2}-2(x/y)^{-1}-1) $ 

k10c12: $ (-2(x/y)^{-3}+2(x/y)^{-2}+2(x/y)^{-1}-1)  +(z/y)^{2}(-3(x/y)^{-4}+5(x/y)^{-3}+5(x/y)^{-2}-3(x/y)^{-1})  +(z/y)^{4}(-(x/y)^{-5}+4(x/y)^{-4}+4(x/y)^{-3}-(x/y)^{-2})  +(z/y)^{6}((x/y)^{-5}+(x/y)^{-4}) $ 

k10c13: $ ((x/y)^{-2}-1+(x/y)-(x/y)^{2}+(x/y)^{3})  +(z/y)^{2}(-2(x/y)^{-2}-(x/y)^{-1}-2(x/y))  +(z/y)^{4}((x/y)^{-2}+(x/y)^{-1}) $ 

k10c14: $ ((x/y)+(x/y)^{2}-(x/y)^{3})  +(z/y)^{2}(3-(x/y)-2(x/y)^{2}+2(x/y)^{3})  +(z/y)^{4}((x/y)^{-1}-3-3(x/y)+(x/y)^{2})  +(z/y)^{6}(-(x/y)^{-1}-1) $ 

k10c15: $ (-2(x/y)^{-2}+3(x/y)^{-1}+1-(x/y))  +(z/y)^{2}(-3(x/y)^{-3}+5(x/y)^{-2}+4(x/y)^{-1}-3)  +(z/y)^{4}(-(x/y)^{-4}+4(x/y)^{-3}+4(x/y)^{-2}-(x/y)^{-1})  +(z/y)^{6}((x/y)^{-4}+(x/y)^{-3}) $ 

k10c16: $ ((x/y)^{-3}-2(x/y)^{-1}+1+(x/y))  +(z/y)^{2}((x/y)^{-4}-(x/y)^{-3}-4(x/y)^{-2}-(x/y)^{-1}+1)  +(z/y)^{4}(-(x/y)^{-4}-2(x/y)^{-3}-(x/y)^{-2}) $ 

k10c17: $ (-2(x/y)^{-1}+5-2(x/y))  +(z/y)^{2}(-7(x/y)^{-2}+16(x/y)^{-1}-7)  +(z/y)^{4}(-5(x/y)^{-3}+17(x/y)^{-2}-5(x/y)^{-1})  +(z/y)^{6}(-(x/y)^{-4}+7(x/y)^{-3}-(x/y)^{-2})  +(x/y)^{-4}(z/y)^{8} $ 

k10c18: $ ((x/y)^{-1}-(x/y)+(x/y)^{2})  +(z/y)^{2}((x/y)^{-2}-(x/y)^{-1}-3+(x/y)^{2})  +(z/y)^{4}(-(x/y)^{-2}-2(x/y)^{-1}-1) $ 

k10c19: $ (-(x/y)^{-1}+3-(x/y))  +(z/y)^{2}(-3(x/y)^{-2}+5(x/y)^{-1}+1-2(x/y))  +(z/y)^{4}(-(x/y)^{-3}+4(x/y)^{-2}+3(x/y)^{-1}-1)  +(z/y)^{6}((x/y)^{-3}+(x/y)^{-2}) $ 

k10c20: $ (2-(x/y)-(x/y)^{3}+(x/y)^{4})  +(z/y)^{2}((x/y)^{-1}-2-(x/y)-2(x/y)^{2}+(x/y)^{3})  +(z/y)^{4}(-(x/y)^{-1}-1-(x/y)) $ 

k10c21: $ ((x/y)+2(x/y)^{2}-3(x/y)^{3}+(x/y)^{4})  +(z/y)^{2}(3-5(x/y)^{2}+3(x/y)^{3})  +(z/y)^{4}((x/y)^{-1}-3-4(x/y)+(x/y)^{2})  +(z/y)^{6}(-(x/y)^{-1}-1) $ 

k10c22: $ (2(x/y)^{-2}-2(x/y)^{-1}-1+2(x/y))  +(z/y)^{2}(3(x/y)^{-3}-5(x/y)^{-2}-5(x/y)^{-1}+3)  +(z/y)^{4}((x/y)^{-4}-4(x/y)^{-3}-4(x/y)^{-2}+(x/y)^{-1})  +(z/y)^{6}(-(x/y)^{-4}-(x/y)^{-3}) $ 

k10c23: $ (-2(x/y)^{-3}+3(x/y)^{-2})  +(z/y)^{2}(-3(x/y)^{-4}+6(x/y)^{-3}+2(x/y)^{-2}-2(x/y)^{-1})  +(z/y)^{4}(-(x/y)^{-5}+4(x/y)^{-4}+3(x/y)^{-3}-(x/y)^{-2})  +(z/y)^{6}((x/y)^{-5}+(x/y)^{-4}) $ 

k10c24: $ (1+(x/y)-(x/y)^{2}-(x/y)^{3}+(x/y)^{4})  +(z/y)^{2}((x/y)^{-1}-3(x/y)-(x/y)^{2}+(x/y)^{3})  +(z/y)^{4}(-(x/y)^{-1}-2-(x/y)) $ 

k10c25: $ (2(x/y)-2(x/y)^{3}+(x/y)^{4})  +(z/y)^{2}(3-2(x/y)-3(x/y)^{2}+2(x/y)^{3})  +(z/y)^{4}((x/y)^{-1}-3-3(x/y)+(x/y)^{2})  +(z/y)^{6}(-(x/y)^{-1}-1) $ 

k10c26: $ (2(x/y)^{-2}-3(x/y)^{-1}+1+(x/y))  +(z/y)^{2}(3(x/y)^{-3}-6(x/y)^{-2}-2(x/y)^{-1}+2)  +(z/y)^{4}((x/y)^{-4}-4(x/y)^{-3}-3(x/y)^{-2}+(x/y)^{-1})  +(z/y)^{6}(-(x/y)^{-4}-(x/y)^{-3}) $ 

k10c27: $ ((x/y)+(x/y)^{2}-(x/y)^{3})  +(z/y)^{2}(-2(x/y)^{-1}+3+3(x/y)-2(x/y)^{2})  +(z/y)^{4}(-(x/y)^{-2}+3(x/y)^{-1}+3-(x/y))  +(z/y)^{6}((x/y)^{-2}+(x/y)^{-1}) $ 

k10c28: $ (-(x/y)^{-3}+3(x/y)^{-1}-1)  +(z/y)^{2}(-(x/y)^{-4}+(x/y)^{-3}+4(x/y)^{-2}-1)  +(z/y)^{4}((x/y)^{-4}+2(x/y)^{-3}+(x/y)^{-2}) $ 

k10c29: $ (2(x/y)^{-1}-2+(x/y)-(x/y)^{2}+(x/y)^{3})  +(z/y)^{2}((x/y)^{-2}-5(x/y)^{-1}+3-4(x/y)+(x/y)^{2})  +(z/y)^{4}(-2(x/y)^{-2}+3(x/y)^{-1}-2)  +(x/y)^{-2}(z/y)^{6} $ 

k10c30: $ (2(x/y)-(x/y)^{2})  +(z/y)^{2}((x/y)^{-1}+1-2(x/y)+(x/y)^{3})  +(z/y)^{4}(-(x/y)^{-1}-2-(x/y)) $ 

k10c31: $ (-(x/y)^{-2}+(x/y)^{-1}+2-(x/y))  +(z/y)^{2}(-(x/y)^{-3}+(x/y)^{-2}+3(x/y)^{-1}-(x/y))  +(z/y)^{4}((x/y)^{-3}+2(x/y)^{-2}+(x/y)^{-1}) $ 

k10c32: $ ((x/y)^{-1}-1+(x/y))  +(z/y)^{2}(2(x/y)^{-2}-3(x/y)^{-1}-2+2(x/y))  +(z/y)^{4}((x/y)^{-3}-3(x/y)^{-2}-3(x/y)^{-1}+1)  +(z/y)^{6}(-(x/y)^{-3}-(x/y)^{-2}) $ 

k10c33: $ 1  +(z/y)^{2}(-(x/y)^{-3}+2(x/y)^{-1}-(x/y))  +(z/y)^{4}((x/y)^{-3}+2(x/y)^{-2}+(x/y)^{-1}) $ 

k10c34: $ (-(x/y)^{-3}+(x/y)^{-2}+2-(x/y))  +(z/y)^{2}(-(x/y)^{-4}+2(x/y)^{-3}+(x/y)^{-2}+2(x/y)^{-1}-1)  +(z/y)^{4}((x/y)^{-4}+(x/y)^{-3}+(x/y)^{-2}) $ 

k10c35: $ ((x/y)^{-3}-(x/y)^{-2}+1-(x/y)+(x/y)^{2})  +(z/y)^{2}(-2(x/y)^{-3}-2)  +(z/y)^{4}((x/y)^{-3}+(x/y)^{-2}) $ 

k10c36: $ (1-(x/y)+2(x/y)^{2}-(x/y)^{3})  +(z/y)^{2}((x/y)^{-1}-1+(x/y)-(x/y)^{2}+(x/y)^{3})  +(z/y)^{4}(-(x/y)^{-1}-1-(x/y)) $ 

k10c37: $ (-(x/y)^{-2}+(x/y)^{-1}+1+(x/y)-(x/y)^{2})  +(z/y)^{2}(-(x/y)^{-3}+(x/y)^{-2}+3(x/y)^{-1}+1-(x/y))  +(z/y)^{4}((x/y)^{-3}+2(x/y)^{-2}+(x/y)^{-1}) $ 

k10c38: $ (1+(x/y)-2(x/y)^{2}+(x/y)^{3})  +(z/y)^{2}((x/y)^{-1}-3(x/y)+(x/y)^{3})  +(z/y)^{4}(-(x/y)^{-1}-2-(x/y)) $ 

k10c39: $ (2(x/y)-(x/y)^{2})  +(z/y)^{2}(3-2(x/y)-2(x/y)^{2}+2(x/y)^{3})  +(z/y)^{4}((x/y)^{-1}-3-3(x/y)+(x/y)^{2})  +(z/y)^{6}(-(x/y)^{-1}-1) $ 

k10c40: $ (-(x/y)^{-3}+3(x/y)^{-1}-1)  +(z/y)^{2}(-2(x/y)^{-4}+3(x/y)^{-3}+4(x/y)^{-2}-2(x/y)^{-1})  +(z/y)^{4}(-(x/y)^{-5}+3(x/y)^{-4}+3(x/y)^{-3}-(x/y)^{-2})  +(z/y)^{6}((x/y)^{-5}+(x/y)^{-4}) $ 

k10c41: $ ((x/y)^{-1}-1+2(x/y)-2(x/y)^{2}+(x/y)^{3})  +(z/y)^{2}((x/y)^{-2}-4(x/y)^{-1}+4-4(x/y)+(x/y)^{2})  +(z/y)^{4}(-2(x/y)^{-2}+3(x/y)^{-1}-2)  +(x/y)^{-2}(z/y)^{6} $ 

k10c42: $ (-(x/y)^{-2}+3(x/y)^{-1}-2+(x/y))  +(z/y)^{2}(-(x/y)^{-3}+4(x/y)^{-2}-5(x/y)^{-1}+3-(x/y))  +(z/y)^{4}(2(x/y)^{-3}-3(x/y)^{-2}+2(x/y)^{-1})  -(x/y)^{-3}(z/y)^{6} $ 

k10c43: $ (-(x/y)^{-2}+2(x/y)^{-1}-1+2(x/y)-(x/y)^{2})  +(z/y)^{2}(-(x/y)^{-3}+4(x/y)^{-2}-4(x/y)^{-1}+4-(x/y))  +(z/y)^{4}(2(x/y)^{-3}-3(x/y)^{-2}+2(x/y)^{-1})  -(x/y)^{-3}(z/y)^{6} $ 

k10c44: $ ((x/y)^{-1}-2+3(x/y)-(x/y)^{2})  +(z/y)^{2}((x/y)^{-2}-4(x/y)^{-1}+5-3(x/y)+(x/y)^{2})  +(z/y)^{4}(-2(x/y)^{-2}+3(x/y)^{-1}-2)  +(x/y)^{-2}(z/y)^{6} $ 

k10c45: $ (2(x/y)^{-1}-3+2(x/y))  +(z/y)^{2}(-(x/y)^{-3}+3(x/y)^{-2}-6(x/y)^{-1}+3-(x/y))  +(z/y)^{4}(2(x/y)^{-3}-3(x/y)^{-2}+2(x/y)^{-1})  -(x/y)^{-3}(z/y)^{6} $ 

k10c46: $ (3(x/y)^{-4}-8(x/y)^{-3}+6(x/y)^{-2})  +(z/y)^{2}(7(x/y)^{-5}-18(x/y)^{-4}+11(x/y)^{-3})  +(z/y)^{4}(5(x/y)^{-6}-17(x/y)^{-5}+6(x/y)^{-4})  +(z/y)^{6}((x/y)^{-7}-7(x/y)^{-6}+(x/y)^{-5})  -(x/y)^{-7}(z/y)^{8} $ 

k10c47: $ (-5(x/y)^{-3}+9(x/y)^{-2}-3(x/y)^{-1})  +(z/y)^{2}(-8(x/y)^{-4}+21(x/y)^{-3}-7(x/y)^{-2})  +(z/y)^{4}(-5(x/y)^{-5}+18(x/y)^{-4}-5(x/y)^{-3})  +(z/y)^{6}(-(x/y)^{-6}+7(x/y)^{-5}-(x/y)^{-4})  +(x/y)^{-6}(z/y)^{8} $ 

k10c48: $ (-4(x/y)^{-1}+9-4(x/y))  +(z/y)^{2}(-8(x/y)^{-2}+20(x/y)^{-1}-8)  +(z/y)^{4}(-5(x/y)^{-3}+18(x/y)^{-2}-5(x/y)^{-1})  +(z/y)^{6}(-(x/y)^{-4}+7(x/y)^{-3}-(x/y)^{-2})  +(x/y)^{-4}(z/y)^{8} $ 

k10c49: $ ((x/y)^{3}+5(x/y)^{4}-7(x/y)^{5}+2(x/y)^{6})  +(z/y)^{2}(4(x/y)^{2}+12(x/y)^{3}-10(x/y)^{4}+(x/y)^{5})  +(z/y)^{4}(4(x/y)+9(x/y)^{2}-3(x/y)^{3})  +(z/y)^{6}(1+2(x/y)) $ 

k10c50: $ (2(x/y)^{-4}-4(x/y)^{-3}+(x/y)^{-2}+2(x/y)^{-1})  +(z/y)^{2}(3(x/y)^{-5}-6(x/y)^{-4}-(x/y)^{-3}+3(x/y)^{-2})  +(z/y)^{4}((x/y)^{-6}-4(x/y)^{-5}-3(x/y)^{-4}+(x/y)^{-3})  +(z/y)^{6}(-(x/y)^{-6}-(x/y)^{-5}) $ 

k10c51: $ (-3(x/y)^{-3}+4(x/y)^{-2}+(x/y)^{-1}-1)  +(z/y)^{2}(-3(x/y)^{-4}+7(x/y)^{-3}+3(x/y)^{-2}-2(x/y)^{-1})  +(z/y)^{4}(-(x/y)^{-5}+4(x/y)^{-4}+3(x/y)^{-3}-(x/y)^{-2})  +(z/y)^{6}((x/y)^{-5}+(x/y)^{-4}) $ 

k10c52: $ (-(x/y)^{-2}+4-2(x/y))  +(z/y)^{2}(-2(x/y)^{-3}+2(x/y)^{-2}+6(x/y)^{-1}-3)  +(z/y)^{4}(-(x/y)^{-4}+3(x/y)^{-3}+4(x/y)^{-2}-(x/y)^{-1})  +(z/y)^{6}((x/y)^{-4}+(x/y)^{-3}) $ 

k10c53: $ (3(x/y)^{3}-3(x/y)^{5}+(x/y)^{6})  +(z/y)^{2}((x/y)+6(x/y)^{2}+2(x/y)^{3}-3(x/y)^{4})  +(z/y)^{4}(1+3(x/y)+2(x/y)^{2}) $ 

k10c54: $ (-2(x/y)^{-2}+2(x/y)^{-1}+3-2(x/y))  +(z/y)^{2}(-3(x/y)^{-3}+5(x/y)^{-2}+5(x/y)^{-1}-3)  +(z/y)^{4}(-(x/y)^{-4}+4(x/y)^{-3}+4(x/y)^{-2}-(x/y)^{-1})  +(z/y)^{6}((x/y)^{-4}+(x/y)^{-3}) $ 

k10c55: $ ((x/y)^{2}+(x/y)^{3}+(x/y)^{4}-3(x/y)^{5}+(x/y)^{6})  +(z/y)^{2}(2(x/y)+3(x/y)^{2}+3(x/y)^{3}-3(x/y)^{4})  +(z/y)^{4}(1+2(x/y)+2(x/y)^{2}) $ 

k10c56: $ ((x/y)^{-4}-2(x/y)^{-3}+2(x/y)^{-1})  +(z/y)^{2}(2(x/y)^{-5}-3(x/y)^{-4}-2(x/y)^{-3}+3(x/y)^{-2})  +(z/y)^{4}((x/y)^{-6}-3(x/y)^{-5}-3(x/y)^{-4}+(x/y)^{-3})  +(z/y)^{6}(-(x/y)^{-6}-(x/y)^{-5}) $ 

k10c57: $ (-2(x/y)^{-3}+2(x/y)^{-2}+2(x/y)^{-1}-1)  +(z/y)^{2}(-2(x/y)^{-4}+4(x/y)^{-3}+4(x/y)^{-2}-2(x/y)^{-1})  +(z/y)^{4}(-(x/y)^{-5}+3(x/y)^{-4}+3(x/y)^{-3}-(x/y)^{-2})  +(z/y)^{6}((x/y)^{-5}+(x/y)^{-4}) $ 

k10c58: $ ((x/y)^{-2}-2+3(x/y)-2(x/y)^{2}+(x/y)^{3})  +(z/y)^{2}(-2(x/y)^{-2}-2(x/y)^{-1}+3-3(x/y))  +(z/y)^{4}((x/y)^{-2}+2(x/y)^{-1}) $ 

k10c59: $ ((x/y)^{-3}-3(x/y)^{-2}+4(x/y)^{-1}-2+(x/y))  +(z/y)^{2}((x/y)^{-4}-4(x/y)^{-3}+5(x/y)^{-2}-4(x/y)^{-1}+1)  +(z/y)^{4}(-2(x/y)^{-4}+3(x/y)^{-3}-2(x/y)^{-2})  +(x/y)^{-4}(z/y)^{6} $ 

k10c60: $ ((x/y)^{-1}-2+4(x/y)-3(x/y)^{2}+(x/y)^{3})  +(z/y)^{2}((x/y)^{-2}-5(x/y)^{-1}+6-3(x/y))  +(z/y)^{4}((x/y)^{-3}-3(x/y)^{-2}+3(x/y)^{-1})  -(x/y)^{-3}(z/y)^{6} $ 

k10c61: $ ((x/y)^{-3}+(x/y)^{-2}-5(x/y)^{-1}+4)  +(z/y)^{2}(3(x/y)^{-4}-3(x/y)^{-3}-8(x/y)^{-2}+4(x/y)^{-1})  +(z/y)^{4}((x/y)^{-5}-4(x/y)^{-4}-5(x/y)^{-3}+(x/y)^{-2})  +(z/y)^{6}(-(x/y)^{-5}-(x/y)^{-4}) $ 

k10c62: $ (-4(x/y)^{-3}+7(x/y)^{-2}-2(x/y)^{-1})  +(z/y)^{2}(-8(x/y)^{-4}+20(x/y)^{-3}-7(x/y)^{-2})  +(z/y)^{4}(-5(x/y)^{-5}+18(x/y)^{-4}-5(x/y)^{-3})  +(z/y)^{6}(-(x/y)^{-6}+7(x/y)^{-5}-(x/y)^{-4})  +(x/y)^{-6}(z/y)^{8} $ 

k10c63: $ ((x/y)^{2}+3(x/y)^{4}-4(x/y)^{5}+(x/y)^{6})  +(z/y)^{2}(2(x/y)+3(x/y)^{2}+4(x/y)^{3}-3(x/y)^{4})  +(z/y)^{4}(1+2(x/y)+2(x/y)^{2}) $ 

k10c64: $ (3(x/y)^{-2}-6(x/y)^{-1}+4)  +(z/y)^{2}(8(x/y)^{-3}-19(x/y)^{-2}+8(x/y)^{-1})  +(z/y)^{4}(5(x/y)^{-4}-18(x/y)^{-3}+5(x/y)^{-2})  +(z/y)^{6}((x/y)^{-5}-7(x/y)^{-4}+(x/y)^{-3})  -(x/y)^{-5}(z/y)^{8} $ 

k10c65: $ (-3(x/y)^{-3}+5(x/y)^{-2}-(x/y)^{-1})  +(z/y)^{2}(-3(x/y)^{-4}+7(x/y)^{-3}+2(x/y)^{-2}-2(x/y)^{-1})  +(z/y)^{4}(-(x/y)^{-5}+4(x/y)^{-4}+3(x/y)^{-3}-(x/y)^{-2})  +(z/y)^{6}((x/y)^{-5}+(x/y)^{-4}) $ 

k10c66: $ (2(x/y)^{3}+2(x/y)^{4}-4(x/y)^{5}+(x/y)^{6})  +(z/y)^{2}(5(x/y)^{2}+9(x/y)^{3}-8(x/y)^{4}+(x/y)^{5})  +(z/y)^{4}(4(x/y)+8(x/y)^{2}-3(x/y)^{3})  +(z/y)^{6}(1+2(x/y)) $ 

k10c67: $ 1  +(z/y)^{2}((x/y)^{-1}-2(x/y)+(x/y)^{3})  +(z/y)^{4}(-(x/y)^{-1}-2-(x/y)) $ 

k10c68: $ ((x/y)+(x/y)^{2}-(x/y)^{3})  +(z/y)^{2}(-(x/y)^{-2}+3+(x/y)-(x/y)^{2})  +(z/y)^{4}((x/y)^{-2}+2(x/y)^{-1}+1) $ 

k10c69: $ (-(x/y)^{-4}+2(x/y)^{-3}-2(x/y)^{-2}+2(x/y)^{-1})  +(z/y)^{2}(3(x/y)^{-4}-5(x/y)^{-3}+5(x/y)^{-2}-(x/y)^{-1})  +(z/y)^{4}(-3(x/y)^{-4}+3(x/y)^{-3}-(x/y)^{-2})  +(x/y)^{-4}(z/y)^{6} $ 

k10c70: $ ((x/y)^{-3}-2(x/y)^{-2}+3(x/y)^{-1}-3+2(x/y))  +(z/y)^{2}((x/y)^{-4}-4(x/y)^{-3}+4(x/y)^{-2}-5(x/y)^{-1}+1)  +(z/y)^{4}(-2(x/y)^{-4}+3(x/y)^{-3}-2(x/y)^{-2})  +(x/y)^{-4}(z/y)^{6} $ 

k10c71: $ (-(x/y)^{-2}+3(x/y)^{-1}-3+3(x/y)-(x/y)^{2})  +(z/y)^{2}(-(x/y)^{-3}+4(x/y)^{-2}-5(x/y)^{-1}+4-(x/y))  +(z/y)^{4}(2(x/y)^{-3}-3(x/y)^{-2}+2(x/y)^{-1})  -(x/y)^{-3}(z/y)^{6} $ 

k10c72: $ (-(x/y)^{-4}+2(x/y)^{-3}-2(x/y)^{-2}+2(x/y)^{-1})  +(z/y)^{2}((x/y)^{-5}+(x/y)^{-4}-3(x/y)^{-3}+3(x/y)^{-2})  +(z/y)^{4}((x/y)^{-6}-2(x/y)^{-5}-3(x/y)^{-4}+(x/y)^{-3})  +(z/y)^{6}(-(x/y)^{-6}-(x/y)^{-5}) $ 

k10c73: $ (3(x/y)-4(x/y)^{2}+3(x/y)^{3}-(x/y)^{4})  +(z/y)^{2}(-(x/y)^{-1}+5-6(x/y)+3(x/y)^{2})  +(z/y)^{4}(-(x/y)^{-2}+3(x/y)^{-1}-3)  +(x/y)^{-2}(z/y)^{6} $ 

k10c74: $ (2(x/y)-2(x/y)^{3}+(x/y)^{4})  +(z/y)^{2}((x/y)^{-1}+1-2(x/y)-(x/y)^{2}+(x/y)^{3})  +(z/y)^{4}(-(x/y)^{-1}-2-(x/y)) $ 

k10c75: $ ((x/y)^{-3}-3(x/y)^{-2}+3(x/y)^{-1})  +(z/y)^{2}(-3(x/y)^{-3}+6(x/y)^{-2}-4(x/y)^{-1}+1)  +(z/y)^{4}(3(x/y)^{-3}-3(x/y)^{-2}+(x/y)^{-1})  -(x/y)^{-3}(z/y)^{6} $ 

k10c76: $ ((x/y)^{-4}-4(x/y)^{-2}+4(x/y)^{-1})  +(z/y)^{2}(2(x/y)^{-5}-2(x/y)^{-4}-6(x/y)^{-3}+4(x/y)^{-2})  +(z/y)^{4}((x/y)^{-6}-3(x/y)^{-5}-4(x/y)^{-4}+(x/y)^{-3})  +(z/y)^{6}(-(x/y)^{-6}-(x/y)^{-5}) $ 

k10c77: $ (-(x/y)^{-3}-(x/y)^{-2}+5(x/y)^{-1}-2)  +(z/y)^{2}(-2(x/y)^{-4}+2(x/y)^{-3}+7(x/y)^{-2}-3(x/y)^{-1})  +(z/y)^{4}(-(x/y)^{-5}+3(x/y)^{-4}+4(x/y)^{-3}-(x/y)^{-2})  +(z/y)^{6}((x/y)^{-5}+(x/y)^{-4}) $ 

k10c78: $ ((x/y)-(x/y)^{2}+4(x/y)^{3}-4(x/y)^{4}+(x/y)^{5})  +(z/y)^{2}(2-3(x/y)+7(x/y)^{2}-3(x/y)^{3})  +(z/y)^{4}((x/y)^{-1}-3+3(x/y))  -(x/y)^{-1}(z/y)^{6} $ 

k10c79: $ (-5(x/y)^{-1}+11-5(x/y))  +(z/y)^{2}(-9(x/y)^{-2}+23(x/y)^{-1}-9)  +(z/y)^{4}(-5(x/y)^{-3}+19(x/y)^{-2}-5(x/y)^{-1})  +(z/y)^{6}(-(x/y)^{-4}+7(x/y)^{-3}-(x/y)^{-2})  +(x/y)^{-4}(z/y)^{8} $ 

k10c80: $ (2(x/y)^{3}+3(x/y)^{4}-6(x/y)^{5}+2(x/y)^{6})  +(z/y)^{2}(5(x/y)^{2}+9(x/y)^{3}-9(x/y)^{4}+(x/y)^{5})  +(z/y)^{4}(4(x/y)+8(x/y)^{2}-3(x/y)^{3})  +(z/y)^{6}(1+2(x/y)) $ 

k10c81: $ (-(x/y)^{-2}+(x/y)^{-1}+1+(x/y)-(x/y)^{2})  +(z/y)^{2}(-(x/y)^{-3}+3(x/y)^{-2}-(x/y)^{-1}+3-(x/y))  +(z/y)^{4}(2(x/y)^{-3}-2(x/y)^{-2}+2(x/y)^{-1})  -(x/y)^{-3}(z/y)^{6} $ 

k10c82: $ 1  +(z/y)^{2}(4(x/y)^{-1}-8+4(x/y))  +(z/y)^{4}(4(x/y)^{-2}-12(x/y)^{-1}+4)  +(z/y)^{6}((x/y)^{-3}-6(x/y)^{-2}+(x/y)^{-1})  -(x/y)^{-3}(z/y)^{8} $ 

k10c83: $ ((x/y)^{-2}-2(x/y)^{-1}+2)  +(z/y)^{2}(2(x/y)^{-3}-4(x/y)^{-2}+1)  +(z/y)^{4}((x/y)^{-4}-3(x/y)^{-3}-2(x/y)^{-2}+(x/y)^{-1})  +(z/y)^{6}(-(x/y)^{-4}-(x/y)^{-3}) $ 

k10c84: $ (-2(x/y)^{-2}+4(x/y)^{-1}-1)  +(z/y)^{2}(-(x/y)^{-4}+5(x/y)^{-2}-2(x/y)^{-1})  +(z/y)^{4}(-(x/y)^{-5}+2(x/y)^{-4}+3(x/y)^{-3}-(x/y)^{-2})  +(z/y)^{6}((x/y)^{-5}+(x/y)^{-4}) $ 

k10c85: $ ((x/y)+(x/y)^{2}-(x/y)^{3})  +(z/y)^{2}(-3+9(x/y)-4(x/y)^{2})  +(z/y)^{4}(-4(x/y)^{-1}+12-4(x/y))  +(z/y)^{6}(-(x/y)^{-2}+6(x/y)^{-1}-1)  +(x/y)^{-2}(z/y)^{8} $ 

k10c86: $ (-(x/y)^{-3}+2(x/y)^{-2}-(x/y)^{-1}+1)  +(z/y)^{2}(-2(x/y)^{-4}+4(x/y)^{-3}-(x/y)^{-1})  +(z/y)^{4}(-(x/y)^{-5}+3(x/y)^{-4}+2(x/y)^{-3}-(x/y)^{-2})  +(z/y)^{6}((x/y)^{-5}+(x/y)^{-4}) $ 

k10c87: $ (-(x/y)^{-2}+3(x/y)^{-1}-2+(x/y))  +(z/y)^{2}((x/y)^{-3}+(x/y)^{-2}-4(x/y)^{-1}+2)  +(z/y)^{4}((x/y)^{-4}-2(x/y)^{-3}-3(x/y)^{-2}+(x/y)^{-1})  +(z/y)^{6}(-(x/y)^{-4}-(x/y)^{-3}) $ 

k10c88: $ ((x/y)^{-1}-1+(x/y))  +(z/y)^{2}(-(x/y)^{-3}+2(x/y)^{-2}-3(x/y)^{-1}+2-(x/y))  +(z/y)^{4}(2(x/y)^{-3}-2(x/y)^{-2}+2(x/y)^{-1})  -(x/y)^{-3}(z/y)^{6} $ 

k10c89: $ (1-(x/y)^{2}+2(x/y)^{3}-(x/y)^{4})  +(z/y)^{2}(2-4(x/y)+3(x/y)^{2})  +(z/y)^{4}(-(x/y)^{-2}+2(x/y)^{-1}-3)  +(x/y)^{-2}(z/y)^{6} $ 

k10c90: $ ((x/y)^{-2}-2+2(x/y))  +(z/y)^{2}(2(x/y)^{-3}-3(x/y)^{-2}-4(x/y)^{-1}+2)  +(z/y)^{4}((x/y)^{-4}-3(x/y)^{-3}-3(x/y)^{-2}+(x/y)^{-1})  +(z/y)^{6}(-(x/y)^{-4}-(x/y)^{-3}) $ 

k10c91: $ (-2(x/y)^{-1}+5-2(x/y))  +(z/y)^{2}(-5(x/y)^{-2}+12(x/y)^{-1}-5)  +(z/y)^{4}(-4(x/y)^{-3}+13(x/y)^{-2}-4(x/y)^{-1})  +(z/y)^{6}(-(x/y)^{-4}+6(x/y)^{-3}-(x/y)^{-2})  +(x/y)^{-4}(z/y)^{8} $ 

k10c92: $ (-(x/y)^{-3}+(x/y)^{-2}+(x/y)^{-1})  +(z/y)^{2}((x/y)^{-5}-(x/y)^{-4}+2(x/y)^{-2})  +(z/y)^{4}((x/y)^{-6}-2(x/y)^{-5}-2(x/y)^{-4}+(x/y)^{-3})  +(z/y)^{6}(-(x/y)^{-6}-(x/y)^{-5}) $ 

k10c93: $ (2(x/y)-(x/y)^{2})  +(z/y)^{2}(-2(x/y)^{-2}+2(x/y)^{-1}+3-2(x/y))  +(z/y)^{4}(-(x/y)^{-3}+3(x/y)^{-2}+3(x/y)^{-1}-1)  +(z/y)^{6}((x/y)^{-3}+(x/y)^{-2}) $ 

k10c94: $ (2(x/y)^{-2}-4(x/y)^{-1}+3)  +(z/y)^{2}(5(x/y)^{-3}-12(x/y)^{-2}+5(x/y)^{-1})  +(z/y)^{4}(4(x/y)^{-4}-13(x/y)^{-3}+4(x/y)^{-2})  +(z/y)^{6}((x/y)^{-5}-6(x/y)^{-4}+(x/y)^{-3})  -(x/y)^{-5}(z/y)^{8} $ 

k10c95: $ (-2(x/y)^{-3}+3(x/y)^{-2})  +(z/y)^{2}(-2(x/y)^{-4}+5(x/y)^{-3}+(x/y)^{-2}-(x/y)^{-1})  +(z/y)^{4}(-(x/y)^{-5}+3(x/y)^{-4}+2(x/y)^{-3}-(x/y)^{-2})  +(z/y)^{6}((x/y)^{-5}+(x/y)^{-4}) $ 

k10c96: $ ((x/y)^{-3}-2(x/y)^{-2}+3(x/y)^{-1}-3+2(x/y))  +(z/y)^{2}(-3(x/y)^{-3}+5(x/y)^{-2}-6(x/y)^{-1}+1)  +(z/y)^{4}(3(x/y)^{-3}-3(x/y)^{-2}+(x/y)^{-1})  -(x/y)^{-3}(z/y)^{6} $ 

k10c97: $ (-(x/y)^{-4}+2(x/y)^{-3}-2(x/y)^{-2}+2(x/y)^{-1})  +(z/y)^{2}((x/y)^{-5}+2(x/y)^{-4}-4(x/y)^{-3}+2(x/y)^{-2}+(x/y)^{-1})  +(z/y)^{4}(-(x/y)^{-5}-3(x/y)^{-4}-(x/y)^{-3}) $ 

k10c98: $ ((x/y)+3(x/y)^{2}-5(x/y)^{3}+2(x/y)^{4})  +(z/y)^{2}(2+(x/y)-5(x/y)^{2}+2(x/y)^{3})  +(z/y)^{4}((x/y)^{-1}-2-3(x/y)+(x/y)^{2})  +(z/y)^{6}(-(x/y)^{-1}-1) $ 

k10c99: $ (-4(x/y)^{-1}+9-4(x/y))  +(z/y)^{2}(-6(x/y)^{-2}+16(x/y)^{-1}-6)  +(z/y)^{4}(-4(x/y)^{-3}+14(x/y)^{-2}-4(x/y)^{-1})  +(z/y)^{6}(-(x/y)^{-4}+6(x/y)^{-3}-(x/y)^{-2})  +(x/y)^{-4}(z/y)^{8} $ 

k10c100: $ (-(x/y)+5(x/y)^{2}-3(x/y)^{3})  +(z/y)^{2}(-4+13(x/y)-5(x/y)^{2})  +(z/y)^{4}(-4(x/y)^{-1}+13-4(x/y))  +(z/y)^{6}(-(x/y)^{-2}+6(x/y)^{-1}-1)  +(x/y)^{-2}(z/y)^{8} $ 

k10c101: $ ((x/y)^{-6}-4(x/y)^{-5}+2(x/y)^{-4}+2(x/y)^{-3})  +(z/y)^{2}(-4(x/y)^{-6}+5(x/y)^{-5}+5(x/y)^{-4}+(x/y)^{-3})  +(z/y)^{4}(3(x/y)^{-6}+3(x/y)^{-5}+(x/y)^{-4}) $ 

k10c102: $ ((x/y)^{-2}-(x/y)^{-1}+(x/y))  +(z/y)^{2}(2(x/y)^{-3}-3(x/y)^{-2}-3(x/y)^{-1}+2)  +(z/y)^{4}((x/y)^{-4}-3(x/y)^{-3}-3(x/y)^{-2}+(x/y)^{-1})  +(z/y)^{6}(-(x/y)^{-4}-(x/y)^{-3}) $ 

k10c103: $ (-1+3(x/y)-(x/y)^{3})  +(z/y)^{2}(-2(x/y)^{-1}+4+3(x/y)-2(x/y)^{2})  +(z/y)^{4}(-(x/y)^{-2}+3(x/y)^{-1}+3-(x/y))  +(z/y)^{6}((x/y)^{-2}+(x/y)^{-1}) $ 

k10c104: $ (-(x/y)^{-1}+3-(x/y))  +(z/y)^{2}(-5(x/y)^{-2}+11(x/y)^{-1}-5)  +(z/y)^{4}(-4(x/y)^{-3}+13(x/y)^{-2}-4(x/y)^{-1})  +(z/y)^{6}(-(x/y)^{-4}+6(x/y)^{-3}-(x/y)^{-2})  +(x/y)^{-4}(z/y)^{8} $ 

k10c105: $ ((x/y)^{-1}-1+(x/y))  +(z/y)^{2}((x/y)^{-4}-2(x/y)^{-3}+2(x/y)^{-2}-3(x/y)^{-1}+1)  +(z/y)^{4}(-2(x/y)^{-4}+2(x/y)^{-3}-2(x/y)^{-2})  +(x/y)^{-4}(z/y)^{6} $ 

k10c106: $ ((x/y)^{-2}-2(x/y)^{-1}+2)  +(z/y)^{2}(5(x/y)^{-3}-11(x/y)^{-2}+5(x/y)^{-1})  +(z/y)^{4}(4(x/y)^{-4}-13(x/y)^{-3}+4(x/y)^{-2})  +(z/y)^{6}((x/y)^{-5}-6(x/y)^{-4}+(x/y)^{-3})  -(x/y)^{-5}(z/y)^{8} $ 

k10c107: $ (-(x/y)^{-2}+2(x/y)^{-1})  +(z/y)^{2}(-(x/y)^{-3}+3(x/y)^{-2}-2(x/y)^{-1}+2-(x/y))  +(z/y)^{4}(2(x/y)^{-3}-2(x/y)^{-2}+2(x/y)^{-1})  -(x/y)^{-3}(z/y)^{6} $ 

k10c108: $ 1  +(z/y)^{2}(-2(x/y)^{-3}+2(x/y)^{-2}+2(x/y)^{-1}-2)  +(z/y)^{4}(-(x/y)^{-4}+3(x/y)^{-3}+3(x/y)^{-2}-(x/y)^{-1})  +(z/y)^{6}((x/y)^{-4}+(x/y)^{-3}) $ 

k10c109: $ (-3(x/y)^{-1}+7-3(x/y))  +(z/y)^{2}(-6(x/y)^{-2}+15(x/y)^{-1}-6)  +(z/y)^{4}(-4(x/y)^{-3}+14(x/y)^{-2}-4(x/y)^{-1})  +(z/y)^{6}(-(x/y)^{-4}+6(x/y)^{-3}-(x/y)^{-2})  +(x/y)^{-4}(z/y)^{8} $ 

k10c110: $ ((x/y)^{-1}-(x/y)^{2}+(x/y)^{3})  +(z/y)^{2}((x/y)^{-2}-3(x/y)^{-1}+1-3(x/y)+(x/y)^{2})  +(z/y)^{4}(-2(x/y)^{-2}+2(x/y)^{-1}-2)  +(x/y)^{-2}(z/y)^{6} $ 

k10c111: $ ((x/y)^{-4}-3(x/y)^{-3}+2(x/y)^{-2}+(x/y)^{-1})  +(z/y)^{2}(2(x/y)^{-5}-4(x/y)^{-4}+(x/y)^{-3}+2(x/y)^{-2})  +(z/y)^{4}((x/y)^{-6}-3(x/y)^{-5}-2(x/y)^{-4}+(x/y)^{-3})  +(z/y)^{6}(-(x/y)^{-6}-(x/y)^{-5}) $ 

k10c112: $ (-1+4(x/y)-2(x/y)^{2})  +(z/y)^{2}((x/y)^{-1}+(x/y))  +(z/y)^{4}(3(x/y)^{-2}-7(x/y)^{-1}+3)  +(z/y)^{6}((x/y)^{-3}-5(x/y)^{-2}+(x/y)^{-1})  -(x/y)^{-3}(z/y)^{8} $ 

k10c113: $ ((x/y)^{-3}-3(x/y)^{-2}+3(x/y)^{-1})  +(z/y)^{2}(-2(x/y)^{-3}+3(x/y)^{-2}-(x/y)^{-1})  +(z/y)^{4}(-(x/y)^{-5}+(x/y)^{-4}+2(x/y)^{-3}-(x/y)^{-2})  +(z/y)^{6}((x/y)^{-5}+(x/y)^{-4}) $ 

k10c114: $ (2(x/y)-(x/y)^{2})  +(z/y)^{2}((x/y)^{-2}-(x/y)^{-1}+(x/y))  +(z/y)^{4}((x/y)^{-3}-2(x/y)^{-2}-2(x/y)^{-1}+1)  +(z/y)^{6}(-(x/y)^{-3}-(x/y)^{-2}) $ 

k10c115: $ (-(x/y)^{-1}+3-(x/y))  +(z/y)^{2}(-(x/y)^{-3}+(x/y)^{-2}+(x/y)^{-1}+1-(x/y))  +(z/y)^{4}(2(x/y)^{-3}-(x/y)^{-2}+2(x/y)^{-1})  -(x/y)^{-3}(z/y)^{6} $ 

k10c116: $ 1  +(z/y)^{2}(2(x/y)^{-1}-4+2(x/y))  +(z/y)^{4}(3(x/y)^{-2}-8(x/y)^{-1}+3)  +(z/y)^{6}((x/y)^{-3}-5(x/y)^{-2}+(x/y)^{-1})  -(x/y)^{-3}(z/y)^{8} $ 

k10c117: $ (-(x/y)^{-3}+(x/y)^{-2}+(x/y)^{-1})  +(z/y)^{2}(-(x/y)^{-4}+2(x/y)^{-3}+2(x/y)^{-2}-(x/y)^{-1})  +(z/y)^{4}(-(x/y)^{-5}+2(x/y)^{-4}+2(x/y)^{-3}-(x/y)^{-2})  +(z/y)^{6}((x/y)^{-5}+(x/y)^{-4}) $ 

k10c118: $ 1  +(z/y)^{2}(-2(x/y)^{-2}+4(x/y)^{-1}-2)  +(z/y)^{4}(-3(x/y)^{-3}+8(x/y)^{-2}-3(x/y)^{-1})  +(z/y)^{6}(-(x/y)^{-4}+5(x/y)^{-3}-(x/y)^{-2})  +(x/y)^{-4}(z/y)^{8} $ 

k10c119: $ ((x/y)^{-1}-1+(x/y))  +(z/y)^{2}((x/y)^{-3}-(x/y)^{-2}-2(x/y)^{-1}+1)  +(z/y)^{4}((x/y)^{-4}-2(x/y)^{-3}-2(x/y)^{-2}+(x/y)^{-1})  +(z/y)^{6}(-(x/y)^{-4}-(x/y)^{-3}) $ 

k10c120: $ (3(x/y)^{3}-3(x/y)^{5}+(x/y)^{6})  +(z/y)^{2}(7(x/y)^{2}+3(x/y)^{3}-4(x/y)^{4})  +(z/y)^{4}(1+4(x/y)+3(x/y)^{2}) $ 

k10c121: $ (1-(x/y)+2(x/y)^{2}-(x/y)^{3})  +(z/y)^{2}(-1+3(x/y)-(x/y)^{2})  +(z/y)^{4}(-(x/y)^{-2}+(x/y)^{-1}+2-(x/y))  +(z/y)^{6}((x/y)^{-2}+(x/y)^{-1}) $ 

k10c122: $ (-2(x/y)^{-2}+4(x/y)^{-1}-1)  +(z/y)^{2}(3(x/y)^{-2}-2(x/y)^{-1}+1)  +(z/y)^{4}((x/y)^{-4}-(x/y)^{-3}-2(x/y)^{-2}+(x/y)^{-1})  +(z/y)^{6}(-(x/y)^{-4}-(x/y)^{-3}) $ 

k10c123: $ (2(x/y)^{-1}-3+2(x/y))  +(z/y)^{2}((x/y)^{-2}-4(x/y)^{-1}+1)  +(z/y)^{4}(-2(x/y)^{-3}+3(x/y)^{-2}-2(x/y)^{-1})  +(z/y)^{6}(-(x/y)^{-4}+4(x/y)^{-3}-(x/y)^{-2})  +(x/y)^{-4}(z/y)^{8} $ 

k10c124: $ (2(x/y)^{-6}-8(x/y)^{-5}+7(x/y)^{-4})  +(z/y)^{2}((x/y)^{-7}-14(x/y)^{-6}+21(x/y)^{-5})  +(z/y)^{4}(-7(x/y)^{-7}+21(x/y)^{-6})  +(z/y)^{6}(-(x/y)^{-8}+8(x/y)^{-7})  +(x/y)^{-8}(z/y)^{8} $ 

k10c125: $ (-3(x/y)^{-1}+7-3(x/y))  +(z/y)^{2}(-4(x/y)^{-2}+11(x/y)^{-1}-4)  +(z/y)^{4}(-(x/y)^{-3}+6(x/y)^{-2}-(x/y)^{-1})  +(x/y)^{-3}(z/y)^{6} $ 

k10c126: $ (-2(x/y)+7(x/y)^{2}-4(x/y)^{3})  +(z/y)^{2}(-3+12(x/y)-4(x/y)^{2})  +(z/y)^{4}(-(x/y)^{-1}+6-(x/y))  +(x/y)^{-1}(z/y)^{6} $ 

k10c127: $ (5(x/y)^{2}-6(x/y)^{3}+2(x/y)^{4})  +(z/y)^{2}(7(x/y)-9(x/y)^{2}+3(x/y)^{3})  +(z/y)^{4}(2-5(x/y)+(x/y)^{2})  -(z/y)^{6} $ 

k10c128: $ ((x/y)^{-6}-4(x/y)^{-5}+2(x/y)^{-4}+2(x/y)^{-3})  +(z/y)^{2}(-5(x/y)^{-6}+6(x/y)^{-5}+6(x/y)^{-4})  +(z/y)^{4}(-(x/y)^{-7}+5(x/y)^{-6}+5(x/y)^{-5})  +(z/y)^{6}((x/y)^{-7}+(x/y)^{-6}) $ 

k10c129: $ (-(x/y)^{-1}+2+(x/y)-(x/y)^{2})  +(z/y)^{2}(-(x/y)^{-2}+2(x/y)^{-1}+2-(x/y))  +(z/y)^{4}((x/y)^{-2}+(x/y)^{-1}) $ 

k10c130: $ (-1+2(x/y)+2(x/y)^{2}-2(x/y)^{3})  +(z/y)^{2}(-(x/y)^{-1}+3+3(x/y)-(x/y)^{2})  +(z/y)^{4}((x/y)^{-1}+1) $ 

k10c131: $ (2(x/y)-2(x/y)^{3}+(x/y)^{4})  +(z/y)^{2}(2-(x/y)-2(x/y)^{2}+(x/y)^{3})  +(z/y)^{4}(-1-(x/y)) $ 

k10c132: $ (3(x/y)^{2}-2(x/y)^{3})  +(z/y)^{2}(4(x/y)-(x/y)^{2})  +(z/y)^{4} $ 

k10c133: $ ((x/y)+2(x/y)^{2}-3(x/y)^{3}+(x/y)^{4})  +(z/y)^{2}(1+2(x/y)-3(x/y)^{2}+(x/y)^{3})  -(x/y)(z/y)^{4} $ 

k10c134: $ ((x/y)^{-6}-3(x/y)^{-5}+3(x/y)^{-3})  +(z/y)^{2}(-4(x/y)^{-6}+3(x/y)^{-5}+7(x/y)^{-4})  +(z/y)^{4}(-(x/y)^{-7}+4(x/y)^{-6}+5(x/y)^{-5})  +(z/y)^{6}((x/y)^{-7}+(x/y)^{-6}) $ 

k10c135: $ (-2(x/y)^{-1}+4-(x/y)^{2})  +(z/y)^{2}(-2(x/y)^{-2}+5(x/y)^{-1}+1-(x/y))  +(z/y)^{4}(2(x/y)^{-2}+(x/y)^{-1}) $ 

k10c136: $ (-(x/y)^{-2}+3(x/y)^{-1}-2+(x/y))  +(z/y)^{2}(2(x/y)^{-2}-3(x/y)^{-1}+1)  -(x/y)^{-2}(z/y)^{4} $ 

k10c137: $ ((x/y)^{-1}-1+2(x/y)-2(x/y)^{2}+(x/y)^{3})  +(z/y)^{2}(-2(x/y)^{-1}+2-2(x/y))  +(x/y)^{-1}(z/y)^{4} $ 

k10c138: $ ((x/y)^{-3}-2(x/y)^{-2}+3(x/y)^{-1}-3+2(x/y))  +(z/y)^{2}(-3(x/y)^{-3}+5(x/y)^{-2}-6(x/y)^{-1}+1)  +(z/y)^{4}(-(x/y)^{-4}+4(x/y)^{-3}-2(x/y)^{-2})  +(x/y)^{-4}(z/y)^{6} $ 

k10c139: $ ((x/y)^{-6}-6(x/y)^{-5}+6(x/y)^{-4})  +(z/y)^{2}((x/y)^{-7}-13(x/y)^{-6}+21(x/y)^{-5})  +(z/y)^{4}(-7(x/y)^{-7}+21(x/y)^{-6})  +(z/y)^{6}(-(x/y)^{-8}+8(x/y)^{-7})  +(x/y)^{-8}(z/y)^{8} $ 

k10c140: $ (1-2(x/y)+4(x/y)^{2}-2(x/y)^{3})  +(z/y)^{2}(-1+4(x/y)-(x/y)^{2})  +(z/y)^{4} $ 

k10c141: $ (2-2(x/y)+(x/y)^{2})  +(z/y)^{2}(3(x/y)^{-1}-7+3(x/y))  +(z/y)^{4}((x/y)^{-2}-5(x/y)^{-1}+1)  -(x/y)^{-2}(z/y)^{6} $ 

k10c142: $ ((x/y)^{-6}-5(x/y)^{-5}+4(x/y)^{-4}+(x/y)^{-3})  +(z/y)^{2}(-5(x/y)^{-6}+7(x/y)^{-5}+6(x/y)^{-4})  +(z/y)^{4}(-(x/y)^{-7}+5(x/y)^{-6}+5(x/y)^{-5})  +(z/y)^{6}((x/y)^{-7}+(x/y)^{-6}) $ 

k10c143: $ (3(x/y)^{2}-2(x/y)^{3})  +(z/y)^{2}(-2+8(x/y)-3(x/y)^{2})  +(z/y)^{4}(-(x/y)^{-1}+5-(x/y))  +(x/y)^{-1}(z/y)^{6} $ 

k10c144: $ (3-4(x/y)+2(x/y)^{2})  +(z/y)^{2}(2(x/y)^{-1}-5+(x/y)^{2})  +(z/y)^{4}(-2(x/y)^{-1}-1) $ 

k10c145: $ (2(x/y)^{2}-(x/y)^{3}+(x/y)^{4}-(x/y)^{5})  +(z/y)^{2}(4(x/y)+(x/y)^{3})  +(z/y)^{4} $ 

k10c146: $ 1  +(z/y)^{2}(-(x/y)^{-2}+(x/y)^{-1}+1-(x/y))  +(z/y)^{4}((x/y)^{-2}+(x/y)^{-1}) $ 

k10c147: $ ((x/y)^{-1}-1+(x/y))  +(z/y)^{2}((x/y)^{-3}-(x/y)^{-2}-2(x/y)^{-1}+1)  +(z/y)^{4}(-(x/y)^{-3}-(x/y)^{-2}) $ 

k10c148: $ (-(x/y)+5(x/y)^{2}-3(x/y)^{3})  +(z/y)^{2}(-2+9(x/y)-3(x/y)^{2})  +(z/y)^{4}(-(x/y)^{-1}+5-(x/y))  +(x/y)^{-1}(z/y)^{6} $ 

k10c149: $ (4(x/y)^{2}-4(x/y)^{3}+(x/y)^{4})  +(z/y)^{2}(6(x/y)-6(x/y)^{2}+2(x/y)^{3})  +(z/y)^{4}(2-4(x/y)+(x/y)^{2})  -(z/y)^{6} $ 

k10c150: $ (-(x/y)^{-2}+2(x/y)^{-1})  +(z/y)^{2}(2(x/y)^{-4}-4(x/y)^{-3}+3(x/y)^{-2})  +(z/y)^{4}((x/y)^{-5}-4(x/y)^{-4}+(x/y)^{-3})  -(x/y)^{-5}(z/y)^{6} $ 

k10c151: $ (-(x/y)^{-3}+3(x/y)^{-1}-1)  +(z/y)^{2}(-(x/y)^{-3}+6(x/y)^{-2}-2(x/y)^{-1})  +(z/y)^{4}(-(x/y)^{-4}+4(x/y)^{-3}-(x/y)^{-2})  +(x/y)^{-4}(z/y)^{6} $ 

k10c152: $ (8(x/y)^{4}-10(x/y)^{5}+3(x/y)^{6})  +(z/y)^{2}(22(x/y)^{3}-17(x/y)^{4}+2(x/y)^{5})  +(z/y)^{4}(21(x/y)^{2}-8(x/y)^{3})  +(z/y)^{6}(8(x/y)-(x/y)^{2})  +(z/y)^{8} $ 

k10c153: $ (-3(x/y)^{-1}+6-(x/y)-(x/y)^{2})  +(z/y)^{2}(-4(x/y)^{-2}+10(x/y)^{-1}-1-(x/y))  +(z/y)^{4}(-(x/y)^{-3}+6(x/y)^{-2})  +(x/y)^{-3}(z/y)^{6} $ 

k10c154: $ ((x/y)^{-6}-2(x/y)^{-5}-2(x/y)^{-4}+4(x/y)^{-3})  +(z/y)^{2}(-2(x/y)^{-6}-2(x/y)^{-5}+9(x/y)^{-4})  +6(x/y)^{-5}(z/y)^{4}  +(x/y)^{-6}(z/y)^{6} $ 

k10c155: $ (2(x/y)^{-2}-4(x/y)^{-1}+3)  +(z/y)^{2}(3(x/y)^{-3}-8(x/y)^{-2}+3(x/y)^{-1})  +(z/y)^{4}((x/y)^{-4}-5(x/y)^{-3}+(x/y)^{-2})  -(x/y)^{-4}(z/y)^{6} $ 

k10c156: $ (2(x/y)-(x/y)^{2})  +(z/y)^{2}(-2(x/y)^{-1}+5-2(x/y))  +(z/y)^{4}(-(x/y)^{-2}+4(x/y)^{-1}-1)  +(x/y)^{-2}(z/y)^{6} $ 

k10c157: $ (-(x/y)^{-4}+2(x/y)^{-2})  +(z/y)^{2}((x/y)^{-5}-2(x/y)^{-4}+5(x/y)^{-3})  +(z/y)^{4}((x/y)^{-6}-3(x/y)^{-5}+2(x/y)^{-4})  -(x/y)^{-6}(z/y)^{6} $ 

k10c158: $ ((x/y)^{-2}-2+2(x/y))  +(z/y)^{2}((x/y)^{-2}-6(x/y)^{-1}+2)  +(z/y)^{4}((x/y)^{-3}-4(x/y)^{-2}+(x/y)^{-1})  -(x/y)^{-3}(z/y)^{6} $ 

k10c159: $ ((x/y)+(x/y)^{2}-(x/y)^{3})  +(z/y)^{2}(-1+5(x/y)-2(x/y)^{2})  +(z/y)^{4}(-(x/y)^{-1}+4-(x/y))  +(x/y)^{-1}(z/y)^{6} $ 

k10c160: $ (-(x/y)^{-4}+(x/y)^{-3}+(x/y)^{-1})  +(z/y)^{2}(3(x/y)^{-4}-3(x/y)^{-3}+3(x/y)^{-2})  +(z/y)^{4}((x/y)^{-5}-4(x/y)^{-4}+(x/y)^{-3})  -(x/y)^{-5}(z/y)^{6} $ 

k10c161: $ (3(x/y)^{3}-(x/y)^{4}-(x/y)^{5})  +(z/y)^{2}(9(x/y)^{2}-(x/y)^{3}-(x/y)^{4})  +6(x/y)(z/y)^{4}  +(z/y)^{6} $ 

k10c162: $ (-(x/y)^{-5}-(x/y)^{-4}+3(x/y)^{-3})  +(z/y)^{2}(-(x/y)^{-6}-(x/y)^{-5}+9(x/y)^{-4})  +6(x/y)^{-5}(z/y)^{4}  +(x/y)^{-6}(z/y)^{6} $ 

k10c163: $ (3-3(x/y)+(x/y)^{3})  +(z/y)^{2}(2(x/y)^{-1}-5-(x/y)+(x/y)^{2})  +(z/y)^{4}(-2(x/y)^{-1}-1) $ 

k10c164: $ (-(x/y)^{-3}+2(x/y)^{-2}-(x/y)^{-1}+1)  +(z/y)^{2}(2(x/y)^{-2}-(x/y)^{-1})  +(z/y)^{4}(-(x/y)^{-4}+3(x/y)^{-3}-(x/y)^{-2})  +(x/y)^{-4}(z/y)^{6} $ 

k10c165: $ (-(x/y)^{-1}+3-(x/y))  +(z/y)^{2}(-2(x/y)^{-2}+4(x/y)^{-1}-(x/y))  +(z/y)^{4}(2(x/y)^{-2}+(x/y)^{-1}) $ 

k10c166: $ (-(x/y)^{-3}+(x/y)^{-2}+(x/y)^{-1})  +(z/y)^{2}((x/y)^{-5}-(x/y)^{-4}+2(x/y)^{-2})  +(z/y)^{4}(-(x/y)^{-5}-(x/y)^{-4}) $ 


%\newpage


\section{F-polynomial}
 $f_{L+} + f_{L-} = x (f_{L0} + f_{L\infty})$,
  $F_L = a^{-w(L)} f_L$ \par
k3c1: $ (-2a^{2}-a^{4}) +x(a^{3}+a^{5}) +x^{2}(a^{2}+a^{4}) $

k4c1: $ (-a^{-2}-1-a^{2}) +x(-a^{-1}-a) +x^{2}(a^{-2}+2+a^{2}) +x^{3}(a^{-1}+a) $

k5c1: $ (3a^{4}+2a^{6}) +x(-2a^{5}-a^{7}+a^{9}) +x^{2}(-4a^{4}-3a^{6}+a^{8}) +x^{3}(a^{5}+a^{7}) +x^{4}(a^{4}+a^{6}) $

k5c2: $ (-a^{2}+a^{4}+a^{6}) +x(-2a^{5}-2a^{7}) +x^{2}(a^{2}-a^{4}-2a^{6}) +x^{3}(a^{3}+2a^{5}+a^{7}) +x^{4}(a^{4}+a^{6}) $

k6c1: $ (-a^{-2}+a^{2}+a^{4}) +x(2a+2a^{3}) +x^{2}(a^{-2}-4a^{2}-3a^{4}) +x^{3}(a^{-1}-2a-3a^{3}) +x^{4}(1+2a^{2}+a^{4}) +x^{5}(a+a^{3}) $

k6c2: $ (2+2a^{2}+a^{4}) +x(-a^{3}-a^{5}) +x^{2}(-3-6a^{2}-2a^{4}+a^{6}) +x^{3}(-2a+2a^{5}) +x^{4}(1+3a^{2}+2a^{4}) +x^{5}(a+a^{3}) $

k6c3: $ (a^{-2}+3+a^{2}) +x(-a^{-3}-2a^{-1}-2a-a^{3}) +x^{2}(-3a^{-2}-6-3a^{2}) +x^{3}(a^{-3}+a^{-1}+a+a^{3}) +x^{4}(2a^{-2}+4+2a^{2}) +x^{5}(a^{-1}+a) $

k7c1: $ (-4a^{6}-3a^{8}) +x(3a^{7}+a^{9}-a^{11}+a^{13}) +x^{2}(10a^{6}+7a^{8}-2a^{10}+a^{12}) +x^{3}(-4a^{7}-3a^{9}+a^{11}) +x^{4}(-6a^{6}-5a^{8}+a^{10}) +x^{5}(a^{7}+a^{9}) +x^{6}(a^{6}+a^{8}) $

k7c2: $ (-a^{2}-a^{6}-a^{8}) +x(3a^{7}+3a^{9}) +x^{2}(a^{2}+3a^{6}+4a^{8}) +x^{3}(a^{3}-a^{5}-6a^{7}-4a^{9}) +x^{4}(a^{4}-3a^{6}-4a^{8}) +x^{5}(a^{5}+2a^{7}+a^{9}) +x^{6}(a^{6}+a^{8}) $

k7c3: $ (-2a^{-8}-2a^{-6}+a^{-4}) +x(-2a^{-11}+a^{-9}+3a^{-7}) +x^{2}(-a^{-10}+6a^{-8}+4a^{-6}-3a^{-4}) +x^{3}(a^{-11}-a^{-9}-4a^{-7}-2a^{-5}) +x^{4}(a^{-10}-3a^{-8}-3a^{-6}+a^{-4}) +x^{5}(a^{-9}+2a^{-7}+a^{-5}) +x^{6}(a^{-8}+a^{-6}) $

k7c4: $ (-a^{-8}+2a^{-4}) +x(4a^{-9}+4a^{-7}) +x^{2}(2a^{-8}-3a^{-6}-4a^{-4}+a^{-2}) +x^{3}(-4a^{-9}-8a^{-7}-2a^{-5}+2a^{-3}) +x^{4}(-3a^{-8}+3a^{-4}) +x^{5}(a^{-9}+3a^{-7}+2a^{-5}) +x^{6}(a^{-8}+a^{-6}) $

k7c5: $ (2a^{4}-a^{8}) +x(-a^{5}+a^{7}+a^{9}-a^{11}) +x^{2}(-3a^{4}+a^{8}-2a^{10}) +x^{3}(-a^{5}-4a^{7}-2a^{9}+a^{11}) +x^{4}(a^{4}-a^{6}+2a^{10}) +x^{5}(a^{5}+3a^{7}+2a^{9}) +x^{6}(a^{6}+a^{8}) $

k7c6: $ (1+a^{2}+2a^{4}+a^{6}) +x(a+2a^{3}-a^{7}) +x^{2}(-2-4a^{2}-4a^{4}-2a^{6}) +x^{3}(-4a-6a^{3}-a^{5}+a^{7}) +x^{4}(1+a^{2}+2a^{4}+2a^{6}) +x^{5}(2a+4a^{3}+2a^{5}) +x^{6}(a^{2}+a^{4}) $

k7c7: $ (a^{-4}+2a^{-2}+2) +x(2a^{-3}+3a^{-1}+a) +x^{2}(-2a^{-4}-6a^{-2}-7-3a^{2}) +x^{3}(-4a^{-3}-8a^{-1}-3a+a^{3}) +x^{4}(a^{-4}+2a^{-2}+4+3a^{2}) +x^{5}(2a^{-3}+5a^{-1}+3a) +x^{6}(a^{-2}+1) $

k8c1: $ (-a^{-2}-a^{4}-a^{6}) +x(-3a^{3}-3a^{5}) +x^{2}(a^{-2}+7a^{4}+6a^{6}) +x^{3}(a^{-1}-a+5a^{3}+7a^{5}) +x^{4}(1-2a^{2}-8a^{4}-5a^{6}) +x^{5}(a-4a^{3}-5a^{5}) +x^{6}(a^{2}+2a^{4}+a^{6}) +x^{7}(a^{3}+a^{5}) $

k8c2: $ (-3a^{2}-3a^{4}-a^{6}) +x(a^{3}+a^{5}-a^{7}-a^{9}) +x^{2}(7a^{2}+12a^{4}+3a^{6}-a^{8}+a^{10}) +x^{3}(3a^{3}-a^{5}-2a^{7}+2a^{9}) +x^{4}(-5a^{2}-12a^{4}-5a^{6}+2a^{8}) +x^{5}(-4a^{3}-2a^{5}+2a^{7}) +x^{6}(a^{2}+3a^{4}+2a^{6}) +x^{7}(a^{3}+a^{5}) $

k8c3: $ (a^{-4}-1+a^{4}) +x(-4a^{-1}-4a) +x^{2}(-3a^{-4}+a^{-2}+8+a^{2}-3a^{4}) +x^{3}(-2a^{-3}+8a^{-1}+8a-2a^{3}) +x^{4}(a^{-4}-2a^{-2}-6-2a^{2}+a^{4}) +x^{5}(a^{-3}-4a^{-1}-4a+a^{3}) +x^{6}(a^{-2}+2+a^{2}) +x^{7}(a^{-1}+a) $

k8c4: $ (-2a^{-2}-2+a^{4}) +x(-a^{-1}+a+2a^{3}) +x^{2}(7a^{-2}+10-a^{2}-3a^{4}+a^{6}) +x^{3}(4a^{-1}-3a-5a^{3}+2a^{5}) +x^{4}(-5a^{-2}-11-3a^{2}+3a^{4}) +x^{5}(-4a^{-1}-a+3a^{3}) +x^{6}(a^{-2}+3+2a^{2}) +x^{7}(a^{-1}+a) $

k8c5: $ (-2a^{-6}-5a^{-4}-4a^{-2}) +x(4a^{-7}+7a^{-5}+3a^{-3}) +x^{2}(a^{-10}-2a^{-8}+4a^{-6}+15a^{-4}+8a^{-2}) +x^{3}(2a^{-9}-8a^{-7}-10a^{-5}) +x^{4}(3a^{-8}-7a^{-6}-15a^{-4}-5a^{-2}) +x^{5}(4a^{-7}+a^{-5}-3a^{-3}) +x^{6}(3a^{-6}+4a^{-4}+a^{-2}) +x^{7}(a^{-5}+a^{-3}) $

k8c6: $ (2+a^{2}-a^{4}-a^{6}) +x(-a-3a^{3}-a^{5}+a^{7}) +x^{2}(-3-2a^{2}+6a^{4}+3a^{6}-2a^{8}) +x^{3}(-a+5a^{3}+2a^{5}-4a^{7}) +x^{4}(1-6a^{4}-4a^{6}+a^{8}) +x^{5}(a-2a^{3}-a^{5}+2a^{7}) +x^{6}(a^{2}+3a^{4}+2a^{6}) +x^{7}(a^{3}+a^{5}) $

k8c7: $ (-2a^{-4}-4a^{-2}-1) +x(-a^{-7}+2a^{-3}+2a^{-1}+a) +x^{2}(-2a^{-6}+4a^{-4}+12a^{-2}+6) +x^{3}(a^{-7}-a^{-5}-2a^{-3}-3a^{-1}-3a) +x^{4}(2a^{-6}-3a^{-4}-12a^{-2}-7) +x^{5}(2a^{-5}-a^{-1}+a) +x^{6}(2a^{-4}+4a^{-2}+2) +x^{7}(a^{-3}+a^{-1}) $

k8c8: $ (-a^{-4}-a^{-2}+2+a^{2}) +x(2a^{-5}+3a^{-3}+a^{-1}-a-a^{3}) +x^{2}(4a^{-4}+5a^{-2}-1-2a^{2}) +x^{3}(-3a^{-5}-5a^{-3}-3a^{-1}+a^{3}) +x^{4}(-6a^{-4}-9a^{-2}-1+2a^{2}) +x^{5}(a^{-5}+a^{-1}+2a) +x^{6}(2a^{-4}+4a^{-2}+2) +x^{7}(a^{-3}+a^{-1}) $

k8c9: $ (-2a^{-2}-3-2a^{2}) +x(a^{-3}+a^{-1}+a+a^{3}) +x^{2}(-2a^{-4}+4a^{-2}+12+4a^{2}-2a^{4}) +x^{3}(-4a^{-3}-a^{-1}-a-4a^{3}) +x^{4}(a^{-4}-4a^{-2}-10-4a^{2}+a^{4}) +x^{5}(2a^{-3}+2a^{3}) +x^{6}(2a^{-2}+4+2a^{2}) +x^{7}(a^{-1}+a) $

k8c10: $ (-3a^{-4}-6a^{-2}-2) +x(-a^{-7}+2a^{-5}+6a^{-3}+5a^{-1}+2a) +x^{2}(-a^{-6}+6a^{-4}+12a^{-2}+5) +x^{3}(a^{-7}-3a^{-5}-9a^{-3}-8a^{-1}-3a) +x^{4}(2a^{-6}-5a^{-4}-13a^{-2}-6) +x^{5}(3a^{-5}+3a^{-3}+a^{-1}+a) +x^{6}(3a^{-4}+5a^{-2}+2) +x^{7}(a^{-3}+a^{-1}) $

k8c11: $ (1-a^{2}-2a^{4}-a^{6}) +x(a^{3}+3a^{5}+2a^{7}) +x^{2}(-2+6a^{4}+2a^{6}-2a^{8}) +x^{3}(-3a-2a^{3}-3a^{5}-4a^{7}) +x^{4}(1-2a^{2}-7a^{4}-3a^{6}+a^{8}) +x^{5}(2a+a^{3}+a^{5}+2a^{7}) +x^{6}(2a^{2}+4a^{4}+2a^{6}) +x^{7}(a^{3}+a^{5}) $

k8c12: $ (a^{-4}+a^{-2}+1+a^{2}+a^{4}) +x(a^{-3}+a^{3}) +x^{2}(-2a^{-4}-2a^{-2}-2a^{2}-2a^{4}) +x^{3}(-3a^{-3}-3a^{-1}-3a-3a^{3}) +x^{4}(a^{-4}-a^{-2}-4-a^{2}+a^{4}) +x^{5}(2a^{-3}+2a^{-1}+2a+2a^{3}) +x^{6}(2a^{-2}+4+2a^{2}) +x^{7}(a^{-1}+a) $

k8c13: $ (-a^{-4}-2a^{-2}) +x(2a^{-5}+4a^{-3}+3a^{-1}+a) +x^{2}(5a^{-4}+7a^{-2}-2a^{2}) +x^{3}(-3a^{-5}-7a^{-3}-9a^{-1}-4a+a^{3}) +x^{4}(-6a^{-4}-11a^{-2}-2+3a^{2}) +x^{5}(a^{-5}+a^{-3}+4a^{-1}+4a) +x^{6}(2a^{-4}+5a^{-2}+3) +x^{7}(a^{-3}+a^{-1}) $

k8c14: $ 1 +x(a+3a^{3}+3a^{5}+a^{7}) +x^{2}(-2-a^{2}+3a^{4}+a^{6}-a^{8}) +x^{3}(-3a-6a^{3}-8a^{5}-5a^{7}) +x^{4}(1-a^{2}-7a^{4}-4a^{6}+a^{8}) +x^{5}(2a+3a^{3}+4a^{5}+3a^{7}) +x^{6}(2a^{2}+5a^{4}+3a^{6}) +x^{7}(a^{3}+a^{5}) $

k8c15: $ (a^{4}-3a^{6}-4a^{8}-a^{10}) +x(6a^{7}+8a^{9}+2a^{11}) +x^{2}(-2a^{4}+5a^{6}+8a^{8}-a^{12}) +x^{3}(-2a^{5}-11a^{7}-14a^{9}-5a^{11}) +x^{4}(a^{4}-5a^{6}-10a^{8}-3a^{10}+a^{12}) +x^{5}(2a^{5}+5a^{7}+6a^{9}+3a^{11}) +x^{6}(3a^{6}+6a^{8}+3a^{10}) +x^{7}(a^{7}+a^{9}) $

k8c16: $ (-2a^{2}-a^{4}) +x(a^{-1}+3a+4a^{3}+2a^{5}) +x^{2}(5+10a^{2}+4a^{4}-a^{6}) +x^{3}(-2a^{-1}-6a-10a^{3}-5a^{5}+a^{7}) +x^{4}(-8-18a^{2}-7a^{4}+3a^{6}) +x^{5}(a^{-1}-a+3a^{3}+5a^{5}) +x^{6}(3+8a^{2}+5a^{4}) +x^{7}(2a+2a^{3}) $

k8c17: $ (-a^{-2}-1-a^{2}) +x(a^{-3}+2a^{-1}+2a+a^{3}) +x^{2}(-a^{-4}+3a^{-2}+8+3a^{2}-a^{4}) +x^{3}(-4a^{-3}-6a^{-1}-6a-4a^{3}) +x^{4}(a^{-4}-6a^{-2}-14-6a^{2}+a^{4}) +x^{5}(3a^{-3}+2a^{-1}+2a+3a^{3}) +x^{6}(4a^{-2}+8+4a^{2}) +x^{7}(2a^{-1}+2a) $

k8c18: $ (a^{-2}+3+a^{2}) +x(a^{-1}+a) +x^{2}(3a^{-2}+6+3a^{2}) +x^{3}(-4a^{-3}-9a^{-1}-9a-4a^{3}) +x^{4}(a^{-4}-9a^{-2}-20-9a^{2}+a^{4}) +x^{5}(4a^{-3}+3a^{-1}+3a+4a^{3}) +x^{6}(6a^{-2}+12+6a^{2}) +x^{7}(3a^{-1}+3a) $

k8c19: $ (-a^{-10}-5a^{-8}-5a^{-6}) +x(5a^{-9}+5a^{-7}) +x^{2}(10a^{-8}+10a^{-6}) +x^{3}(-5a^{-9}-5a^{-7}) +x^{4}(-6a^{-8}-6a^{-6}) +x^{5}(a^{-9}+a^{-7}) +x^{6}(a^{-8}+a^{-6}) $

k8c20: $ (-1-4a^{2}-2a^{4}) +x(a^{-1}+3a+5a^{3}+3a^{5}) +x^{2}(2+6a^{2}+4a^{4}) +x^{3}(-3a-7a^{3}-4a^{5}) +x^{4}(-4a^{2}-4a^{4}) +x^{5}(a+2a^{3}+a^{5}) +x^{6}(a^{2}+a^{4}) $

k8c21: $ (-3a^{2}-3a^{4}-a^{6}) +x(2a^{3}+4a^{5}+2a^{7}) +x^{2}(3a^{2}+5a^{4}-2a^{8}) +x^{3}(-a^{3}-6a^{5}-5a^{7}) +x^{4}(-2a^{4}-a^{6}+a^{8}) +x^{5}(a^{3}+3a^{5}+2a^{7}) +x^{6}(a^{4}+a^{6}) $

k9c1: $ (5a^{8}+4a^{10}) +x(-4a^{9}-a^{11}+a^{13}-a^{15}+a^{17}) +x^{2}(-20a^{8}-14a^{10}+3a^{12}-2a^{14}+a^{16}) +x^{3}(10a^{9}+6a^{11}-3a^{13}+a^{15}) +x^{4}(21a^{8}+16a^{10}-4a^{12}+a^{14}) +x^{5}(-6a^{9}-5a^{11}+a^{13}) +x^{6}(-8a^{8}-7a^{10}+a^{12}) +x^{7}(a^{9}+a^{11}) +x^{8}(a^{8}+a^{10}) $

k9c2: $ (-a^{2}+a^{8}+a^{10}) +x(-4a^{9}-4a^{11}) +x^{2}(a^{2}-6a^{8}-7a^{10}) +x^{3}(a^{3}-a^{5}+a^{7}+13a^{9}+10a^{11}) +x^{4}(a^{4}-2a^{6}+8a^{8}+11a^{10}) +x^{5}(a^{5}-3a^{7}-10a^{9}-6a^{11}) +x^{6}(a^{6}-5a^{8}-6a^{10}) +x^{7}(a^{7}+2a^{9}+a^{11}) +x^{8}(a^{8}+a^{10}) $

k9c3: $ (3a^{-10}+3a^{-8}-a^{-6}) +x(-2a^{-15}+a^{-13}-a^{-11}-4a^{-9}) +x^{2}(-a^{-14}+3a^{-12}-11a^{-10}-9a^{-8}+6a^{-6}) +x^{3}(a^{-15}-a^{-13}+4a^{-11}+9a^{-9}+3a^{-7}) +x^{4}(a^{-14}-2a^{-12}+11a^{-10}+9a^{-8}-5a^{-6}) +x^{5}(a^{-13}-3a^{-11}-8a^{-9}-4a^{-7}) +x^{6}(a^{-12}-5a^{-10}-5a^{-8}+a^{-6}) +x^{7}(a^{-11}+2a^{-9}+a^{-7}) +x^{8}(a^{-10}+a^{-8}) $

k9c4: $ (a^{4}+2a^{8}+2a^{10}) +x(-4a^{9}-a^{11}+3a^{13}) +x^{2}(-3a^{4}+a^{6}-7a^{8}-10a^{10}+a^{12}) +x^{3}(-2a^{5}+4a^{7}+12a^{9}+2a^{11}-4a^{13}) +x^{4}(a^{4}-2a^{6}+11a^{8}+11a^{10}-3a^{12}) +x^{5}(a^{5}-3a^{7}-8a^{9}-3a^{11}+a^{13}) +x^{6}(a^{6}-5a^{8}-5a^{10}+a^{12}) +x^{7}(a^{7}+2a^{9}+a^{11}) +x^{8}(a^{8}+a^{10}) $

k9c5: $ (a^{-10}-a^{-6}+a^{-4}) +x(-6a^{-11}-6a^{-9}) +x^{2}(-3a^{-10}+4a^{-8}+3a^{-6}-3a^{-4}+a^{-2}) +x^{3}(11a^{-11}+18a^{-9}+a^{-7}-4a^{-5}+2a^{-3}) +x^{4}(7a^{-10}-3a^{-8}-7a^{-6}+3a^{-4}) +x^{5}(-6a^{-11}-14a^{-9}-5a^{-7}+3a^{-5}) +x^{6}(-5a^{-10}-2a^{-8}+3a^{-6}) +x^{7}(a^{-11}+3a^{-9}+2a^{-7}) +x^{8}(a^{-10}+a^{-8}) $

k9c6: $ (-3a^{6}-a^{8}+a^{10}) +x(2a^{7}-a^{9}-2a^{11}-a^{15}) +x^{2}(7a^{6}+a^{8}-3a^{10}+a^{12}-2a^{14}) +x^{3}(8a^{9}+6a^{11}-a^{13}+a^{15}) +x^{4}(-5a^{6}+a^{8}+2a^{10}-2a^{12}+2a^{14}) +x^{5}(-3a^{7}-10a^{9}-5a^{11}+2a^{13}) +x^{6}(a^{6}-3a^{8}-2a^{10}+2a^{12}) +x^{7}(a^{7}+3a^{9}+2a^{11}) +x^{8}(a^{8}+a^{10}) $

k9c7: $ (2a^{4}+a^{6}+a^{8}+a^{10}) +x(-a^{5}-a^{7}-3a^{9}-2a^{11}+a^{13}) +x^{2}(-3a^{4}-2a^{6}-4a^{8}-2a^{10}+3a^{12}) +x^{3}(-a^{5}+2a^{7}+11a^{9}+5a^{11}-3a^{13}) +x^{4}(a^{4}+7a^{8}+2a^{10}-6a^{12}) +x^{5}(a^{5}-a^{7}-9a^{9}-6a^{11}+a^{13}) +x^{6}(a^{6}-3a^{8}-2a^{10}+2a^{12}) +x^{7}(a^{7}+3a^{9}+2a^{11}) +x^{8}(a^{8}+a^{10}) $

k9c8: $ (-a^{-2}-1+2a^{4}+a^{6}) +x(-2a^{-1}-3a-a^{3}-a^{5}-a^{7}) +x^{2}(4a^{-2}+7+2a^{2}-3a^{4}-2a^{6}) +x^{3}(8a^{-1}+11a+2a^{3}+a^{7}) +x^{4}(-4a^{-2}-6-4a^{2}+2a^{6}) +x^{5}(-8a^{-1}-13a-3a^{3}+2a^{5}) +x^{6}(a^{-2}-1+2a^{4}) +x^{7}(2a^{-1}+4a+2a^{3}) +x^{8}(1+a^{2}) $

k9c9: $ (-2a^{6}+a^{8}+2a^{10}) +x(a^{7}-2a^{9}+2a^{13}-a^{15}) +x^{2}(7a^{6}-3a^{8}-6a^{10}+3a^{12}-a^{14}) +x^{3}(a^{7}+5a^{9}-3a^{13}+a^{15}) +x^{4}(-5a^{6}+3a^{8}+2a^{10}-4a^{12}+2a^{14}) +x^{5}(-3a^{7}-8a^{9}-2a^{11}+3a^{13}) +x^{6}(a^{6}-3a^{8}-a^{10}+3a^{12}) +x^{7}(a^{7}+3a^{9}+2a^{11}) +x^{8}(a^{8}+a^{10}) $

k9c10: $ (2a^{-10}+a^{-8}-2a^{-6}) +x(4a^{-13}-4a^{-9}) +x^{2}(-11a^{-10}-2a^{-8}+7a^{-6}-2a^{-4}) +x^{3}(-4a^{-13}-a^{-11}+9a^{-9}+3a^{-7}-3a^{-5}) +x^{4}(-2a^{-12}+9a^{-10}+3a^{-8}-7a^{-6}+a^{-4}) +x^{5}(a^{-13}-a^{-11}-7a^{-9}-3a^{-7}+2a^{-5}) +x^{6}(a^{-12}-3a^{-10}-a^{-8}+3a^{-6}) +x^{7}(a^{-11}+3a^{-9}+2a^{-7}) +x^{8}(a^{-10}+a^{-8}) $

k9c11: $ (-2a^{-8}-3a^{-6}-a^{-4}-a^{-2}) +x(-a^{-11}+2a^{-9}+2a^{-7}-2a^{-5}-a^{-3}) +x^{2}(-a^{-10}+4a^{-8}+6a^{-6}+5a^{-4}+4a^{-2}) +x^{3}(a^{-11}-3a^{-9}-3a^{-7}+9a^{-5}+8a^{-3}) +x^{4}(2a^{-10}-4a^{-8}-7a^{-6}-5a^{-4}-4a^{-2}) +x^{5}(3a^{-9}-a^{-7}-12a^{-5}-8a^{-3}) +x^{6}(3a^{-8}+a^{-6}-a^{-4}+a^{-2}) +x^{7}(2a^{-7}+4a^{-5}+2a^{-3}) +x^{8}(a^{-6}+a^{-4}) $

k9c12: $ (1-a^{4}-2a^{6}-a^{8}) +x(-2a^{3}-4a^{5}-a^{7}+a^{9}) +x^{2}(-2-2a^{2}+3a^{4}+7a^{6}+4a^{8}) +x^{3}(-3a+4a^{3}+13a^{5}+3a^{7}-3a^{9}) +x^{4}(1-a^{2}-a^{4}-5a^{6}-6a^{8}) +x^{5}(2a-3a^{3}-11a^{5}-5a^{7}+a^{9}) +x^{6}(2a^{2}+2a^{8}) +x^{7}(2a^{3}+4a^{5}+2a^{7}) +x^{8}(a^{4}+a^{6}) $

k9c13: $ (a^{-10}-a^{-8}-3a^{-6}) +x(2a^{-13}-2a^{-11}-3a^{-9}+a^{-7}) +x^{2}(2a^{-12}-2a^{-10}+6a^{-8}+8a^{-6}-2a^{-4}) +x^{3}(-3a^{-13}+2a^{-11}+9a^{-9}+a^{-7}-3a^{-5}) +x^{4}(-5a^{-12}-a^{-10}-4a^{-8}-7a^{-6}+a^{-4}) +x^{5}(a^{-13}-4a^{-11}-9a^{-9}-2a^{-7}+2a^{-5}) +x^{6}(2a^{-12}+a^{-8}+3a^{-6}) +x^{7}(2a^{-11}+4a^{-9}+2a^{-7}) +x^{8}(a^{-10}+a^{-8}) $

k9c14: $ (-a^{-6}-2a^{-4}-a^{-2}+1) +x(-3a^{-5}-5a^{-3}-2a^{-1}) +x^{2}(4a^{-6}+10a^{-4}+8a^{-2}-2a^{2}) +x^{3}(9a^{-5}+15a^{-3}+2a^{-1}-3a+a^{3}) +x^{4}(-4a^{-6}-9a^{-4}-12a^{-2}-4+3a^{2}) +x^{5}(-8a^{-5}-16a^{-3}-4a^{-1}+4a) +x^{6}(a^{-6}+3a^{-2}+4) +x^{7}(2a^{-5}+5a^{-3}+3a^{-1}) +x^{8}(a^{-4}+a^{-2}) $

k9c15: $ (-a^{-8}-a^{-6}+a^{-4}+a^{-2}+1) +x(2a^{-9}+a^{-7}-a^{-5}+a^{-3}+a^{-1}) +x^{2}(3a^{-8}+2a^{-6}-2a^{-4}-3a^{-2}-2) +x^{3}(-3a^{-9}-a^{-7}+5a^{-5}-3a^{-1}) +x^{4}(-5a^{-8}-4a^{-6}+1) +x^{5}(a^{-9}-3a^{-7}-7a^{-5}-a^{-3}+2a^{-1}) +x^{6}(2a^{-8}+a^{-6}+a^{-4}+2a^{-2}) +x^{7}(2a^{-7}+4a^{-5}+2a^{-3}) +x^{8}(a^{-6}+a^{-4}) $

k9c16: $ (-3a^{-8}-4a^{-6}) +x(2a^{-13}+2a^{-11}+4a^{-9}+4a^{-7}) +x^{2}(-a^{-14}+2a^{-12}+a^{-10}+6a^{-8}+8a^{-6}) +x^{3}(a^{-15}-5a^{-13}-5a^{-11}-a^{-9}-2a^{-7}) +x^{4}(3a^{-14}-6a^{-12}-8a^{-10}-4a^{-8}-5a^{-6}) +x^{5}(5a^{-13}-a^{-11}-8a^{-9}-2a^{-7}) +x^{6}(5a^{-12}+3a^{-10}-a^{-8}+a^{-6}) +x^{7}(3a^{-11}+4a^{-9}+a^{-7}) +x^{8}(a^{-10}+a^{-8}) $

k9c17: $ (-2a^{-2}-3-2a^{2}) +x(-a^{-1}+a+3a^{3}+a^{5}) +x^{2}(5a^{-2}+13+9a^{2}-a^{4}-2a^{6}) +x^{3}(6a^{-1}+6a-4a^{3}-3a^{5}+a^{7}) +x^{4}(-4a^{-2}-12-14a^{2}-3a^{4}+3a^{6}) +x^{5}(-7a^{-1}-13a-2a^{3}+4a^{5}) +x^{6}(a^{-2}+1+4a^{2}+4a^{4}) +x^{7}(2a^{-1}+5a+3a^{3}) +x^{8}(1+a^{2}) $

k9c18: $ (a^{4}-a^{6}+a^{10}) +x(2a^{7}+2a^{13}) +x^{2}(-2a^{4}+3a^{6}-2a^{10}+3a^{12}) +x^{3}(-2a^{5}-4a^{7}+a^{9}-3a^{13}) +x^{4}(a^{4}-4a^{6}-2a^{8}-2a^{10}-5a^{12}) +x^{5}(2a^{5}+a^{7}-5a^{9}-3a^{11}+a^{13}) +x^{6}(3a^{6}+2a^{8}+a^{10}+2a^{12}) +x^{7}(2a^{7}+4a^{9}+2a^{11}) +x^{8}(a^{8}+a^{10}) $

k9c19: $ (a^{-4}+a^{-2}-a^{2}) +x(a^{-3}-a^{-1}-3a-a^{3}) +x^{2}(-2a^{-4}-3a^{-2}+3+8a^{2}+4a^{4}) +x^{3}(-3a^{-3}+a^{-1}+10a+4a^{3}-2a^{5}) +x^{4}(a^{-4}-4-11a^{2}-8a^{4}) +x^{5}(2a^{-3}-a^{-1}-11a-7a^{3}+a^{5}) +x^{6}(2a^{-2}+2+3a^{2}+3a^{4}) +x^{7}(2a^{-1}+5a+3a^{3}) +x^{8}(1+a^{2}) $

k9c20: $ (-2a^{2}-2a^{4}-2a^{6}-a^{8}) +x(2a^{7}+2a^{9}) +x^{2}(5a^{2}+11a^{4}+10a^{6}+3a^{8}-a^{10}) +x^{3}(6a^{3}+5a^{5}-7a^{7}-5a^{9}+a^{11}) +x^{4}(-4a^{2}-11a^{4}-16a^{6}-6a^{8}+3a^{10}) +x^{5}(-7a^{3}-12a^{5}+5a^{9}) +x^{6}(a^{2}+a^{4}+5a^{6}+5a^{8}) +x^{7}(2a^{3}+5a^{5}+3a^{7}) +x^{8}(a^{4}+a^{6}) $

k9c21: $ (-a^{-8}-a^{-6}-a^{-2}) +x(2a^{-9}-3a^{-5}-a^{-3}) +x^{2}(3a^{-8}+5a^{-6}+6a^{-4}+3a^{-2}-1) +x^{3}(-3a^{-9}+9a^{-5}+2a^{-3}-4a^{-1}) +x^{4}(-5a^{-8}-7a^{-6}-9a^{-4}-6a^{-2}+1) +x^{5}(a^{-9}-3a^{-7}-10a^{-5}-3a^{-3}+3a^{-1}) +x^{6}(2a^{-8}+2a^{-6}+4a^{-4}+4a^{-2}) +x^{7}(2a^{-7}+5a^{-5}+3a^{-3}) +x^{8}(a^{-6}+a^{-4}) $

k9c22: $ (-a^{-4}-4a^{-2}-4-2a^{2}) +x(a^{-5}+a^{-3}-2a^{-1}-2a) +x^{2}(-a^{-6}+5a^{-4}+17a^{-2}+16+5a^{2}) +x^{3}(a^{-7}-4a^{-5}-2a^{-3}+10a^{-1}+7a) +x^{4}(3a^{-6}-9a^{-4}-23a^{-2}-15-4a^{2}) +x^{5}(5a^{-5}-4a^{-3}-16a^{-1}-7a) +x^{6}(6a^{-4}+7a^{-2}+2+a^{2}) +x^{7}(4a^{-3}+6a^{-1}+2a) +x^{8}(a^{-2}+1) $

k9c23: $ (a^{4}-2a^{6}-2a^{8}) +x(4a^{7}+4a^{9}+a^{11}+a^{13}) +x^{2}(-2a^{4}+4a^{6}+6a^{8}+3a^{10}+3a^{12}) +x^{3}(-2a^{5}-6a^{7}-2a^{9}-2a^{13}) +x^{4}(a^{4}-4a^{6}-8a^{8}-10a^{10}-7a^{12}) +x^{5}(2a^{5}+2a^{7}-6a^{9}-5a^{11}+a^{13}) +x^{6}(3a^{6}+4a^{8}+4a^{10}+3a^{12}) +x^{7}(2a^{7}+5a^{9}+3a^{11}) +x^{8}(a^{8}+a^{10}) $

k9c24: $ (-a^{-2}-3-5a^{2}-2a^{4}) +x(a^{-3}+2a^{-1}+2a+3a^{3}+2a^{5}) +x^{2}(-a^{-4}+2a^{-2}+9+10a^{2}+4a^{4}) +x^{3}(-4a^{-3}-3a^{-1}+a-3a^{3}-3a^{5}) +x^{4}(a^{-4}-5a^{-2}-11-10a^{2}-5a^{4}) +x^{5}(3a^{-3}-a^{-1}-7a-2a^{3}+a^{5}) +x^{6}(4a^{-2}+5+3a^{2}+2a^{4}) +x^{7}(3a^{-1}+5a+2a^{3}) +x^{8}(1+a^{2}) $

k9c25: $ (1-a^{2}-3a^{4}-3a^{6}-a^{8}) +x(-a^{3}-a^{5}+a^{7}+a^{9}) +x^{2}(-2+2a^{2}+13a^{4}+13a^{6}+4a^{8}) +x^{3}(-2a+3a^{3}+5a^{5}-2a^{7}-2a^{9}) +x^{4}(1-3a^{2}-15a^{4}-18a^{6}-7a^{8}) +x^{5}(2a-3a^{3}-10a^{5}-4a^{7}+a^{9}) +x^{6}(3a^{2}+6a^{4}+6a^{6}+3a^{8}) +x^{7}(3a^{3}+6a^{5}+3a^{7}) +x^{8}(a^{4}+a^{6}) $

k9c26: $ (-a^{-6}-3a^{-4}-3a^{-2}) +x(a^{-7}+a^{-5}-a^{-3}-a^{-1}) +x^{2}(-a^{-8}+2a^{-6}+11a^{-4}+13a^{-2}+5) +x^{3}(-4a^{-7}-2a^{-5}+7a^{-3}+3a^{-1}-2a) +x^{4}(a^{-8}-5a^{-6}-14a^{-4}-16a^{-2}-8) +x^{5}(3a^{-7}-a^{-5}-11a^{-3}-6a^{-1}+a) +x^{6}(4a^{-6}+6a^{-4}+5a^{-2}+3) +x^{7}(3a^{-5}+6a^{-3}+3a^{-1}) +x^{8}(a^{-4}+a^{-2}) $

k9c27: $ (-a^{-2}-2-3a^{2}-a^{4}) +x(a^{-3}+2a^{-1}+2a+2a^{3}+a^{5}) +x^{2}(-a^{-4}+3a^{-2}+12+12a^{2}+4a^{4}) +x^{3}(-4a^{-3}-4a^{-1}-2a^{3}-2a^{5}) +x^{4}(a^{-4}-5a^{-2}-16-17a^{2}-7a^{4}) +x^{5}(3a^{-3}-8a-4a^{3}+a^{5}) +x^{6}(4a^{-2}+7+6a^{2}+3a^{4}) +x^{7}(3a^{-1}+6a+3a^{3}) +x^{8}(1+a^{2}) $

k9c28: $ (-1-5a^{2}-4a^{4}-a^{6}) +x(a^{-1}+3a+6a^{3}+6a^{5}+2a^{7}) +x^{2}(5+14a^{2}+12a^{4}+2a^{6}-a^{8}) +x^{3}(-2a^{-1}-4a-7a^{3}-9a^{5}-4a^{7}) +x^{4}(-7-19a^{2}-17a^{4}-4a^{6}+a^{8}) +x^{5}(a^{-1}-3a-5a^{3}+2a^{5}+3a^{7}) +x^{6}(3+7a^{2}+8a^{4}+4a^{6}) +x^{7}(3a+6a^{3}+3a^{5}) +x^{8}(a^{2}+a^{4}) $

k9c29: $ (-a^{-2}-3-5a^{2}-2a^{4}) +x(-a^{-1}-a+2a^{3}+2a^{5}) +x^{2}(3a^{-2}+12+17a^{2}+8a^{4}) +x^{3}(9a^{-1}+14a-a^{3}-5a^{5}+a^{7}) +x^{4}(-3a^{-2}-11-24a^{2}-13a^{4}+3a^{6}) +x^{5}(-10a^{-1}-24a-8a^{3}+6a^{5}) +x^{6}(a^{-2}-1+6a^{2}+8a^{4}) +x^{7}(3a^{-1}+9a+6a^{3}) +x^{8}(2+2a^{2}) $

k9c30: $ (-2a^{-2}-4-4a^{2}-a^{4}) +x(a^{-3}+a^{-1}+a+2a^{3}+a^{5}) +x^{2}(-a^{-4}+5a^{-2}+17+16a^{2}+5a^{4}) +x^{3}(-3a^{-3}-2a^{-1}-3a^{3}-2a^{5}) +x^{4}(a^{-4}-7a^{-2}-23-22a^{2}-7a^{4}) +x^{5}(3a^{-3}-2a^{-1}-9a-3a^{3}+a^{5}) +x^{6}(5a^{-2}+10+8a^{2}+3a^{4}) +x^{7}(4a^{-1}+7a+3a^{3}) +x^{8}(1+a^{2}) $

k9c31: $ (-1-4a^{2}-2a^{4}) +x(a^{-1}+3a+5a^{3}+3a^{5}) +x^{2}(5+15a^{2}+13a^{4}+3a^{6}) +x^{3}(-2a^{-1}-3a-5a^{3}-8a^{5}-4a^{7}) +x^{4}(-7-21a^{2}-23a^{4}-8a^{6}+a^{8}) +x^{5}(a^{-1}-3a-7a^{3}+a^{5}+4a^{7}) +x^{6}(3+8a^{2}+11a^{4}+6a^{6}) +x^{7}(3a+7a^{3}+4a^{5}) +x^{8}(a^{2}+a^{4}) $

k9c32: $ (-a^{-6}-2a^{-4}-a^{-2}+1) +x(a^{-7}-2a^{-3}-a^{-1}) +x^{2}(-a^{-8}+4a^{-6}+12a^{-4}+10a^{-2}+3) +x^{3}(-3a^{-7}+2a^{-5}+9a^{-3}+3a^{-1}-a) +x^{4}(a^{-8}-6a^{-6}-18a^{-4}-19a^{-2}-8) +x^{5}(3a^{-7}-5a^{-5}-18a^{-3}-9a^{-1}+a) +x^{6}(5a^{-6}+7a^{-4}+6a^{-2}+4) +x^{7}(5a^{-5}+10a^{-3}+5a^{-1}) +x^{8}(2a^{-4}+2a^{-2}) $

k9c33: $ (-2a^{2}-a^{4}) +x(a^{3}+a^{5}) +x^{2}(3a^{-2}+9+10a^{2}+4a^{4}) +x^{3}(-3a^{-3}-a^{-1}+5a+a^{3}-2a^{5}) +x^{4}(a^{-4}-9a^{-2}-20-16a^{2}-6a^{4}) +x^{5}(4a^{-3}-5a^{-1}-16a-6a^{3}+a^{5}) +x^{6}(7a^{-2}+9+5a^{2}+3a^{4}) +x^{7}(6a^{-1}+10a+4a^{3}) +x^{8}(2+2a^{2}) $

k9c34: $ (-a^{-2}-1-a^{2}) +x(-a^{-1}-a) +x^{2}(4a^{-2}+11+10a^{2}+3a^{4}) +x^{3}(-2a^{-3}+4a^{-1}+12a+5a^{3}-a^{5}) +x^{4}(a^{-4}-10a^{-2}-23-19a^{2}-7a^{4}) +x^{5}(4a^{-3}-10a^{-1}-26a-11a^{3}+a^{5}) +x^{6}(8a^{-2}+9+5a^{2}+4a^{4}) +x^{7}(8a^{-1}+14a+6a^{3}) +x^{8}(3+3a^{2}) $

k9c35: $ (-3a^{6}-a^{8}+a^{10}) +x(-a^{7}-9a^{9}-8a^{11}) +x^{2}(a^{2}-2a^{4}+12a^{6}+16a^{8}+a^{10}) +x^{3}(2a^{3}-6a^{5}+3a^{7}+23a^{9}+12a^{11}) +x^{4}(3a^{4}-15a^{6}-15a^{8}+3a^{10}) +x^{5}(4a^{5}-8a^{7}-18a^{9}-6a^{11}) +x^{6}(5a^{6}+a^{8}-4a^{10}) +x^{7}(3a^{7}+4a^{9}+a^{11}) +x^{8}(a^{8}+a^{10}) $

k9c36: $ (-2a^{-8}-4a^{-6}-3a^{-4}-2a^{-2}) +x(-a^{-11}+a^{-9}+a^{-7}-2a^{-5}-a^{-3}) +x^{2}(-a^{-10}+7a^{-8}+15a^{-6}+12a^{-4}+5a^{-2}) +x^{3}(a^{-11}-2a^{-9}+9a^{-5}+6a^{-3}) +x^{4}(2a^{-10}-7a^{-8}-17a^{-6}-12a^{-4}-4a^{-2}) +x^{5}(3a^{-9}-4a^{-7}-14a^{-5}-7a^{-3}) +x^{6}(4a^{-8}+4a^{-6}+a^{-4}+a^{-2}) +x^{7}(3a^{-7}+5a^{-5}+2a^{-3}) +x^{8}(a^{-6}+a^{-4}) $

k9c37: $ (a^{-4}-2-2a^{2}) +x(-5a^{-1}-7a-2a^{3}) +x^{2}(-2a^{-4}+a^{-2}+12+14a^{2}+5a^{4}) +x^{3}(-2a^{-3}+6a^{-1}+13a+3a^{3}-2a^{5}) +x^{4}(a^{-4}-3a^{-2}-13-17a^{2}-8a^{4}) +x^{5}(2a^{-3}-4a^{-1}-13a-6a^{3}+a^{5}) +x^{6}(3a^{-2}+5+5a^{2}+3a^{4}) +x^{7}(3a^{-1}+6a+3a^{3}) +x^{8}(1+a^{2}) $

k9c38: $ (-4a^{6}-3a^{8}) +x(3a^{7}+a^{9}-a^{11}+a^{13}) +x^{2}(-a^{4}+9a^{6}+10a^{8}+3a^{10}+3a^{12}) +x^{3}(-2a^{5}-2a^{7}+5a^{9}+3a^{11}-2a^{13}) +x^{4}(a^{4}-10a^{6}-15a^{8}-10a^{10}-6a^{12}) +x^{5}(3a^{5}-4a^{7}-15a^{9}-7a^{11}+a^{13}) +x^{6}(6a^{6}+6a^{8}+3a^{10}+3a^{12}) +x^{7}(5a^{7}+9a^{9}+4a^{11}) +x^{8}(2a^{8}+2a^{10}) $

k9c39: $ (-a^{-8}-2a^{-6}-2a^{-4}-2a^{-2}) +x(a^{-9}-a^{-7}-3a^{-5}-a^{-3}) +x^{2}(3a^{-8}+9a^{-6}+12a^{-4}+5a^{-2}-1) +x^{3}(-2a^{-9}+2a^{-7}+12a^{-5}+5a^{-3}-3a^{-1}) +x^{4}(-6a^{-8}-13a^{-6}-15a^{-4}-7a^{-2}+1) +x^{5}(a^{-9}-7a^{-7}-18a^{-5}-7a^{-3}+3a^{-1}) +x^{6}(3a^{-8}+3a^{-6}+5a^{-4}+5a^{-2}) +x^{7}(4a^{-7}+9a^{-5}+5a^{-3}) +x^{8}(2a^{-6}+2a^{-4}) $

k9c40: $ (2+2a^{2}+a^{4}) +x(-a^{3}-a^{5}) +x^{2}(3a^{2}+7a^{4}+4a^{6}) +x^{3}(6a+14a^{3}+6a^{5}-2a^{7}) +x^{4}(-7-17a^{2}-20a^{4}-9a^{6}+a^{8}) +x^{5}(a^{-1}-15a-32a^{3}-12a^{5}+4a^{7}) +x^{6}(5+4a^{2}+7a^{4}+8a^{6}) +x^{7}(8a+17a^{3}+9a^{5}) +x^{8}(4a^{2}+4a^{4}) $

k9c41: $ (-3a^{2}-3a^{4}-a^{6}) +x(-2a-4a^{3}-2a^{5}) +x^{2}(-a^{-2}+6+17a^{2}+13a^{4}+3a^{6}) +x^{3}(a^{-3}-3a^{-1}+6a+19a^{3}+9a^{5}) +x^{4}(3a^{-2}-11-23a^{2}-12a^{4}-3a^{6}) +x^{5}(5a^{-1}-11a-26a^{3}-10a^{5}) +x^{6}(7+5a^{2}-a^{4}+a^{6}) +x^{7}(6a+9a^{3}+3a^{5}) +x^{8}(2a^{2}+2a^{4}) $

k9c42: $ (-2a^{-2}-3-2a^{2}) +x(-2a^{-1}-2a) +x^{2}(6a^{-2}+12+6a^{2}) +x^{3}(6a^{-1}+6a) +x^{4}(-5a^{-2}-10-5a^{2}) +x^{5}(-5a^{-1}-5a) +x^{6}(a^{-2}+2+a^{2}) +x^{7}(a^{-1}+a) $

k9c43: $ (-a^{-8}-3a^{-6}-4a^{-4}-3a^{-2}) +x(a^{-9}+a^{-7}) +x^{2}(2a^{-8}+9a^{-6}+14a^{-4}+7a^{-2}) +x^{3}(-2a^{-7}+a^{-5}+3a^{-3}) +x^{4}(-8a^{-6}-13a^{-4}-5a^{-2}) +x^{5}(a^{-7}-3a^{-5}-4a^{-3}) +x^{6}(2a^{-6}+3a^{-4}+a^{-2}) +x^{7}(a^{-5}+a^{-3}) $

k9c44: $ (-a^{-2}-2-3a^{2}-a^{4}) +x(-a^{-1}-a+a^{3}+a^{5}) +x^{2}(a^{-2}+6+10a^{2}+5a^{4}) +x^{3}(2a^{-1}+4a-a^{3}-3a^{5}) +x^{4}(-3-10a^{2}-7a^{4}) +x^{5}(-3a-2a^{3}+a^{5}) +x^{6}(1+3a^{2}+2a^{4}) +x^{7}(a+a^{3}) $

k9c45: $ (-2a^{2}-2a^{4}-2a^{6}-a^{8}) +x(2a^{7}+2a^{9}) +x^{2}(3a^{2}+6a^{4}+7a^{6}+4a^{8}) +x^{3}(a^{3}-a^{5}-5a^{7}-3a^{9}) +x^{4}(-4a^{4}-10a^{6}-6a^{8}) +x^{5}(a^{3}+a^{9}) +x^{6}(2a^{4}+4a^{6}+2a^{8}) +x^{7}(a^{5}+a^{7}) $

k9c46: $ (2+a^{2}-a^{4}-a^{6}) +x(-2a-6a^{3}-4a^{5}) +x^{2}(3a^{2}+9a^{4}+6a^{6}) +x^{3}(a+8a^{3}+7a^{5}) +x^{4}(-4a^{2}-9a^{4}-5a^{6}) +x^{5}(-5a^{3}-5a^{5}) +x^{6}(a^{2}+2a^{4}+a^{6}) +x^{7}(a^{3}+a^{5}) $

k9c47: $ (-a^{-6}-2a^{-4}-a^{-2}+1) +x(-3a^{-5}-5a^{-3}-2a^{-1}) +x^{2}(3a^{-6}+9a^{-4}+11a^{-2}+5) +x^{3}(3a^{-5}+6a^{-3}+a^{-1}-2a) +x^{4}(-7a^{-4}-16a^{-2}-9) +x^{5}(a^{-5}-4a^{-3}-4a^{-1}+a) +x^{6}(3a^{-4}+6a^{-2}+3) +x^{7}(2a^{-3}+2a^{-1}) $

k9c48: $ (2a^{-6}+3a^{-4}) +x(-4a^{-7}-5a^{-5}-a^{-3}) +x^{2}(-a^{-6}+2a^{-4}+2a^{-2}-1) +x^{3}(3a^{-7}+5a^{-5}-3a^{-3}-5a^{-1}) +x^{4}(-6a^{-4}-5a^{-2}+1) +x^{5}(-a^{-5}+2a^{-3}+3a^{-1}) +x^{6}(a^{-6}+4a^{-4}+3a^{-2}) +x^{7}(a^{-5}+a^{-3}) $

k9c49: $ (-3a^{-8}-4a^{-6}) +x(-4a^{-11}-2a^{-9}+2a^{-7}) +x^{2}(-a^{-10}+10a^{-8}+9a^{-6}-2a^{-4}) +x^{3}(3a^{-11}+3a^{-9}-3a^{-7}-3a^{-5}) +x^{4}(-9a^{-8}-8a^{-6}+a^{-4}) +x^{5}(-a^{-9}+a^{-7}+2a^{-5}) +x^{6}(a^{-10}+4a^{-8}+3a^{-6}) +x^{7}(a^{-9}+a^{-7}) $

k10c1: $ (-a^{-2}+a^{6}+a^{8}) +x(4a^{5}+4a^{7}) +x^{2}(a^{-2}-11a^{6}-10a^{8}) +x^{3}(a^{-1}-a+a^{3}-11a^{5}-14a^{7}) +x^{4}(1-2a^{2}+3a^{4}+21a^{6}+15a^{8}) +x^{5}(a-3a^{3}+12a^{5}+16a^{7}) +x^{6}(a^{2}-4a^{4}-12a^{6}-7a^{8}) +x^{7}(a^{3}-6a^{5}-7a^{7}) +x^{8}(a^{4}+2a^{6}+a^{8}) +x^{9}(a^{5}+a^{7}) $

k10c2: $ (4a^{4}+4a^{6}+a^{8}) +x(-2a^{5}-a^{7}+a^{9}-a^{11}-a^{13}) +x^{2}(-14a^{4}-21a^{6}-5a^{8}-a^{12}+a^{14}) +x^{3}(-3a^{5}+3a^{7}+2a^{9}-2a^{11}+2a^{13}) +x^{4}(16a^{4}+33a^{6}+11a^{8}-4a^{10}+2a^{12}) +x^{5}(10a^{5}+2a^{7}-6a^{9}+2a^{11}) +x^{6}(-7a^{4}-18a^{6}-9a^{8}+2a^{10}) +x^{7}(-6a^{5}-4a^{7}+2a^{9}) +x^{8}(a^{4}+3a^{6}+2a^{8}) +x^{9}(a^{5}+a^{7}) $

k10c3: $ (a^{-4}+a^{2}-a^{6}) +x(6a+6a^{3}) +x^{2}(-3a^{-4}+a^{-2}-12a^{2}-2a^{4}+6a^{6}) +x^{3}(-2a^{-3}+4a^{-1}-15a-18a^{3}+3a^{5}) +x^{4}(a^{-4}-2a^{-2}+6+18a^{2}+4a^{4}-5a^{6}) +x^{5}(a^{-3}-3a^{-1}+15a+15a^{3}-4a^{5}) +x^{6}(a^{-2}-4-10a^{2}-4a^{4}+a^{6}) +x^{7}(a^{-1}-6a-6a^{3}+a^{5}) +x^{8}(1+2a^{2}+a^{4}) +x^{9}(a+a^{3}) $

k10c4: $ (2a^{-4}+2a^{-2}+a^{4}) +x(2a^{-3}-a^{-1}-3a) +x^{2}(-13a^{-4}-16a^{-2}+1-3a^{4}+a^{6}) +x^{3}(-7a^{-3}+7a^{-1}+8a-4a^{3}+2a^{5}) +x^{4}(16a^{-4}+29a^{-2}+4-6a^{2}+3a^{4}) +x^{5}(11a^{-3}-2a^{-1}-10a+3a^{3}) +x^{6}(-7a^{-4}-17a^{-2}-7+3a^{2}) +x^{7}(-6a^{-3}-3a^{-1}+3a) +x^{8}(a^{-4}+3a^{-2}+2) +x^{9}(a^{-3}+a^{-1}) $

k10c5: $ (3a^{-6}+5a^{-4}+a^{-2}) +x(-a^{-11}-a^{-7}-3a^{-5}-2a^{-3}-a^{-1}) +x^{2}(-2a^{-10}+a^{-8}-9a^{-6}-22a^{-4}-10a^{-2}) +x^{3}(a^{-11}-a^{-9}+3a^{-7}+6a^{-5}+7a^{-3}+6a^{-1}) +x^{4}(2a^{-10}-2a^{-8}+10a^{-6}+32a^{-4}+18a^{-2}) +x^{5}(2a^{-9}-4a^{-7}-3a^{-5}-2a^{-3}-5a^{-1}) +x^{6}(2a^{-8}-7a^{-6}-20a^{-4}-11a^{-2}) +x^{7}(2a^{-7}-2a^{-5}-3a^{-3}+a^{-1}) +x^{8}(2a^{-6}+4a^{-4}+2a^{-2}) +x^{9}(a^{-5}+a^{-3}) $

k10c6: $ (-3a^{2}-2a^{4}+a^{6}+a^{8}) +x(2a^{3}+3a^{5}+a^{11}) +x^{2}(7a^{2}+5a^{4}-10a^{6}-5a^{8}+a^{10}-2a^{12}) +x^{3}(-10a^{5}-2a^{7}+4a^{9}-4a^{11}) +x^{4}(-5a^{2}-3a^{4}+18a^{6}+12a^{8}-3a^{10}+a^{12}) +x^{5}(-3a^{3}+8a^{5}+5a^{7}-4a^{9}+2a^{11}) +x^{6}(a^{2}-2a^{4}-12a^{6}-7a^{8}+2a^{10}) +x^{7}(a^{3}-4a^{5}-3a^{7}+2a^{9}) +x^{8}(a^{4}+3a^{6}+2a^{8}) +x^{9}(a^{5}+a^{7}) $

k10c7: $ (1+a^{4}+2a^{6}+a^{8}) +x(-2a^{5}-5a^{7}-3a^{9}) +x^{2}(-2-2a^{2}-4a^{4}-10a^{6}-3a^{8}+3a^{10}) +x^{3}(-3a+a^{3}+6a^{5}+10a^{7}+8a^{9}) +x^{4}(1-a^{2}+8a^{4}+20a^{6}+6a^{8}-4a^{10}) +x^{5}(2a-2a^{3}-2a^{5}-6a^{7}-8a^{9}) +x^{6}(2a^{2}-5a^{4}-15a^{6}-7a^{8}+a^{10}) +x^{7}(2a^{3}-a^{5}-a^{7}+2a^{9}) +x^{8}(2a^{4}+4a^{6}+2a^{8}) +x^{9}(a^{5}+a^{7}) $

k10c8: $ (3+3a^{2}-a^{6}) +x(-a^{3}+a^{5}+2a^{7}) +x^{2}(-13-18a^{2}+3a^{4}+5a^{6}-2a^{8}+a^{10}) +x^{3}(-6a+5a^{3}+2a^{5}-7a^{7}+2a^{9}) +x^{4}(16+30a^{2}+a^{4}-10a^{6}+3a^{8}) +x^{5}(11a-a^{3}-8a^{5}+4a^{7}) +x^{6}(-7-17a^{2}-6a^{4}+4a^{6}) +x^{7}(-6a-3a^{3}+3a^{5}) +x^{8}(1+3a^{2}+2a^{4}) +x^{9}(a+a^{3}) $

k10c9: $ (2a^{-4}+4a^{-2}+3) +x(a^{-7}-2a^{-3}-2a^{-1}-a) +x^{2}(-2a^{-8}+a^{-6}-8a^{-4}-22a^{-2}-8+3a^{2}) +x^{3}(-4a^{-7}+4a^{-5}+5a^{-3}+4a^{-1}+7a) +x^{4}(a^{-8}-3a^{-6}+13a^{-4}+31a^{-2}+10-4a^{2}) +x^{5}(2a^{-7}-4a^{-5}-2a^{-1}-8a) +x^{6}(2a^{-6}-7a^{-4}-18a^{-2}-8+a^{2}) +x^{7}(2a^{-5}-2a^{-3}-2a^{-1}+2a) +x^{8}(2a^{-4}+4a^{-2}+2) +x^{9}(a^{-3}+a^{-1}) $

k10c10: $ (a^{-6}+2a^{-4}+a^{-2}+1) +x(-3a^{-7}-6a^{-5}-4a^{-3}-a^{-1}) +x^{2}(-8a^{-6}-12a^{-4}-4a^{-2}-2-2a^{2}) +x^{3}(7a^{-7}+17a^{-5}+17a^{-3}+3a^{-1}-3a+a^{3}) +x^{4}(15a^{-6}+26a^{-4}+5a^{-2}-3+3a^{2}) +x^{5}(-5a^{-7}-10a^{-5}-16a^{-3}-7a^{-1}+4a) +x^{6}(-10a^{-6}-21a^{-4}-7a^{-2}+4) +x^{7}(a^{-7}-a^{-5}+2a^{-3}+4a^{-1}) +x^{8}(2a^{-6}+5a^{-4}+3a^{-2}) +x^{9}(a^{-5}+a^{-3}) $

k10c11: $ (-2a^{-2}-1+a^{2}-a^{6}) +x(a^{-1}+5a+2a^{3}-2a^{5}) +x^{2}(7a^{-2}+2-12a^{2}+5a^{6}-2a^{8}) +x^{3}(a^{-1}-16a-5a^{3}+9a^{5}-3a^{7}) +x^{4}(-5a^{-2}-1+16a^{2}+5a^{4}-6a^{6}+a^{8}) +x^{5}(-3a^{-1}+11a+5a^{3}-7a^{5}+2a^{7}) +x^{6}(a^{-2}-2-10a^{2}-4a^{4}+3a^{6}) +x^{7}(a^{-1}-4a-2a^{3}+3a^{5}) +x^{8}(1+3a^{2}+2a^{4}) +x^{9}(a+a^{3}) $

k10c12: $ (2a^{-6}+2a^{-4}-2a^{-2}-1) +x(2a^{-9}-3a^{-5}-a^{-3}+a^{-1}+a) +x^{2}(2a^{-8}-8a^{-6}-12a^{-4}+2a^{-2}+4) +x^{3}(-3a^{-9}-a^{-7}+4a^{-5}+5a^{-3}-3a) +x^{4}(-5a^{-8}+8a^{-6}+23a^{-4}+4a^{-2}-6) +x^{5}(a^{-9}-3a^{-7}-a^{-3}-4a^{-1}+a) +x^{6}(2a^{-8}-5a^{-6}-14a^{-4}-5a^{-2}+2) +x^{7}(2a^{-7}-a^{-5}-a^{-3}+2a^{-1}) +x^{8}(2a^{-6}+4a^{-4}+2a^{-2}) +x^{9}(a^{-5}+a^{-3}) $

k10c13: $ (a^{-4}-1-a^{2}-a^{4}-a^{6}) +x(-2a^{-1}+a^{3}-a^{5}) +x^{2}(-2a^{-4}+a^{-2}+4-a^{2}+2a^{4}+4a^{6}) +x^{3}(-2a^{-3}+3a^{-1}+a^{3}+6a^{5}) +x^{4}(a^{-4}-3a^{-2}-3+6a^{2}+a^{4}-4a^{6}) +x^{5}(2a^{-3}-3a^{-1}-2a-4a^{3}-7a^{5}) +x^{6}(3a^{-2}-9a^{2}-5a^{4}+a^{6}) +x^{7}(3a^{-1}+a+2a^{5}) +x^{8}(2+4a^{2}+2a^{4}) +x^{9}(a+a^{3}) $

k10c14: $ (-a^{2}+a^{4}+a^{6}) +x(-a^{3}-4a^{5}-2a^{7}+2a^{9}+a^{11}) +x^{2}(4a^{2}-a^{4}-9a^{6}-3a^{8}-a^{12}) +x^{3}(6a^{3}+10a^{5}+8a^{7}-4a^{11}) +x^{4}(-4a^{2}+2a^{4}+16a^{6}+5a^{8}-4a^{10}+a^{12}) +x^{5}(-7a^{3}-9a^{5}-9a^{7}-4a^{9}+3a^{11}) +x^{6}(a^{2}-5a^{4}-14a^{6}-4a^{8}+4a^{10}) +x^{7}(2a^{3}+a^{5}+3a^{7}+4a^{9}) +x^{8}(2a^{4}+5a^{6}+3a^{8}) +x^{9}(a^{5}+a^{7}) $

k10c15: $ (-2a^{-4}-3a^{-2}+1+a^{2}) +x(-a^{-7}+a^{-5}+3a^{-3}-3a-2a^{3}) +x^{2}(-a^{-6}+7a^{-4}+8a^{-2}-7-7a^{2}) +x^{3}(a^{-7}-2a^{-5}-3a^{-3}+a^{-1}+8a+7a^{3}) +x^{4}(2a^{-6}-7a^{-4}-8a^{-2}+16+15a^{2}) +x^{5}(3a^{-5}-3a^{-3}-5a^{-1}-4a-5a^{3}) +x^{6}(4a^{-4}-a^{-2}-15-10a^{2}) +x^{7}(3a^{-3}-2a+a^{3}) +x^{8}(2a^{-2}+4+2a^{2}) +x^{9}(a^{-1}+a) $

k10c16: $ (-a^{-6}+2a^{-2}+1-a^{2}) +x(-4a^{-5}-4a^{-3}) +x^{2}(-2a^{-8}+5a^{-6}+2a^{-4}-11a^{-2}-2+4a^{2}) +x^{3}(-3a^{-7}+10a^{-5}+8a^{-3}+5a) +x^{4}(a^{-8}-6a^{-6}+2a^{-4}+17a^{-2}+4-4a^{2}) +x^{5}(2a^{-7}-7a^{-5}-4a^{-3}-2a^{-1}-7a) +x^{6}(3a^{-6}-3a^{-4}-13a^{-2}-6+a^{2}) +x^{7}(3a^{-5}-a^{-1}+2a) +x^{8}(2a^{-4}+4a^{-2}+2) +x^{9}(a^{-3}+a^{-1}) $

k10c17: $ (2a^{-2}+5+2a^{2}) +x(a^{-5}-3a^{-1}-3a+a^{5}) +x^{2}(3a^{-4}-8a^{-2}-22-8a^{2}+3a^{4}) +x^{3}(-3a^{-5}+2a^{-3}+6a^{-1}+6a+2a^{3}-3a^{5}) +x^{4}(-6a^{-4}+11a^{-2}+34+11a^{2}-6a^{4}) +x^{5}(a^{-5}-5a^{-3}-5a^{3}+a^{5}) +x^{6}(2a^{-4}-7a^{-2}-18-7a^{2}+2a^{4}) +x^{7}(2a^{-3}-2a^{-1}-2a+2a^{3}) +x^{8}(2a^{-2}+4+2a^{2}) +x^{9}(a^{-1}+a) $

k10c18: $ (-a^{-2}+a^{2}+a^{4}) +x(-2a^{-1}-4a-4a^{3}-2a^{5}) +x^{2}(4a^{-2}+1-8a^{2}-3a^{4}+a^{6}-a^{8}) +x^{3}(6a^{-1}+11a+14a^{3}+5a^{5}-4a^{7}) +x^{4}(-4a^{-2}+1+17a^{2}+6a^{4}-5a^{6}+a^{8}) +x^{5}(-7a^{-1}-10a-12a^{3}-6a^{5}+3a^{7}) +x^{6}(a^{-2}-5-15a^{2}-5a^{4}+4a^{6}) +x^{7}(2a^{-1}+a+3a^{3}+4a^{5}) +x^{8}(2+5a^{2}+3a^{4}) +x^{9}(a+a^{3}) $

k10c19: $ (a^{-2}+3+a^{2}) +x(-2a^{-3}-4a^{-1}-2a+a^{3}+a^{5}) +x^{2}(-9a^{-2}-13+3a^{4}-a^{6}) +x^{3}(7a^{-3}+13a^{-1}+11a-4a^{5}+a^{7}) +x^{4}(16a^{-2}+23-4a^{2}-8a^{4}+3a^{6}) +x^{5}(-5a^{-3}-8a^{-1}-15a-7a^{3}+5a^{5}) +x^{6}(-10a^{-2}-19-3a^{2}+6a^{4}) +x^{7}(a^{-3}-a^{-1}+3a+5a^{3}) +x^{8}(2a^{-2}+5+3a^{2}) +x^{9}(a^{-1}+a) $

k10c20: $ (2+a^{2}+a^{6}+a^{8}) +x(-a-a^{3}+3a^{5}+2a^{7}-a^{9}) +x^{2}(-3-2a^{2}-9a^{6}-5a^{8}+3a^{10}) +x^{3}(-a+2a^{3}-8a^{5}-4a^{7}+7a^{9}) +x^{4}(1+3a^{4}+17a^{6}+9a^{8}-4a^{10}) +x^{5}(a-a^{3}+9a^{5}+3a^{7}-8a^{9}) +x^{6}(a^{2}-2a^{4}-12a^{6}-8a^{8}+a^{10}) +x^{7}(a^{3}-4a^{5}-3a^{7}+2a^{9}) +x^{8}(a^{4}+3a^{6}+2a^{8}) +x^{9}(a^{5}+a^{7}) $

k10c21: $ (-a^{2}+2a^{4}+3a^{6}+a^{8}) +x(-2a^{5}-a^{7}+3a^{9}+2a^{11}) +x^{2}(4a^{2}-3a^{4}-14a^{6}-5a^{8}-2a^{12}) +x^{3}(5a^{3}+3a^{5}+2a^{7}-4a^{11}) +x^{4}(-4a^{2}+4a^{4}+20a^{6}+9a^{8}-2a^{10}+a^{12}) +x^{5}(-7a^{3}-3a^{5}-2a^{9}+2a^{11}) +x^{6}(a^{2}-6a^{4}-14a^{6}-5a^{8}+2a^{10}) +x^{7}(2a^{3}-a^{5}-a^{7}+2a^{9}) +x^{8}(2a^{4}+4a^{6}+2a^{8}) +x^{9}(a^{5}+a^{7}) $

k10c22: $ (2a^{-4}+2a^{-2}-1-2a^{2}) +x(-a^{-5}+a^{-3}+a^{-1}-a) +x^{2}(4a^{-6}-6a^{-4}-12a^{-2}+6+6a^{2}-2a^{4}) +x^{3}(6a^{-5}-4a^{-3}+7a-3a^{3}) +x^{4}(-4a^{-6}+6a^{-4}+16a^{-2}-1-6a^{2}+a^{4}) +x^{5}(-7a^{-5}-a^{-1}-6a+2a^{3}) +x^{6}(a^{-6}-6a^{-4}-12a^{-2}-2+3a^{2}) +x^{7}(2a^{-5}-a^{-3}+3a) +x^{8}(2a^{-4}+4a^{-2}+2) +x^{9}(a^{-3}+a^{-1}) $

k10c23: $ (2a^{-6}+3a^{-4}) +x(2a^{-9}+a^{-7}-2a^{-5}-2a^{-3}-a^{-1}) +x^{2}(3a^{-8}-6a^{-6}-13a^{-4}-a^{-2}+3) +x^{3}(-3a^{-9}-2a^{-7}+3a^{-5}+9a^{-3}+5a^{-1}-2a) +x^{4}(-5a^{-8}+5a^{-6}+20a^{-4}+3a^{-2}-7) +x^{5}(a^{-9}-2a^{-7}-2a^{-5}-9a^{-3}-9a^{-1}+a) +x^{6}(2a^{-8}-3a^{-6}-13a^{-4}-5a^{-2}+3) +x^{7}(2a^{-7}+a^{-5}+3a^{-3}+4a^{-1}) +x^{8}(2a^{-6}+5a^{-4}+3a^{-2}) +x^{9}(a^{-5}+a^{-3}) $

k10c24: $ (1-a^{2}-a^{4}+a^{6}+a^{8}) +x(2a^{3}+4a^{5}-2a^{9}) +x^{2}(-2+2a^{2}+5a^{4}-5a^{6}-2a^{8}+4a^{10}) +x^{3}(-2a-7a^{5}-2a^{7}+7a^{9}) +x^{4}(1-3a^{2}-5a^{4}+6a^{6}+3a^{8}-4a^{10}) +x^{5}(2a-2a^{3}+a^{5}-2a^{7}-7a^{9}) +x^{6}(3a^{2}+a^{4}-8a^{6}-5a^{8}+a^{10}) +x^{7}(3a^{3}+a^{5}+2a^{9}) +x^{8}(2a^{4}+4a^{6}+2a^{8}) +x^{9}(a^{5}+a^{7}) $

k10c25: $ (-2a^{2}+2a^{6}+a^{8}) +x(a^{3}-2a^{7}+a^{11}) +x^{2}(5a^{2}+4a^{4}-4a^{6}+a^{8}+3a^{10}-a^{12}) +x^{3}(4a^{3}+2a^{5}+3a^{7}+2a^{9}-3a^{11}) +x^{4}(-4a^{2}-3a^{4}+3a^{6}-5a^{8}-6a^{10}+a^{12}) +x^{5}(-6a^{3}-7a^{5}-9a^{7}-5a^{9}+3a^{11}) +x^{6}(a^{2}-3a^{4}-8a^{6}+a^{8}+5a^{10}) +x^{7}(2a^{3}+2a^{5}+5a^{7}+5a^{9}) +x^{8}(2a^{4}+5a^{6}+3a^{8}) +x^{9}(a^{5}+a^{7}) $

k10c26: $ (2a^{-4}+3a^{-2}+1-a^{2}) +x(-2a^{-5}-2a^{-3}-a^{-1}-a) +x^{2}(4a^{-6}-4a^{-4}-12a^{-2}+1+4a^{2}-a^{4}) +x^{3}(7a^{-5}+4a^{-3}+5a^{-1}+5a-3a^{3}) +x^{4}(-4a^{-6}+4a^{-4}+14a^{-2}-2-7a^{2}+a^{4}) +x^{5}(-7a^{-5}-6a^{-3}-9a^{-1}-7a+3a^{3}) +x^{6}(a^{-6}-5a^{-4}-12a^{-2}-1+5a^{2}) +x^{7}(2a^{-5}+a^{-3}+4a^{-1}+5a) +x^{8}(2a^{-4}+5a^{-2}+3) +x^{9}(a^{-3}+a^{-1}) $

k10c27: $ (-a^{2}+a^{4}+a^{6}) +x(-a-2a^{3}-2a^{5}+a^{9}) +x^{2}(4+4a^{2}-4a^{4}-a^{6}+3a^{8}) +x^{3}(-2a^{-1}+5a+11a^{3}+7a^{5}+a^{7}-2a^{9}) +x^{4}(-7-3a^{2}+7a^{4}-3a^{6}-6a^{8}) +x^{5}(a^{-1}-8a-14a^{3}-12a^{5}-6a^{7}+a^{9}) +x^{6}(3-2a^{2}-9a^{4}-a^{6}+3a^{8}) +x^{7}(4a+6a^{3}+6a^{5}+4a^{7}) +x^{8}(3a^{2}+6a^{4}+3a^{6}) +x^{9}(a^{3}+a^{5}) $

k10c28: $ (a^{-6}-3a^{-2}-1) +x(-4a^{-7}-6a^{-5}-2a^{-3}+a^{-1}+a) +x^{2}(-5a^{-6}+10a^{-2}+4-a^{2}) +x^{3}(8a^{-7}+18a^{-5}+13a^{-3}-2a^{-1}-4a+a^{3}) +x^{4}(12a^{-6}+11a^{-4}-12a^{-2}-8+3a^{2}) +x^{5}(-5a^{-7}-12a^{-5}-18a^{-3}-6a^{-1}+5a) +x^{6}(-9a^{-6}-16a^{-4}-a^{-2}+6) +x^{7}(a^{-7}+4a^{-3}+5a^{-1}) +x^{8}(2a^{-6}+5a^{-4}+3a^{-2}) +x^{9}(a^{-5}+a^{-3}) $

k10c29: $ (-2a^{-2}-2-a^{2}-a^{4}-a^{6}) +x(2a-2a^{5}) +x^{2}(5a^{-2}+6+4a^{4}+4a^{6}-a^{8}) +x^{3}(4a^{-1}+2a+7a^{3}+6a^{5}-3a^{7}) +x^{4}(-4a^{-2}-4+3a^{2}-5a^{4}-7a^{6}+a^{8}) +x^{5}(-6a^{-1}-8a-12a^{3}-7a^{5}+3a^{7}) +x^{6}(a^{-2}-3-9a^{2}+5a^{6}) +x^{7}(2a^{-1}+2a+5a^{3}+5a^{5}) +x^{8}(2+5a^{2}+3a^{4}) +x^{9}(a+a^{3}) $

k10c30: $ (-2a^{2}-a^{4}) +x(-a^{3}-5a^{5}-6a^{7}-2a^{9}) +x^{2}(-1+5a^{2}+9a^{4}+2a^{6}+a^{8}+2a^{10}) +x^{3}(-3a+4a^{3}+16a^{5}+18a^{7}+9a^{9}) +x^{4}(1-7a^{2}-11a^{4}+2a^{6}+2a^{8}-3a^{10}) +x^{5}(3a-6a^{3}-19a^{5}-20a^{7}-10a^{9}) +x^{6}(5a^{2}+2a^{4}-11a^{6}-7a^{8}+a^{10}) +x^{7}(5a^{3}+7a^{5}+5a^{7}+3a^{9}) +x^{8}(3a^{4}+6a^{6}+3a^{8}) +x^{9}(a^{5}+a^{7}) $

k10c31: $ (-a^{-4}-a^{-2}+2+a^{2}) +x(2a^{-5}+2a^{-3}-2a^{-1}-4a-2a^{3}) +x^{2}(3a^{-4}-2a^{-2}-10-3a^{2}+2a^{4}) +x^{3}(-3a^{-5}-3a^{-3}+6a^{-1}+15a+7a^{3}-2a^{5}) +x^{4}(-5a^{-4}+3a^{-2}+20+5a^{2}-7a^{4}) +x^{5}(a^{-5}-2a^{-3}-4a^{-1}-12a-10a^{3}+a^{5}) +x^{6}(2a^{-4}-3a^{-2}-14-6a^{2}+3a^{4}) +x^{7}(2a^{-3}+a^{-1}+3a+4a^{3}) +x^{8}(2a^{-2}+5+3a^{2}) +x^{9}(a^{-1}+a) $

k10c32: $ (-a^{-2}-1-a^{2}) +x(a^{-3}+a^{-1}-a-2a^{3}-a^{5}) +x^{2}(-a^{-4}+4a^{-2}+7+2a^{6}) +x^{3}(-3a^{-3}+7a+13a^{3}+9a^{5}) +x^{4}(a^{-4}-6a^{-2}-11+2a^{2}+3a^{4}-3a^{6}) +x^{5}(3a^{-3}-4a^{-1}-15a-18a^{3}-10a^{5}) +x^{6}(5a^{-2}+3-10a^{2}-7a^{4}+a^{6}) +x^{7}(5a^{-1}+7a+5a^{3}+3a^{5}) +x^{8}(3+6a^{2}+3a^{4}) +x^{9}(a+a^{3}) $

k10c33: $ 1 +x(-2a^{-3}-6a^{-1}-6a-2a^{3}) +x^{2}(3a^{-4}-6+3a^{4}) +x^{3}(-2a^{-5}+6a^{-3}+18a^{-1}+18a+6a^{3}-2a^{5}) +x^{4}(-7a^{-4}+a^{-2}+16+a^{2}-7a^{4}) +x^{5}(a^{-5}-9a^{-3}-16a^{-1}-16a-9a^{3}+a^{5}) +x^{6}(3a^{-4}-4a^{-2}-14-4a^{2}+3a^{4}) +x^{7}(4a^{-3}+5a^{-1}+5a+4a^{3}) +x^{8}(3a^{-2}+6+3a^{2}) +x^{9}(a^{-1}+a) $

k10c34: $ (a^{-6}+a^{-4}+2+a^{2}) +x(-3a^{-7}-4a^{-5}-a^{-3}-a-a^{3}) +x^{2}(-6a^{-6}-8a^{-4}-3a^{-2}-3-2a^{2}) +x^{3}(7a^{-7}+12a^{-5}+5a^{-3}-a^{-1}+a^{3}) +x^{4}(14a^{-6}+20a^{-4}+4a^{-2}+2a^{2}) +x^{5}(-5a^{-7}-6a^{-5}-5a^{-3}-2a^{-1}+2a) +x^{6}(-10a^{-6}-17a^{-4}-5a^{-2}+2) +x^{7}(a^{-7}-2a^{-5}-a^{-3}+2a^{-1}) +x^{8}(2a^{-6}+4a^{-4}+2a^{-2}) +x^{9}(a^{-5}+a^{-3}) $

k10c35: $ (-a^{-6}-a^{-4}+1+a^{2}+a^{4}) +x(-2a^{-5}-a^{-3}+a^{-1}+a+a^{3}) +x^{2}(4a^{-6}+3a^{-4}-3a^{-2}-3-3a^{2}-2a^{4}) +x^{3}(6a^{-5}+5a^{-3}-2a-3a^{3}) +x^{4}(-4a^{-6}+10a^{-2}+5+a^{4}) +x^{5}(-7a^{-5}-6a^{-3}-a^{-1}+2a^{3}) +x^{6}(a^{-6}-5a^{-4}-11a^{-2}-3+2a^{2}) +x^{7}(2a^{-5}+2a) +x^{8}(2a^{-4}+4a^{-2}+2) +x^{9}(a^{-3}+a^{-1}) $

k10c36: $ (1+a^{2}+2a^{4}+a^{6}) +x(a+a^{3}-3a^{5}-4a^{7}-a^{9}) +x^{2}(-2-3a^{2}-6a^{4}-8a^{6}-2a^{8}+a^{10}) +x^{3}(-3a-2a^{3}+8a^{5}+16a^{7}+9a^{9}) +x^{4}(1+6a^{4}+18a^{6}+8a^{8}-3a^{10}) +x^{5}(2a-6a^{5}-15a^{7}-11a^{9}) +x^{6}(2a^{2}-3a^{4}-16a^{6}-10a^{8}+a^{10}) +x^{7}(2a^{3}+a^{5}+2a^{7}+3a^{9}) +x^{8}(2a^{4}+5a^{6}+3a^{8}) +x^{9}(a^{5}+a^{7}) $

k10c37: $ (-a^{-4}-a^{-2}+1-a^{2}-a^{4}) +x(2a^{-5}+2a^{-3}-a^{-1}-a+2a^{3}+2a^{5}) +x^{2}(3a^{-4}-6+3a^{4}) +x^{3}(-3a^{-5}-3a^{-3}+a^{-1}+a-3a^{3}-3a^{5}) +x^{4}(-5a^{-4}+2a^{-2}+14+2a^{2}-5a^{4}) +x^{5}(a^{-5}-2a^{-3}-2a^{3}+a^{5}) +x^{6}(2a^{-4}-3a^{-2}-10-3a^{2}+2a^{4}) +x^{7}(2a^{-3}+2a^{3}) +x^{8}(2a^{-2}+4+2a^{2}) +x^{9}(a^{-1}+a) $

k10c38: $ (1-a^{2}-2a^{4}-a^{6}) +x(-a^{7}-a^{9}) +x^{2}(-2+2a^{2}+8a^{4}+2a^{6}+2a^{10}) +x^{3}(-2a+a^{3}+3a^{5}+8a^{7}+8a^{9}) +x^{4}(1-3a^{2}-8a^{4}+3a^{6}+4a^{8}-3a^{10}) +x^{5}(2a-2a^{3}-7a^{5}-13a^{7}-10a^{9}) +x^{6}(3a^{2}+2a^{4}-10a^{6}-8a^{8}+a^{10}) +x^{7}(3a^{3}+3a^{5}+3a^{7}+3a^{9}) +x^{8}(2a^{4}+5a^{6}+3a^{8}) +x^{9}(a^{5}+a^{7}) $

k10c39: $ (-2a^{2}-a^{4}) +x(-a^{5}+2a^{9}+a^{11}) +x^{2}(5a^{2}+5a^{4}-a^{6}+a^{8}+a^{10}-a^{12}) +x^{3}(4a^{3}+5a^{5}+4a^{7}-a^{9}-4a^{11}) +x^{4}(-4a^{2}-4a^{4}+5a^{6}-4a^{10}+a^{12}) +x^{5}(-6a^{3}-9a^{5}-9a^{7}-3a^{9}+3a^{11}) +x^{6}(a^{2}-3a^{4}-10a^{6}-2a^{8}+4a^{10}) +x^{7}(2a^{3}+2a^{5}+4a^{7}+4a^{9}) +x^{8}(2a^{4}+5a^{6}+3a^{8}) +x^{9}(a^{5}+a^{7}) $

k10c40: $ (a^{-6}-3a^{-2}-1) +x(a^{-9}+2a^{-3}+2a^{-1}+a) +x^{2}(3a^{-8}+a^{-6}+a^{-4}+7a^{-2}+4) +x^{3}(-2a^{-9}+2a^{-7}+6a^{-5}+3a^{-3}-a^{-1}-2a) +x^{4}(-6a^{-8}-5a^{-6}-2a^{-4}-9a^{-2}-6) +x^{5}(a^{-9}-6a^{-7}-13a^{-5}-12a^{-3}-5a^{-1}+a) +x^{6}(3a^{-8}-5a^{-4}+a^{-2}+3) +x^{7}(4a^{-7}+7a^{-5}+7a^{-3}+4a^{-1}) +x^{8}(3a^{-6}+6a^{-4}+3a^{-2}) +x^{9}(a^{-5}+a^{-3}) $

k10c41: $ (-a^{-2}-1-2a^{2}-2a^{4}-a^{6}) +x(-a^{-1}-2a-2a^{3}+a^{7}) +x^{2}(3a^{-2}+7+9a^{2}+10a^{4}+4a^{6}-a^{8}) +x^{3}(7a^{-1}+13a+10a^{3}+a^{5}-3a^{7}) +x^{4}(-3a^{-2}-4-8a^{2}-14a^{4}-6a^{6}+a^{8}) +x^{5}(-9a^{-1}-20a-18a^{3}-4a^{5}+3a^{7}) +x^{6}(a^{-2}-5-7a^{2}+4a^{4}+5a^{6}) +x^{7}(3a^{-1}+6a+8a^{3}+5a^{5}) +x^{8}(3+6a^{2}+3a^{4}) +x^{9}(a+a^{3}) $

k10c42: $ (-a^{-4}-3a^{-2}-2-a^{2}) +x(a^{-5}+a^{-3}-a^{-1}-a) +x^{2}(4a^{-4}+9a^{-2}+9+6a^{2}+2a^{4}) +x^{3}(-2a^{-5}+10a^{-1}+14a+5a^{3}-a^{5}) +x^{4}(-6a^{-4}-11a^{-2}-8-10a^{2}-7a^{4}) +x^{5}(a^{-5}-5a^{-3}-18a^{-1}-24a-11a^{3}+a^{5}) +x^{6}(3a^{-4}+2a^{-2}-5+4a^{4}) +x^{7}(4a^{-3}+9a^{-1}+11a+6a^{3}) +x^{8}(3a^{-2}+7+4a^{2}) +x^{9}(a^{-1}+a) $

k10c43: $ (-a^{-4}-2a^{-2}-1-2a^{2}-a^{4}) +x(a^{-5}-3a^{-1}-3a+a^{5}) +x^{2}(3a^{-4}+7a^{-2}+8+7a^{2}+3a^{4}) +x^{3}(-2a^{-5}+a^{-3}+12a^{-1}+12a+a^{3}-2a^{5}) +x^{4}(-6a^{-4}-8a^{-2}-4-8a^{2}-6a^{4}) +x^{5}(a^{-5}-6a^{-3}-16a^{-1}-16a-6a^{3}+a^{5}) +x^{6}(3a^{-4}-6+3a^{4}) +x^{7}(4a^{-3}+7a^{-1}+7a+4a^{3}) +x^{8}(3a^{-2}+6+3a^{2}) +x^{9}(a^{-1}+a) $

k10c44: $ (-a^{-2}-2-3a^{2}-a^{4}) +x(-2a^{-1}-4a-2a^{3}) +x^{2}(3a^{-2}+9+13a^{2}+10a^{4}+3a^{6}) +x^{3}(8a^{-1}+20a+15a^{3}-3a^{7}) +x^{4}(-3a^{-2}-6-12a^{2}-18a^{4}-8a^{6}+a^{8}) +x^{5}(-9a^{-1}-26a-27a^{3}-6a^{5}+4a^{7}) +x^{6}(a^{-2}-4-7a^{2}+5a^{4}+7a^{6}) +x^{7}(3a^{-1}+8a+12a^{3}+7a^{5}) +x^{8}(3+7a^{2}+4a^{4}) +x^{9}(a+a^{3}) $

k10c45: $ (-2a^{-2}-3-2a^{2}) +x(-a^{-3}-5a^{-1}-5a-a^{3}) +x^{2}(3a^{-4}+12a^{-2}+18+12a^{2}+3a^{4}) +x^{3}(-a^{-5}+5a^{-3}+21a^{-1}+21a+5a^{3}-a^{5}) +x^{4}(-7a^{-4}-17a^{-2}-20-17a^{2}-7a^{4}) +x^{5}(a^{-5}-10a^{-3}-31a^{-1}-31a-10a^{3}+a^{5}) +x^{6}(4a^{-4}+3a^{-2}-2+3a^{2}+4a^{4}) +x^{7}(6a^{-3}+14a^{-1}+14a+6a^{3}) +x^{8}(4a^{-2}+8+4a^{2}) +x^{9}(a^{-1}+a) $

k10c46: $ (3a^{-8}+8a^{-6}+6a^{-4}) +x(2a^{-11}-2a^{-9}-10a^{-7}-6a^{-5}) +x^{2}(a^{-14}-2a^{-12}+2a^{-10}-7a^{-8}-29a^{-6}-17a^{-4}) +x^{3}(2a^{-13}-7a^{-11}+9a^{-9}+23a^{-7}+5a^{-5}) +x^{4}(3a^{-12}-9a^{-10}+13a^{-8}+42a^{-6}+17a^{-4}) +x^{5}(4a^{-11}-13a^{-9}-12a^{-7}+5a^{-5}) +x^{6}(4a^{-10}-12a^{-8}-23a^{-6}-7a^{-4}) +x^{7}(4a^{-9}-a^{-7}-5a^{-5}) +x^{8}(3a^{-8}+4a^{-6}+a^{-4}) +x^{9}(a^{-7}+a^{-5}) $

k10c47: $ (5a^{-6}+9a^{-4}+3a^{-2}) +x(-a^{-11}+2a^{-9}-a^{-7}-9a^{-5}-8a^{-3}-3a^{-1}) +x^{2}(-a^{-10}+a^{-8}-15a^{-6}-26a^{-4}-9a^{-2}) +x^{3}(a^{-11}-3a^{-9}+2a^{-7}+19a^{-5}+20a^{-3}+7a^{-1}) +x^{4}(2a^{-10}-3a^{-8}+15a^{-6}+35a^{-4}+15a^{-2}) +x^{5}(3a^{-9}-5a^{-7}-14a^{-5}-11a^{-3}-5a^{-1}) +x^{6}(3a^{-8}-10a^{-6}-23a^{-4}-10a^{-2}) +x^{7}(3a^{-7}+a^{-5}-a^{-3}+a^{-1}) +x^{8}(3a^{-6}+5a^{-4}+2a^{-2}) +x^{9}(a^{-5}+a^{-3}) $

k10c48: $ (4a^{-2}+9+4a^{2}) +x(a^{-5}-3a^{-3}-9a^{-1}-7a+2a^{5}) +x^{2}(a^{-4}-13a^{-2}-27-11a^{2}+2a^{4}) +x^{3}(-3a^{-5}+8a^{-3}+21a^{-1}+12a-a^{3}-3a^{5}) +x^{4}(-5a^{-4}+18a^{-2}+37+9a^{2}-5a^{4}) +x^{5}(a^{-5}-9a^{-3}-11a^{-1}-5a-3a^{3}+a^{5}) +x^{6}(2a^{-4}-11a^{-2}-20-5a^{2}+2a^{4}) +x^{7}(3a^{-3}+a^{-1}+2a^{3}) +x^{8}(3a^{-2}+5+2a^{2}) +x^{9}(a^{-1}+a) $

k10c49: $ (-a^{6}+5a^{8}+7a^{10}+2a^{12}) +x(-9a^{9}-10a^{11}+a^{15}) +x^{2}(4a^{6}-13a^{8}-20a^{10}-2a^{12}-a^{16}) +x^{3}(3a^{7}+22a^{9}+24a^{11}+a^{13}-4a^{15}) +x^{4}(-4a^{6}+15a^{8}+26a^{10}+2a^{12}-4a^{14}+a^{16}) +x^{5}(-6a^{7}-18a^{9}-19a^{11}-4a^{13}+3a^{15}) +x^{6}(a^{6}-11a^{8}-19a^{10}-3a^{12}+4a^{14}) +x^{7}(2a^{7}+3a^{9}+5a^{11}+4a^{13}) +x^{8}(3a^{8}+6a^{10}+3a^{12}) +x^{9}(a^{9}+a^{11}) $

k10c50: $ (2a^{-8}+4a^{-6}+a^{-4}-2a^{-2}) +x(-6a^{-9}-10a^{-7}-3a^{-5}+a^{-3}) +x^{2}(-2a^{-12}+3a^{-10}-3a^{-8}-13a^{-6}+5a^{-2}) +x^{3}(-3a^{-11}+16a^{-9}+22a^{-7}+6a^{-5}+3a^{-3}) +x^{4}(a^{-12}-5a^{-10}+9a^{-8}+18a^{-6}-a^{-4}-4a^{-2}) +x^{5}(2a^{-11}-11a^{-9}-15a^{-7}-8a^{-5}-6a^{-3}) +x^{6}(3a^{-10}-7a^{-8}-15a^{-6}-4a^{-4}+a^{-2}) +x^{7}(4a^{-9}+3a^{-7}+a^{-5}+2a^{-3}) +x^{8}(3a^{-8}+5a^{-6}+2a^{-4}) +x^{9}(a^{-7}+a^{-5}) $

k10c51: $ (3a^{-6}+4a^{-4}-a^{-2}-1) +x(2a^{-9}-3a^{-7}-9a^{-5}-5a^{-3}+a) +x^{2}(a^{-8}-8a^{-6}-8a^{-4}+4a^{-2}+3) +x^{3}(-3a^{-9}+5a^{-7}+21a^{-5}+15a^{-3}-2a) +x^{4}(-4a^{-8}+9a^{-6}+13a^{-4}-6a^{-2}-6) +x^{5}(a^{-9}-6a^{-7}-16a^{-5}-16a^{-3}-6a^{-1}+a) +x^{6}(2a^{-8}-6a^{-6}-12a^{-4}-a^{-2}+3) +x^{7}(3a^{-7}+5a^{-5}+6a^{-3}+4a^{-1}) +x^{8}(3a^{-6}+6a^{-4}+3a^{-2}) +x^{9}(a^{-5}+a^{-3}) $

k10c52: $ (-a^{-4}+4+2a^{2}) +x(2a^{-5}-7a^{-1}-9a-4a^{3}) +x^{2}(6a^{-4}+4a^{-2}-9-7a^{2}) +x^{3}(a^{-7}-5a^{-5}+2a^{-3}+24a^{-1}+24a+8a^{3}) +x^{4}(3a^{-6}-12a^{-4}-9a^{-2}+19+13a^{2}) +x^{5}(6a^{-5}-11a^{-3}-28a^{-1}-16a-5a^{3}) +x^{6}(8a^{-4}-3a^{-2}-20-9a^{2}) +x^{7}(7a^{-3}+7a^{-1}+a+a^{3}) +x^{8}(4a^{-2}+6+2a^{2}) +x^{9}(a^{-1}+a) $

k10c53: $ (-3a^{6}+3a^{10}+a^{12}) +x(a^{7}-7a^{9}-11a^{11}-3a^{13}) +x^{2}(-a^{4}+8a^{6}+4a^{8}-5a^{10}+2a^{12}+2a^{14}) +x^{3}(-2a^{5}+a^{7}+21a^{9}+28a^{11}+10a^{13}) +x^{4}(a^{4}-9a^{6}-7a^{8}+6a^{10}-3a^{14}) +x^{5}(3a^{5}-6a^{7}-26a^{9}-27a^{11}-10a^{13}) +x^{6}(6a^{6}-13a^{10}-6a^{12}+a^{14}) +x^{7}(6a^{7}+10a^{9}+7a^{11}+3a^{13}) +x^{8}(4a^{8}+7a^{10}+3a^{12}) +x^{9}(a^{9}+a^{11}) $

k10c54: $ (-2a^{-4}-2a^{-2}+3+2a^{2}) +x(-a^{-7}+a^{-5}+a^{-3}-5a^{-1}-8a-4a^{3}) +x^{2}(-a^{-6}+5a^{-4}+5a^{-2}-7-6a^{2}) +x^{3}(a^{-7}-2a^{-5}+2a^{-3}+17a^{-1}+20a+8a^{3}) +x^{4}(2a^{-6}-6a^{-4}-3a^{-2}+17+12a^{2}) +x^{5}(3a^{-5}-7a^{-3}-18a^{-1}-13a-5a^{3}) +x^{6}(4a^{-4}-5a^{-2}-18-9a^{2}) +x^{7}(4a^{-3}+3a^{-1}+a^{3}) +x^{8}(3a^{-2}+5+2a^{2}) +x^{9}(a^{-1}+a) $

k10c55: $ (a^{4}-a^{6}+a^{8}+3a^{10}+a^{12}) +x(2a^{7}-4a^{9}-9a^{11}-3a^{13}) +x^{2}(-2a^{4}+2a^{6}-3a^{8}-8a^{10}+a^{12}+2a^{14}) +x^{3}(-2a^{5}-2a^{7}+15a^{9}+24a^{11}+9a^{13}) +x^{4}(a^{4}-3a^{6}+5a^{8}+13a^{10}+a^{12}-3a^{14}) +x^{5}(2a^{5}-a^{7}-16a^{9}-23a^{11}-10a^{13}) +x^{6}(3a^{6}-4a^{8}-15a^{10}-7a^{12}+a^{14}) +x^{7}(3a^{7}+5a^{9}+5a^{11}+3a^{13}) +x^{8}(3a^{8}+6a^{10}+3a^{12}) +x^{9}(a^{9}+a^{11}) $

k10c56: $ (a^{-8}+2a^{-6}-2a^{-2}) +x(-4a^{-9}-8a^{-7}-4a^{-5}) +x^{2}(-a^{-12}+2a^{-10}-2a^{-8}-7a^{-6}+3a^{-4}+5a^{-2}) +x^{3}(-3a^{-11}+11a^{-9}+21a^{-7}+11a^{-5}+4a^{-3}) +x^{4}(a^{-12}-6a^{-10}+4a^{-8}+12a^{-6}-3a^{-4}-4a^{-2}) +x^{5}(3a^{-11}-11a^{-9}-21a^{-7}-13a^{-5}-6a^{-3}) +x^{6}(5a^{-10}-5a^{-8}-14a^{-6}-3a^{-4}+a^{-2}) +x^{7}(6a^{-9}+7a^{-7}+3a^{-5}+2a^{-3}) +x^{8}(4a^{-8}+6a^{-6}+2a^{-4}) +x^{9}(a^{-7}+a^{-5}) $

k10c57: $ (2a^{-6}+2a^{-4}-2a^{-2}-1) +x(a^{-9}-3a^{-7}-6a^{-5}-2a^{-3}+a^{-1}+a) +x^{2}(2a^{-8}-2a^{-6}+8a^{-2}+4) +x^{3}(-2a^{-9}+6a^{-7}+18a^{-5}+12a^{-3}-2a) +x^{4}(-5a^{-8}-a^{-6}-a^{-4}-11a^{-2}-6) +x^{5}(a^{-9}-9a^{-7}-23a^{-5}-19a^{-3}-5a^{-1}+a) +x^{6}(3a^{-8}-3a^{-6}-7a^{-4}+2a^{-2}+3) +x^{7}(5a^{-7}+10a^{-5}+9a^{-3}+4a^{-1}) +x^{8}(4a^{-6}+7a^{-4}+3a^{-2}) +x^{9}(a^{-5}+a^{-3}) $

k10c58: $ (a^{-4}-2-3a^{2}-2a^{4}-a^{6}) +x(-4a^{-1}-6a-4a^{3}-2a^{5}) +x^{2}(-2a^{-4}+8+10a^{2}+7a^{4}+3a^{6}) +x^{3}(-2a^{-3}+8a^{-1}+21a+18a^{3}+7a^{5}) +x^{4}(a^{-4}-2a^{-2}-5-4a^{2}-5a^{4}-3a^{6}) +x^{5}(2a^{-3}-6a^{-1}-22a-23a^{3}-9a^{5}) +x^{6}(3a^{-2}-1-10a^{2}-5a^{4}+a^{6}) +x^{7}(4a^{-1}+7a+6a^{3}+3a^{5}) +x^{8}(3+6a^{2}+3a^{4}) +x^{9}(a+a^{3}) $

k10c59: $ (-a^{-6}-3a^{-4}-4a^{-2}-2-a^{2}) +x(a^{-7}-4a^{-3}-5a^{-1}-2a) +x^{2}(-a^{-8}+3a^{-6}+10a^{-4}+11a^{-2}+8+3a^{2}) +x^{3}(-3a^{-7}+4a^{-5}+20a^{-3}+21a^{-1}+8a) +x^{4}(a^{-8}-5a^{-6}-11a^{-4}-8a^{-2}-6-3a^{2}) +x^{5}(3a^{-7}-7a^{-5}-28a^{-3}-27a^{-1}-9a) +x^{6}(5a^{-6}+a^{-4}-9a^{-2}-4+a^{2}) +x^{7}(6a^{-5}+11a^{-3}+8a^{-1}+3a) +x^{8}(4a^{-4}+7a^{-2}+3) +x^{9}(a^{-3}+a^{-1}) $

k10c60: $ (-a^{-2}-2-4a^{2}-3a^{4}-a^{6}) +x(-2a^{-1}-6a-7a^{3}-3a^{5}) +x^{2}(4a^{-2}+14+18a^{2}+11a^{4}+3a^{6}) +x^{3}(-2a^{-3}+5a^{-1}+25a+27a^{3}+9a^{5}) +x^{4}(a^{-4}-9a^{-2}-22-17a^{2}-8a^{4}-3a^{6}) +x^{5}(4a^{-3}-11a^{-1}-38a-32a^{3}-9a^{5}) +x^{6}(8a^{-2}+5-7a^{2}-3a^{4}+a^{6}) +x^{7}(9a^{-1}+16a+10a^{3}+3a^{5}) +x^{8}(5+8a^{2}+3a^{4}) +x^{9}(a+a^{3}) $

k10c61: $ (-a^{-6}+a^{-4}+5a^{-2}+4) +x(-6a^{-5}-8a^{-3}-2a^{-1}) +x^{2}(a^{-10}-2a^{-8}+6a^{-6}+a^{-4}-24a^{-2}-16) +x^{3}(2a^{-9}-6a^{-7}+17a^{-5}+26a^{-3}+a^{-1}) +x^{4}(3a^{-8}-13a^{-6}+5a^{-4}+38a^{-2}+17) +x^{5}(4a^{-7}-18a^{-5}-16a^{-3}+6a^{-1}) +x^{6}(5a^{-6}-10a^{-4}-22a^{-2}-7) +x^{7}(5a^{-5}-5a^{-1}) +x^{8}(3a^{-4}+4a^{-2}+1) +x^{9}(a^{-3}+a^{-1}) $

k10c62: $ (4a^{-6}+7a^{-4}+2a^{-2}) +x(-a^{-11}+a^{-9}-a^{-7}-6a^{-5}-5a^{-3}-2a^{-1}) +x^{2}(-a^{-10}+4a^{-8}-8a^{-6}-23a^{-4}-10a^{-2}) +x^{3}(a^{-11}-2a^{-9}+5a^{-7}+16a^{-5}+15a^{-3}+7a^{-1}) +x^{4}(2a^{-10}-6a^{-8}+6a^{-6}+30a^{-4}+16a^{-2}) +x^{5}(3a^{-9}-8a^{-7}-15a^{-5}-9a^{-3}-5a^{-1}) +x^{6}(4a^{-8}-7a^{-6}-21a^{-4}-10a^{-2}) +x^{7}(4a^{-7}+2a^{-5}-a^{-3}+a^{-1}) +x^{8}(3a^{-6}+5a^{-4}+2a^{-2}) +x^{9}(a^{-5}+a^{-3}) $

k10c63: $ (a^{4}+3a^{8}+4a^{10}+a^{12}) +x(-8a^{9}-10a^{11}-2a^{13}) +x^{2}(-2a^{4}+a^{6}-10a^{8}-16a^{10}-2a^{12}+a^{14}) +x^{3}(-2a^{5}+20a^{9}+28a^{11}+10a^{13}) +x^{4}(a^{4}-3a^{6}+11a^{8}+24a^{10}+6a^{12}-3a^{14}) +x^{5}(2a^{5}-2a^{7}-16a^{9}-23a^{11}-11a^{13}) +x^{6}(3a^{6}-6a^{8}-19a^{10}-9a^{12}+a^{14}) +x^{7}(3a^{7}+4a^{9}+4a^{11}+3a^{13}) +x^{8}(3a^{8}+6a^{10}+3a^{12}) +x^{9}(a^{9}+a^{11}) $

k10c64: $ (3a^{-4}+6a^{-2}+4) +x(-4a^{-5}-6a^{-3}-3a^{-1}-a) +x^{2}(-2a^{-8}+3a^{-6}-8a^{-4}-26a^{-2}-9+4a^{2}) +x^{3}(-3a^{-7}+15a^{-5}+16a^{-3}+4a^{-1}+6a) +x^{4}(a^{-8}-5a^{-6}+13a^{-4}+30a^{-2}+7-4a^{2}) +x^{5}(2a^{-7}-11a^{-5}-11a^{-3}-5a^{-1}-7a) +x^{6}(3a^{-6}-8a^{-4}-18a^{-2}-6+a^{2}) +x^{7}(4a^{-5}+2a^{-3}+2a) +x^{8}(3a^{-4}+5a^{-2}+2) +x^{9}(a^{-3}+a^{-1}) $

k10c65: $ (3a^{-6}+5a^{-4}+a^{-2}) +x(2a^{-9}-2a^{-7}-8a^{-5}-6a^{-3}-2a^{-1}) +x^{2}(a^{-8}-12a^{-6}-17a^{-4}-a^{-2}+3) +x^{3}(-3a^{-9}+4a^{-7}+19a^{-5}+20a^{-3}+6a^{-1}-2a) +x^{4}(-4a^{-8}+12a^{-6}+24a^{-4}+a^{-2}-7) +x^{5}(a^{-9}-6a^{-7}-14a^{-5}-17a^{-3}-9a^{-1}+a) +x^{6}(2a^{-8}-7a^{-6}-16a^{-4}-4a^{-2}+3) +x^{7}(3a^{-7}+4a^{-5}+5a^{-3}+4a^{-1}) +x^{8}(3a^{-6}+6a^{-4}+3a^{-2}) +x^{9}(a^{-5}+a^{-3}) $

k10c66: $ (-2a^{6}+2a^{8}+4a^{10}+a^{12}) +x(a^{7}-5a^{9}-6a^{11}) +x^{2}(5a^{6}-6a^{8}-8a^{10}+5a^{12}+2a^{14}) +x^{3}(2a^{7}+20a^{9}+22a^{11}+a^{13}-3a^{15}) +x^{4}(-4a^{6}+8a^{8}+8a^{10}-13a^{12}-8a^{14}+a^{16}) +x^{5}(-5a^{7}-22a^{9}-28a^{11}-7a^{13}+4a^{15}) +x^{6}(a^{6}-8a^{8}-13a^{10}+3a^{12}+7a^{14}) +x^{7}(2a^{7}+6a^{9}+11a^{11}+7a^{13}) +x^{8}(3a^{8}+7a^{10}+4a^{12}) +x^{9}(a^{9}+a^{11}) $

k10c67: $ 1 +x(-2a^{3}-6a^{5}-6a^{7}-2a^{9}) +x^{2}(-2+2a^{4}-2a^{6}+2a^{10}) +x^{3}(-2a+7a^{3}+19a^{5}+19a^{7}+9a^{9}) +x^{4}(1-2a^{2}-a^{4}+7a^{6}+2a^{8}-3a^{10}) +x^{5}(2a-6a^{3}-19a^{5}-21a^{7}-10a^{9}) +x^{6}(3a^{2}-2a^{4}-13a^{6}-7a^{8}+a^{10}) +x^{7}(4a^{3}+6a^{5}+5a^{7}+3a^{9}) +x^{8}(3a^{4}+6a^{6}+3a^{8}) +x^{9}(a^{5}+a^{7}) $

k10c68: $ (-a^{2}+a^{4}+a^{6}) +x(-2a-6a^{3}-8a^{5}-4a^{7}) +x^{2}(-a^{-2}+4+7a^{2}-5a^{4}-7a^{6}) +x^{3}(a^{-3}-3a^{-1}+8a+27a^{3}+23a^{5}+8a^{7}) +x^{4}(3a^{-2}-10-9a^{2}+17a^{4}+13a^{6}) +x^{5}(5a^{-1}-14a-30a^{3}-16a^{5}-5a^{7}) +x^{6}(7-4a^{2}-20a^{4}-9a^{6}) +x^{7}(7a+7a^{3}+a^{5}+a^{7}) +x^{8}(4a^{2}+6a^{4}+2a^{6}) +x^{9}(a^{3}+a^{5}) $

k10c69: $ (-a^{-8}-2a^{-6}-2a^{-4}-2a^{-2}) +x(a^{-9}-2a^{-7}-6a^{-5}-4a^{-3}-a^{-1}) +x^{2}(3a^{-8}+7a^{-6}+12a^{-4}+11a^{-2}+3) +x^{3}(-2a^{-9}+5a^{-7}+23a^{-5}+22a^{-3}+5a^{-1}-a) +x^{4}(-5a^{-8}-9a^{-6}-14a^{-4}-17a^{-2}-7) +x^{5}(a^{-9}-8a^{-7}-30a^{-5}-32a^{-3}-10a^{-1}+a) +x^{6}(3a^{-8}-4a^{-4}+3a^{-2}+4) +x^{7}(5a^{-7}+13a^{-5}+14a^{-3}+6a^{-1}) +x^{8}(4a^{-6}+8a^{-4}+4a^{-2}) +x^{9}(a^{-5}+a^{-3}) $

k10c70: $ (-a^{-6}-2a^{-4}-3a^{-2}-3-2a^{2}) +x(a^{-7}+a^{-5}+a^{-3}-a) +x^{2}(-a^{-8}+4a^{-6}+9a^{-4}+9a^{-2}+10+5a^{2}) +x^{3}(-3a^{-7}+2a^{-3}+4a^{-1}+5a) +x^{4}(a^{-8}-6a^{-6}-12a^{-4}-8a^{-2}-7-4a^{2}) +x^{5}(3a^{-7}-4a^{-5}-11a^{-3}-10a^{-1}-6a) +x^{6}(5a^{-6}+3a^{-4}-5a^{-2}-2+a^{2}) +x^{7}(5a^{-5}+6a^{-3}+3a^{-1}+2a) +x^{8}(3a^{-4}+5a^{-2}+2) +x^{9}(a^{-3}+a^{-1}) $

k10c71: $ (-a^{-4}-3a^{-2}-3-3a^{2}-a^{4}) +x(a^{-5}+a^{-3}-a^{-1}-a+a^{3}+a^{5}) +x^{2}(4a^{-4}+10a^{-2}+12+10a^{2}+4a^{4}) +x^{3}(-2a^{-5}+7a^{-1}+7a-2a^{5}) +x^{4}(-6a^{-4}-12a^{-2}-12-12a^{2}-6a^{4}) +x^{5}(a^{-5}-5a^{-3}-15a^{-1}-15a-5a^{3}+a^{5}) +x^{6}(3a^{-4}+2a^{-2}-2+2a^{2}+3a^{4}) +x^{7}(4a^{-3}+8a^{-1}+8a+4a^{3}) +x^{8}(3a^{-2}+6+3a^{2}) +x^{9}(a^{-1}+a) $

k10c72: $ (-a^{-8}-2a^{-6}-2a^{-4}-2a^{-2}) +x(a^{-9}-a^{-7}-3a^{-5}-a^{-3}) +x^{2}(2a^{-10}+6a^{-8}+7a^{-6}+8a^{-4}+5a^{-2}) +x^{3}(-3a^{-11}+9a^{-7}+11a^{-5}+5a^{-3}) +x^{4}(a^{-12}-8a^{-10}-11a^{-8}-4a^{-6}-6a^{-4}-4a^{-2}) +x^{5}(4a^{-11}-7a^{-9}-19a^{-7}-14a^{-5}-6a^{-3}) +x^{6}(7a^{-10}+2a^{-8}-8a^{-6}-2a^{-4}+a^{-2}) +x^{7}(7a^{-9}+9a^{-7}+4a^{-5}+2a^{-3}) +x^{8}(4a^{-8}+6a^{-6}+2a^{-4}) +x^{9}(a^{-7}+a^{-5}) $

k10c73: $ (-3a^{2}-4a^{4}-3a^{6}-a^{8}) +x(-a-3a^{3}-3a^{5}+a^{9}) +x^{2}(3+12a^{2}+17a^{4}+12a^{6}+4a^{8}) +x^{3}(-a^{-1}+4a+16a^{3}+14a^{5}+a^{7}-2a^{9}) +x^{4}(-7-16a^{2}-17a^{4}-14a^{6}-6a^{8}) +x^{5}(a^{-1}-10a-26a^{3}-21a^{5}-5a^{7}+a^{9}) +x^{6}(4+2a^{2}-2a^{4}+3a^{6}+3a^{8}) +x^{7}(6a+12a^{3}+10a^{5}+4a^{7}) +x^{8}(4a^{2}+7a^{4}+3a^{6}) +x^{9}(a^{3}+a^{5}) $

k10c74: $ (-2a^{2}+2a^{6}+a^{8}) +x(-4a^{5}-8a^{7}-4a^{9}) +x^{2}(-1+5a^{2}+8a^{4}-a^{6}+a^{8}+4a^{10}) +x^{3}(-3a+3a^{3}+9a^{5}+11a^{7}+8a^{9}) +x^{4}(1-7a^{2}-9a^{4}+3a^{6}-4a^{10}) +x^{5}(3a-6a^{3}-12a^{5}-10a^{7}-7a^{9}) +x^{6}(5a^{2}+a^{4}-9a^{6}-4a^{8}+a^{10}) +x^{7}(5a^{3}+5a^{5}+2a^{7}+2a^{9}) +x^{8}(3a^{4}+5a^{6}+2a^{8}) +x^{9}(a^{5}+a^{7}) $

k10c75: $ (-a^{-6}-3a^{-4}-3a^{-2}) +x(-3a^{-5}-7a^{-3}-5a^{-1}-a) +x^{2}(3a^{-6}+12a^{-4}+20a^{-2}+15+4a^{2}) +x^{3}(9a^{-5}+24a^{-3}+17a^{-1}-a-3a^{3}) +x^{4}(-3a^{-6}-9a^{-4}-21a^{-2}-24-8a^{2}+a^{4}) +x^{5}(-9a^{-5}-29a^{-3}-29a^{-1}-5a+4a^{3}) +x^{6}(a^{-6}-3a^{-4}-4a^{-2}+7+7a^{2}) +x^{7}(3a^{-5}+9a^{-3}+13a^{-1}+7a) +x^{8}(3a^{-4}+7a^{-2}+4) +x^{9}(a^{-3}+a^{-1}) $

k10c76: $ (a^{-8}-4a^{-4}-4a^{-2}) +x(-2a^{-9}+2a^{-7}+8a^{-5}+4a^{-3}) +x^{2}(-a^{-12}+3a^{-10}-4a^{-8}-7a^{-6}+9a^{-4}+8a^{-2}) +x^{3}(-3a^{-11}+7a^{-9}-3a^{-7}-15a^{-5}-2a^{-3}) +x^{4}(a^{-12}-7a^{-10}+4a^{-8}+10a^{-6}-7a^{-4}-5a^{-2}) +x^{5}(3a^{-11}-8a^{-9}-2a^{-7}+7a^{-5}-2a^{-3}) +x^{6}(5a^{-10}-3a^{-8}-9a^{-6}+a^{-2}) +x^{7}(5a^{-9}+2a^{-7}-2a^{-5}+a^{-3}) +x^{8}(3a^{-8}+4a^{-6}+a^{-4}) +x^{9}(a^{-7}+a^{-5}) $

k10c77: $ (a^{-6}-a^{-4}-5a^{-2}-2) +x(a^{-9}-a^{-7}-a^{-5}+3a^{-3}+4a^{-1}+2a) +x^{2}(2a^{-8}-2a^{-6}-a^{-4}+7a^{-2}+4) +x^{3}(-2a^{-9}+2a^{-7}+6a^{-5}-5a^{-1}-3a) +x^{4}(-6a^{-8}+8a^{-4}-3a^{-2}-5) +x^{5}(a^{-9}-7a^{-7}-9a^{-5}-3a^{-3}-a^{-1}+a) +x^{6}(3a^{-8}-3a^{-6}-9a^{-4}-a^{-2}+2) +x^{7}(4a^{-7}+4a^{-5}+2a^{-3}+2a^{-1}) +x^{8}(3a^{-6}+5a^{-4}+2a^{-2}) +x^{9}(a^{-5}+a^{-3}) $

k10c78: $ (-a^{2}-a^{4}-4a^{6}-4a^{8}-a^{10}) +x(-a^{3}-3a^{5}+2a^{7}+6a^{9}+2a^{11}) +x^{2}(3a^{2}+6a^{4}+11a^{6}+10a^{8}+a^{10}-a^{12}) +x^{3}(7a^{3}+15a^{5}+5a^{7}-7a^{9}-4a^{11}) +x^{4}(-3a^{2}-4a^{4}-7a^{6}-10a^{8}-3a^{10}+a^{12}) +x^{5}(-9a^{3}-21a^{5}-15a^{7}+3a^{11}) +x^{6}(a^{2}-5a^{4}-8a^{6}+2a^{8}+4a^{10}) +x^{7}(3a^{3}+6a^{5}+7a^{7}+4a^{9}) +x^{8}(3a^{4}+6a^{6}+3a^{8}) +x^{9}(a^{5}+a^{7}) $

k10c79: $ (5a^{-2}+11+5a^{2}) +x(2a^{-5}-2a^{-3}-11a^{-1}-11a-2a^{3}+2a^{5}) +x^{2}(a^{-4}-13a^{-2}-28-13a^{2}+a^{4}) +x^{3}(-3a^{-5}+4a^{-3}+22a^{-1}+22a+4a^{3}-3a^{5}) +x^{4}(-4a^{-4}+12a^{-2}+32+12a^{2}-4a^{4}) +x^{5}(a^{-5}-6a^{-3}-15a^{-1}-15a-6a^{3}+a^{5}) +x^{6}(2a^{-4}-7a^{-2}-18-7a^{2}+2a^{4}) +x^{7}(3a^{-3}+4a^{-1}+4a+3a^{3}) +x^{8}(3a^{-2}+6+3a^{2}) +x^{9}(a^{-1}+a) $

k10c80: $ (-2a^{6}+3a^{8}+6a^{10}+2a^{12}) +x(a^{7}-8a^{9}-12a^{11}-2a^{13}+a^{15}) +x^{2}(5a^{6}-7a^{8}-13a^{10}+2a^{12}+2a^{14}-a^{16}) +x^{3}(2a^{7}+22a^{9}+29a^{11}+6a^{13}-3a^{15}) +x^{4}(-4a^{6}+8a^{8}+13a^{10}-5a^{12}-5a^{14}+a^{16}) +x^{5}(-5a^{7}-23a^{9}-29a^{11}-8a^{13}+3a^{15}) +x^{6}(a^{6}-8a^{8}-15a^{10}-a^{12}+5a^{14}) +x^{7}(2a^{7}+6a^{9}+10a^{11}+6a^{13}) +x^{8}(3a^{8}+7a^{10}+4a^{12}) +x^{9}(a^{9}+a^{11}) $

k10c81: $ (-a^{-4}-a^{-2}+1-a^{2}-a^{4}) +x(a^{-5}-2a^{-3}-8a^{-1}-8a-2a^{3}+a^{5}) +x^{2}(3a^{-4}+6a^{-2}+6+6a^{2}+3a^{4}) +x^{3}(-2a^{-5}+5a^{-3}+25a^{-1}+25a+5a^{3}-2a^{5}) +x^{4}(-5a^{-4}-9a^{-2}-8-9a^{2}-5a^{4}) +x^{5}(a^{-5}-8a^{-3}-31a^{-1}-31a-8a^{3}+a^{5}) +x^{6}(3a^{-4}-6+3a^{4}) +x^{7}(5a^{-3}+13a^{-1}+13a+5a^{3}) +x^{8}(4a^{-2}+8+4a^{2}) +x^{9}(a^{-1}+a) $

k10c82: $ 1 +x(-a^{-1}-2a+2a^{5}+a^{7}) +x^{2}(a^{-2}-6-13a^{2}-5a^{4}-a^{8}) +x^{3}(7a^{-1}+10a+5a^{3}-2a^{5}-4a^{7}) +x^{4}(-3a^{-2}+14+32a^{2}+10a^{4}-4a^{6}+a^{8}) +x^{5}(-10a^{-1}-8a-4a^{3}-3a^{5}+3a^{7}) +x^{6}(a^{-2}-14-27a^{2}-8a^{4}+4a^{6}) +x^{7}(3a^{-1}-2a-a^{3}+4a^{5}) +x^{8}(4+8a^{2}+4a^{4}) +x^{9}(2a+2a^{3}) $

k10c83: $ (a^{-4}+2a^{-2}+2) +x(-2a^{-5}-4a^{-3}-3a^{-1}-a) +x^{2}(2a^{-6}-2a^{-4}-6a^{-2}+1+3a^{2}) +x^{3}(8a^{-5}+15a^{-3}+13a^{-1}+4a-2a^{3}) +x^{4}(-3a^{-6}+7a^{-4}+13a^{-2}-7-9a^{2}+a^{4}) +x^{5}(-9a^{-5}-17a^{-3}-23a^{-1}-11a+4a^{3}) +x^{6}(a^{-6}-10a^{-4}-21a^{-2}-2+8a^{2}) +x^{7}(3a^{-5}+3a^{-3}+9a^{-1}+9a) +x^{8}(4a^{-4}+10a^{-2}+6) +x^{9}(2a^{-3}+2a^{-1}) $

k10c84: $ (-2a^{-4}-4a^{-2}-1) +x(-a^{-7}+2a^{-3}+2a^{-1}+a) +x^{2}(a^{-8}-a^{-6}+a^{-4}+7a^{-2}+4) +x^{3}(-a^{-9}+6a^{-7}+11a^{-5}+4a^{-3}-2a^{-1}-2a) +x^{4}(-6a^{-8}+2a^{-6}+9a^{-4}-5a^{-2}-6) +x^{5}(a^{-9}-13a^{-7}-20a^{-5}-11a^{-3}-4a^{-1}+a) +x^{6}(4a^{-8}-8a^{-6}-17a^{-4}-2a^{-2}+3) +x^{7}(7a^{-7}+8a^{-5}+5a^{-3}+4a^{-1}) +x^{8}(6a^{-6}+10a^{-4}+4a^{-2}) +x^{9}(2a^{-5}+2a^{-3}) $

k10c85: $ (-a^{2}+a^{4}+a^{6}) +x(-a-2a^{3}-2a^{5}+a^{9}) +x^{2}(-7a^{2}-14a^{4}-5a^{6}+a^{8}-a^{10}) +x^{3}(4a+11a^{3}+14a^{5}+2a^{7}-4a^{9}+a^{11}) +x^{4}(19a^{2}+37a^{4}+8a^{6}-7a^{8}+3a^{10}) +x^{5}(-4a-4a^{3}-15a^{5}-10a^{7}+5a^{9}) +x^{6}(-14a^{2}-32a^{4}-12a^{6}+6a^{8}) +x^{7}(a-5a^{3}+6a^{7}) +x^{8}(3a^{2}+8a^{4}+5a^{6}) +x^{9}(2a^{3}+2a^{5}) $

k10c86: $ (a^{-6}+2a^{-4}+a^{-2}+1) +x(-3a^{-7}-6a^{-5}-4a^{-3}-a^{-1}) +x^{2}(2a^{-8}-4a^{-6}-10a^{-4}-2a^{-2}+2) +x^{3}(-2a^{-9}+9a^{-7}+20a^{-5}+13a^{-3}+3a^{-1}-a) +x^{4}(-6a^{-8}+10a^{-6}+22a^{-4}-a^{-2}-7) +x^{5}(a^{-9}-11a^{-7}-18a^{-5}-17a^{-3}-10a^{-1}+a) +x^{6}(3a^{-8}-10a^{-6}-22a^{-4}-5a^{-2}+4) +x^{7}(5a^{-7}+5a^{-5}+6a^{-3}+6a^{-1}) +x^{8}(5a^{-6}+10a^{-4}+5a^{-2}) +x^{9}(2a^{-5}+2a^{-3}) $

k10c87: $ (-a^{-4}-3a^{-2}-2-a^{2}) +x(-a^{-3}-a^{-1}+a+a^{3}) +x^{2}(a^{-6}+a^{-4}+3a^{-2}+7+3a^{2}-a^{4}) +x^{3}(7a^{-5}+15a^{-3}+13a^{-1}+2a-3a^{3}) +x^{4}(-2a^{-6}+5a^{-4}+8a^{-2}-5-5a^{2}+a^{4}) +x^{5}(-11a^{-5}-23a^{-3}-21a^{-1}-6a+3a^{3}) +x^{6}(a^{-6}-12a^{-4}-21a^{-2}-3+5a^{2}) +x^{7}(4a^{-5}+5a^{-3}+7a^{-1}+6a) +x^{8}(5a^{-4}+10a^{-2}+5) +x^{9}(2a^{-3}+2a^{-1}) $

k10c88: $ (-a^{-2}-1-a^{2}) +x(-a^{-3}-4a^{-1}-4a-a^{3}) +x^{2}(3a^{-4}+7a^{-2}+8+7a^{2}+3a^{4}) +x^{3}(-a^{-5}+6a^{-3}+19a^{-1}+19a+6a^{3}-a^{5}) +x^{4}(-6a^{-4}-10a^{-2}-8-10a^{2}-6a^{4}) +x^{5}(a^{-5}-11a^{-3}-32a^{-1}-32a-11a^{3}+a^{5}) +x^{6}(4a^{-4}-2a^{-2}-12-2a^{2}+4a^{4}) +x^{7}(7a^{-3}+14a^{-1}+14a+7a^{3}) +x^{8}(6a^{-2}+12+6a^{2}) +x^{9}(2a^{-1}+2a) $

k10c89: $ (1-a^{4}-2a^{6}-a^{8}) +x(-2a^{3}-4a^{5}-a^{7}+a^{9}) +x^{2}(3a^{2}+6a^{4}+6a^{6}+3a^{8}) +x^{3}(5a+19a^{3}+20a^{5}+4a^{7}-2a^{9}) +x^{4}(-6-9a^{2}-2a^{4}-4a^{6}-5a^{8}) +x^{5}(a^{-1}-15a-35a^{3}-27a^{5}-7a^{7}+a^{9}) +x^{6}(5-4a^{2}-15a^{4}-3a^{6}+3a^{8}) +x^{7}(9a+15a^{3}+11a^{5}+5a^{7}) +x^{8}(7a^{2}+12a^{4}+5a^{6}) +x^{9}(2a^{3}+2a^{5}) $

k10c90: $ (a^{-4}-2-2a^{2}) +x(-a^{-5}-a^{-3}-2a^{-1}-2a) +x^{2}(2a^{-6}-4a^{-4}-5a^{-2}+8+6a^{2}-a^{4}) +x^{3}(7a^{-5}+7a^{-3}+9a^{-1}+7a-2a^{3}) +x^{4}(-3a^{-6}+9a^{-4}+15a^{-2}-6-8a^{2}+a^{4}) +x^{5}(-9a^{-5}-11a^{-3}-15a^{-1}-10a+3a^{3}) +x^{6}(a^{-6}-11a^{-4}-21a^{-2}-3+6a^{2}) +x^{7}(3a^{-5}+a^{-3}+5a^{-1}+7a) +x^{8}(4a^{-4}+9a^{-2}+5) +x^{9}(2a^{-3}+2a^{-1}) $

k10c91: $ (2a^{-2}+5+2a^{2}) +x(-3a^{-3}-6a^{-1}-4a+a^{5}) +x^{2}(a^{-4}-9a^{-2}-19-7a^{2}+2a^{4}) +x^{3}(-2a^{-5}+9a^{-3}+18a^{-1}+9a-2a^{5}) +x^{4}(-6a^{-4}+16a^{-2}+35+7a^{2}-6a^{4}) +x^{5}(a^{-5}-12a^{-3}-13a^{-1}-7a-6a^{3}+a^{5}) +x^{6}(3a^{-4}-13a^{-2}-26-7a^{2}+3a^{4}) +x^{7}(5a^{-3}+2a^{-1}+a+4a^{3}) +x^{8}(5a^{-2}+9+4a^{2}) +x^{9}(2a^{-1}+2a) $

k10c92: $ (a^{-6}+a^{-4}-a^{-2}) +x(-a^{-9}-5a^{-7}-5a^{-5}-a^{-3}) +x^{2}(2a^{-10}+2a^{-8}-2a^{-6}+a^{-4}+3a^{-2}) +x^{3}(-2a^{-11}+7a^{-9}+21a^{-7}+18a^{-5}+6a^{-3}) +x^{4}(a^{-12}-8a^{-10}-4a^{-8}+10a^{-6}+2a^{-4}-3a^{-2}) +x^{5}(4a^{-11}-14a^{-9}-32a^{-7}-22a^{-5}-8a^{-3}) +x^{6}(8a^{-10}-5a^{-8}-22a^{-6}-8a^{-4}+a^{-2}) +x^{7}(10a^{-9}+12a^{-7}+5a^{-5}+3a^{-3}) +x^{8}(7a^{-8}+11a^{-6}+4a^{-4}) +x^{9}(2a^{-7}+2a^{-5}) $

k10c93: $ (-2a^{2}-a^{4}) +x(-2a^{-3}-6a^{-1}-6a-a^{3}+a^{5}) +x^{2}(-6a^{-2}-6+7a^{2}+7a^{4}) +x^{3}(5a^{-3}+18a^{-1}+25a+7a^{3}-4a^{5}+a^{7}) +x^{4}(17a^{-2}+28-6a^{2}-14a^{4}+3a^{6}) +x^{5}(-4a^{-3}-10a^{-1}-29a-17a^{3}+6a^{5}) +x^{6}(-13a^{-2}-31-9a^{2}+9a^{4}) +x^{7}(a^{-3}-3a^{-1}+5a+9a^{3}) +x^{8}(3a^{-2}+9+6a^{2}) +x^{9}(2a^{-1}+2a) $

k10c94: $ (2a^{-4}+4a^{-2}+3) +x(-3a^{-5}-5a^{-3}-3a^{-1}-a) +x^{2}(-a^{-8}+2a^{-6}-6a^{-4}-18a^{-2}-7+2a^{2}) +x^{3}(-3a^{-7}+9a^{-5}+16a^{-3}+10a^{-1}+6a) +x^{4}(a^{-8}-6a^{-6}+10a^{-4}+31a^{-2}+11-3a^{2}) +x^{5}(3a^{-7}-10a^{-5}-15a^{-3}-11a^{-1}-9a) +x^{6}(5a^{-6}-9a^{-4}-27a^{-2}-12+a^{2}) +x^{7}(6a^{-5}+3a^{-3}+3a) +x^{8}(5a^{-4}+9a^{-2}+4) +x^{9}(2a^{-3}+2a^{-1}) $

k10c95: $ (2a^{-6}+3a^{-4}) +x(a^{-9}-2a^{-7}-5a^{-5}-3a^{-3}-a^{-1}) +x^{2}(2a^{-8}-4a^{-6}-7a^{-4}+a^{-2}+2) +x^{3}(-2a^{-9}+5a^{-7}+17a^{-5}+16a^{-3}+5a^{-1}-a) +x^{4}(-5a^{-8}+4a^{-6}+13a^{-4}-2a^{-2}-6) +x^{5}(a^{-9}-8a^{-7}-21a^{-5}-25a^{-3}-12a^{-1}+a) +x^{6}(3a^{-8}-6a^{-6}-19a^{-4}-6a^{-2}+4) +x^{7}(5a^{-7}+8a^{-5}+10a^{-3}+7a^{-1}) +x^{8}(5a^{-6}+11a^{-4}+6a^{-2}) +x^{9}(2a^{-5}+2a^{-3}) $

k10c96: $ (-a^{-6}-2a^{-4}-3a^{-2}-3-2a^{2}) +x(-2a^{-5}-2a^{-3}-a^{-1}-a) +x^{2}(3a^{-6}+6a^{-4}+10a^{-2}+12+5a^{2}) +x^{3}(7a^{-5}+16a^{-3}+17a^{-1}+7a-a^{3}) +x^{4}(-3a^{-6}-a^{-4}-4a^{-2}-17-10a^{2}+a^{4}) +x^{5}(-8a^{-5}-23a^{-3}-34a^{-1}-15a+4a^{3}) +x^{6}(a^{-6}-7a^{-4}-17a^{-2}+9a^{2}) +x^{7}(3a^{-5}+6a^{-3}+14a^{-1}+11a) +x^{8}(4a^{-4}+11a^{-2}+7) +x^{9}(2a^{-3}+2a^{-1}) $

k10c97: $ (-a^{-8}-2a^{-6}-2a^{-4}-2a^{-2}) +x(-4a^{-7}-6a^{-5}-2a^{-3}) +x^{2}(a^{-10}+a^{-8}+3a^{-6}+10a^{-4}+6a^{-2}-1) +x^{3}(8a^{-9}+20a^{-7}+24a^{-5}+10a^{-3}-2a^{-1}) +x^{4}(-2a^{-10}+4a^{-8}+5a^{-6}-9a^{-4}-7a^{-2}+1) +x^{5}(-11a^{-9}-28a^{-7}-32a^{-5}-12a^{-3}+3a^{-1}) +x^{6}(a^{-10}-11a^{-8}-21a^{-6}-3a^{-4}+6a^{-2}) +x^{7}(4a^{-9}+7a^{-7}+11a^{-5}+8a^{-3}) +x^{8}(5a^{-8}+11a^{-6}+6a^{-4}) +x^{9}(2a^{-7}+2a^{-5}) $

k10c98: $ (-a^{2}+3a^{4}+5a^{6}+2a^{8}) +x(-6a^{5}-12a^{7}-6a^{9}) +x^{2}(3a^{2}-2a^{4}-10a^{6}+4a^{10}-a^{12}) +x^{3}(5a^{3}+14a^{5}+25a^{7}+14a^{9}-2a^{11}) +x^{4}(-3a^{2}+4a^{4}+17a^{6}+2a^{8}-7a^{10}+a^{12}) +x^{5}(-8a^{3}-17a^{5}-26a^{7}-14a^{9}+3a^{11}) +x^{6}(a^{2}-9a^{4}-23a^{6}-7a^{8}+6a^{10}) +x^{7}(3a^{3}+3a^{5}+8a^{7}+8a^{9}) +x^{8}(4a^{4}+10a^{6}+6a^{8}) +x^{9}(2a^{5}+2a^{7}) $

k10c99: $ (4a^{-2}+9+4a^{2}) +x(a^{-5}-3a^{-3}-10a^{-1}-10a-3a^{3}+a^{5}) +x^{2}(a^{-4}-8a^{-2}-18-8a^{2}+a^{4}) +x^{3}(-2a^{-5}+5a^{-3}+21a^{-1}+21a+5a^{3}-2a^{5}) +x^{4}(-5a^{-4}+9a^{-2}+28+9a^{2}-5a^{4}) +x^{5}(a^{-5}-9a^{-3}-18a^{-1}-18a-9a^{3}+a^{5}) +x^{6}(3a^{-4}-9a^{-2}-24-9a^{2}+3a^{4}) +x^{7}(5a^{-3}+5a^{-1}+5a+5a^{3}) +x^{8}(5a^{-2}+10+5a^{2}) +x^{9}(2a^{-1}+2a) $

k10c100: $ (a^{2}+5a^{4}+3a^{6}) +x(-2a-6a^{3}-8a^{5}-2a^{7}+2a^{9}) +x^{2}(-7a^{2}-17a^{4}-6a^{6}+4a^{8}) +x^{3}(5a+20a^{3}+26a^{5}+5a^{7}-5a^{9}+a^{11}) +x^{4}(17a^{2}+36a^{4}+5a^{6}-11a^{8}+3a^{10}) +x^{5}(-4a-11a^{3}-27a^{5}-14a^{7}+6a^{9}) +x^{6}(-13a^{2}-33a^{4}-12a^{6}+8a^{8}) +x^{7}(a-3a^{3}+4a^{5}+8a^{7}) +x^{8}(3a^{2}+9a^{4}+6a^{6}) +x^{9}(2a^{3}+2a^{5}) $

k10c101: $ (a^{-12}+4a^{-10}+2a^{-8}-2a^{-6}) +x(-a^{-13}-9a^{-11}-8a^{-9}) +x^{2}(a^{-14}+a^{-12}-9a^{-10}-a^{-8}+7a^{-6}-a^{-4}) +x^{3}(8a^{-13}+28a^{-11}+26a^{-9}+4a^{-7}-2a^{-5}) +x^{4}(-2a^{-14}+3a^{-12}+15a^{-10}+a^{-8}-8a^{-6}+a^{-4}) +x^{5}(-11a^{-13}-31a^{-11}-31a^{-9}-8a^{-7}+3a^{-5}) +x^{6}(a^{-14}-11a^{-12}-24a^{-10}-6a^{-8}+6a^{-6}) +x^{7}(4a^{-13}+7a^{-11}+10a^{-9}+7a^{-7}) +x^{8}(5a^{-12}+11a^{-10}+6a^{-8}) +x^{9}(2a^{-11}+2a^{-9}) $

k10c102: $ (a^{-4}+a^{-2}-a^{2}) +x(-2a^{-5}-4a^{-3}-4a^{-1}-2a) +x^{2}(2a^{-6}-4a^{-4}-8a^{-2}+2+3a^{2}-a^{4}) +x^{3}(7a^{-5}+13a^{-3}+16a^{-1}+7a-3a^{3}) +x^{4}(-3a^{-6}+8a^{-4}+21a^{-2}+3-6a^{2}+a^{4}) +x^{5}(-9a^{-5}-14a^{-3}-17a^{-1}-9a+3a^{3}) +x^{6}(a^{-6}-11a^{-4}-24a^{-2}-7+5a^{2}) +x^{7}(3a^{-5}+a^{-3}+4a^{-1}+6a) +x^{8}(4a^{-4}+9a^{-2}+5) +x^{9}(2a^{-3}+2a^{-1}) $

k10c103: $ (-1-3a^{2}+a^{6}) +x(a^{-1}+a-2a^{3}-6a^{5}-4a^{7}) +x^{2}(3+2a^{2}-8a^{4}-6a^{6}+a^{8}) +x^{3}(-2a^{-1}-2a+9a^{3}+21a^{5}+10a^{7}-2a^{9}) +x^{4}(-6+25a^{4}+13a^{6}-6a^{8}) +x^{5}(a^{-1}-5a-9a^{3}-16a^{5}-12a^{7}+a^{9}) +x^{6}(3-5a^{2}-23a^{4}-12a^{6}+3a^{8}) +x^{7}(4a+2a^{3}+3a^{5}+5a^{7}) +x^{8}(4a^{2}+9a^{4}+5a^{6}) +x^{9}(2a^{3}+2a^{5}) $

k10c104: $ (a^{-2}+3+a^{2}) +x(-2a^{-3}-4a^{-1}-2a+a^{3}+a^{5}) +x^{2}(2a^{-4}-6a^{-2}-15-4a^{2}+3a^{4}) +x^{3}(-2a^{-5}+8a^{-3}+13a^{-1}+4a-a^{3}-2a^{5}) +x^{4}(-6a^{-4}+12a^{-2}+27+3a^{2}-6a^{4}) +x^{5}(a^{-5}-11a^{-3}-12a^{-1}-6a-5a^{3}+a^{5}) +x^{6}(3a^{-4}-11a^{-2}-22-5a^{2}+3a^{4}) +x^{7}(5a^{-3}+3a^{-1}+2a+4a^{3}) +x^{8}(5a^{-2}+9+4a^{2}) +x^{9}(2a^{-1}+2a) $

k10c105: $ (-a^{-2}-1-a^{2}) +x(-a^{-5}-3a^{-3}-4a^{-1}-2a) +x^{2}(3a^{-6}+5a^{-4}+4a^{-2}+5+3a^{2}) +x^{3}(-2a^{-7}+6a^{-5}+19a^{-3}+18a^{-1}+7a) +x^{4}(a^{-8}-8a^{-6}-9a^{-4}+2a^{-2}-1-3a^{2}) +x^{5}(4a^{-7}-13a^{-5}-33a^{-3}-24a^{-1}-8a) +x^{6}(8a^{-6}-3a^{-4}-19a^{-2}-7+a^{2}) +x^{7}(10a^{-5}+13a^{-3}+6a^{-1}+3a) +x^{8}(7a^{-4}+11a^{-2}+4) +x^{9}(2a^{-3}+2a^{-1}) $

k10c106: $ (a^{-4}+2a^{-2}+2) +x(a^{-7}+a^{-5}-a^{-3}-2a^{-1}-a) +x^{2}(-a^{-8}+2a^{-6}-3a^{-4}-13a^{-2}-5+2a^{2}) +x^{3}(-3a^{-7}+3a^{-5}+8a^{-3}+9a^{-1}+7a) +x^{4}(a^{-8}-5a^{-6}+4a^{-4}+22a^{-2}+9-3a^{2}) +x^{5}(3a^{-7}-7a^{-5}-13a^{-3}-12a^{-1}-9a) +x^{6}(5a^{-6}-6a^{-4}-23a^{-2}-11+a^{2}) +x^{7}(6a^{-5}+4a^{-3}+a^{-1}+3a) +x^{8}(5a^{-4}+9a^{-2}+4) +x^{9}(2a^{-3}+2a^{-1}) $

k10c107: $ (-a^{-4}-2a^{-2}) +x(a^{-5}-3a^{-1}-3a-a^{3}) +x^{2}(3a^{-4}+3a^{-2}+2a^{2}+2a^{4}) +x^{3}(-2a^{-5}+3a^{-3}+15a^{-1}+17a+6a^{3}-a^{5}) +x^{4}(-5a^{-4}-2a^{-2}+5-4a^{2}-6a^{4}) +x^{5}(a^{-5}-7a^{-3}-22a^{-1}-27a-12a^{3}+a^{5}) +x^{6}(3a^{-4}-4a^{-2}-16-5a^{2}+4a^{4}) +x^{7}(5a^{-3}+9a^{-1}+11a+7a^{3}) +x^{8}(5a^{-2}+11+6a^{2}) +x^{9}(2a^{-1}+2a) $

k10c108: $ 1 +x(-2a^{-3}-6a^{-1}-6a-2a^{3}) +x^{2}(-a^{-6}+2a^{-4}-10-7a^{2}) +x^{3}(a^{-7}-3a^{-5}+10a^{-3}+28a^{-1}+19a+5a^{3}) +x^{4}(3a^{-6}-9a^{-4}+4a^{-2}+33+17a^{2}) +x^{5}(5a^{-5}-17a^{-3}-29a^{-1}-11a-4a^{3}) +x^{6}(7a^{-4}-13a^{-2}-33-13a^{2}) +x^{7}(8a^{-3}+4a^{-1}-3a+a^{3}) +x^{8}(6a^{-2}+9+3a^{2}) +x^{9}(2a^{-1}+2a) $

k10c109: $ (3a^{-2}+7+3a^{2}) +x(a^{-5}-a^{-3}-5a^{-1}-5a-a^{3}+a^{5}) +x^{2}(2a^{-4}-7a^{-2}-18-7a^{2}+2a^{4}) +x^{3}(-2a^{-5}+4a^{-3}+13a^{-1}+13a+4a^{3}-2a^{5}) +x^{4}(-5a^{-4}+6a^{-2}+22+6a^{2}-5a^{4}) +x^{5}(a^{-5}-8a^{-3}-16a^{-1}-16a-8a^{3}+a^{5}) +x^{6}(3a^{-4}-7a^{-2}-20-7a^{2}+3a^{4}) +x^{7}(5a^{-3}+6a^{-1}+6a+5a^{3}) +x^{8}(5a^{-2}+10+5a^{2}) +x^{9}(2a^{-1}+2a) $

k10c110: $ (-a^{-2}-a^{4}-a^{6}) +x(-a^{-1}-3a-6a^{3}-4a^{5}) +x^{2}(3a^{-2}+2-a^{2}+6a^{4}+5a^{6}-a^{8}) +x^{3}(6a^{-1}+13a+21a^{3}+12a^{5}-2a^{7}) +x^{4}(-3a^{-2}+1+8a^{2}-4a^{4}-7a^{6}+a^{8}) +x^{5}(-8a^{-1}-19a-27a^{3}-13a^{5}+3a^{7}) +x^{6}(a^{-2}-8-20a^{2}-5a^{4}+6a^{6}) +x^{7}(3a^{-1}+4a+9a^{3}+8a^{5}) +x^{8}(4+10a^{2}+6a^{4}) +x^{9}(2a+2a^{3}) $

k10c111: $ (a^{-8}+3a^{-6}+2a^{-4}-a^{-2}) +x(-4a^{-9}-10a^{-7}-7a^{-5}-a^{-3}) +x^{2}(-a^{-12}+a^{-10}-3a^{-8}-10a^{-6}-2a^{-4}+3a^{-2}) +x^{3}(-3a^{-11}+13a^{-9}+30a^{-7}+19a^{-5}+5a^{-3}) +x^{4}(a^{-12}-5a^{-10}+10a^{-8}+22a^{-6}+3a^{-4}-3a^{-2}) +x^{5}(3a^{-11}-13a^{-9}-28a^{-7}-20a^{-5}-8a^{-3}) +x^{6}(5a^{-10}-11a^{-8}-26a^{-6}-9a^{-4}+a^{-2}) +x^{7}(7a^{-9}+7a^{-7}+3a^{-5}+3a^{-3}) +x^{8}(6a^{-8}+10a^{-6}+4a^{-4}) +x^{9}(2a^{-7}+2a^{-5}) $

k10c112: $ (-1-4a^{2}-2a^{4}) +x(2a^{3}+2a^{5}) +x^{2}(-3-3a^{2}+a^{4}+a^{6}) +x^{3}(6a^{-1}+13a+9a^{3}-a^{5}-3a^{7}) +x^{4}(-2a^{-2}+15+28a^{2}+3a^{4}-7a^{6}+a^{8}) +x^{5}(-11a^{-1}-16a-17a^{3}-8a^{5}+4a^{7}) +x^{6}(a^{-2}-18-35a^{2}-9a^{4}+7a^{6}) +x^{7}(4a^{-1}+4a^{3}+8a^{5}) +x^{8}(6+13a^{2}+7a^{4}) +x^{9}(3a+3a^{3}) $

k10c113: $ (-a^{-6}-3a^{-4}-3a^{-2}) +x(a^{-7}+a^{-5}-a^{-3}-a^{-1}) +x^{2}(3a^{-6}+8a^{-4}+8a^{-2}+3) +x^{3}(5a^{-7}+16a^{-5}+17a^{-3}+5a^{-1}-a) +x^{4}(-5a^{-8}-4a^{-6}+a^{-4}-6a^{-2}-6) +x^{5}(a^{-9}-16a^{-7}-36a^{-5}-30a^{-3}-10a^{-1}+a) +x^{6}(5a^{-8}-9a^{-6}-23a^{-4}-5a^{-2}+4) +x^{7}(10a^{-7}+15a^{-5}+12a^{-3}+7a^{-1}) +x^{8}(9a^{-6}+16a^{-4}+7a^{-2}) +x^{9}(3a^{-5}+3a^{-3}) $

k10c114: $ (-2a^{2}-a^{4}) +x(-a^{-1}-3a-2a^{3}) +x^{2}(2a^{-2}-5a^{2}-3a^{4}) +x^{3}(-2a^{-3}+5a^{-1}+18a+18a^{3}+7a^{5}) +x^{4}(a^{-4}-8a^{-2}+1+26a^{2}+14a^{4}-2a^{6}) +x^{5}(4a^{-3}-13a^{-1}-27a-21a^{3}-11a^{5}) +x^{6}(8a^{-2}-9-35a^{2}-17a^{4}+a^{6}) +x^{7}(10a^{-1}+8a+2a^{3}+4a^{5}) +x^{8}(8+14a^{2}+6a^{4}) +x^{9}(3a+3a^{3}) $

k10c115: $ (a^{-2}+3+a^{2}) +x(-2a^{-3}-5a^{-1}-5a-2a^{3}) +x^{2}(2a^{-4}-a^{-2}-6-a^{2}+2a^{4}) +x^{3}(-a^{-5}+8a^{-3}+22a^{-1}+22a+8a^{3}-a^{5}) +x^{4}(-5a^{-4}+a^{-2}+12+a^{2}-5a^{4}) +x^{5}(a^{-5}-13a^{-3}-34a^{-1}-34a-13a^{3}+a^{5}) +x^{6}(4a^{-4}-9a^{-2}-26-9a^{2}+4a^{4}) +x^{7}(8a^{-3}+13a^{-1}+13a+8a^{3}) +x^{8}(8a^{-2}+16+8a^{2}) +x^{9}(3a^{-1}+3a) $

k10c116: $ 1 +x(-a^{-1}-3a-3a^{3}-a^{5}) +x^{2}(a^{-2}-1-3a^{2}+a^{4}+2a^{6}) +x^{3}(6a^{-1}+17a+19a^{3}+6a^{5}-2a^{7}) +x^{4}(-2a^{-2}+9+19a^{2}-a^{4}-8a^{6}+a^{8}) +x^{5}(-10a^{-1}-22a-29a^{3}-13a^{5}+4a^{7}) +x^{6}(a^{-2}-15-32a^{2}-8a^{4}+8a^{6}) +x^{7}(4a^{-1}+3a+9a^{3}+10a^{5}) +x^{8}(6+14a^{2}+8a^{4}) +x^{9}(3a+3a^{3}) $

k10c117: $ (a^{-6}+a^{-4}-a^{-2}) +x(-3a^{-7}-5a^{-5}-3a^{-3}-a^{-1}) +x^{2}(a^{-8}-3a^{-6}-4a^{-4}+2a^{-2}+2) +x^{3}(-a^{-9}+8a^{-7}+21a^{-5}+18a^{-3}+5a^{-1}-a) +x^{4}(-5a^{-8}+6a^{-6}+17a^{-4}-6) +x^{5}(a^{-9}-14a^{-7}-29a^{-5}-26a^{-3}-11a^{-1}+a) +x^{6}(4a^{-8}-12a^{-6}-28a^{-4}-8a^{-2}+4) +x^{7}(8a^{-7}+10a^{-5}+9a^{-3}+7a^{-1}) +x^{8}(8a^{-6}+15a^{-4}+7a^{-2}) +x^{9}(3a^{-5}+3a^{-3}) $

k10c118: $ 1 +x(-a^{-3}-3a^{-1}-3a-a^{3}) +x^{2}(a^{-4}-2a^{-2}-6-2a^{2}+a^{4}) +x^{3}(-a^{-5}+5a^{-3}+15a^{-1}+15a+5a^{3}-a^{5}) +x^{4}(-6a^{-4}+6a^{-2}+24+6a^{2}-6a^{4}) +x^{5}(a^{-5}-12a^{-3}-20a^{-1}-20a-12a^{3}+a^{5}) +x^{6}(4a^{-4}-11a^{-2}-30-11a^{2}+4a^{4}) +x^{7}(7a^{-3}+6a^{-1}+6a+7a^{3}) +x^{8}(7a^{-2}+14+7a^{2}) +x^{9}(3a^{-1}+3a) $

k10c119: $ (-a^{-2}-1-a^{2}) +x(-a^{-5}-3a^{-3}-4a^{-1}-2a) +x^{2}(a^{-6}+a^{-2}+6+4a^{2}) +x^{3}(7a^{-5}+19a^{-3}+22a^{-1}+9a-a^{3}) +x^{4}(-2a^{-6}+8a^{-4}+13a^{-2}-7-9a^{2}+a^{4}) +x^{5}(-10a^{-5}-26a^{-3}-37a^{-1}-17a+4a^{3}) +x^{6}(a^{-6}-14a^{-4}-31a^{-2}-7+9a^{2}) +x^{7}(4a^{-5}+5a^{-3}+13a^{-1}+12a) +x^{8}(6a^{-4}+15a^{-2}+9) +x^{9}(3a^{-3}+3a^{-1}) $

k10c120: $ (-3a^{6}+3a^{10}+a^{12}) +x(2a^{7}-4a^{9}-8a^{11}-2a^{13}) +x^{2}(7a^{6}-7a^{10}+a^{12}+a^{14}) +x^{3}(5a^{7}+26a^{9}+29a^{11}+8a^{13}) +x^{4}(a^{4}-11a^{6}-3a^{8}+17a^{10}+6a^{12}-2a^{14}) +x^{5}(4a^{5}-17a^{7}-44a^{9}-33a^{11}-10a^{13}) +x^{6}(10a^{6}-9a^{8}-33a^{10}-13a^{12}+a^{14}) +x^{7}(13a^{7}+16a^{9}+7a^{11}+4a^{13}) +x^{8}(10a^{8}+16a^{10}+6a^{12}) +x^{9}(3a^{9}+3a^{11}) $

k10c121: $ (1+a^{2}+2a^{4}+a^{6}) +x(-a^{3}-3a^{5}-2a^{7}) +x^{2}(-3a^{2}-7a^{4}-3a^{6}+a^{8}) +x^{3}(4a+14a^{3}+19a^{5}+8a^{7}-a^{9}) +x^{4}(-5+3a^{2}+22a^{4}+9a^{6}-5a^{8}) +x^{5}(a^{-1}-15a-30a^{3}-28a^{5}-13a^{7}+a^{9}) +x^{6}(5-13a^{2}-36a^{4}-14a^{6}+4a^{8}) +x^{7}(10a+11a^{3}+9a^{5}+8a^{7}) +x^{8}(10a^{2}+19a^{4}+9a^{6}) +x^{9}(4a^{3}+4a^{5}) $

k10c122: $ (-2a^{-4}-4a^{-2}-1) +x(2a^{-5}+2a^{-3}) +x^{2}(2+2a^{2}) +x^{3}(4a^{-5}+14a^{-3}+18a^{-1}+6a-2a^{3}) +x^{4}(-a^{-6}+12a^{-4}+24a^{-2}+3-7a^{2}+a^{4}) +x^{5}(-11a^{-5}-25a^{-3}-32a^{-1}-14a+4a^{3}) +x^{6}(a^{-6}-20a^{-4}-42a^{-2}-13+8a^{2}) +x^{7}(5a^{-5}+3a^{-3}+9a^{-1}+11a) +x^{8}(8a^{-4}+18a^{-2}+10) +x^{9}(4a^{-3}+4a^{-1}) $

k10c123: $ (-2a^{-2}-3-2a^{2}) +x(-2a^{-1}-2a) +x^{2}(6a^{-2}+12+6a^{2}) +x^{3}(5a^{-3}+21a^{-1}+21a+5a^{3}) +x^{4}(-5a^{-4}-3a^{-2}+4-3a^{2}-5a^{4}) +x^{5}(a^{-5}-15a^{-3}-38a^{-1}-38a-15a^{3}+a^{5}) +x^{6}(5a^{-4}-11a^{-2}-32-11a^{2}+5a^{4}) +x^{7}(10a^{-3}+14a^{-1}+14a+10a^{3}) +x^{8}(10a^{-2}+20+10a^{2}) +x^{9}(4a^{-1}+4a) $

k10c124: $ (2a^{-12}+8a^{-10}+7a^{-8}) +x(-8a^{-11}-8a^{-9}) +x^{2}(-a^{-12}-22a^{-10}-21a^{-8}) +x^{3}(14a^{-11}+14a^{-9}) +x^{4}(21a^{-10}+21a^{-8}) +x^{5}(-7a^{-11}-7a^{-9}) +x^{6}(-8a^{-10}-8a^{-8}) +x^{7}(a^{-11}+a^{-9}) +x^{8}(a^{-10}+a^{-8}) $

k10c125: $ (3a^{-2}+7+3a^{2}) +x(a^{-5}-a^{-3}-6a^{-1}-8a-4a^{3}) +x^{2}(a^{-4}-6a^{-2}-15-8a^{2}) +x^{3}(a^{-3}+8a^{-1}+17a+10a^{3}) +x^{4}(2a^{-2}+13+11a^{2}) +x^{5}(-5a^{-1}-11a-6a^{3}) +x^{6}(-6-6a^{2}) +x^{7}(a^{-1}+2a+a^{3}) +x^{8}(1+a^{2}) $

k10c126: $ (2a^{2}+7a^{4}+4a^{6}) +x(-2a-6a^{3}-8a^{5}-a^{7}+3a^{9}) +x^{2}(-4a^{2}-16a^{4}-11a^{6}+a^{8}) +x^{3}(a+11a^{3}+16a^{5}+2a^{7}-4a^{9}) +x^{4}(2a^{2}+16a^{4}+11a^{6}-3a^{8}) +x^{5}(-5a^{3}-9a^{5}-3a^{7}+a^{9}) +x^{6}(-6a^{4}-5a^{6}+a^{8}) +x^{7}(a^{3}+2a^{5}+a^{7}) +x^{8}(a^{4}+a^{6}) $

k10c127: $ (5a^{4}+6a^{6}+2a^{8}) +x(-5a^{5}-8a^{7}-2a^{9}+a^{11}) +x^{2}(-9a^{4}-14a^{6}-2a^{8}+a^{10}-2a^{12}) +x^{3}(5a^{5}+16a^{7}+7a^{9}-4a^{11}) +x^{4}(3a^{4}+11a^{6}+4a^{8}-3a^{10}+a^{12}) +x^{5}(-3a^{5}-10a^{7}-5a^{9}+2a^{11}) +x^{6}(-4a^{6}-2a^{8}+2a^{10}) +x^{7}(a^{5}+3a^{7}+2a^{9}) +x^{8}(a^{6}+a^{8}) $

k10c128: $ (a^{-12}+4a^{-10}+2a^{-8}-2a^{-6}) +x(-6a^{-11}-5a^{-9}+a^{-7}) +x^{2}(-11a^{-10}-5a^{-8}+6a^{-6}) +x^{3}(11a^{-11}+13a^{-9}+2a^{-7}) +x^{4}(12a^{-10}+7a^{-8}-5a^{-6}) +x^{5}(-6a^{-11}-10a^{-9}-4a^{-7}) +x^{6}(-6a^{-10}-5a^{-8}+a^{-6}) +x^{7}(a^{-11}+2a^{-9}+a^{-7}) +x^{8}(a^{-10}+a^{-8}) $

k10c129: $ (a^{-2}+2-a^{2}-a^{4}) +x(-2a^{-3}-5a^{-1}-5a-a^{3}+a^{5}) +x^{2}(-3a^{-2}-4+2a^{2}+3a^{4}) +x^{3}(a^{-3}+9a^{-1}+15a+4a^{3}-3a^{5}) +x^{4}(2a^{-2}+8-6a^{4}) +x^{5}(-4a^{-1}-11a-6a^{3}+a^{5}) +x^{6}(-4-2a^{2}+2a^{4}) +x^{7}(a^{-1}+3a+2a^{3}) +x^{8}(1+a^{2}) $

k10c130: $ (-1-2a^{2}+2a^{4}+2a^{6}) +x(a^{-1}+a-3a^{3}-9a^{5}-6a^{7}) +x^{2}(2+6a^{2}-4a^{6}) +x^{3}(-2a+8a^{3}+21a^{5}+11a^{7}) +x^{4}(-7a^{2}+7a^{6}) +x^{5}(a-8a^{3}-15a^{5}-6a^{7}) +x^{6}(2a^{2}-3a^{4}-5a^{6}) +x^{7}(2a^{3}+3a^{5}+a^{7}) +x^{8}(a^{4}+a^{6}) $

k10c131: $ (-2a^{2}+2a^{6}+a^{8}) +x(a^{3}-a^{5}-5a^{7}-3a^{9}) +x^{2}(3a^{2}+2a^{4}-3a^{6}+2a^{8}+4a^{10}) +x^{3}(a^{3}+2a^{5}+10a^{7}+9a^{9}) +x^{4}(-2a^{4}-2a^{6}-4a^{8}-4a^{10}) +x^{5}(a^{3}-3a^{5}-12a^{7}-8a^{9}) +x^{6}(2a^{4}-a^{8}+a^{10}) +x^{7}(2a^{5}+4a^{7}+2a^{9}) +x^{8}(a^{6}+a^{8}) $

k10c132: $ (3a^{4}+2a^{6}) +x(-a-4a^{3}-8a^{5}-5a^{7}) +x^{2}(-a^{2}-7a^{4}-6a^{6}) +x^{3}(9a^{3}+19a^{5}+10a^{7}) +x^{4}(10a^{4}+10a^{6}) +x^{5}(-6a^{3}-12a^{5}-6a^{7}) +x^{6}(-6a^{4}-6a^{6}) +x^{7}(a^{3}+2a^{5}+a^{7}) +x^{8}(a^{4}+a^{6}) $

k10c133: $ (-a^{2}+2a^{4}+3a^{6}+a^{8}) +x(-4a^{5}-7a^{7}-3a^{9}) +x^{2}(a^{2}-3a^{4}-6a^{6}+a^{8}+3a^{10}) +x^{3}(a^{3}+7a^{5}+16a^{7}+10a^{9}) +x^{4}(2a^{4}+6a^{6}-4a^{10}) +x^{5}(-4a^{5}-13a^{7}-9a^{9}) +x^{6}(-4a^{6}-3a^{8}+a^{10}) +x^{7}(a^{5}+3a^{7}+2a^{9}) +x^{8}(a^{6}+a^{8}) $

k10c134: $ (a^{-12}+3a^{-10}-3a^{-6}) +x(-2a^{-13}-8a^{-11}-4a^{-9}+2a^{-7}) +x^{2}(a^{-14}+a^{-12}-7a^{-10}+7a^{-6}) +x^{3}(3a^{-13}+14a^{-11}+11a^{-9}) +x^{4}(-a^{-12}+5a^{-10}+a^{-8}-5a^{-6}) +x^{5}(-8a^{-11}-11a^{-9}-3a^{-7}) +x^{6}(a^{-12}-3a^{-10}-3a^{-8}+a^{-6}) +x^{7}(2a^{-11}+3a^{-9}+a^{-7}) +x^{8}(a^{-10}+a^{-8}) $

k10c135: $ (2a^{-2}+4-a^{4}) +x(-3a^{-3}-6a^{-1}-4a+a^{3}+2a^{5}) +x^{2}(-4a^{-2}-6+a^{2}+3a^{4}) +x^{3}(3a^{-3}+9a^{-1}+8a-a^{3}-3a^{5}) +x^{4}(2a^{-2}+3-4a^{2}-5a^{4}) +x^{5}(-4a^{-1}-8a-3a^{3}+a^{5}) +x^{6}(a^{-2}+a^{2}+2a^{4}) +x^{7}(2a^{-1}+4a+2a^{3}) +x^{8}(1+a^{2}) $

k10c136: $ (-a^{-4}-3a^{-2}-2-a^{2}) +x(-2a^{-3}-4a^{-1}-2a) +x^{2}(a^{-4}+4a^{-2}+6+3a^{2}) +x^{3}(7a^{-3}+16a^{-1}+9a) +x^{4}(2a^{-2}-2-4a^{2}) +x^{5}(-5a^{-3}-14a^{-1}-9a) +x^{6}(-4a^{-2}-3+a^{2}) +x^{7}(a^{-3}+3a^{-1}+2a) +x^{8}(a^{-2}+1) $

k10c137: $ (-a^{-2}-1-2a^{2}-2a^{4}-a^{6}) +x(-a^{-1}-3a-5a^{3}-3a^{5}) +x^{2}(a^{-2}+4+7a^{2}+8a^{4}+4a^{6}) +x^{3}(2a^{-1}+9a+15a^{3}+8a^{5}) +x^{4}(-2-5a^{2}-7a^{4}-4a^{6}) +x^{5}(-7a-15a^{3}-8a^{5}) +x^{6}(1-a^{2}-a^{4}+a^{6}) +x^{7}(2a+4a^{3}+2a^{5}) +x^{8}(a^{2}+a^{4}) $

k10c138: $ (-a^{-6}-2a^{-4}-3a^{-2}-3-2a^{2}) +x(-2a^{-5}-2a^{-3}-a^{-1}-a) +x^{2}(3a^{-6}+6a^{-4}+10a^{-2}+12+5a^{2}) +x^{3}(3a^{-5}+5a^{-3}+8a^{-1}+6a) +x^{4}(-5a^{-4}-13a^{-2}-12-4a^{2}) +x^{5}(a^{-5}-6a^{-3}-14a^{-1}-7a) +x^{6}(3a^{-4}+3a^{-2}+1+a^{2}) +x^{7}(3a^{-3}+5a^{-1}+2a) +x^{8}(a^{-2}+1) $

k10c139: $ (a^{-12}+6a^{-10}+6a^{-8}) +x(-2a^{-15}-a^{-13}-5a^{-11}-6a^{-9}) +x^{2}(-2a^{-14}-19a^{-10}-21a^{-8}) +x^{3}(a^{-15}+a^{-13}+13a^{-11}+13a^{-9}) +x^{4}(a^{-14}+20a^{-10}+21a^{-8}) +x^{5}(-7a^{-11}-7a^{-9}) +x^{6}(-8a^{-10}-8a^{-8}) +x^{7}(a^{-11}+a^{-9}) +x^{8}(a^{-10}+a^{-8}) $

k10c140: $ (1+2a^{2}+4a^{4}+2a^{6}) +x(-2a^{3}-6a^{5}-4a^{7}) +x^{2}(-4a^{2}-12a^{4}-8a^{6}) +x^{3}(6a^{3}+16a^{5}+10a^{7}) +x^{4}(a^{2}+12a^{4}+11a^{6}) +x^{5}(-5a^{3}-11a^{5}-6a^{7}) +x^{6}(-6a^{4}-6a^{6}) +x^{7}(a^{3}+2a^{5}+a^{7}) +x^{8}(a^{4}+a^{6}) $

k10c141: $ (2+2a^{2}+a^{4}) +x(-a^{-1}-3a-4a^{3}-2a^{5}) +x^{2}(a^{-2}-4-9a^{2}-a^{4}+3a^{6}) +x^{3}(2a^{-1}+5a+13a^{3}+10a^{5}) +x^{4}(3+8a^{2}+a^{4}-4a^{6}) +x^{5}(-3a-12a^{3}-9a^{5}) +x^{6}(-4a^{2}-3a^{4}+a^{6}) +x^{7}(a+3a^{3}+2a^{5}) +x^{8}(a^{2}+a^{4}) $

k10c142: $ (a^{-12}+5a^{-10}+4a^{-8}-a^{-6}) +x(2a^{-13}-4a^{-11}-6a^{-9}) +x^{2}(-a^{-12}-17a^{-10}-10a^{-8}+6a^{-6}) +x^{3}(9a^{-11}+12a^{-9}+3a^{-7}) +x^{4}(a^{-12}+15a^{-10}+9a^{-8}-5a^{-6}) +x^{5}(-5a^{-11}-9a^{-9}-4a^{-7}) +x^{6}(-6a^{-10}-5a^{-8}+a^{-6}) +x^{7}(a^{-11}+2a^{-9}+a^{-7}) +x^{8}(a^{-10}+a^{-8}) $

k10c143: $ (3a^{4}+2a^{6}) +x(-a-3a^{3}-5a^{5}-2a^{7}+a^{9}) +x^{2}(-4a^{2}-10a^{4}-3a^{6}+3a^{8}) +x^{3}(a+7a^{3}+14a^{5}+5a^{7}-3a^{9}) +x^{4}(3a^{2}+11a^{4}+2a^{6}-6a^{8}) +x^{5}(-3a^{3}-10a^{5}-6a^{7}+a^{9}) +x^{6}(-4a^{4}-2a^{6}+2a^{8}) +x^{7}(a^{3}+3a^{5}+2a^{7}) +x^{8}(a^{4}+a^{6}) $

k10c144: $ (3+4a^{2}+2a^{4}) +x(-2a^{3}-2a^{5}) +x^{2}(-7-12a^{2}-2a^{4}+2a^{6}-a^{8}) +x^{3}(8a^{3}+4a^{5}-4a^{7}) +x^{4}(3+8a^{2}-2a^{4}-6a^{6}+a^{8}) +x^{5}(-a-8a^{3}-4a^{5}+3a^{7}) +x^{6}(-2a^{2}+2a^{4}+4a^{6}) +x^{7}(a+4a^{3}+3a^{5}) +x^{8}(a^{2}+a^{4}) $

k10c145: $ (2a^{4}+a^{6}+a^{8}+a^{10}) +x(-a^{5}-2a^{7}-6a^{9}-5a^{11}) +x^{2}(-4a^{4}-2a^{6}-4a^{8}-6a^{10}) +x^{3}(8a^{7}+18a^{9}+10a^{11}) +x^{4}(a^{4}+9a^{8}+10a^{10}) +x^{5}(-6a^{7}-12a^{9}-6a^{11}) +x^{6}(-6a^{8}-6a^{10}) +x^{7}(a^{7}+2a^{9}+a^{11}) +x^{8}(a^{8}+a^{10}) $

k10c146: $ 1 +x(-a^{-3}-3a^{-1}-3a-a^{3}) +x^{2}(-3a^{-2}-3+3a^{2}+3a^{4}) +x^{3}(a^{-3}+6a^{-1}+12a+5a^{3}-2a^{5}) +x^{4}(3a^{-2}+5-6a^{2}-8a^{4}) +x^{5}(-2a^{-1}-11a-8a^{3}+a^{5}) +x^{6}(-2+a^{2}+3a^{4}) +x^{7}(a^{-1}+4a+3a^{3}) +x^{8}(1+a^{2}) $

k10c147: $ (-a^{-2}-1-a^{2}) +x(-a^{-5}-3a^{-3}-4a^{-1}-2a) +x^{2}(a^{-6}+a^{-2}+6+4a^{2}) +x^{3}(3a^{-5}+8a^{-3}+13a^{-1}+8a) +x^{4}(-2a^{-2}-6-4a^{2}) +x^{5}(-6a^{-3}-14a^{-1}-8a) +x^{6}(a^{-4}-a^{-2}-1+a^{2}) +x^{7}(2a^{-3}+4a^{-1}+2a) +x^{8}(a^{-2}+1) $

k10c148: $ (a^{2}+5a^{4}+3a^{6}) +x(-a-3a^{3}-5a^{5}-a^{7}+2a^{9}) +x^{2}(-3a^{2}-11a^{4}-6a^{6}+2a^{8}) +x^{3}(a+6a^{3}+9a^{5}+a^{7}-3a^{9}) +x^{4}(3a^{2}+10a^{4}+2a^{6}-5a^{8}) +x^{5}(-2a^{3}-7a^{5}-4a^{7}+a^{9}) +x^{6}(-3a^{4}-a^{6}+2a^{8}) +x^{7}(a^{3}+3a^{5}+2a^{7}) +x^{8}(a^{4}+a^{6}) $

k10c149: $ (4a^{4}+4a^{6}+a^{8}) +x(-3a^{5}-3a^{7}+a^{9}+a^{11}) +x^{2}(-7a^{4}-9a^{6}+a^{10}-a^{12}) +x^{3}(2a^{5}+5a^{7}-a^{9}-4a^{11}) +x^{4}(3a^{4}+5a^{6}-4a^{8}-5a^{10}+a^{12}) +x^{5}(-a^{5}-6a^{7}-2a^{9}+3a^{11}) +x^{6}(-a^{6}+3a^{8}+4a^{10}) +x^{7}(a^{5}+4a^{7}+3a^{9}) +x^{8}(a^{6}+a^{8}) $

k10c150: $ (-a^{-4}-2a^{-2}) +x(-a^{-9}-3a^{-7}-2a^{-5}) +x^{2}(a^{-10}+a^{-8}+3a^{-6}+8a^{-4}+5a^{-2}) +x^{3}(3a^{-9}+6a^{-7}+8a^{-5}+5a^{-3}) +x^{4}(-5a^{-6}-9a^{-4}-4a^{-2}) +x^{5}(-5a^{-7}-12a^{-5}-7a^{-3}) +x^{6}(a^{-8}+a^{-2}) +x^{7}(2a^{-7}+4a^{-5}+2a^{-3}) +x^{8}(a^{-6}+a^{-4}) $

k10c151: $ (a^{-6}-3a^{-2}-1) +x(-3a^{-7}-3a^{-5}+a^{-3}+2a^{-1}+a) +x^{2}(-2a^{-6}+4a^{-4}+10a^{-2}+4) +x^{3}(3a^{-7}+5a^{-5}+a^{-3}-3a^{-1}-2a) +x^{4}(2a^{-6}-6a^{-4}-15a^{-2}-7) +x^{5}(-2a^{-5}-7a^{-3}-4a^{-1}+a) +x^{6}(a^{-6}+3a^{-4}+5a^{-2}+3) +x^{7}(2a^{-5}+5a^{-3}+3a^{-1}) +x^{8}(a^{-4}+a^{-2}) $

k10c152: $ (8a^{8}+10a^{10}+3a^{12}) +x(-10a^{9}-11a^{11}+a^{13}+2a^{15}) +x^{2}(-22a^{8}-26a^{10}-3a^{12}-a^{14}-2a^{16}) +x^{3}(17a^{9}+19a^{11}-3a^{13}-5a^{15}) +x^{4}(21a^{8}+25a^{10}+2a^{12}-a^{14}+a^{16}) +x^{5}(-8a^{9}-8a^{11}+2a^{13}+2a^{15}) +x^{6}(-8a^{8}-9a^{10}+a^{14}) +x^{7}(a^{9}+a^{11}) +x^{8}(a^{8}+a^{10}) $

k10c153: $ (3a^{-2}+6+a^{2}-a^{4}) +x(-5a^{-3}-10a^{-1}-6a+2a^{3}+3a^{5}) +x^{2}(-7a^{-2}-12-2a^{2}+3a^{4}) +x^{3}(10a^{-3}+22a^{-1}+12a-4a^{3}-4a^{5}) +x^{4}(10a^{-2}+14-4a^{4}) +x^{5}(-6a^{-3}-13a^{-1}-7a+a^{3}+a^{5}) +x^{6}(-6a^{-2}-7+a^{4}) +x^{7}(a^{-3}+2a^{-1}+a) +x^{8}(a^{-2}+1) $

k10c154: $ (a^{-12}+2a^{-10}-2a^{-8}-4a^{-6}) +x(-4a^{-13}-10a^{-11}-3a^{-9}+3a^{-7}) +x^{2}(3a^{-14}+2a^{-12}-5a^{-10}+5a^{-8}+9a^{-6}) +x^{3}(10a^{-13}+21a^{-11}+9a^{-9}-2a^{-7}) +x^{4}(-4a^{-14}-a^{-12}+7a^{-10}-2a^{-8}-6a^{-6}) +x^{5}(-9a^{-13}-15a^{-11}-6a^{-9}) +x^{6}(a^{-14}-3a^{-12}-5a^{-10}+a^{-6}) +x^{7}(2a^{-13}+3a^{-11}+a^{-9}) +x^{8}(a^{-12}+a^{-10}) $

k10c155: $ (2a^{-4}+4a^{-2}+3) +x(-2a^{-5}-2a^{-3}) +x^{2}(4a^{-6}-a^{-4}-11a^{-2}-5+a^{2}) +x^{3}(8a^{-5}+6a^{-3}+2a) +x^{4}(-4a^{-6}-a^{-4}+7a^{-2}+4) +x^{5}(-8a^{-5}-9a^{-3}-a^{-1}) +x^{6}(a^{-6}-2a^{-4}-3a^{-2}) +x^{7}(2a^{-5}+3a^{-3}+a^{-1}) +x^{8}(a^{-4}+a^{-2}) $

k10c156: $ (-2a^{2}-a^{4}) +x(-a-2a^{3}-2a^{5}-a^{7}) +x^{2}(4+7a^{2}+a^{4}-2a^{6}) +x^{3}(-2a^{-1}+3a+8a^{3}+4a^{5}+a^{7}) +x^{4}(-8-9a^{2}+2a^{4}+3a^{6}) +x^{5}(a^{-1}-7a-9a^{3}-a^{5}) +x^{6}(3+2a^{2}-a^{4}) +x^{7}(3a+4a^{3}+a^{5}) +x^{8}(a^{2}+a^{4}) $

k10c157: $ (-a^{-8}+2a^{-4}) +x(4a^{-9}+4a^{-7}) +x^{2}(2a^{-10}+7a^{-8}-5a^{-4}) +x^{3}(-4a^{-11}-8a^{-9}-6a^{-7}-2a^{-5}) +x^{4}(a^{-12}-8a^{-10}-15a^{-8}-3a^{-6}+3a^{-4}) +x^{5}(4a^{-11}-3a^{-7}+a^{-5}) +x^{6}(6a^{-10}+8a^{-8}+2a^{-6}) +x^{7}(4a^{-9}+5a^{-7}+a^{-5}) +x^{8}(a^{-8}+a^{-6}) $

k10c158: $ (a^{-4}-2-2a^{2}) +x(2a^{-3}+a^{-1}-a) +x^{2}(-5a^{-4}-2a^{-2}+9+5a^{2}-a^{4}) +x^{3}(-4a^{-3}+2a^{-1}+3a-3a^{3}) +x^{4}(3a^{-4}-a^{-2}-13-8a^{2}+a^{4}) +x^{5}(a^{-3}-7a^{-1}-5a+3a^{3}) +x^{6}(a^{-2}+6+5a^{2}) +x^{7}(a^{-3}+5a^{-1}+4a) +x^{8}(a^{-2}+1) $

k10c159: $ (-a^{2}+a^{4}+a^{6}) +x(a^{3}+a^{5}+a^{7}+a^{9}) +x^{2}(-2a^{2}-4a^{4}+a^{6}+3a^{8}) +x^{3}(a-a^{7}-2a^{9}) +x^{4}(4a^{2}+3a^{4}-8a^{6}-7a^{8}) +x^{5}(a^{3}-5a^{5}-5a^{7}+a^{9}) +x^{6}(3a^{6}+3a^{8}) +x^{7}(a^{3}+4a^{5}+3a^{7}) +x^{8}(a^{4}+a^{6}) $

k10c160: $ (-a^{-8}-a^{-6}-a^{-2}) +x(2a^{-9}-3a^{-5}-a^{-3}) +x^{2}(a^{-8}+3a^{-4}+4a^{-2}) +x^{3}(3a^{-7}+10a^{-5}+7a^{-3}) +x^{4}(a^{-8}+2a^{-6}-3a^{-4}-4a^{-2}) +x^{5}(-3a^{-7}-11a^{-5}-8a^{-3}) +x^{6}(-3a^{-6}-2a^{-4}+a^{-2}) +x^{7}(a^{-7}+3a^{-5}+2a^{-3}) +x^{8}(a^{-6}+a^{-4}) $

k10c161: $ (-3a^{6}-a^{8}+a^{10}) +x(2a^{7}+a^{11}+3a^{13}) +x^{2}(9a^{6}+3a^{8}-3a^{10}+3a^{12}) +x^{3}(-a^{7}-3a^{11}-4a^{13}) +x^{4}(-6a^{6}-a^{8}+a^{10}-4a^{12}) +x^{5}(a^{11}+a^{13}) +x^{6}(a^{6}+a^{12}) $

k10c162: $ (a^{-10}-a^{-8}-3a^{-6}) +x(3a^{-13}+a^{-11}+2a^{-7}) +x^{2}(3a^{-12}-3a^{-10}+3a^{-8}+9a^{-6}) +x^{3}(-4a^{-13}-3a^{-11}-a^{-7}) +x^{4}(-4a^{-12}+a^{-10}-a^{-8}-6a^{-6}) +x^{5}(a^{-13}+a^{-11}) +x^{6}(a^{-12}+a^{-6}) $

k10c163: $ (3+3a^{2}-a^{6}) +x(-2a-7a^{3}-5a^{5}) +x^{2}(-7-9a^{2}+5a^{4}+5a^{6}-2a^{8}) +x^{3}(15a^{3}+12a^{5}-3a^{7}) +x^{4}(3+6a^{2}-4a^{4}-6a^{6}+a^{8}) +x^{5}(-a-11a^{3}-8a^{5}+2a^{7}) +x^{6}(-2a^{2}+a^{4}+3a^{6}) +x^{7}(a+4a^{3}+3a^{5}) +x^{8}(a^{2}+a^{4}) $

k10c164: $ (a^{-6}+2a^{-4}+a^{-2}+1) +x(-2a^{-7}-3a^{-5}-a^{-3}) +x^{2}(-4a^{-6}-4a^{-4}+2a^{-2}+2) +x^{3}(3a^{-7}+7a^{-5}+8a^{-3}+3a^{-1}-a) +x^{4}(4a^{-6}+a^{-4}-11a^{-2}-8) +x^{5}(-4a^{-5}-15a^{-3}-10a^{-1}+a) +x^{6}(a^{-6}+3a^{-2}+4) +x^{7}(3a^{-5}+8a^{-3}+5a^{-1}) +x^{8}(2a^{-4}+2a^{-2}) $

k10c165: $ (a^{-2}+3+a^{2}) +x(-2a^{-3}-5a^{-1}-5a-2a^{3}) +x^{2}(-6a^{-2}-9+3a^{4}) +x^{3}(3a^{-3}+10a^{-1}+16a+7a^{3}-2a^{5}) +x^{4}(4a^{-2}+8-3a^{2}-7a^{4}) +x^{5}(-6a^{-1}-17a-10a^{3}+a^{5}) +x^{6}(a^{-2}-3-a^{2}+3a^{4}) +x^{7}(3a^{-1}+7a+4a^{3}) +x^{8}(2+2a^{2}) $

k10c166: $ (a^{-6}+a^{-4}-a^{-2}) +x(-a^{-9}-5a^{-7}-5a^{-5}-a^{-3}) +x^{2}(2a^{-10}+2a^{-8}-2a^{-6}+a^{-4}+3a^{-2}) +x^{3}(10a^{-9}+18a^{-7}+11a^{-5}+3a^{-3}) +x^{4}(-3a^{-10}-2a^{-8}-2a^{-6}-3a^{-4}) +x^{5}(-11a^{-9}-22a^{-7}-10a^{-5}+a^{-3}) +x^{6}(a^{-10}-4a^{-8}-2a^{-6}+3a^{-4}) +x^{7}(3a^{-9}+7a^{-7}+4a^{-5}) +x^{8}(2a^{-8}+2a^{-6}) $


\newpage


\section{Q-polynomial}
$Q_{L+} + Q_{L-} = x (Q_{L0} + Q_{L\infty})$ \par
k3c1: $ -3+2x+2x^{2} $ 

k4c1: $ -3-2x+4x^{2}+2x^{3} $ 

k5c1: $ 5-2x-6x^{2}+2x^{3}+2x^{4} $ 

k5c2: $ 1-4x-2x^{2}+4x^{3}+2x^{4} $ 

k6c1: $ 1+4x-6x^{2}-4x^{3}+4x^{4}+2x^{5} $ 

k6c2: $ 5-2x-10x^{2}+6x^{4}+2x^{5} $ 

k6c3: $ 5-6x-12x^{2}+4x^{3}+8x^{4}+2x^{5} $ 

k7c1: $ -7+4x+16x^{2}-6x^{3}-10x^{4}+2x^{5}+2x^{6} $ 

k7c2: $ -3+6x+8x^{2}-10x^{3}-6x^{4}+4x^{5}+2x^{6} $ 

k7c3: $ -3+2x+6x^{2}-6x^{3}-4x^{4}+4x^{5}+2x^{6} $ 

k7c4: $ 1+8x-4x^{2}-12x^{3}+6x^{5}+2x^{6} $ 

k7c5: $ 1-4x^{2}-6x^{3}+2x^{4}+6x^{5}+2x^{6} $ 

k7c6: $ 5+2x-12x^{2}-10x^{3}+6x^{4}+8x^{5}+2x^{6} $ 

k7c7: $ 5+6x-18x^{2}-14x^{3}+10x^{4}+10x^{5}+2x^{6} $ 

k8c1: $ -3-6x+14x^{2}+12x^{3}-14x^{4}-8x^{5}+4x^{6}+2x^{7} $ 

k8c2: $ -7+22x^{2}+2x^{3}-20x^{4}-4x^{5}+6x^{6}+2x^{7} $ 

k8c3: $ 1-8x+4x^{2}+12x^{3}-8x^{4}-6x^{5}+4x^{6}+2x^{7} $ 

k8c4: $ -3+2x+14x^{2}-2x^{3}-16x^{4}-2x^{5}+6x^{6}+2x^{7} $ 

k8c5: $ -11+14x+26x^{2}-16x^{3}-24x^{4}+2x^{5}+8x^{6}+2x^{7} $ 

k8c6: $ 1-4x+2x^{2}+2x^{3}-8x^{4}+6x^{6}+2x^{7} $ 

k8c7: $ -7+4x+20x^{2}-8x^{3}-20x^{4}+2x^{5}+8x^{6}+2x^{7} $ 

k8c8: $ 1+4x+6x^{2}-10x^{3}-14x^{4}+4x^{5}+8x^{6}+2x^{7} $ 

k8c9: $ -7+4x+16x^{2}-10x^{3}-16x^{4}+4x^{5}+8x^{6}+2x^{7} $ 

k8c10: $ -11+14x+22x^{2}-22x^{3}-22x^{4}+8x^{5}+10x^{6}+2x^{7} $ 

k8c11: $ -3+6x+4x^{2}-12x^{3}-10x^{4}+6x^{5}+8x^{6}+2x^{7} $ 

k8c12: $ 5+2x-8x^{2}-12x^{3}-4x^{4}+8x^{5}+8x^{6}+2x^{7} $ 

k8c13: $ -3+10x+10x^{2}-22x^{3}-16x^{4}+10x^{5}+10x^{6}+2x^{7} $ 

k8c14: $ 1+8x-22x^{3}-10x^{4}+12x^{5}+10x^{6}+2x^{7} $ 

k8c15: $ -7+16x+10x^{2}-32x^{3}-16x^{4}+16x^{5}+12x^{6}+2x^{7} $ 

k8c16: $ -3+10x+18x^{2}-22x^{3}-30x^{4}+8x^{5}+16x^{6}+4x^{7} $ 

k8c17: $ -3+6x+12x^{2}-20x^{3}-24x^{4}+10x^{5}+16x^{6}+4x^{7} $ 

k8c18: $ 5+2x+12x^{2}-26x^{3}-36x^{4}+14x^{5}+24x^{6}+6x^{7} $ 

k8c19: $ -11+10x+20x^{2}-10x^{3}-12x^{4}+2x^{5}+2x^{6} $ 

k8c20: $ -7+12x+12x^{2}-14x^{3}-8x^{4}+4x^{5}+2x^{6} $ 

k8c21: $ -7+8x+6x^{2}-12x^{3}-2x^{4}+6x^{5}+2x^{6} $ 

k9c1: $ 9-4x-32x^{2}+14x^{3}+34x^{4}-10x^{5}-14x^{6}+2x^{7}+2x^{8} $ 

k9c2: $ 1-8x-12x^{2}+24x^{3}+18x^{4}-18x^{5}-10x^{6}+4x^{7}+2x^{8} $ 

k9c3: $ 5-6x-12x^{2}+16x^{3}+14x^{4}-14x^{5}-8x^{6}+4x^{7}+2x^{8} $ 

k9c4: $ 5-2x-18x^{2}+12x^{3}+18x^{4}-12x^{5}-8x^{6}+4x^{7}+2x^{8} $ 

k9c5: $ 1-12x+2x^{2}+28x^{3}-22x^{5}-4x^{6}+6x^{7}+2x^{8} $ 

k9c6: $ -3-2x+4x^{2}+14x^{3}-2x^{4}-16x^{5}-2x^{6}+6x^{7}+2x^{8} $ 

k9c7: $ 5-6x-8x^{2}+14x^{3}+4x^{4}-14x^{5}-2x^{6}+6x^{7}+2x^{8} $ 

k9c8: $ 1-8x+8x^{2}+22x^{3}-12x^{4}-22x^{5}+2x^{6}+8x^{7}+2x^{8} $ 

k9c9: $ 1+4x^{3}-2x^{4}-10x^{5}+6x^{7}+2x^{8} $ 

k9c10: $ 1-8x^{2}+4x^{3}+4x^{4}-8x^{5}+6x^{7}+2x^{8} $ 

k9c11: $ -7+18x^{2}+12x^{3}-18x^{4}-18x^{5}+4x^{6}+8x^{7}+2x^{8} $ 

k9c12: $ -3-6x+10x^{2}+14x^{3}-12x^{4}-16x^{5}+4x^{6}+8x^{7}+2x^{8} $ 

k9c13: $ -3-2x+12x^{2}+6x^{3}-16x^{4}-12x^{5}+6x^{6}+8x^{7}+2x^{8} $ 

k9c14: $ -3-10x+20x^{2}+24x^{3}-26x^{4}-24x^{5}+8x^{6}+10x^{7}+2x^{8} $ 

k9c15: $ 1+4x-2x^{2}-2x^{3}-8x^{4}-8x^{5}+6x^{6}+8x^{7}+2x^{8} $ 

k9c16: $ -7+12x+16x^{2}-12x^{3}-20x^{4}-6x^{5}+8x^{6}+8x^{7}+2x^{8} $ 

k9c17: $ -7+4x+24x^{2}+6x^{3}-30x^{4}-18x^{5}+10x^{6}+10x^{7}+2x^{8} $ 

k9c18: $ 1+4x+2x^{2}-8x^{3}-12x^{4}-4x^{5}+8x^{6}+8x^{7}+2x^{8} $ 

k9c19: $ 1-4x+10x^{2}+10x^{3}-22x^{4}-16x^{5}+10x^{6}+10x^{7}+2x^{8} $ 

k9c20: $ -7+4x+28x^{2}-34x^{4}-14x^{5}+12x^{6}+10x^{7}+2x^{8} $ 

k9c21: $ -3-2x+16x^{2}+4x^{3}-26x^{4}-12x^{5}+12x^{6}+10x^{7}+2x^{8} $ 

k9c22: $ -11-2x+42x^{2}+12x^{3}-48x^{4}-22x^{5}+16x^{6}+12x^{7}+2x^{8} $ 

k9c23: $ -3+10x+14x^{2}-12x^{3}-28x^{4}-6x^{5}+14x^{6}+10x^{7}+2x^{8} $ 

k9c24: $ -11+10x+24x^{2}-12x^{3}-30x^{4}-6x^{5}+14x^{6}+10x^{7}+2x^{8} $ 

k9c25: $ -7+30x^{2}+2x^{3}-42x^{4}-14x^{5}+18x^{6}+12x^{7}+2x^{8} $ 

k9c26: $ -7+30x^{2}+2x^{3}-42x^{4}-14x^{5}+18x^{6}+12x^{7}+2x^{8} $ 

k9c27: $ -7+8x+30x^{2}-12x^{3}-44x^{4}-8x^{5}+20x^{6}+12x^{7}+2x^{8} $ 

k9c28: $ -11+18x+32x^{2}-26x^{3}-46x^{4}-2x^{5}+22x^{6}+12x^{7}+2x^{8} $ 

k9c29: $ -11+2x+40x^{2}+18x^{3}-48x^{4}-36x^{5}+14x^{6}+18x^{7}+4x^{8} $ 

k9c30: $ -11+6x+42x^{2}-10x^{3}-58x^{4}-10x^{5}+26x^{6}+14x^{7}+2x^{8} $ 

k9c31: $ -7+12x+36x^{2}-22x^{3}-58x^{4}-4x^{5}+28x^{6}+14x^{7}+2x^{8} $ 

k9c32: $ -3-2x+28x^{2}+10x^{3}-50x^{4}-28x^{5}+22x^{6}+20x^{7}+4x^{8} $ 

k9c33: $ -3+2x+26x^{2}-50x^{4}-22x^{5}+24x^{6}+20x^{7}+4x^{8} $ 

k9c34: $ -3-2x+28x^{2}+18x^{3}-58x^{4}-42x^{5}+26x^{6}+28x^{7}+6x^{8} $ 

k9c35: $ -3-18x+28x^{2}+34x^{3}-24x^{4}-28x^{5}+2x^{6}+8x^{7}+2x^{8} $ 

k9c36: $ -11-2x+38x^{2}+14x^{3}-38x^{4}-22x^{5}+10x^{6}+10x^{7}+2x^{8} $ 

k9c37: $ -3-14x+30x^{2}+18x^{3}-40x^{4}-20x^{5}+16x^{6}+12x^{7}+2x^{8} $ 

k9c38: $ -7+4x+24x^{2}+2x^{3}-40x^{4}-22x^{5}+18x^{6}+18x^{7}+4x^{8} $ 

k9c39: $ -7-4x+28x^{2}+14x^{3}-40x^{4}-28x^{5}+16x^{6}+18x^{7}+4x^{8} $ 

k9c40: $ 5-2x+14x^{2}+24x^{3}-52x^{4}-54x^{5}+24x^{6}+34x^{7}+8x^{8} $ 

k9c41: $ -7-8x+38x^{2}+32x^{3}-46x^{4}-42x^{5}+12x^{6}+18x^{7}+4x^{8} $ 

k9c42: $ -7-4x+24x^{2}+12x^{3}-20x^{4}-10x^{5}+4x^{6}+2x^{7} $ 

k9c43: $ -11+2x+32x^{2}+2x^{3}-26x^{4}-6x^{5}+6x^{6}+2x^{7} $ 

k9c44: $ -7+22x^{2}+2x^{3}-20x^{4}-4x^{5}+6x^{6}+2x^{7} $ 

k9c45: $ -7+4x+20x^{2}-8x^{3}-20x^{4}+2x^{5}+8x^{6}+2x^{7} $ 

k9c46: $ 1-12x+18x^{2}+16x^{3}-18x^{4}-10x^{5}+4x^{6}+2x^{7} $ 

k9c47: $ -3-10x+28x^{2}+8x^{3}-32x^{4}-6x^{5}+12x^{6}+4x^{7} $ 

k9c48: $ 5-10x+2x^{2}-10x^{4}+4x^{5}+8x^{6}+2x^{7} $ 

k9c49: $ -7-4x+16x^{2}-16x^{4}+2x^{5}+8x^{6}+2x^{7} $ 

k10c1: $ 1+8x-20x^{2}-24x^{3}+38x^{4}+26x^{5}-22x^{6}-12x^{7}+4x^{8}+2x^{9} $ 

k10c2: $ 9-4x-40x^{2}+2x^{3}+58x^{4}+8x^{5}-32x^{6}-8x^{7}+6x^{8}+2x^{9} $ 

k10c3: $ 1+12x-10x^{2}-28x^{3}+22x^{4}+24x^{5}-16x^{6}-10x^{7}+4x^{8}+2x^{9} $ 

k10c4: $ 5-2x-30x^{2}+6x^{3}+46x^{4}+2x^{5}-28x^{6}-6x^{7}+6x^{8}+2x^{9} $ 

k10c5: $ 9-8x-42x^{2}+22x^{3}+60x^{4}-12x^{5}-36x^{6}-2x^{7}+8x^{8}+2x^{9} $ 

k10c6: $ -3+6x-4x^{2}-12x^{3}+20x^{4}+8x^{5}-18x^{6}-4x^{7}+6x^{8}+2x^{9} $ 

k10c7: $ 5-10x-18x^{2}+22x^{3}+30x^{4}-16x^{5}-24x^{6}+2x^{7}+8x^{8}+2x^{9} $ 

k10c8: $ 5+2x-24x^{2}-4x^{3}+40x^{4}+6x^{5}-26x^{6}-6x^{7}+6x^{8}+2x^{9} $ 

k10c9: $ 9-4x-36x^{2}+16x^{3}+48x^{4}-12x^{5}-30x^{6}+8x^{8}+2x^{9} $ 

k10c10: $ 5-14x-28x^{2}+42x^{3}+46x^{4}-34x^{5}-34x^{6}+6x^{7}+10x^{8}+2x^{9} $ 

k10c11: $ -3+6x-14x^{3}+10x^{4}+8x^{5}-12x^{6}-2x^{7}+6x^{8}+2x^{9} $ 

k10c12: $ 1-12x^{2}+2x^{3}+24x^{4}-6x^{5}-20x^{6}+2x^{7}+8x^{8}+2x^{9} $ 

k10c13: $ -3-2x+8x^{2}+8x^{3}-2x^{4}-14x^{5}-10x^{6}+6x^{7}+8x^{8}+2x^{9} $ 

k10c14: $ 1-4x-10x^{2}+20x^{3}+16x^{4}-26x^{5}-18x^{6}+10x^{7}+10x^{8}+2x^{9} $ 

k10c15: $ -3-2x+12x^{3}+18x^{4}-14x^{5}-22x^{6}+2x^{7}+8x^{8}+2x^{9} $ 

k10c16: $ 1-8x-4x^{2}+20x^{3}+14x^{4}-18x^{5}-18x^{6}+4x^{7}+8x^{8}+2x^{9} $ 

k10c17: $ 9-4x-32x^{2}+10x^{3}+44x^{4}-8x^{5}-28x^{6}+8x^{8}+2x^{9} $ 

k10c18: $ 1-12x-6x^{2}+32x^{3}+16x^{4}-32x^{5}-20x^{6}+10x^{7}+10x^{8}+2x^{9} $ 

k10c19: $ 5-6x-20x^{2}+28x^{3}+30x^{4}-30x^{5}-26x^{6}+8x^{7}+10x^{8}+2x^{9} $ 

k10c20: $ 5+2x-16x^{2}-4x^{3}+26x^{4}+4x^{5}-20x^{6}-4x^{7}+6x^{8}+2x^{9} $ 

k10c21: $ 5+2x-20x^{2}+6x^{3}+28x^{4}-10x^{5}-22x^{6}+2x^{7}+8x^{8}+2x^{9} $ 

k10c22: $ 1-4x^{2}+6x^{3}+12x^{4}-12x^{5}-16x^{6}+4x^{7}+8x^{8}+2x^{9} $ 

k10c23: $ 5-2x-14x^{2}+10x^{3}+16x^{4}-20x^{5}-16x^{6}+10x^{7}+10x^{8}+2x^{9} $ 

k10c24: $ 1+4x+2x^{2}-4x^{3}-2x^{4}-8x^{5}-8x^{6}+6x^{7}+8x^{8}+2x^{9} $ 

k10c25: $ 1+8x^{2}+8x^{3}-14x^{4}-24x^{5}-4x^{6}+14x^{7}+10x^{8}+2x^{9} $ 

k10c26: $ 5-6x-8x^{2}+18x^{3}+6x^{4}-26x^{5}-12x^{6}+12x^{7}+10x^{8}+2x^{9} $ 

k10c27: $ 1-4x+6x^{2}+20x^{3}-12x^{4}-38x^{5}-6x^{6}+20x^{7}+12x^{8}+2x^{9} $ 

k10c28: $ -3-10x+8x^{2}+34x^{3}+6x^{4}-36x^{5}-20x^{6}+10x^{7}+10x^{8}+2x^{9} $ 

k10c29: $ -7+18x^{2}+16x^{3}-16x^{4}-30x^{5}-6x^{6}+14x^{7}+10x^{8}+2x^{9} $ 

k10c30: $ -3-14x+18x^{2}+44x^{3}-16x^{4}-52x^{5}-10x^{6}+20x^{7}+12x^{8}+2x^{9} $ 

k10c31: $ 1-4x-10x^{2}+20x^{3}+16x^{4}-26x^{5}-18x^{6}+10x^{7}+10x^{8}+2x^{9} $ 

k10c32: $ -3-2x+12x^{2}+26x^{3}-14x^{4}-44x^{5}-8x^{6}+20x^{7}+12x^{8}+2x^{9} $ 

k10c33: $ 1-16x+44x^{3}+4x^{4}-48x^{5}-16x^{6}+18x^{7}+12x^{8}+2x^{9} $ 

k10c34: $ 5-10x-22x^{2}+24x^{3}+40x^{4}-16x^{5}-30x^{6}+8x^{8}+2x^{9} $ 

k10c35: $ 1-4x^{2}+6x^{3}+12x^{4}-12x^{5}-16x^{6}+4x^{7}+8x^{8}+2x^{9} $ 

k10c36: $ 5-6x-20x^{2}+28x^{3}+30x^{4}-30x^{5}-26x^{6}+8x^{7}+10x^{8}+2x^{9} $ 

k10c37: $ -3+6x-10x^{3}+8x^{4}-2x^{5}-12x^{6}+4x^{7}+8x^{8}+2x^{9} $ 

k10c38: $ -3-2x+12x^{2}+18x^{3}-6x^{4}-30x^{5}-12x^{6}+12x^{7}+10x^{8}+2x^{9} $ 

k10c39: $ -3+2x+10x^{2}+8x^{3}-6x^{4}-24x^{5}-10x^{6}+12x^{7}+10x^{8}+2x^{9} $ 

k10c40: $ -3+6x+16x^{2}+6x^{3}-28x^{4}-34x^{5}+2x^{6}+22x^{7}+12x^{8}+2x^{9} $ 

k10c41: $ -7-4x+32x^{2}+28x^{3}-34x^{4}-48x^{5}-2x^{6}+22x^{7}+12x^{8}+2x^{9} $ 

k10c42: $ -7+30x^{2}+26x^{3}-42x^{4}-56x^{5}+4x^{6}+30x^{7}+14x^{8}+2x^{9} $ 

k10c43: $ -7-4x+28x^{2}+22x^{3}-32x^{4}-42x^{5}+22x^{7}+12x^{8}+2x^{9} $ 

k10c44: $ -7-8x+38x^{2}+40x^{3}-46x^{4}-64x^{5}+2x^{6}+30x^{7}+14x^{8}+2x^{9} $ 

k10c45: $ -7-12x+48x^{2}+50x^{3}-68x^{4}-80x^{5}+12x^{6}+40x^{7}+16x^{8}+2x^{9} $ 

k10c46: $ 17-16x-52x^{2}+32x^{3}+66x^{4}-16x^{5}-38x^{6}-2x^{7}+8x^{8}+2x^{9} $ 

k10c47: $ 17-20x-50x^{2}+46x^{3}+64x^{4}-32x^{5}-40x^{6}+4x^{7}+10x^{8}+2x^{9} $ 

k10c48: $ 17-16x-48x^{2}+34x^{3}+54x^{4}-26x^{5}-32x^{6}+6x^{7}+10x^{8}+2x^{9} $ 

k10c49: $ 13-18x-32x^{2}+46x^{3}+36x^{4}-44x^{5}-28x^{6}+14x^{7}+12x^{8}+2x^{9} $ 

k10c50: $ 5-18x-10x^{2}+44x^{3}+18x^{4}-38x^{5}-22x^{6}+10x^{7}+10x^{8}+2x^{9} $ 

k10c51: $ 5-14x-8x^{2}+36x^{3}+6x^{4}-42x^{5}-14x^{6}+18x^{7}+12x^{8}+2x^{9} $ 

k10c52: $ 5-18x-6x^{2}+54x^{3}+14x^{4}-54x^{5}-24x^{6}+16x^{7}+12x^{8}+2x^{9} $ 

k10c53: $ 1-20x+10x^{2}+58x^{3}-12x^{4}-66x^{5}-12x^{6}+26x^{7}+14x^{8}+2x^{9} $ 

k10c54: $ 1-16x-4x^{2}+46x^{3}+22x^{4}-40x^{5}-28x^{6}+8x^{7}+10x^{8}+2x^{9} $ 

k10c55: $ 5-14x-8x^{2}+44x^{3}+14x^{4}-48x^{5}-22x^{6}+16x^{7}+12x^{8}+2x^{9} $ 

k10c56: $ 1-16x+44x^{3}+4x^{4}-48x^{5}-16x^{6}+18x^{7}+12x^{8}+2x^{9} $ 

k10c57: $ 1-8x+12x^{2}+32x^{3}-24x^{4}-54x^{5}-2x^{6}+28x^{7}+14x^{8}+2x^{9} $ 

k10c58: $ -7-16x+26x^{2}+52x^{3}-18x^{4}-58x^{5}-12x^{6}+20x^{7}+12x^{8}+2x^{9} $ 

k10c59: $ -11-10x+34x^{2}+50x^{3}-32x^{4}-68x^{5}-6x^{6}+28x^{7}+14x^{8}+2x^{9} $ 

k10c60: $ -11-18x+50x^{2}+64x^{3}-58x^{4}-86x^{5}+4x^{6}+38x^{7}+16x^{8}+2x^{9} $ 

k10c61: $ 9-16x-34x^{2}+40x^{3}+50x^{4}-24x^{5}-34x^{6}+8x^{8}+2x^{9} $ 

k10c62: $ 13-14x-38x^{2}+42x^{3}+48x^{4}-34x^{5}-34x^{6}+6x^{7}+10x^{8}+2x^{9} $ 

k10c63: $ 9-20x-28x^{2}+56x^{3}+36x^{4}-50x^{5}-30x^{6}+14x^{7}+12x^{8}+2x^{9} $ 

k10c64: $ 13-14x-38x^{2}+38x^{3}+42x^{4}-32x^{5}-28x^{6}+8x^{7}+10x^{8}+2x^{9} $ 

k10c65: $ 9-16x-26x^{2}+44x^{3}+26x^{4}-44x^{5}-22x^{6}+16x^{7}+12x^{8}+2x^{9} $ 

k10c66: $ 5-10x-2x^{2}+42x^{3}-8x^{4}-58x^{5}-10x^{6}+26x^{7}+14x^{8}+2x^{9} $ 

k10c67: $ 1-16x+52x^{3}+4x^{4}-54x^{5}-18x^{6}+18x^{7}+12x^{8}+2x^{9} $ 

k10c68: $ 1-20x-2x^{2}+64x^{3}+14x^{4}-60x^{5}-26x^{6}+16x^{7}+12x^{8}+2x^{9} $ 

k10c69: $ -7-12x+36x^{2}+52x^{3}-52x^{4}-78x^{5}+6x^{6}+38x^{7}+16x^{8}+2x^{9} $ 

k10c70: $ -11+2x+36x^{2}+8x^{3}-36x^{4}-28x^{5}+2x^{6}+16x^{7}+10x^{8}+2x^{9} $ 

k10c71: $ -11+2x+40x^{2}+10x^{3}-48x^{4}-38x^{5}+8x^{6}+24x^{7}+12x^{8}+2x^{9} $ 

k10c72: $ -7-4x+28x^{2}+22x^{3}-32x^{4}-42x^{5}+22x^{7}+12x^{8}+2x^{9} $ 

k10c73: $ -11-6x+48x^{2}+32x^{3}-60x^{4}-60x^{5}+10x^{6}+32x^{7}+14x^{8}+2x^{9} $ 

k10c74: $ 1-16x+16x^{2}+28x^{3}-16x^{4}-32x^{5}-6x^{6}+14x^{7}+10x^{8}+2x^{9} $ 

k10c75: $ -7-16x+54x^{2}+46x^{3}-64x^{4}-68x^{5}+8x^{6}+32x^{7}+14x^{8}+2x^{9} $ 

k10c76: $ -7+12x+8x^{2}-16x^{3}-4x^{4}-2x^{5}-6x^{6}+6x^{7}+8x^{8}+2x^{9} $ 

k10c77: $ -7+8x+10x^{2}-2x^{3}-6x^{4}-18x^{5}-8x^{6}+12x^{7}+10x^{8}+2x^{9} $ 

k10c78: $ -11+6x+30x^{2}+16x^{3}-26x^{4}-42x^{5}-6x^{6}+20x^{7}+12x^{8}+2x^{9} $ 

k10c79: $ 21-22x-52x^{2}+46x^{3}+48x^{4}-40x^{5}-28x^{6}+14x^{7}+12x^{8}+2x^{9} $ 

k10c80: $ 9-20x-12x^{2}+56x^{3}+8x^{4}-62x^{5}-18x^{6}+24x^{7}+14x^{8}+2x^{9} $ 

k10c81: $ -3-18x+24x^{2}+56x^{3}-36x^{4}-76x^{5}+36x^{7}+16x^{8}+2x^{9} $ 

k10c82: $ 1-24x^{2}+16x^{3}+50x^{4}-22x^{5}-44x^{6}+4x^{7}+16x^{8}+4x^{9} $ 

k10c83: $ 5-10x-2x^{2}+38x^{3}+2x^{4}-56x^{5}-24x^{6}+24x^{7}+20x^{8}+4x^{9} $ 

k10c84: $ -7+4x+12x^{2}+16x^{3}-6x^{4}-46x^{5}-20x^{6}+24x^{7}+20x^{8}+4x^{9} $ 

k10c85: $ 1-4x-26x^{2}+28x^{3}+60x^{4}-28x^{5}-52x^{6}+2x^{7}+16x^{8}+4x^{9} $ 

k10c86: $ 5-14x-12x^{2}+42x^{3}+18x^{4}-54x^{5}-30x^{6}+22x^{7}+20x^{8}+4x^{9} $ 

k10c87: $ -7+14x^{2}+34x^{3}+2x^{4}-58x^{5}-30x^{6}+22x^{7}+20x^{8}+4x^{9} $ 

k10c88: $ -3-10x+28x^{2}+48x^{3}-40x^{4}-84x^{5}-8x^{6}+42x^{7}+24x^{8}+4x^{9} $ 

k10c89: $ -3-6x+18x^{2}+46x^{3}-26x^{4}-82x^{5}-14x^{6}+40x^{7}+24x^{8}+4x^{9} $ 

k10c90: $ -3-6x+6x^{2}+28x^{3}+8x^{4}-42x^{5}-28x^{6}+16x^{7}+18x^{8}+4x^{9} $ 

k10c91: $ 9-12x-32x^{2}+32x^{3}+46x^{4}-36x^{5}-40x^{6}+12x^{7}+18x^{8}+4x^{9} $ 

k10c92: $ 1-12x+6x^{2}+50x^{3}-2x^{4}-72x^{5}-26x^{6}+30x^{7}+22x^{8}+4x^{9} $ 

k10c93: $ -3-14x+2x^{2}+52x^{3}+28x^{4}-54x^{5}-44x^{6}+12x^{7}+18x^{8}+4x^{9} $ 

k10c94: $ 9-12x-28x^{2}+38x^{3}+44x^{4}-42x^{5}-42x^{6}+12x^{7}+18x^{8}+4x^{9} $ 

k10c95: $ 5-10x-6x^{2}+40x^{3}+4x^{4}-64x^{5}-24x^{6}+30x^{7}+22x^{8}+4x^{9} $ 

k10c96: $ -11-6x+36x^{2}+46x^{3}-34x^{4}-76x^{5}-14x^{6}+34x^{7}+22x^{8}+4x^{9} $ 

k10c97: $ -7-12x+20x^{2}+60x^{3}-8x^{4}-80x^{5}-28x^{6}+30x^{7}+22x^{8}+4x^{9} $ 

k10c98: $ 9-24x-6x^{2}+56x^{3}+14x^{4}-62x^{5}-32x^{6}+22x^{7}+20x^{8}+4x^{9} $ 

k10c99: $ 17-24x-32x^{2}+48x^{3}+36x^{4}-52x^{5}-36x^{6}+20x^{7}+20x^{8}+4x^{9} $ 

k10c100: $ 9-16x-26x^{2}+52x^{3}+50x^{4}-50x^{5}-50x^{6}+10x^{7}+18x^{8}+4x^{9} $ 

k10c101: $ 5-18x-2x^{2}+64x^{3}+10x^{4}-78x^{5}-34x^{6}+28x^{7}+22x^{8}+4x^{9} $ 

k10c102: $ 1-12x-6x^{2}+40x^{3}+24x^{4}-46x^{5}-36x^{6}+14x^{7}+18x^{8}+4x^{9} $ 

k10c103: $ -3-10x-8x^{2}+34x^{3}+26x^{4}-40x^{5}-34x^{6}+14x^{7}+18x^{8}+4x^{9} $ 

k10c104: $ 5-6x-20x^{2}+20x^{3}+30x^{4}-32x^{5}-32x^{6}+14x^{7}+18x^{8}+4x^{9} $ 

k10c105: $ -3-10x+20x^{2}+48x^{3}-18x^{4}-74x^{5}-20x^{6}+32x^{7}+22x^{8}+4x^{9} $ 

k10c106: $ 5-2x-18x^{2}+24x^{3}+28x^{4}-38x^{5}-34x^{6}+14x^{7}+18x^{8}+4x^{9} $ 

k10c107: $ -3-6x+10x^{2}+38x^{3}-12x^{4}-66x^{5}-18x^{6}+32x^{7}+22x^{8}+4x^{9} $ 

k10c108: $ 1-16x-16x^{2}+60x^{3}+48x^{4}-56x^{5}-52x^{6}+10x^{7}+18x^{8}+4x^{9} $ 

k10c109: $ 13-10x-28x^{2}+30x^{3}+24x^{4}-46x^{5}-28x^{6}+22x^{7}+20x^{8}+4x^{9} $ 

k10c110: $ -3-14x+14x^{2}+50x^{3}-4x^{4}-64x^{5}-26x^{6}+24x^{7}+20x^{8}+4x^{9} $ 

k10c111: $ 5-22x-12x^{2}+64x^{3}+28x^{4}-66x^{5}-40x^{6}+20x^{7}+20x^{8}+4x^{9} $ 

k10c112: $ -7+4x-4x^{2}+24x^{3}+38x^{4}-48x^{5}-54x^{6}+16x^{7}+26x^{8}+6x^{9} $ 

k10c113: $ -7+22x^{2}+42x^{3}-20x^{4}-90x^{5}-28x^{6}+44x^{7}+32x^{8}+6x^{9} $ 

k10c114: $ -3-6x-6x^{2}+46x^{3}+32x^{4}-68x^{5}-52x^{6}+24x^{7}+28x^{8}+6x^{9} $ 

k10c115: $ 5-14x-4x^{2}+58x^{3}+4x^{4}-92x^{5}-36x^{6}+42x^{7}+32x^{8}+6x^{9} $ 

k10c116: $ 1-8x+46x^{3}+18x^{4}-70x^{5}-46x^{6}+26x^{7}+28x^{8}+6x^{9} $ 

k10c117: $ 1-12x-2x^{2}+50x^{3}+12x^{4}-78x^{5}-40x^{6}+34x^{7}+30x^{8}+6x^{9} $ 

k10c118: $ 1-8x-8x^{2}+38x^{3}+24x^{4}-62x^{5}-44x^{6}+26x^{7}+28x^{8}+6x^{9} $ 

k10c119: $ -3-10x+12x^{2}+56x^{3}+4x^{4}-86x^{5}-42x^{6}+34x^{7}+30x^{8}+6x^{9} $ 

k10c120: $ 1-12x+2x^{2}+68x^{3}+8x^{4}-100x^{5}-44x^{6}+40x^{7}+32x^{8}+6x^{9} $ 

k10c121: $ 5-6x-12x^{2}+44x^{3}+24x^{4}-84x^{5}-54x^{6}+38x^{7}+38x^{8}+8x^{9} $ 

k10c122: $ -7+4x+4x^{2}+40x^{3}+32x^{4}-78x^{5}-66x^{6}+28x^{7}+36x^{8}+8x^{9} $ 

k10c123: $ -7-4x+24x^{2}+52x^{3}-12x^{4}-104x^{5}-44x^{6}+48x^{7}+40x^{8}+8x^{9} $ 

k10c124: $ 17-16x-44x^{2}+28x^{3}+42x^{4}-14x^{5}-16x^{6}+2x^{7}+2x^{8} $ 

k10c125: $ 13-18x-28x^{2}+36x^{3}+26x^{4}-22x^{5}-12x^{6}+4x^{7}+2x^{8} $ 

k10c126: $ 13-14x-30x^{2}+26x^{3}+26x^{4}-16x^{5}-10x^{6}+4x^{7}+2x^{8} $ 

k10c127: $ 13-14x-26x^{2}+24x^{3}+16x^{4}-16x^{5}-4x^{6}+6x^{7}+2x^{8} $ 

k10c128: $ 5-10x-10x^{2}+26x^{3}+14x^{4}-20x^{5}-10x^{6}+4x^{7}+2x^{8} $ 

k10c129: $ 1-12x-2x^{2}+26x^{3}+4x^{4}-20x^{5}-4x^{6}+6x^{7}+2x^{8} $ 

k10c130: $ 1-16x+4x^{2}+38x^{3}-28x^{5}-6x^{6}+6x^{7}+2x^{8} $ 

k10c131: $ 1-8x+8x^{2}+22x^{3}-12x^{4}-22x^{5}+2x^{6}+8x^{7}+2x^{8} $ 

k10c132: $ 5-18x-14x^{2}+38x^{3}+20x^{4}-24x^{5}-12x^{6}+4x^{7}+2x^{8} $ 

k10c133: $ 5-14x-4x^{2}+34x^{3}+4x^{4}-26x^{5}-6x^{6}+6x^{7}+2x^{8} $ 

k10c134: $ 1-12x+2x^{2}+28x^{3}-22x^{5}-4x^{6}+6x^{7}+2x^{8} $ 

k10c135: $ 5-10x-6x^{2}+16x^{3}-4x^{4}-14x^{5}+4x^{6}+8x^{7}+2x^{8} $ 

k10c136: $ -7-8x+14x^{2}+32x^{3}-4x^{4}-28x^{5}-6x^{6}+6x^{7}+2x^{8} $ 

k10c137: $ -7-12x+24x^{2}+34x^{3}-18x^{4}-30x^{5}+8x^{7}+2x^{8} $ 

k10c138: $ -11-6x+36x^{2}+22x^{3}-34x^{4}-26x^{5}+8x^{6}+10x^{7}+2x^{8} $ 

k10c139: $ 13-14x-42x^{2}+28x^{3}+42x^{4}-14x^{5}-16x^{6}+2x^{7}+2x^{8} $ 

k10c140: $ 9-12x-24x^{2}+32x^{3}+24x^{4}-22x^{5}-12x^{6}+4x^{7}+2x^{8} $ 

k10c141: $ 5-10x-10x^{2}+30x^{3}+8x^{4}-24x^{5}-6x^{6}+6x^{7}+2x^{8} $ 

k10c142: $ 9-8x-22x^{2}+24x^{3}+20x^{4}-18x^{5}-10x^{6}+4x^{7}+2x^{8} $ 

k10c143: $ 5-10x-14x^{2}+24x^{3}+10x^{4}-18x^{5}-4x^{6}+6x^{7}+2x^{8} $ 

k10c144: $ 9-4x-20x^{2}+8x^{3}+4x^{4}-10x^{5}+4x^{6}+8x^{7}+2x^{8} $ 

k10c145: $ 5-14x-16x^{2}+36x^{3}+20x^{4}-24x^{5}-12x^{6}+4x^{7}+2x^{8} $ 

k10c146: $ 1-8x+22x^{3}-6x^{4}-20x^{5}+2x^{6}+8x^{7}+2x^{8} $ 

k10c147: $ -3-10x+12x^{2}+32x^{3}-12x^{4}-28x^{5}+8x^{7}+2x^{8} $ 

k10c148: $ 9-8x-18x^{2}+14x^{3}+10x^{4}-12x^{5}-2x^{6}+6x^{7}+2x^{8} $ 

k10c149: $ 9-4x-16x^{2}+2x^{3}-6x^{5}+6x^{6}+8x^{7}+2x^{8} $ 

k10c150: $ -3-6x+18x^{2}+22x^{3}-18x^{4}-24x^{5}+2x^{6}+8x^{7}+2x^{8} $ 

k10c151: $ -3-2x+16x^{2}+4x^{3}-26x^{4}-12x^{5}+12x^{6}+10x^{7}+2x^{8} $ 

k10c152: $ 21-18x-54x^{2}+28x^{3}+48x^{4}-12x^{5}-16x^{6}+2x^{7}+2x^{8} $ 

k10c153: $ 9-16x-18x^{2}+36x^{3}+20x^{4}-24x^{5}-12x^{6}+4x^{7}+2x^{8} $ 

k10c154: $ -3-14x+14x^{2}+38x^{3}-6x^{4}-30x^{5}-6x^{6}+6x^{7}+2x^{8} $ 

k10c155: $ 9-4x-12x^{2}+16x^{3}+6x^{4}-18x^{5}-4x^{6}+6x^{7}+2x^{8} $ 

k10c156: $ -3-6x+10x^{2}+14x^{3}-12x^{4}-16x^{5}+4x^{6}+8x^{7}+2x^{8} $ 

k10c157: $ 1+8x+4x^{2}-20x^{3}-22x^{4}+2x^{5}+16x^{6}+10x^{7}+2x^{8} $ 

k10c158: $ -3+2x+6x^{2}-2x^{3}-18x^{4}-8x^{5}+12x^{6}+10x^{7}+2x^{8} $ 

k10c159: $ 1+4x-2x^{2}-2x^{3}-8x^{4}-8x^{5}+6x^{6}+8x^{7}+2x^{8} $ 

k10c160: $ -3-2x+8x^{2}+20x^{3}-4x^{4}-22x^{5}-4x^{6}+6x^{7}+2x^{8} $ 

k10c161: $ -3+6x+12x^{2}-8x^{3}-10x^{4}+2x^{5}+2x^{6} $ 

k10c162: $ -3+6x+12x^{2}-8x^{3}-10x^{4}+2x^{5}+2x^{6} $ 

k10c163: $ 5-14x-8x^{2}+24x^{3}-18x^{5}+2x^{6}+8x^{7}+2x^{8} $ 

k10c164: $ 5-6x-4x^{2}+20x^{3}-14x^{4}-28x^{5}+8x^{6}+16x^{7}+4x^{8} $ 

k10c165: $ 5-14x-12x^{2}+34x^{3}+2x^{4}-32x^{5}+14x^{7}+4x^{8} $ 

k10c166: $ 1-12x+6x^{2}+42x^{3}-10x^{4}-42x^{5}-2x^{6}+14x^{7}+4x^{8} $ 


%\newpage

\end{document}
